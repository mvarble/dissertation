\begin{color}{gray}
  Here I put a summary of chapter, along with a short history.
  It will include the following important notes.
  \begin{itemize}
    \item
      Chapter addresses important fundamental results of affine processes.
    \item
      Chapter addresses consequences of mgf results that are important for us, though not specified exactly much in the literature
  \end{itemize}
\end{color}


\section{Mogulskii's theorem}
% make Mogulskii understandable on our path space
A surprisingly powerful theorem in the theory of large deviations of stochastic processes is that of Mogulskii (see \cite[Theorems 5.1.2 and 5.1.19 and Exercise 5.122]{dembo2010}).
Fixing a family $(\V_j)_{j\in\bbN}$ of independent quantities distributing with common distribution $\affDist$ with light tails,
\begin{equation}
  \label{eq:cumulant-generating-function}
  \ode_\affDist(\momentVar) \defeq \log\int_\vecSpace e^\prj\momentVar\markVar \affDist(\rmd\markVar) < \infty, \quad \momentVar \in \vecSpace
\end{equation}
this theorem provides a large deviation principle for the laws associated to quantities $Y^\epsilon$ as below.
\begin{equation*}
  Y^\epsilon_t = \epsilon \sum_{j=1}^{[t/\epsilon]} \V_j, \quad t \in [0,\tau]
\end{equation*}
It states that the associated laws $(\Prb^\epsilon)_{\epsilon>0}$ satisfy a large deviation principle on the space $\bbL^\infty[0,\tau]$ of bounded functions $[0,\tau] \rightarrow \vecSpace$, equipped with the supremum norm.
The rate function, like ours, is an integral of the Fenchel-Legendre transform of $\ode_\affDist$.
\begin{equation*}
  \pathVar \mapsto \left\{\begin{array}{ll}
    \displaystyle\int_0^\tau \ode_\affDist^*\big(\dot\pathVar(t)\big) \rmd t & \pathVar(0) = 0, ~ \pathVar \in \acpathSpace{[0,\tau]}{\vecSpace}\\
    \infty & \text{otherwise}
  \end{array}\right.
\end{equation*}
Very minor adjustments can actually make this theorem similar to the context of our principle.
Firstly, the principle may be lifted to the space $\bbL^\infty_\loc[0,\infty)$ of locally bounded functions $[0,\infty) \rightarrow \vecSpace$, equipped with the weighted supremum norm,
\begin{equation*}
  (\pathVar, \pathVar') \mapsto \sup_{t\in [0,\infty)} e^{-t} |\pathVar(t) - \pathVar'(t)|,
\end{equation*}
for this metric is consistent with $\pathVar_n \rightarrow \pathVar$ if and only if $\pathVar_n|_{[0,\tau]} \rightarrow \pathVar|_{[0,\tau]}$ uniformly for all $\tau \geq 0$, which is the same as the projective limit space induced by the restriction maps.
\[
  (\pathVar_\tau)_{\tau > 0} \in \lim\limits_{\leftarrow\tau} \bbL^\infty[0,\tau] \quad\xleftrightarrow{\pathVar_\tau = \pathVar|_{[0,\tau]}}\quad \pathVar \in \bbL^\infty_\loc[0,\infty)
\]
Applying Dawson-G\"artner \cite[Theorem 4.6.1]{dembo2010}, the rate function over this space is as follows.
\[
  \pathVar \mapsto \left\{ \begin{array}{ll}
    \displaystyle\sup_{\tau>0} \int_0^\tau \ode_\affDist^*\big(\dot\pathVar(t)\big) \rmd t & \pathVar(0) = 0, ~ \pathVar \in \acpathSpace{[0,\tau)}{\vecSpace} \text{ for all } \tau > 0 \\
    \infty & \text{otherwise}
  \end{array}\right.
\]
From here, we recognize that each process $Y^\epsilon$ is c\`adl\`ag; if $\nu$ is supported on $\stateSpace$, the process takes values in $\pathSpace{[0,\infty)}{\stateSpace}$, and so we may restrict our principle (see \cite[Lemma 4.1.5(b)]{dembo2010}).
Our rate function then takes the same form (recall the local definition of absolute continuity $\acpathSpace{[0,\infty)}{\stateSpace}$).
\begin{equation}
  \label{eq:rf-mogulskii}
  \pathVar \mapsto \left\{\begin{array}{ll}
    \displaystyle\int_0^\infty \ode_\affDist^*\big(\dot\pathVar(t)\big) \rmd t & \pathVar(0) = 0, ~\pathVar \in \acpathSpace{[0,\infty)}{\stateSpace} \\
    \infty & \text{otherwise}
  \end{array}\right.
\end{equation}

% apply Mogulskii to get Brownian motion
\begin{example}[Brownian motion]
  \label{example:brownian}
  Applying Mogulskii's theorem when our increment distribution $\affDist$ is $\operatorname{Normal}(0,\id_\vecSpace)$, the integral in our rate function in (\ref{eq:rf-mogulskii}) becomes the following.
  \begin{equation}
    \label{eq:rf-brownian}
    \int_0^\infty \frac12\big|\dot\pathVar(t)\big|^2 \rmd t
  \end{equation}
  Furthermore, for a Brownian motion $\W$, the process $\sqrt\epsilon\W$ ends up being exponentially equivalent to $Y^\epsilon$,
  \begin{equation*}
    \limsup_{\epsilon\rightarrow0} \epsilon\log \Prb\big( |\sqrt\epsilon\W - Y^\epsilon| \geq \delta \big) = -\infty,
  \end{equation*}
  which makes the family $\sqrt\epsilon\W$ satisfy the large deviation principle with rate function (\ref{eq:rf-brownian}); this result is known as Schilder's theorem (see \cite[Theorem 5.2.3]{dembo2010}).

  Note that $(\sqrt\epsilon\W)_{\epsilon>0}$ is a family of affine processes covered Theorem \ref{theorem:ldp-integral}.
  We have $\Xe = \sqrt\epsilon\W$, where the base process $\X$ has special differential characteristics $(0, \id_\vecSpace, 0)$.
  The easiest way to see this is by considering Proposition \ref{proposition:sde-asymptotics} with initial state $\stateVar = 0$.
  Our theorem also immediately resolves (\ref{eq:rf-integral}) the same rate function.
  \begin{equation*}
    \ode^*(\diffVar, \stateVar)
    = \sup_{\momentVar \in \vecSpace} \bigg( \prj{\momentVar}{\diffVar} - \frac12\prj{\momentVar}{\id_\vecSpace\cdot \momentVar} \bigg) = \frac12 |\diffVar|
  \end{equation*}
\end{example}


% apply Mogulskii to get Poisson
\begin{example}[Poisson]
  \label{example:poisson}
  One may apply a very similar argument for when our increment distribution $\affDist$ is $\operatorname{Poisson}(1)$.
  In this case, the integral in the rate function in (\ref{eq:rf-mogulskii}) evaluates to
  \begin{equation}
    \label{eq:rf-poisson}
    \int_0^\infty \Big( \dot\pathVar(t)  \log\big(\dot\pathVar(t)\big) - \dot\pathVar(t) + 1 \Big) \rmd t,
  \end{equation}
  so long as $\pathVar(t) \geq 0$ for Lebesgue-almost-every $t \geq 0$ (otherwise, it is infinite).
  In the case that $\pathVar(t) = 0$, we are taking the continuous extension of the integrand, i.e.\ $0\log(0) \defeq 0$.
  Similar to the work of Schilder's theorem, we may show, for a standard Poisson process $\N$, $\epsilon\N_{\cdot/\epsilon}$ is exponentially equivalent to this $Y^\epsilon$, which makes the family satisfy a large deviation principle with rate function (\ref{eq:rf-poisson}).
  In fact and exercise of our reference text, \cite[Exercise 5.2.12]{dembo2010}, suggests the reader to show just this.

  Again, such a family $(\epsilon\N_{\cdot/\epsilon})_{\epsilon>0}$ is covered by Theorem \ref{theorem:ldp-integral}.
  To see this, consider a base affine process $\X$ on $(\bbR, \scrB(\bbR))$ with special differential characteristics as below, where $\delta_1$ denotes the degenerate distribution at $1 \in \bbR$.
  \begin{equation*}
    \affDrift(\stateVar) = 1, \quad \affDiff(\stateVar) = 0, \quad \affJump(\stateVar, \rmd\markVar) = \delta_1(\rmd\markVar)
  \end{equation*}
  Setting the initial state $\stateVar = 0$ and looking at Proposition \ref{proposition:sde-asymptotics}, we may say that $\Xe$ can be realized as follows.
  \begin{align*}
    \Xe_t 
    &= t + \epsilon 1_{[0,1]}(\id_\bbR) \ast \compensate{\poisRM^\epsilon}_t \\
    &= t + \epsilon 1_{[0,1]}(\id_\bbR) \ast \poisRM^\epsilon_t - \epsilon 1_{[0,1]}(\id_\bbR) \ast \predproj{\poisRM^\epsilon}_t \\
    &= t + \epsilon \poisRM([0,t/\epsilon] \times [0,1]) - \int_0^{t/\epsilon} \int_\bbR \epsilon 1_{[0,1]}(\markVar) \rmd\markVar \rmd s \\
    &= \epsilon \poisRM([0,t/\epsilon] \times [0,1])
  \end{align*}
  As stated in \cite[Theorem II.4.8]{jacod2003}, this Poisson random measure $\poisRM$ is a Poisson point process with Lebesgue intensity.
  This means that, for each $t \geq 0$, $\N_t \defeq \poisRM([0,t] \times [0,1])$ distributes $\operatorname{Poisson}(t)$, and $\N_t - \N_s = \poisRM((s,t] \times [0,1])$ is independent of $\N_s = \poisRM([0,s], [0,1])$ for each $0 \leq s < t$.
  In other words, $\N$ is a standard Poisson process and
  \begin{equation*}
    \Xe_t = \epsilon \poisRM([0,t/\epsilon] \times [0,1]) = \epsilon\N_{t/\epsilon}.
  \end{equation*}
  As with the normal increments, our rate function (\ref{eq:rf-integral}) resolves this immediately.
  \begin{align*}
    \ode^*(\diffVar, \stateVar) 
    &= \sup_{\momentVar\in\vecSpace} \bigg( \momentVar\diffVar - \momentVar - \int_\bbR \big( e^{\momentVar\markVar} - 1 - \momentVar\markVar \big) \delta_1(\rmd\markVar) \bigg)  \\
    &= \sup_{\momentVar\in\vecSpace} \bigg( \momentVar\diffVar - e^\momentVar + 1 \bigg) \\
    &= \left\{\begin{array}{ll}
      \diffVar \log\diffVar - \diffVar + 1, & \diffVar \geq 0 \\
      \infty, & \text{otherwise}
    \end{array}\right.
  \end{align*}
\end{example}



\section{Transformations}
% discuss how transforms + measure changes allow for state-dependent processes
While Mogulskii's theorem specifies that the processes $(Y^\epsilon)_{\epsilon>0}$---by design---have independent increments, we may use transformation arguments to produce large deviation principles for families of processes with state-dependent increments.
The two ways of leveraging this are via the contraction principle or measure-change arguments.

% discuss transforms
The contraction principle states that mapping the quantities $F(Y^\epsilon)$ via a continuous map $F$ produces a large deviation principle for the family $(\push{\Prb^\epsilon}{F})_{\epsilon>0}$ of measures $\push{\Prb^\epsilon}{F}$ associated with these respective quantities $F(Y^\epsilon)$ (see \cite[Theorem 4.2.1]{dembo2010}).
Seeing as this section serves as a survey for intuition on rate functions, we will digress from discussing the specifics of continuity of $F$ on restricted spaces and/or exponentially equivalent families in our example below.

% introduce drift to Brownian motion
\begin{example}[Diffusions]
  \label{example:diffusion}
  We can leverage Example \ref{example:brownian} to a family of processes $(\Xe)_{\epsilon>0}$,
  \begin{equation*}
    \Xe = \stateVar + \affDrift(\Xe) \shortInt \leb + \sqrt\epsilon \shortInt \W,
  \end{equation*}
  where the drift $\affDrift: \vecSpace \rightarrow \vecSpace$ is bounded and Lipschitz.
  Having a map $F_\affDrift$ which implicitly solves the equation,
  \begin{equation}
    \label{eq:F-brownian-drift}
    F_\affDrift(\omega) = \pathVar, \quad \pathVar(t) = \stateVar + \affDrift(\pathVar) \shortInt \leb_t + \omega_t,
  \end{equation}
  will make $F_\affDrift(\sqrt\epsilon\W) = \Xe$ for each $\epsilon > 0$, so the contraction principle states that the distributions of $(\Xe)_{\epsilon>0}$ satisfy a large deviation principle in which the rate function $\rf_\affDrift$ is derived from that $\rf_\W$ from Example \ref{example:brownian}.
  \begin{align*}
    \rf_\affDrift(\pathVar) &\defeq \inf\Big\{ \rf_\W(\omega) : F_\affDrift(\omega) = \pathVar \Big\},\\
    \rf_\W(\omega) &\defeq \left\{\begin{array}{ll}
      \displaystyle\int_0^\infty \frac12\big|\dot\omega(t)\big|^2 \rmd t, & \omega(0) = 0, ~ \omega \in \acpathSpace{[0,\infty)}{\vecSpace}, \\
      \infty, & \text{otherwise}
    \end{array}\right.
  \end{align*}
  When $F_\affDrift(\omega) = \pathVar$, equation (\ref{eq:F-brownian-drift}) tells us $\pathVar(0) = \stateVar$ and $\dot\omega = \dot\pathVar - \affDrift(\pathVar)$, and so we have the following.
  \begin{equation*}
    \rf_\affDrift(\pathVar) = \left\{\begin{array}{ll}
      \displaystyle \int_0^\infty \frac12 \Big|\dot\pathVar - \affDrift\big(\pathVar(t)\big)\Big| \rmd t, & \pathVar(0) = \stateVar, ~ \pathVar \in \acpathSpace{[0,\infty)}{\vecSpace}, \\
      \infty, & \text{otherwise}
    \end{array}\right.
  \end{equation*}
  Similarly, we may introduce a bounded, Lipschitz diffusion $\affDiff = \affDiffSqrt\affDiffSqrt^*: \vecSpace \rightarrow \bbL(\vecSpace)$ in which each $\affDiff(\stateVar)$ is invertible, so that the dynamics become as follows.
  \begin{equation}
    \label{eq:F-brownian-drift-diff}
    \Xe = \stateVar + \affDrift(\Xe) \shortInt \leb + \sqrt\epsilon \affDiffSqrt(\Xe) \shortInt \W,
  \end{equation}
  Having a map $F_{\affDrift,\affDiff}$ which implicitly solves the equation,
  \begin{equation*}
    F_{\affDrift,\affDiff}(\omega) = \pathVar, \quad \pathVar(t) = \affDrift(\pathVar) \shortInt \leb_t + \affDiffSqrt(\pathVar) \shortInt \omega_t,
  \end{equation*}
  will allow us to repeat the above argument to get a large deviation principle for $(\Xe)_{\epsilon>0}$ with rate function $\rf_{\affDrift,\affDiff}$; when $\pathVar(0) = \stateVar$, $\pathVar \in \acpathSpace{[0,\infty)}{\vecSpace}$, we get the following.
  \begin{equation*}
    \rf_{\affDrift,\affDiff}(\pathVar) = \int_0^\infty \frac12 \bprj{\Big(\dot\pathVar(t)-\affDrift\big(\pathVar(t)\big)\Big)}{\affDiff\big(\pathVar(t)\big)^{-1}\Big(\dot\pathVar(t)-\affDrift\big(\pathVar(t)\big)\Big)} \rmd t
  \end{equation*}

  The true details of this result, attributed to Freidlin-Wentzel \cite[Theorems 5.6.3 and 5.6.7]{dembo2010}, are rather complicated, and the above argument is just a heuristic.
  Also, note that this result does not apply to the general class of affine diffusions, as $\affDrift, \affDiff$ are generally not bounded or Lipschitz, and $\affDiff$ need not be invertible.
  However, \cite{kang2014}---a paper which inspires parts of our proof---first proved that affine (jump-)diffusions with special differential characteristics $(\affDrift, \affDiff, 0)$ satisfy a large deviation principle with rate function similar to that above.
  Our rate function (\ref{eq:rf-integral}) from Theorem \ref{theorem:ldp-integral} immediately resolves an identical representation.
  \begin{align*}
    \ode^*(\diffVar, \stateVar)
    &= \sup_{\momentVar \in \vecSpace} \bigg( \prj{\momentVar}{\diffVar} - \bprj{\momentVar}{\affDrift(\stateVar)} - \frac12\bprj{\momentVar}{\affDiff(\stateVar)\momentVar} \bigg) \\
    &= \left\{\begin{array}{ll}
      \displaystyle \frac12\Bprj{\big(\diffVar-\affDrift(\stateVar)\big)}{\affDiff(\stateVar)^\dagger\big(\diffVar-\affDrift(\stateVar)\big)}, & \diffVar - \affDrift(\stateVar) \in \operatorname{image}\affDiff(\stateVar), \\
      \infty, & \text{otherwise}
    \end{array}\right.
  \end{align*}
  Above, $a^\dagger \in \bbL(\vecSpace)$ denotes the pseudoinverse of $a \in \bbL(\vecSpace)$.
\end{example}


% discuss measure changes
The above result leveraged a large deviation principle for Brownian motions $(\sqrt\epsilon\W)_{\epsilon>0}$ to one on state-dependent diffusions $(\Xe)_{\epsilon>0}$ via mappings $\Xe = F(\sqrt\epsilon\W)$.
The analogue of the Brownian motion $\W$ for jump processes---in the sense of homogeneous independent-increments---is the Poisson process $\N$.
To introduce state-dependence to our sequence $(\epsilon\N_{\cdot/\epsilon})_{\epsilon>0}$ from Example \ref{example:poisson}, we may perform a simple measure change argument.

% introduce intensity to Poisson
\begin{example}[Continuous-branching/Hawkes]
  Consider our sequence $(\epsilon\N_{\cdot/\epsilon})_{\epsilon>0}$ derived from a Poisson process $\N$, as in Example \ref{example:poisson}.
  Denoting $\Prb^\epsilon$ the distribution of each $J^\epsilon \defeq \epsilon\N_{\cdot/\epsilon}$, we may construct a measure $\mgPrb^\epsilon \sim \Prb^\epsilon$ in which $\frac1\epsilon J^\epsilon$ has $\mgPrb^\epsilon$ intensity $\frac1\epsilon\affInt(J^\epsilon)$ for affine function $\affInt$.
  \begin{equation*}
    \affInt: \bbR_+ \rightarrow \bbR_+, \quad \affInt(\stateVar) = \affIntPart_0 + \affIntPart_1 \stateVar
  \end{equation*}
  The martingale that induces this change of measure is familiar from Theorem \ref{theorem:LK-exponential-martingale}, selecting $H = \log\affInt(J^\epsilon)$.
  The L\'evy-Khintchine map associated with $\frac1\epsilon J^\epsilon$ is $\ode(\momentVar, \stateVar) = \frac1\epsilon (e^\momentVar - 1)$, and it resolves to the following.
  \begin{equation*}
    \mgDensity^\epsilon \defeq \exp\bigg( \frac1\epsilon \log\affInt(J^\epsilon_-) \shortInt J^\epsilon + \frac1\epsilon\big( 1 - \affInt(J^\epsilon) \big) \shortInt \leb \bigg)
  \end{equation*}
  The associated measure $\mgPrb^\epsilon(\rmd\omega) \defeq \mgDensity^\epsilon(\omega) \cdot \Prb^\epsilon(\rmd\omega)$ makes $\frac1\epsilon J^\epsilon$ have the desired intensity $\frac1\epsilon\affInt(J^\epsilon)$.
  The large deviation principle associated with $(\mgPrb^\epsilon)_{\epsilon>0}$ then comes from that of $(\Prb^\epsilon)_{\epsilon>0}$.
  The change in rate function comes from exponential corrections of our martingale term.
  Indeed, we observe that if $J^\epsilon$ is uniformly within $\delta > 0$ of some absolutely continuous increasing $\pathVar$ on some compact interval $[0,\tau]$, then we have the following inequalities.
  \begin{align*}
    \bigg| \int_0^\tau \Big( 1 - \affInt\big(J^\epsilon_t\big) \Big) \rmd t - \int_0^\tau \Big( 1 - \affInt\big( \pathVar(t) \big) \Big) \rmd t \bigg|
    &\leq \ell_1 \delta \tau \\
    %
    \bigg| \int_0^\tau \log\affInt\big(J^\epsilon_{t-}\big) \rmd J^\epsilon_t - \int_0^\tau \log\affInt\big(\pathVar(t-)\big) \rmd J^\epsilon_t \bigg| 
    &\leq \sup_{t\in[0,\tau]} \Big| \log\affInt\big(J^\epsilon_t\big) - \log\affInt\big(\pathVar(t)\big) \Big| \cdot J^\epsilon_T \\
    &\leq \frac{\affIntPart_1}{\affIntPart_0} \delta \Big( \pathVar(\tau) + \delta \Big), \\
    %
    \bigg| \int_0^\tau \log\affInt\big(\pathVar(t-)\big) \rmd J^\epsilon_t - \int_0^\tau \log\affInt\big(\pathVar(t-)\big) \rmd \pathVar(t)  \bigg| 
    &\leq \log\affInt\big(\pathVar(\tau)\big) \delta \\
    %
    \bigg| \int_0^\tau \log\affInt\big(J^\epsilon_{t-}\big) \rmd J^\epsilon_t - \int_0^\tau \log\affInt\big(\pathVar(t-)\big) \rmd \pathVar(t) \bigg| 
    &\leq \frac{\affIntPart_1}{\affIntPart_0} \delta \Big( \pathVar(\tau) + \delta \Big) + \log\affInt\big(\pathVar(\tau)\big) \delta 
  \end{align*}
  Then these give us the following identity, revealing the rate function.
  \begin{align*}
    &\lim_{\delta\rightarrow0}\lim_{\epsilon\rightarrow0}\epsilon\log\mgPrb^\epsilon\big( J^\epsilon \in \ball{\pathVar}{\delta} \big) \\
    &= \lim_{\delta\rightarrow0}\lim_{\epsilon\rightarrow0} \epsilon\log\Exp_{\Prb^\epsilon}\bigg( \exp\Big(\frac1\epsilon\int_0^\tau \log\affInt(J^\epsilon_{s-}) \rmd J^\epsilon_s + \frac1\epsilon\int_0^\tau \big(1 - \affInt(J^\epsilon_s) \big) \rmd s  \Big) 1_{J^\epsilon \in \ball{\pathVar}{\delta}} \bigg) \\
    &= \begin{aligned}[t]
      & \int_0^\tau \log\affInt\big(\pathVar(s-)\big) \rmd\pathVar(s) + \int_0^\tau \Big(1 - \affInt\big(\pathVar(s)\big) \Big) \rmd s \\
      &+\lim_{\delta\rightarrow0} \Big( \affIntPart_1 \delta \tau + \frac{\affIntPart_1}{\affIntPart_0} \delta \big( \pathVar(\tau) + \delta \big) + \log\affInt\big(\pathVar(\tau)\big) \delta \Big) 
      +\lim_{\delta\rightarrow0}\lim_{\epsilon\rightarrow0} \epsilon \log \Prb\Big( \epsilon N_{\cdot/\epsilon} \in \ball{\pathVar}{\delta} \Big)
    \end{aligned} \\
    &= \int_0^\tau \Big( \dot\pathVar(s) \log \affInt\big(\pathVar(s)\big) - \affInt\big(\pathVar(s)\big) + 1 \Big) \rmd s - \int_0^\tau \Big(\dot\pathVar(t) \log \dot\pathVar(t) - \dot\pathVar(t) + 1 \Big) \rmd s \\
    &= -\int_0^\tau \bigg( \dot\pathVar(s) \log\Big(\frac{\dot\pathVar(s)}{\affInt\big(\pathVar(s)\big)} \Big) - \dot\pathVar(s) + \affInt\big(\pathVar(s)\big) \bigg) \rmd s
  \end{align*}

  Of course, many technical details are missing in the above argument, but we again reiterate that this is merely to gather intuition on rate functions.
  For a rigorous proof involving such analysis, we refer the reader to \cite{Gao2018b}, where they prove a large deviation principle for an asymptotic family of nonlinear Hawkes processes.
  In any case, we again note that Theorem \ref{theorem:ldp-integral} covers the large deviation principle and rate function mentioned above.
  A continuous-branching process $\X$ with intensity $\affInt(\X)$ has special semimartingale decomposition as below,
  \begin{equation*}
    \X = 1 \ast \jumpMeas^{\X} = 1 \ast \predproj{\jumpMeas^{\X}} + 1 \ast \compensate{\jumpMeas^{\X}} = \affInt(\X) \shortInt \leb + 1 \ast \compensate{\jumpMeas^{\X}}
  \end{equation*}
  which lends itself to the following special differential characteristics $(\affDrift, \affDiff, \affJump)$,
  \begin{equation*}
    \affDrift(\stateVar) = \affInt(\stateVar), \quad
    \affDiff(\stateVar) = 0, \quad
    \affJump(\stateVar, \rmd\markVar) = \affInt(\stateVar) \delta_1(\rmd\markVar)
  \end{equation*}
  and so the corresponding integrand in (\ref{eq:rf-integral}) evaluates the following function.
  \begin{align*}
    \ode^*\big(\diffVar, \stateVar \big) 
    &= \sup_{\momentVar\in\bbR}\bigg( \momentVar \diffVar - \momentVar\affInt(\stateVar) - \int_\bbR \big( e^{\momentVar\markVar} - 1 - \momentVar\markVar \big) \affInt(\stateVar)\delta_1(\rmd\markVar) \bigg) \\
    &= \sup_{\momentVar\in\bbR}\bigg( \momentVar \diffVar - \affInt(\stateVar) e^\momentVar + \affInt(\stateVar) \bigg) \\
    &= \left\{\begin{array}{ll}
      \displaystyle \diffVar \log\Big(\frac{\diffVar}{\affInt(\stateVar)}\Big) - \diffVar + \affInt(\stateVar), & \diffVar \geq 0, ~ \affInt(\stateVar) \geq 0 \\
      \infty, & \text{otherwise}
    \end{array}\right.
  \end{align*}

  We may even extend this to Hawkes processes $(\Xe)_{\epsilon>0}$ induced by base process,
  \begin{equation*}
    \X = \kappa\big(\mu- \X_t \big) \shortInt \leb + \N, \quad \N \text{ intensity } \affInt(\X),
  \end{equation*}
  for this process $\X$ has affine special differential characteristics.
  \begin{equation*}
    \affDrift(\stateVar) = \kappa(\mu - \stateVar) + \affInt(\stateVar), \quad
    \affDiff(\stateVar) = 0, \quad
    \affJump(\stateVar, \rmd\markVar) = \affInt(\stateVar) \delta_1(\rmd\markVar)
  \end{equation*}
  The rate function then involves the following expression,
  \begin{align*}
    \ode^*(\diffVar, \stateVar) 
    &= \sup_{\momentVar\in\bbR} \bigg( \momentVar\diffVar - \momentVar\Big(\kappa(\mu - \stateVar) - \affInt(\stateVar)\Big) - \int_\bbR \big( e^{\momentVar\markVar} - 1 - \momentVar\markVar \big) \affInt(\stateVar) \delta_1(\rmd\markVar) \bigg) \\
    &= \sup_{\momentVar\in\bbR} \bigg( \momentVar\Big(\diffVar - \kappa(\mu-\stateVar)\Big) - \affInt(\stateVar)e^\momentVar + \affInt(\stateVar) \bigg) \\
    &= \left\{\begin{array}{ll}
      \displaystyle \diffVar \log\Big(\frac{\diffVar-\kappa(\mu-\stateVar)}{\affInt(\stateVar)}\Big) - \diffVar + \affInt(\stateVar), & \diffVar - \kappa(\mu-\stateVar) \geq 0, ~ \affInt(\stateVar) \geq 0, \\
      \infty, & \text{otherwise}
    \end{array}\right.
  \end{align*}
  which is similar to the linear case of \cite{Gao2018b}.
\end{example}



\section{Coupling}

\begin{enumerate}
    \begin{color}{gray}
    \item
      {\bfseries Mogulskii's theorem.}
      Purpose of section is to familiarize with rate functions we already have and set the stage for how we operate with our theorem.
      \begin{enumerate}
        \item
          Cite Mogulskii's theorem.
        \item
          Brownian motion of our regime: achieved with Mogulskii's theorem with Normal increments and exponential tightness (Schilder).
        \item
          Poisson process of our regime: achieved with Mogulskii's theorem with Poisson increments and use exponential tightness.
      \end{enumerate}
    \end{color}
  \item
    {\bfseries Simple contraction maps.}
    \begin{enumerate}
      \item
        Friedlin Wentzel (not so much our regime): use contraction mapping principles
      \item
        Birth rate process: Can we similarly contraction map Poisson to get this?
      \item
        Extension of Freidlin Wentzel to affine: KK
    \end{enumerate}
  \item
    {\bfseries Coupling states.}
    Indicate that when contraction mappings are not sufficient, we may \emph{couple correlated states}, in the sense of looking at LDPs of joint processes.
    \begin{enumerate}
      \item
        Compound Poisson: Two \emph{sources} of randomness; the arrivals and the jump sizes. Appeal to Duffy results for heuristical calculations.
      \item
        Compound linear Hawkes: Similarly two \emph{sources} of randomness. Duffy also gives us the calculations.
        Note on Zhu paper for \emph{sidestep}; less general jumps, more general nonlinear relationship of arrivals.
      \item
        The jumps of a general jump-diffusion do not have a well-posed notion of arrivals and jump sizes; we thus turn our focus to locally countable jump-diffusions, in which the three \emph{sources} of randomness are the continuous local martingale, the arrival times, and the jump sizes.
        Note how this is discussed in next section.
    \end{enumerate}
  \item
    {\bfseries Locally countable affine processes.}
    Perform the necessary calculus and proceed to show our general formulation.
    \begin{enumerate}
      \item
        State result in numerous flavors, depending on which \emph{base} quantities in which we choose to focus the large deviations.
        \begin{align*}
          \X &= \affDrift(\X)\shortInt \leb + \X^\rmc + \id_\vecSpace \ast \compensate{\jumpMeas^\X} \\
          \N^\X &= 1 \ast \jumpMeas^\X \\
          V^\X &= \id_\vecSpace \ast \jumpMeas^\X 
        \end{align*}
        \begin{enumerate}
          \item
            \textbf{overdetermined flavor.}
            $(\X, \X^\rmc, \N^\X, V^\X)$ produces an overdetermined system which requires another condition for $\rf(\pathVar, \omega, \eta, \gamma)$ to be finite.
            \[
              \dot\pathVar(t) = \affDrift\big(\pathVar(t)\big) + \dot\omega(t) + \dot\eta(t)\dot\gamma(t)
            \]
            However, the rate function is very simple to understand.
          \item
            \textbf{determine-continuous-noise flavor.}
            Normal term gets messy
          \item
            \textbf{determine-arrivals flavor.}
            Poisson and jump-term-denominator gets messy
          \item
            \textbf{determine-jumps flavor.}
            This is the one we have already presented; the jump-term-numerator gets messy.
        \end{enumerate}
      \item
        Discuss how the deviations of $\X$ from the dynamical system $\X = \affDrift(\X) \shortInt \leb$ are imposed from \emph{continuous deviations} $\X^\rmc$ and \emph{discontinuous deviations} $\id_\vecSpace \ast \compensate{\jumpMeas^\X}$.
      \item
        Discuss how four quantities $\X, \X^\rmc, \N^\X, V^\X$ heuristically relate in simple infinitesimal equality $(\X, \X^\rmc, \N^\X, V^\X) \approx (\pathVar, \omega, \eta, \gamma)$.
        \begin{align*}
          \dot\pathVar(t) = \affDrift\big(\pathVar(t)\big) + \dot\omega(t) + \dot\eta(t) \cdot \dot\gamma(t)
        \end{align*}
      \item
        Each of the primitive deviations have a simple analogy, when we think of infinitesimals.
        \begin{align*}
          &\dot\omega(t)  & \text{\emph{normal deviations of covariance }} \affDiff\big(\pathVar(t)\big) \\
          &\dot\eta(t) & \text{\emph{Poisson deviations of rate }} \affInt\big(\pathVar(t)\big) \\
          &\dot\gamma(t) & \text{\emph{jump deviations from }} \affDist\big(\pathVar(t), \rmd\markVar\big) \\
          \pathVar(t) &=\affDrift\big(\pathVar(t)\big) + \dot\omega(t) + \dot\eta(t)\cdot\dot\gamma(t) & \text{\emph{all combined deviations}}
        \end{align*}
      \item
        Think of results from first section in this regard.
        \begin{enumerate}
          \item
            birth is  $\dot\pathVar(t) = \dot\eta(t)$, so we only need $\pathVar \approx \X$.
          \item
            diffusion is $\dot\pathVar(t) = \affDrift\big(\pathVar(t)\big) + \dot\omega(t)$, and so we only need $\pathVar\approx\X$ and rate function includes $\pathVar(t) - \affDrift\big(\pathVar(t)\big)$ where $\dot\omega$ is.
          \item
            compound Poisson is $\dot\pathVar(t) = \dot\eta(t) \cdot \dot\gamma(t)$, so we choose one of the following pairs $(\pathVar, \eta) \approx (\X, \N^\X)$, $(\pathVar, \gamma) \approx (\X, V^\X)$, or $(\eta, \gamma) \approx (\N^\X, V^\X)$.
          \item
            compound linear Hawkes is $\dot\pathVar(t) = \affDrift\big(\pathVar(t)\big) + \dot\eta(t) \cdot \dot\gamma(t)$, so we can choose $(\pathVar, \eta) \approx (\X, \N^\X)$ or $(\pathVar, \dot\gamma) \approx (\X, V^\X)$.
        \end{enumerate}
    \end{enumerate}
\end{enumerate}
