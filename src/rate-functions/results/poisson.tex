\begin{example}[Poisson]
  \label{example:poisson}
  One may apply a very similar argument for when our increment distribution $\affDist$ is $\operatorname{Poisson}(1)$.
  In this case, the integral in the rate function in (\ref{eq:rf-mogulskii}) evaluates to
  \begin{equation}
    \label{eq:rf-poisson}
    \int_0^\infty \Big( \dot\pathVar(t)  \log\big(\dot\pathVar(t)\big) - \dot\pathVar(t) + 1 \Big) \rmd t,
  \end{equation}
  so long as $\pathVar(t) \geq 0$ for Lebesgue-almost-every $t \geq 0$ (otherwise, it is infinite).
  In the case that $\pathVar(t) = 0$, we are taking the continuous extension of the integrand, i.e.\ $0\log(0) \defeq 0$.
  Similar to the work of Schilder's theorem, we may show, for a standard Poisson process $\N$, $\epsilon\N_{\cdot/\epsilon}$ is exponentially equivalent to this $Y^\epsilon$, which makes the family satisfy a large deviation principle with rate function (\ref{eq:rf-poisson}).
  In fact and exercise of our reference text, \cite[Exercise 5.2.12]{dembo2010}, suggests the reader to show just this.

  Again, such a family $(\epsilon\N_{\cdot/\epsilon})_{\epsilon>0}$ is covered by Theorem \ref{theorem:ldp-integral}.
  To see this, consider a base affine process $\X$ on $(\bbR, \scrB(\bbR))$ with special differential characteristics as below, where $\delta_1$ denotes the degenerate distribution at $1 \in \bbR$.
  \begin{equation*}
    \affDrift(\stateVar) = 1, \quad \affDiff(\stateVar) = 0, \quad \affJump(\stateVar, \rmd\markVar) = \delta_1(\rmd\markVar)
  \end{equation*}
  Setting the initial state $\stateVar = 0$ and looking at Proposition \ref{proposition:sde-asymptotics}, we may say that $\Xe$ can be realized as follows.
  \begin{align*}
    \Xe_t 
    &= t + \epsilon 1_{[0,1]}(\id_\bbR) \ast \compensate{\poisRM^\epsilon}_t \\
    &= t + \epsilon 1_{[0,1]}(\id_\bbR) \ast \poisRM^\epsilon_t - \epsilon 1_{[0,1]}(\id_\bbR) \ast \predproj{\poisRM^\epsilon}_t \\
    &= t + \epsilon \poisRM([0,t/\epsilon] \times [0,1]) - \int_0^{t/\epsilon} \int_\bbR \epsilon 1_{[0,1]}(\markVar) \rmd\markVar \rmd s \\
    &= \epsilon \poisRM([0,t/\epsilon] \times [0,1])
  \end{align*}
  As stated in \cite[Theorem II.4.8]{jacod2003}, this Poisson random measure $\poisRM$ is a Poisson point process with Lebesgue intensity.
  This means that, for each $t \geq 0$, $\N_t \defeq \poisRM([0,t] \times [0,1])$ distributes $\operatorname{Poisson}(t)$, and $\N_t - \N_s = \poisRM((s,t] \times [0,1])$ is independent of $\N_s = \poisRM([0,s], [0,1])$ for each $0 \leq s < t$.
  In other words, $\N$ is a standard Poisson process and
  \begin{equation*}
    \Xe_t = \epsilon \poisRM([0,t/\epsilon] \times [0,1]) = \epsilon\N_{t/\epsilon}.
  \end{equation*}
  As with the normal increments, our rate function (\ref{eq:rf-integral}) resolves this immediately.
  \begin{align*}
    \ode^*(\diffVar, \stateVar) 
    &= \sup_{\momentVar\in\vecSpace} \bigg( \momentVar\diffVar - \momentVar - \int_\bbR \big( e^{\momentVar\markVar} - 1 - \momentVar\markVar \big) \delta_1(\rmd\markVar) \bigg)  \\
    &= \sup_{\momentVar\in\vecSpace} \bigg( \momentVar\diffVar - e^\momentVar + 1 \bigg) \\
    &= \left\{\begin{array}{ll}
      \diffVar \log\diffVar - \diffVar + 1, & \diffVar \geq 0 \\
      \infty, & \text{otherwise}
    \end{array}\right.
  \end{align*}
\end{example}
