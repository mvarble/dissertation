We have now gathered enough familiarity with rate functions that appear in the large deviations literature and the various tools we may use to lift existing principles to new ones.
This will allow us to more easily understand the nature and derivation of our semi-closed form for (\ref{eq:rf-integral}) in Theorem \ref{theorem:ldp-integral}.
Let us now elaborate on the family on which we derive our semi-closed form.

Firstly, in order for us to perform the tricks of coupling we explored in the previous section, we need to keep track of our arrivals and/or accumulated jumps.
We will need these objects to be locally integrable, so we impose that our base process $\X$ is \emph{locally countable} in the sense of our definition in Appendix \ref{jump-diffusions:countable}.
In particular, suppose that our base process $\X$ has special differential characteristics $(\affDrift, \affDiff, \affJump)$, where $\affJump$ has the finiteness condition below.
\begin{equation}
  \label{eq:assumption-affJump-finite}
  \affJump(\stateVar, \vecSpace) < \infty, \quad \stateVar \in \stateSpace
\end{equation}
Note that Proposition \ref{proposition:locally-countable} thus provides us with a factoring 
\begin{equation}
  \label{eq:affJump-factoring}
  \affJump(\stateVar,\rmd\markVar) = \affInt(\stateVar) \affDist(\stateVar, \rmd\markVar)
\end{equation}
into an intensity function $\affInt \in \scrB(\stateSpace)/\scrB(\bbR_+)$ and conditional jump distribution $\affDist$ from $(\stateSpace, \scrB(\stateSpace))$ to $(\vecSpace, \scrB(\vecSpace))$.
Furthermore, the following process is locally integrable.
\begin{equation}
  \label{eq:arrivals}
  \N^\X \defeq 1 \ast \jumpMeas^\X
\end{equation}
Note that this is a stronger restriction than local integrability of the compensated jumps
\begin{equation}
  \label{eq:jumps}
  \V^\X \defeq \id_\vecSpace \ast \compensate{\jumpMeas^\X},
\end{equation}
which we already have from the special property of $\X$; nonetheless, we intend to use this object as well.

\begin{theorem}
  \label{eq:ldp-closed-form}
  Let $\X$ be an affine process as introduced above with Assumption \ref{assumption:joint-affine} satisfied.
  Then $\hat\X$ is a base affine process with which we may parameterize $\hat\Xe$ with distribution as in Section \ref{large-deviations:asymptotics}.
  Fixing $\hat\stateVar \in \hat\stateSpace^\circ$ and denoting $\hat\Prb^\epsilon_{\hat\stateVar}$ the distribution associated with $\hat\Xe$ starting at $\hat\stateVar$, we have a large deviation principle for $(\hat\Prb^\epsilon_{\hat\stateVar})_{\epsilon>0}$.
  The rate function $\rf_{\hat\stateVar}$ simplifies to the following semi-closed form, where we denote the components of an arbitrary function $\hat\pathVar$ by $\hat\pathVar \defeq (\pathVar, \omega, \gamma, \eta)$.
  \begin{equation*}
    \rf_{\hat\stateVar}(\hat\pathVar) = 
    \begin{aligned}[t]
      &\int_0^\infty \frac12 \Bprj{\dot\omega(t)}{\affDiff\big(\pathVar(t)\big)^\dagger \dot\omega(t)} \rmd t 
      + \int_0^\infty \bigg( \dot\eta(t)\log\Big(\frac{\dot\eta(t)}{\affInt\big(\pathVar(t)\big)} \Big) - \dot\eta(t) + \affInt\big(\pathVar(t)\big) \bigg) \rmd t  \\
      &\qquad+ \int_0^\infty \dot\eta(t) \ode_{\affDist(\pathVar(t),\cdot)}\bigg( \frac{\dot\gamma(t) + \affInt\big(\pathVar(t)\big)\overline{\affDist(\pathVar(t), \cdot)}}{\dot\eta(t)} \bigg) \rmd t
    \end{aligned}
  \end{equation*}
  In the evaluation above, we are insisting that $\hat\pathVar$ satisfies the following properties below, where statements involving $t$ are taken Lebesgue-almost-everywhere; otherwise $\rf_{\hat\stateVar}(\hat\pathVar) = \infty$.
  \begin{itemize}
    \item
      $\hat\pathVar(0) = \hat\stateVar$,
    \item
      $\pathVar \in \acpathSpace{[0,\infty)}{\stateSpace}$, $\omega \in \acpathSpace{[0,\infty)}{\vecSpace}$, $\gamma \in \acpathSpace{[0,\infty)}{\vecSpace}$, and $\eta \in \acpathSpace{[0,\infty)}{\bbR_+}$,
    \item
      $\dot\pathVar(t) = \affDrift\big(\pathVar(t)\big) + \dot\omega(t) + \dot\gamma(t)$,
    \item
      $\dot\omega(t) \in \operatorname{range}\big(\affDiff(\pathVar(t))\big)$,
    \item
      $\dot\eta(t) \geq 0$,
    \item
      $\affInt\big(\pathVar(t)\big) \geq 0$.
  \end{itemize}
\end{theorem}
\begin{proof}
  Assumption \ref{assumption:joint-affine} is precisely the Riccati equation with which the following process
  \[
    t \mapsto \exp\Big( \hat\affPart_0(\tau-t, \hat\momentVar) + \Bprj{\hat\affPart(\tau-t, \hat\momentVar)}{\hat\X_t} \Big)
  \]
  is a martingale which gives us our desired transform formula.
  This makes $\hat\X$ an affine process, so Lemma \ref{lemma:joint-cone} and Proposition \ref{proposition:joint-characteristics} tell us that $\hat\X$ satisfies the assumptions of Section \ref{large-deviations:assumptions}.
  Thus, the parameterization $\hat\Xe$ is a family described by Theorem \ref{theorem:ldp-integral}, and so we have a large deviation principle for $(\hat\Prb^\epsilon_{\hat\stateVar})_{\epsilon>0}$.
  Again using Proposition \ref{proposition:joint-characteristics}, we see that the integrand evaluates the following function.
  \begin{align*}
    \hat\ode^*(\hat{\diffVar}, \hat\stateVar) 
    &= \sup_{\hat\momentVar \in \hat\vecSpace} \bigg( \bprj{\hat\momentVar}{\hat\diffVar} - \hat\ode\big(\hat\momentVar, \hat\stateVar\big) \bigg) \\
    &= \sup_{\hat\momentVar \in \hat\vecSpace} \bigg(\begin{aligned}[t]
      &\prj{\momentVar_1}{\diffVar_1} 
      + \prj{\momentVar_2}{\diffVar_2} 
      + \prj{\momentVar_3}{\diffVar_3} 
      + \momentVar_4\diffVar_4
      - \prj{\momentVar_1}{\affDrift(\stateVar_1)}  \\
      &- \frac12\bprj{\momentVar_1+\momentVar_2}{\affDiff(\stateVar)(\momentVar_1+\momentVar_2)} 
      - \affInt(\stateVar_1)\exp\Big(\ode_{\affDist(\stateVar_1,\cdot)}(\momentVar_1+\momentVar_3) + \momentVar_4\Big) \\
      &+ \affInt(\stateVar_1) 
      + \bprj{\momentVar_1+\momentVar_3}{\affInt(\stateVar_1)\overline{\affDist(\stateVar_1,\cdot)}}  \bigg)
    \end{aligned}
  \end{align*}
  We now break the problem into cases.
  \begin{enumerate}
    \item
      First, assume $\diffVar_1 = \affDrift(\stateVar_1) + \diffVar_2 + \diffVar_3$, $\diffVar_2 \in \operatorname{range}(\affDiff(\stateVar_1))$, $\diffVar_4 \geq 0$, and $\affInt(\stateVar_1) \geq 0$.
      Using Lemma \ref{lemma:ode-differentiable}, we are able to take the gradients of the above expression to get the following system of equations.
      \begin{align*}
        0 &= \begin{aligned}[t]
          &\diffVar_1 - \affDrift(\stateVar_1) - \affDiff(\stateVar_1) (\momentVar_1 + \momentVar_2) \\
          &- \affInt(\stateVar_1) \exp\Big( \ode_{\affDist(\stateVar_1,\cdot)}(\momentVar_1+\momentVar_3) + \momentVar_4\Big) \nabla\ode_{\affDist(\stateVar_1,\cdot)}(\momentVar_1+\momentVar_3) + \affInt(\stateVar_1) \overline{\affDist(\stateVar_1, \cdot)}
        \end{aligned} \\
        0 &= \diffVar_2 - \affDiff(\stateVar_1) ( \momentVar_1 + \momentVar_2 ) \\
        0 &= \diffVar_3 - \affInt(\stateVar_1) \exp\Big( \ode_{\affDist(\stateVar_1,\cdot)}(\momentVar_1+\momentVar_3) + \momentVar_4\Big) \nabla\ode_{\affDist(\stateVar_1,\cdot)}(\momentVar_1+\momentVar_3) + \affInt(\stateVar_1) \overline{\affDist(\stateVar_1, \cdot)} \\
        0 &= \diffVar_4 - \affInt(\stateVar_1) \exp\Big( \ode_{\affDist(\stateVar_1,\cdot)}(\momentVar_1+\momentVar_3) + \momentVar_4\Big) 
      \end{align*}
      These equations yield the following via simple algebra.
      \begin{equation}
        \label{eq:semi-closed-algebra}
        \begin{aligned}
          \momentVar_1 + \momentVar_2 &= \affDiff(\stateVar_1)^\dagger \diffVar_2 \\
          \diffVar_4 &= \affInt(\stateVar_1) \exp\Big( \ode_{\affDist(\stateVar_1,\cdot)}(\momentVar_1+\momentVar_3) + \momentVar_4\Big)  \\
          \diffVar_4\log\Big(\frac{\diffVar_4}{\affInt(\stateVar_1)}\Big) &=   \diffVar_4\ode_{\affDist(\stateVar_1,\cdot)}(\momentVar_1+\momentVar_3) + \momentVar_4\diffVar_4 \\
          \frac{\diffVar_3 + \affInt(\stateVar_1)\overline{\affDist(\stateVar_1,\cdot)}}{\diffVar_4} &= \nabla\ode_{\affDist(\stateVar_1,\cdot)}(\momentVar_1+\momentVar_3)
        \end{aligned}
      \end{equation}
      Seeing as $\ode_{\affDist(\stateVar_1,\cdot)}$ is a convex function, the last equality of (\ref{eq:semi-closed-algebra}) corresponds to the unique extreme point of the Fenchel-Legendre transform.
      \begin{equation*}
        \Bprj{\momentVar_1 + \momentVar_3}{\frac{\diffVar_3+\affInt(\stateVar_1)\overline{\affDist(\stateVar_1,\cdot)}}{\diffVar_4}} - \ode_{\affDist(\stateVar_1,\cdot)}(\momentVar_1+\momentVar_3) = \ode_{\affDist(\stateVar_1,\cdot)}^*\Big(\frac{\diffVar_3+\affInt(\stateVar_1)\overline{\affDist(\stateVar_1,\cdot)}}{\diffVar_4}\Big)
      \end{equation*}
      Combining this equality with those of (\ref{eq:semi-closed-algebra}) now gives us the following identity.
      \begin{align*}
        \bprj{\hat\momentVar}{\hat\diffVar} - \hat\ode\big(\hat\momentVar, \hat\stateVar\big) 
        &=\begin{aligned}[t]
          &\prj{\momentVar_1}{\diffVar_1} 
          + \prj{\momentVar_2}{\diffVar_2} 
          + \prj{\momentVar_3}{\diffVar_3} 
          + \momentVar_4\diffVar_4
          - \prj{\momentVar_1}{\affDrift(\stateVar_1)}  \\
          &\quad- \frac12\bprj{\momentVar_1+\momentVar_2}{\affDiff(\stateVar)(\momentVar_1+\momentVar_2)} \\
          &\quad- \affInt(\stateVar_1)\exp\Big(\ode_{\affDist(\stateVar_1,\cdot)}(\momentVar_1+\momentVar_3) + \momentVar_4\Big) \\
          &\quad+ \affInt(\stateVar_1) 
          + \bprj{\momentVar_1+\momentVar_3}{\affInt(\stateVar_1)\overline{\affDist(\stateVar_1,\cdot)}}
        \end{aligned} \\
        &= \begin{aligned}[t]
          &\frac12\Bprj{\diffVar_2}{\affDiff(\stateVar_1)^\dagger\diffVar_2} + \diffVar_4 \log\Big(\frac{\diffVar_4}{\affInt(\stateVar_1)}\Big) - \diffVar_4 + \affInt(\stateVar_1) \\
          &+ \ode_{\affDist(\stateVar_1,\cdot)}^*\Big(\frac{\diffVar_3+\affInt(\stateVar_1)\overline{\affDist(\stateVar_1,\cdot)}}{\diffVar_4}\Big)
        \end{aligned}
      \end{align*}
    \item
  \end{enumerate}
\end{proof}


\begin{color}{gray}
  {\bfseries Locally countable affine processes.}
  Perform the necessary calculus and proceed to show our general formulation.
  \begin{enumerate}
    \item
      State result in numerous flavors, depending on which \emph{base} quantities in which we choose to focus the large deviations.
      \begin{align*}
        \X &= \affDrift(\X)\shortInt \leb + \X^\rmc + \id_\vecSpace \ast \compensate{\jumpMeas^\X} \\
        \N^\X &= 1 \ast \jumpMeas^\X \\
        V^\X &= \id_\vecSpace \ast \jumpMeas^\X 
      \end{align*}
      \begin{enumerate}
        \item
          \textbf{overdetermined flavor.}
          $(\X, \X^\rmc, \N^\X, V^\X)$ produces an overdetermined system which requires another condition for $\rf(\pathVar, \omega, \eta, \gamma)$ to be finite.
          \[
            \dot\pathVar(t) = \affDrift\big(\pathVar(t)\big) + \dot\omega(t) + \dot\eta(t)\dot\gamma(t)
          \]
          However, the rate function is very simple to understand.
        \item
          \textbf{determine-continuous-noise flavor.}
          Normal term gets messy
        \item
          \textbf{determine-arrivals flavor.}
          Poisson and jump-term-denominator gets messy
        \item
          \textbf{determine-jumps flavor.}
          This is the one we have already presented; the jump-term-numerator gets messy.
      \end{enumerate}
    \item
      Discuss how the deviations of $\X$ from the dynamical system $\X = \affDrift(\X) \shortInt \leb$ are imposed from \emph{continuous deviations} $\X^\rmc$ and \emph{discontinuous deviations} $\id_\vecSpace \ast \compensate{\jumpMeas^\X}$.
    \item
      Discuss how four quantities $\X, \X^\rmc, \N^\X, V^\X$ heuristically relate in simple infinitesimal equality $(\X, \X^\rmc, \N^\X, V^\X) \approx (\pathVar, \omega, \eta, \gamma)$.
      \begin{align*}
        \dot\pathVar(t) = \affDrift\big(\pathVar(t)\big) + \dot\omega(t) + \dot\eta(t) \cdot \dot\gamma(t)
      \end{align*}
    \item
      Each of the primitive deviations have a simple analogy, when we think of infinitesimals.
      \begin{align*}
        &\dot\omega(t)  & \text{\emph{normal deviations of covariance }} \affDiff\big(\pathVar(t)\big) \\
        &\dot\eta(t) & \text{\emph{Poisson deviations of rate }} \affInt\big(\pathVar(t)\big) \\
        &\dot\gamma(t) & \text{\emph{jump deviations from }} \affDist\big(\pathVar(t), \rmd\markVar\big) \\
        \pathVar(t) &=\affDrift\big(\pathVar(t)\big) + \dot\omega(t) + \dot\eta(t)\cdot\dot\gamma(t) & \text{\emph{all combined deviations}}
      \end{align*}
    \item
      Think of results from first section in this regard.
      \begin{enumerate}
        \item
          birth is  $\dot\pathVar(t) = \dot\eta(t)$, so we only need $\pathVar \approx \X$.
        \item
          diffusion is $\dot\pathVar(t) = \affDrift\big(\pathVar(t)\big) + \dot\omega(t)$, and so we only need $\pathVar\approx\X$ and rate function includes $\pathVar(t) - \affDrift\big(\pathVar(t)\big)$ where $\dot\omega$ is.
        \item
          compound Poisson is $\dot\pathVar(t) = \dot\eta(t) \cdot \dot\gamma(t)$, so we choose one of the following pairs $(\pathVar, \eta) \approx (\X, \N^\X)$, $(\pathVar, \gamma) \approx (\X, V^\X)$, or $(\eta, \gamma) \approx (\N^\X, V^\X)$.
        \item
          compound linear Hawkes is $\dot\pathVar(t) = \affDrift\big(\pathVar(t)\big) + \dot\eta(t) \cdot \dot\gamma(t)$, so we can choose $(\pathVar, \eta) \approx (\X, \N^\X)$ or $(\pathVar, \dot\gamma) \approx (\X, V^\X)$.
      \end{enumerate}
  \end{enumerate}
\end{color}
