\begin{theorem}
  \label{theorem:ldp-closed-form}
  Let $\X$ be an affine process as introduced above with Assumption \ref{assumption:joint-affine} satisfied.
  Then $\hat\X$ is a base affine process with which we may parameterize $\hat\Xe$ with distribution as in Section \ref{large-deviations:asymptotics}.
  Fixing $\hat\stateVar \in \hat\stateSpace^\circ$ and denoting $\hat\Prb^\epsilon_{\hat\stateVar}$ the distribution associated with $\hat\Xe$ starting at $\hat\stateVar$, we have a large deviation principle for $(\hat\Prb^\epsilon_{\hat\stateVar})_{\epsilon>0}$.
  The rate function $\rf_{\hat\stateVar}$ simplifies to the following semi-closed form, where we denote the components of an arbitrary function $\hat\pathVar$ by $\hat\pathVar \defeq (\pathVar, \omega, \gamma, \eta)$.
  \begin{equation*}
    \rf_{\hat\stateVar}(\hat\pathVar) = 
    \begin{aligned}[t]
      &\int_0^\infty \frac12 \Bprj{\dot\omega(t)}{\affDiff\big(\pathVar(t)\big)^\dagger \dot\omega(t)} \rmd t \\
      &+ \int_0^\infty \bigg( \dot\eta(t)\log\Big(\frac{\dot\eta(t)}{\affInt\big(\pathVar(t)\big)} \Big) - \dot\eta(t) + \affInt\big(\pathVar(t)\big) \bigg) \rmd t  \\
      &\qquad+ \int_0^\infty \dot\eta(t) \ode_{\affDist(\pathVar(t),\cdot)}^*\bigg( \frac{\dot\gamma(t) + \affInt\big(\pathVar(t)\big)\overline{\affDist(\pathVar(t), \cdot)}}{\dot\eta(t)} \bigg) \rmd t
    \end{aligned}
  \end{equation*}
  In the evaluation above, we are insisting $0\log 0 = 0$, $0 \cdot \ode_{\affDist(\cdot,\cdot)}(\cdot) = 0$, and that $\hat\pathVar$ satisfies the following properties below, where statements involving $t$ are taken Lebesgue-almost-everywhere; otherwise $\rf_{\hat\stateVar}(\hat\pathVar) = \infty$.
  \begin{itemize}
    \item
      $\hat\pathVar(0) = \hat\stateVar$,
    \item
      $\pathVar \in \acpathSpace{[0,\infty)}{\stateSpace}$, $\omega \in \acpathSpace{[0,\infty)}{\vecSpace}$, $\gamma \in \acpathSpace{[0,\infty)}{\vecSpace}$, and $\eta \in \acpathSpace{[0,\infty)}{\bbR_+}$,
    \item
      $\dot\pathVar(t) = \affDrift\big(\pathVar(t)\big) + \dot\omega(t) + \dot\gamma(t)$,
    \item
      $\dot\omega(t) \in \operatorname{image}\big(\affDiff(\pathVar(t))\big)$,
    \item
      $\affInt\big(\pathVar(t)\big) > 0$.
    \item
      $\dot\eta(t) \geq 0$,
  \end{itemize}
\end{theorem}
\begin{proof}
  \label{proof:theorem:ldp-closed-form}
  Assumption \ref{assumption:joint-affine} is another way of rewriting the system in Proposition \ref{proposition:affine-resolution}, as in Remark \ref{remark:riccati-affine-formulation}.
  This makes $\hat\X$ an affine process, so Lemma \ref{lemma:joint-cone} and Proposition \ref{proposition:joint-characteristics} tell us that $\hat\X$ satisfies the assumptions of Section \ref{large-deviations:assumptions}.
  Thus, the parameterization $\hat\Xe$ is a family described by Theorem \ref{theorem:ldp-integral}, and so we have a large deviation principle for $(\hat\Prb^\epsilon_{\hat\stateVar})_{\epsilon>0}$.
  Again using Proposition \ref{proposition:joint-characteristics}, we see that the integrand evaluates the following function.
  \begin{align}
    \notag
    \hat\ode^*(\hat{\diffVar}, \hat\stateVar) 
    &= \sup_{\hat\momentVar \in \hat\vecSpace} \bigg( \bprj{\hat\momentVar}{\hat\diffVar} - \hat\ode\big(\hat\momentVar, \hat\stateVar\big) \bigg) \\
    \notag
    &= \sup_{\hat\momentVar \in \hat\vecSpace} \bigg(\begin{aligned}[t]
      &\prj{\momentVar_1}{\diffVar_1} 
      + \prj{\momentVar_2}{\diffVar_2} 
      + \prj{\momentVar_3}{\diffVar_3} 
      + \momentVar_4\diffVar_4
      - \prj{\momentVar_1}{\affDrift(\stateVar_1)}  \\
      &- \frac12\bprj{\momentVar_1+\momentVar_2}{\affDiff(\stateVar_1)(\momentVar_1+\momentVar_2)} 
      - \affInt(\stateVar_1)\exp\Big(\ode_{\affDist(\stateVar_1,\cdot)}(\momentVar_1+\momentVar_3) + \momentVar_4\Big) \\
      &+ \affInt(\stateVar_1) 
      + \bprj{\momentVar_1+\momentVar_3}{\affInt(\stateVar_1)\overline{\affDist(\stateVar_1,\cdot)}}  \bigg) 
    \end{aligned} \\
    \label{eq:optimizing-integrand}
    &= \sup_{\hat\momentVar\in\hat\vecSpace} \bigg(\begin{aligned}[t]
      & \bprj{\momentVar_1}{\diffVar_1 - \affDrift(\stateVar_1) - \diffVar_2 - \diffVar_3} 
      + \Bprj{\momentVar_1+\momentVar_2}{\diffVar_2 
      - \frac12\affDiff(\stateVar_1)(\momentVar_1+\momentVar_2)} \\
      &+ \Bprj{\momentVar_1+\momentVar_3}{\diffVar_3 + \affInt(\stateVar_1)\overline{\affDist(\stateVar_1,\cdot)}} 
      + \momentVar_4\diffVar_4 \\
      &- \affInt(\stateVar_1) \exp\Big(\ode_{\affDist(\stateVar_1,\cdot)}(\momentVar_1+\momentVar_3) + \momentVar_4\Big)
      + \affInt(\stateVar_1) \bigg)
    \end{aligned}
  \end{align}
  We now evaluate the expression depending on different scenarios.
  \begin{enumerate}
    \item
      Suppose we had $\diffVar_1 \neq \affDrift(\stateVar_1) + \diffVar_2 + \diffVar_3$.
      Observe that the expression in (\ref{eq:optimizing-integrand}) can have all expressions but the first summand canceled, if $\momentVar_1 = -\momentVar_2 = -\momentVar_3$ and $\momentVar_4 = 0$.
      This shows us the following.
      \begin{align*}
        \hat\ode^*(\hat{\diffVar}, \hat\stateVar) 
        &= \sup_{\hat\momentVar \in \hat\vecSpace} \bigg( \bprj{\hat\momentVar}{\hat\diffVar} - \hat\ode\big(\hat\momentVar, \hat\stateVar\big) \bigg) \\
        &\geq \sup_{\substack{\hat\momentVar=(\momentVar, -\momentVar, -\momentVar, 0) \\ \momentVar = \rho (\diffVar_1 - \affDrift(\stateVar_1) - \diffVar_2 - \diffVar_3) \\ \rho \in \bbR}} \bigg( \bprj{\hat\momentVar}{\hat\diffVar} - \hat\ode\big(\hat\momentVar, \hat\stateVar\big) \bigg) \\
        &= \sup_{\rho\in\bbR} \rho \Big| \diffVar_1 - \affDrift(\stateVar_1) - \diffVar_2 - \diffVar_3 \Big|^2 \\
        &= \infty
      \end{align*}
    \item
      Now assume $\diffVar_2 \not\in \operatorname{image}\affDiff(\stateVar_1)$.
      Observe that the expression in (\ref{eq:optimizing-integrand}) can have all expressions other than the quadratic term canceled, if we set $\momentVar_1 = \momentVar_3 = \momentVar_4 = 0$. 
      Denote $w_2, w_2^\bot \in \vecSpace$ the projections of $\diffVar_2$ onto $\operatorname{image}(\affDiff(\stateVar_1))$ and its orthogonal complement, respectively.
      Select $\tilde w_2 \in \vecSpace$ such that $w_2 = \frac12\affDiff(\stateVar_1) \tilde w_2$,
      \begin{equation*}
        \diffVar_2 = w_2 + w_2^\bot = \frac12 \affDiff(\stateVar_1) \tilde w_2 + w_2^\bot,
      \end{equation*}
      and observe the following.
      \begin{align*}
        \hat\ode^*(\hat{\diffVar}, \hat\stateVar) 
        &= \sup_{\hat\momentVar \in \hat\vecSpace} \bigg( \bprj{\hat\momentVar}{\hat\diffVar} - \hat\ode\big(\hat\momentVar, \hat\stateVar\big) \bigg) \\
        &\geq \sup_{\substack{\hat\momentVar=(0, \momentVar, 0, 0) \\ \momentVar = \rho w_2^\bot \\ \rho \in \bbR}} \bigg( \bprj{\hat\momentVar}{\hat\diffVar} - \hat\ode\big(\hat\momentVar, \hat\stateVar\big) \bigg) \\
        &= \sup_{\rho\in\bbR} \Bprj{\rho w_2^\bot}{\diffVar_2 - \frac12\affDiff(\stateVar_1) \cdot \rho w_2^\bot} \\
        &= \sup_{\rho\in\bbR} \bigg( \frac12\Bprj{\rho w_2^\bot}{\affDiff(\stateVar_1)(\tilde w_2 - \rho w_2^\bot)} + \Bprj{\rho w_2^\bot}{w_2^\bot} \bigg) \\
        &= \sup_{\rho\in\bbR} \rho \big|w_2^\bot\big|^2 \\
        &= \infty
      \end{align*}
    \item
      Now suppose $\affInt(\stateVar_1) = 0$.
      Then, (\ref{eq:optimizing-integrand}) shows us that evaluating at $\momentVar_1=\momentVar_2=\momentVar_4=0$ will cancel all terms not involving $\diffVar_3$.
      \begin{align*}
        \hat\ode^*(\hat{\diffVar}, \hat\stateVar) 
        = \sup_{\hat\momentVar \in \hat\vecSpace} \bigg( \bprj{\hat\momentVar}{\hat\diffVar} - \hat\ode\big(\hat\momentVar, \hat\stateVar\big) \bigg) 
        &\geq \sup_{\substack{\hat\momentVar=(0, 0, \momentVar, 0)  \\ \momentVar = \rho \diffVar_3 \\ \rho \in \bbR}} \bigg( \bprj{\hat\momentVar}{\hat\diffVar} - \hat\ode\big(\hat\momentVar, \hat\stateVar\big) \bigg) \\
        &= \sup_{\rho\in\bbR} \rho\big|\diffVar_3\big|^2 \\
        &= \infty
      \end{align*}
    \item
      Now suppose $\affInt(\stateVar_1) < 0$.
      It is clear that selecting $\momentVar_1 = \momentVar_2 = \momentVar_3 = 0$ in (\ref{eq:optimizing-integrand}) cancels many of the terms unrelated to $\diffVar_4$, which lends us to the following argument.
      \begin{align*}
        \hat\ode^*(\hat{\diffVar}, \hat\stateVar) 
        = \sup_{\hat\momentVar \in \hat\vecSpace} \bigg( \bprj{\hat\momentVar}{\hat\diffVar} - \hat\ode\big(\hat\momentVar, \hat\stateVar\big) \bigg) 
        &\geq \sup_{\substack{\hat\momentVar=(0, 0, 0, \rho)  \\ \rho \in \bbR}} \bigg( \bprj{\hat\momentVar}{\hat\diffVar} - \hat\ode\big(\hat\momentVar, \hat\stateVar\big) \bigg) \\
        &= \sup_{\rho\in\bbR} \bigg( \rho\diffVar_4 + \affInt(\stateVar_1) (1 - e^\rho) \bigg) \\
        &= \lim_{\rho\rightarrow-\infty} \bigg( \rho\diffVar_4 + \affInt(\stateVar_1) (1 - e^\rho) \bigg) \\
        &= \infty
      \end{align*}
    \item
      Now suppose $\affInt(\stateVar_1) > 0$ and $\diffVar_4 < 0$.
      Again using the same selection as the preceding part, we have another infinite value.
      \begin{align*}
        \hat\ode^*(\hat{\diffVar}, \hat\stateVar) 
        = \sup_{\hat\momentVar \in \hat\vecSpace} \bigg( \bprj{\hat\momentVar}{\hat\diffVar} - \hat\ode\big(\hat\momentVar, \hat\stateVar\big) \bigg) 
        &\geq \sup_{\substack{\hat\momentVar=(0, 0, 0, \rho)  \\ \rho \in \bbR}} \bigg( \bprj{\hat\momentVar}{\hat\diffVar} - \hat\ode\big(\hat\momentVar, \hat\stateVar\big) \bigg) \\
        &= \sup_{\rho\in\bbR} \bigg( \rho\diffVar_4 + \affInt(\stateVar_1) (1 - e^\rho) \bigg) \\
        &= \lim_{\rho\rightarrow\infty} \bigg( \rho\diffVar_4 + \affInt(\stateVar_1) (1 - e^\rho) \bigg) \\
        &= \infty
      \end{align*}
    \item
      Now suppose $\diffVar_1 = \affDrift(\stateVar_1) + \diffVar_2 + \diffVar_3$, $\diffVar_2 \in \operatorname{image}(\affDiff(\stateVar_1))$, $\affInt(\stateVar_1) > 0$, and $\diffVar_4 > 0$.
      Using Lemma \ref{lemma:ode-differentiable}, we are able to take the gradients of the expression in (\ref{eq:optimizing-integrand}) to get the following system of equations.
      \begin{align*}
        0 &= \begin{aligned}[t]
          &\diffVar_1 - \affDrift(\stateVar_1) - \affDiff(\stateVar_1) (\momentVar_1 + \momentVar_2) \\
          &- \affInt(\stateVar_1) \exp\Big( \ode_{\affDist(\stateVar_1,\cdot)}(\momentVar_1+\momentVar_3) + \momentVar_4\Big) \nabla\ode_{\affDist(\stateVar_1,\cdot)}(\momentVar_1+\momentVar_3) \\
          &+ \affInt(\stateVar_1) \overline{\affDist(\stateVar_1, \cdot)}
        \end{aligned} \\
        0 &= \diffVar_2 - \affDiff(\stateVar_1) ( \momentVar_1 + \momentVar_2 ) \\
        0 &= \begin{aligned}[t]
          &\diffVar_3 - \affInt(\stateVar_1) \exp\Big( \ode_{\affDist(\stateVar_1,\cdot)}(\momentVar_1+\momentVar_3) + \momentVar_4\Big) \nabla\ode_{\affDist(\stateVar_1,\cdot)}(\momentVar_1+\momentVar_3) \\
          &+ \affInt(\stateVar_1) \overline{\affDist(\stateVar_1, \cdot)} 
        \end{aligned} \\
        0 &= \diffVar_4 - \affInt(\stateVar_1) \exp\Big( \ode_{\affDist(\stateVar_1,\cdot)}(\momentVar_1+\momentVar_3) + \momentVar_4\Big) 
      \end{align*}
      These equations yield the following via simple algebra.
      \begin{equation}
        \label{eq:semi-closed-algebra}
        \begin{aligned}
          \momentVar_1 + \momentVar_2 &= \affDiff(\stateVar_1)^\dagger \diffVar_2 \\
          \diffVar_4 &= \affInt(\stateVar_1) \exp\Big( \ode_{\affDist(\stateVar_1,\cdot)}(\momentVar_1+\momentVar_3) + \momentVar_4\Big)  \\
          \diffVar_4\log\Big(\frac{\diffVar_4}{\affInt(\stateVar_1)}\Big) &=   \diffVar_4\ode_{\affDist(\stateVar_1,\cdot)}(\momentVar_1+\momentVar_3) + \momentVar_4\diffVar_4 \\
          \frac{\diffVar_3 + \affInt(\stateVar_1)\overline{\affDist(\stateVar_1,\cdot)}}{\diffVar_4} &= \nabla\ode_{\affDist(\stateVar_1,\cdot)}(\momentVar_1+\momentVar_3)
        \end{aligned}
      \end{equation}
      Seeing as $\ode_{\affDist(\stateVar_1,\cdot)}$ is a convex function, the last equality of (\ref{eq:semi-closed-algebra}) corresponds to the unique extreme point of the Fenchel-Legendre transform.
      \begin{align*}
        &\Bprj{\momentVar_1 + \momentVar_3}{\frac{\diffVar_3+\affInt(\stateVar_1)\overline{\affDist(\stateVar_1,\cdot)}}{\diffVar_4}} - \ode_{\affDist(\stateVar_1,\cdot)}(\momentVar_1+\momentVar_3) \\
        &\hspace{55mm}= \ode_{\affDist(\stateVar_1,\cdot)}^*\Big(\frac{\diffVar_3+\affInt(\stateVar_1)\overline{\affDist(\stateVar_1,\cdot)}}{\diffVar_4}\Big)
      \end{align*}
      Combining this equality with those of (\ref{eq:semi-closed-algebra}) now gives us the following identity.
      \begin{align*}
        \bprj{\hat\momentVar}{\hat\diffVar} - \hat\ode\big(\hat\momentVar, \hat\stateVar\big) 
        &=\begin{aligned}[t]
          &\prj{\momentVar_1}{\diffVar_1} 
          + \prj{\momentVar_2}{\diffVar_2} 
          + \prj{\momentVar_3}{\diffVar_3} 
          + \momentVar_4\diffVar_4
          - \prj{\momentVar_1}{\affDrift(\stateVar_1)}  \\
          &\quad- \frac12\bprj{\momentVar_1+\momentVar_2}{\affDiff(\stateVar)(\momentVar_1+\momentVar_2)} \\
          &\quad- \affInt(\stateVar_1)\exp\Big(\ode_{\affDist(\stateVar_1,\cdot)}(\momentVar_1+\momentVar_3) + \momentVar_4\Big) \\
          &\quad+ \affInt(\stateVar_1) 
          + \bprj{\momentVar_1+\momentVar_3}{\affInt(\stateVar_1)\overline{\affDist(\stateVar_1,\cdot)}}
        \end{aligned} \\
        &= \begin{aligned}[t]
          &\frac12\Bprj{\diffVar_2}{\affDiff(\stateVar_1)^\dagger\diffVar_2} + \diffVar_4 \log\Big(\frac{\diffVar_4}{\affInt(\stateVar_1)}\Big) - \diffVar_4 + \affInt(\stateVar_1) \\
          &+ \ode_{\affDist(\stateVar_1,\cdot)}^*\Big(\frac{\diffVar_3+\affInt(\stateVar_1)\overline{\affDist(\stateVar_1,\cdot)}}{\diffVar_4}\Big)
        \end{aligned}
      \end{align*}
      Observe that since $\hat\ode(\cdot,\hat\stateVar)$ convex, the critical point we have solved is a global extremum, evaluating $\hat\ode^*(\hat\diffVar,\hat\stateVar)$.
    \item
      The last case we must consider is when $\diffVar_1 = \affDrift(\stateVar_1) + \diffVar_2 + \diffVar_3$, $\diffVar_2 \in \operatorname{image}(\affDiff(\stateVar_1))$, $\affInt(\stateVar_1) > 0$, and $\diffVar_4 = 0$.
      For this, we start by greedily optimizing in the $\momentVar_4$ coordinate in (\ref{eq:optimizing-integrand}).
      \begin{align*}
        \hat\ode^*(\hat{\diffVar}, \hat\stateVar) 
        &= \sup_{\hat\momentVar \in \hat\vecSpace} \bigg( \bprj{\hat\momentVar}{\hat\diffVar} - \hat\ode\big(\hat\momentVar, \hat\stateVar\big) \bigg)  \\
        &= \sup_{\momentVar_1,\momentVar_2,\momentVar_3 \in \vecSpace} \bigg[ \begin{aligned}[t]
          & \bprj{\momentVar_1}{\diffVar_1 - \affDrift(\stateVar_1) - \diffVar_2 - \diffVar_3} \\
          &+ \Bprj{\momentVar_1+\momentVar_2}{\diffVar_2 
          - \frac12\affDiff(\stateVar_1)(\momentVar_1+\momentVar_2)} \\
          &+ \Bprj{\momentVar_1+\momentVar_3}{\diffVar_3 + \affInt(\stateVar_1)\overline{\affDist(\stateVar_1,\cdot)}} 
          + \affInt(\stateVar_1) \\
          &+ \sup_{\momentVar_4 \in \bbR} \bigg(- \affInt(\stateVar_1) \exp\Big(\ode_{\affDist(\stateVar_1,\cdot)}(\momentVar_1+\momentVar_3) + \momentVar_4\Big) \bigg)
          \bigg]
        \end{aligned} \\
        &= \sup_{\momentVar_1,\momentVar_2,\momentVar_3 \in \vecSpace} \bigg[ \begin{aligned}[t]
          & \bprj{\momentVar_1}{\diffVar_1 - \affDrift(\stateVar_1) - \diffVar_2 - \diffVar_3}  \\
          &+ \Bprj{\momentVar_1+\momentVar_2}{\diffVar_2 
          - \frac12\affDiff(\stateVar_1)(\momentVar_1+\momentVar_2)} \\
          &+ \Bprj{\momentVar_1+\momentVar_3}{\diffVar_3 + \affInt(\stateVar_1)\overline{\affDist(\stateVar_1,\cdot)}} 
          + \affInt(\stateVar_1) 
          \bigg]
        \end{aligned}
      \end{align*}
      From here, we are left with yet another convex function we intend to optimize; observe that the critical point now easily evaluates our expression.
      \begin{align*}
        &\begin{aligned}
          0 &= \diffVar_1 - \affDrift(\stateVar_1) - \affDiff(\stateVar_1) (\momentVar_1 + \momentVar_2) + \affInt(\stateVar_1) \overline{\affDist(\stateVar_1, \cdot)} \\
          0 &= \diffVar_2 - \affDiff(\stateVar_1) ( \momentVar_1 + \momentVar_2 ) \\
          0 &= \diffVar_3 + \affInt(\stateVar_1) \overline{\affDist(\stateVar_1, \cdot)} 
        \end{aligned} \\
        &\hspace{35mm}\Longrightarrow\quad
        \begin{aligned}[t]
          \prj{\hat\momentVar}{\hat\diffVar} - \hat\ode(\hat\momentVar,\hat\stateVar)
          &= \frac12 \bprj{\diffVar_2}{\affDiff(\stateVar_1)^\dagger\diffVar_2} + \affInt(\stateVar_1)
        \end{aligned}
      \end{align*}
      Thus, this expression solves $\ode^*(\hat\diffVar,\hat\stateVar)$.
      Note that this expression is identical to the previous case, when taking convention that $0 \log 0 = 0$ and $0 \cdot \ode_{\affDist(\cdot,\cdot)}(\cdot) = 0$
  \end{enumerate}
  These cases simplify the nature of $\ode^*$ in (\ref{eq:rf-integral}), finishing the proof.
\end{proof}
