\begin{example}[Compound-Poisson]
  \label{example:compound-poisson}
  Returning to the example above, let us consider the following family $(\Xe)_{\epsilon>0}$ determined by base path $\X$.
  \begin{equation*}
    \Xe_t = \epsilon\sum_{k=1}^{\N_{t/\epsilon}} \V_k
  \end{equation*}
  To understand the deviations of this object, we observe how it appears to be a composition of two processes.
  \begin{equation*}
    \Xe_t = \epsilon A^\epsilon_{\epsilon N^\epsilon_t} = \Big((\epsilon A^\epsilon) \circ (\epsilon \N^\epsilon)\Big)_t, \quad
    \epsilon A^\epsilon_t \defeq \epsilon \sum_{k=1}^{[t/\epsilon]} \V_k, \quad
    \epsilon \N^\epsilon_t \defeq \epsilon\N_{t/\epsilon}
  \end{equation*}
  Note that the deviations of $\epsilon A^\epsilon$ are understood by Mogulskii's theorem, while $\epsilon \N^\epsilon$ are understood from Example \ref{example:poisson}.
  Moreover, these processes are independent, so it is easy to see that $(\epsilon A^\epsilon, \epsilon \N^\epsilon)_{\epsilon>0}$ satisfies a large deviation principle with rate function in which the finite points evaluate as the following integral.
  \begin{equation*}
    \rf_{A\N}(\gamma, \eta) = \int_0^\infty \bigg( \dot\eta(t)\log\dot\eta(t) - \dot\eta(t) + 1 + \ode_\affDist^*\big(\dot\gamma(t)\big) \bigg) \rmd t
  \end{equation*}
  From here, $(\Xe, \epsilon \N^\epsilon)_{\epsilon>0}$ should satisfy a large deviation principle via the contraction principle of $\Xe = \epsilon A^\epsilon \circ \epsilon\N^\epsilon$.
  The rate function is then as follows.
  \begin{equation*}
    \rf_{\X\N}(\pathVar, \eta) = \inf\big\{ \rf_{AB}(\gamma, \eta) : \pathVar = \gamma \circ \eta \big\}
  \end{equation*}
  From here, we recognize that the condition $\pathVar = \gamma \circ \eta$ implies $\dot\gamma = \frac{\dot\pathVar \circ \eta^{-1} }{\dot\eta \circ \eta^{-1}}$ and so substituting this in $\rf_{AB}$ and making a time-change $t \leftarrow \eta^{-1}(t)$ (on only the final term) will reduce to the following expression for finite points.
  \begin{equation}
    \label{eq:rf-compound-poisson}
    \rf_{\X\N}(\pathVar, \eta) =
      \int_0^\infty \bigg( \dot\eta(t)\log\dot\eta(t) - \dot\eta(t) + 1 + \dot\eta(t) \ode_\affDist^*\Big( \frac{\dot\pathVar(t)}{\dot\eta(t)} \Big) \bigg) \rmd t
  \end{equation}
  This integrand should make intuitive sense when we recognize that $(\pathVar, \eta)$ serves as a proxy for $(\X, \N)$.
  The first three terms accumulate rate for deviations in the arrivals of $\N$, while the last term is accumulating rate for deviations in the jump sizes.
  The term $\dot\pathVar(t)/\dot\eta(t)$ is a deviation of $\X$ normalized against one in $\N$, which is effectively a deviation from a jump $\V_k$.
  Deviations of a jump $\V_k$ are measured by $\ode_\affDist^*$, and they are arriving at a rate of $\N$, hence the final expression in the rate function.
  For more detail on how a large deviation principle could be proven with this type of argument, we refer the reader to \cite{duffy2004}.
  In fact, Theorem 4 in this paper inspired us to use coupling as an argument.

  In any case, we may still refer to Theorem \ref{theorem:ldp-integral} for a rigorous argument.
  Denote $(\Omega, \Sigma, \Prb)$ the space that includes all these quantities, $\scrF^\N$ the filtration associated with $\N$, and $\hat\Xee = (\Xee, \N_{\cdot/\epsilon})$.
  It is clear that $\hat\Xee$ is a pure jump process,
  \begin{equation*}
    \hat\Xee_t = \id_{\vecSpace\times\bbR_+} \ast \jumpMeas^{\hat\Xee}_t,
  \end{equation*}
  and that any predictable $H: \Omega \times \bbR_+ \times (\vecSpace \times \bbR_+) \rightarrow \bbR_+$ is such that the following holds.
  \begin{align*}
    \Exp_\Prb \Big( H \ast \jumpMeas^{\hat\Xee}_\infty \Big)
    &= \Exp_\Prb \int_0^\infty H(\cdot, s, \Delta \X_s, 1) \rmd\N_{s/\epsilon}  \\
    &= \Exp_\Prb \int_0^\infty H(\cdot, \epsilon s, \Delta \X_{\epsilon s}, 1) \rmd\N_s  \\
    &= \Exp_\Prb \bigg( \Exp_\Prb \bigg( \int_0^\infty H(\cdot, \epsilon s, \Delta\X_{\epsilon s}, 1) \rmd\N_s  | \scrF^N_\infty \bigg) \bigg) \\
    &= \Exp_\Prb \bigg( \int_\vecSpace\int_0^\infty H(\cdot, \epsilon s, \markVar_1, 1) \rmd\N_s \affDist(\rmd\markVar_1) \bigg) \\
    &= \int_0^\infty \int_\vecSpace H(\cdot, \epsilon s, \markVar_1, 1) \affDist(\rmd\markVar_1)  \rmd s \\
    &= \int_0^\infty \int_\vecSpace H(\cdot, s, \markVar_1, \markVar_2) \frac1\epsilon \affDist(\rmd\markVar_1) \delta_1(\rmd\markVar_2) \rmd s, & s \leftarrow \epsilon s
  \end{align*}
  This means that we have the following predictable compensator,
  \begin{equation*}
    \predproj{\jumpMeas^{\hat\Xee}}(\rmd s, \rmd\hat\markVar) = \frac1\epsilon \affDist(\rmd\markVar_1) \delta_1(\rmd\markVar_2)\rmd s
  \end{equation*}
  which by our light-tails (\ref{eq:cumulant-generating-function}) and Proposition \ref{proposition:affine-special} means that $\hat\Xee$ is special and has the following decomposition.
  \begin{equation*}
    \hat\Xee
    = \id_{\vecSpace\times\bbR_+} \ast \predproj{\jumpMeas^{\hat\Xee}}_t + \id_{\vecSpace\times\bbR_+} \ast \compensate{\jumpMeas^{\hat\Xee}}_t 
    = \frac1\epsilon\begin{pmatrix} 
      \overline\affDist \\
      1
    \end{pmatrix} \shortInt \leb_t + \id_{\vecSpace\times\bbR_+} \ast \compensate{\jumpMeas^{\hat\Xee}}_t
  \end{equation*}
  This means that the special differential characteristics $(\affDrift^\epsilon, \affDiff^\epsilon, \affJump^\epsilon)$ of $\hat\Xee$ are as follows.
  \begin{equation*}
    \affDrift^\epsilon(\hat\stateVar) = \frac1\epsilon \begin{pmatrix} \overline\affDist \\ 1 \end{pmatrix}, \quad
    \affDiff^\epsilon(\hat\stateVar) = 0, \quad
    \affJump^\epsilon(\hat\stateVar, \rmd\hat\markVar) = \frac1\epsilon \affDist(\rmd\markVar_1) \delta_1(\rmd\markVar_2)
  \end{equation*}
  This parameterization, along with our light-tails (\ref{eq:cumulant-generating-function}) means that we have the hypotheses of Theorem \ref{theorem:ldp-integral}.
  Note that the associated L\'evy-Khintchine map $\ode$ is as follows.
  \begin{align*}
    \ode(\hat\momentVar, \hat\stateVar) 
    &= \prj{\momentVar_1}{\overline\affDist} + \momentVar_2 + \int_{\vecSpace \times \bbR_+} \Big( e^{\prj{\momentVar_1}{\markVar_1} + \momentVar_2} - 1 - \prj{\momentVar_1}{\markVar_1} - \momentVar_2\markVar_2 \Big) \affDist(\rmd\markVar_1)\delta_1(\rmd\markVar_2) \\
    &= e^{\ode_\affDist(\momentVar_1)+\momentVar_2} - 1
  \end{align*}
  The integrand of our rate function \ref{eq:rf-integral} thus involves the following expression.
  \begin{equation*}
    \ode^*\big(\hat\diffVar, \hat\stateVar\big)
    = \sup_{\hat\momentVar\in\vecSpace \times \bbR_+} \bigg( \prj{\momentVar_1}{\diffVar_1} - \momentVar_2\diffVar_2 - e^{\ode_\affDist(\momentVar_1) + \momentVar_2} + 1 \bigg)
  \end{equation*}
  Setting the gradients to zero results in the system of equations.
  \begin{align*}
    0 &= \diffVar_1 - e^{\ode_\affDist(\momentVar_1)+\momentVar_2} \nabla\ode_\affDist(\momentVar_1) \\
    0 &= \diffVar_2 - e^{\ode_\affDist(\momentVar_1)+\momentVar_2}
  \end{align*}
  Some immediate results come from this; firstly, the second equation gives us that $\diffVar_2 > 0$ and the following.
  \begin{equation}
    \label{eq:compound-poisson-solve-1}
    \diffVar_2 = e^{\ode_\affDist(\momentVar_1)+\momentVar_2}, \quad \diffVar_2 \ode_\affDist(\momentVar_1) = \diffVar_2\log\diffVar_2 - \momentVar_2\diffVar_2
  \end{equation}
  Dividing the common expressions between the two equations results to the following.
  \begin{equation*}
    \nabla\ode_\affDist(\momentVar_1) = \frac{\diffVar_1}{\diffVar_2}
  \end{equation*}
  Seeing as $\ode$ is convex, this is the unique solution to its Fenchel-Legendre transform.
  \begin{equation*}
    \ode_\affDist^*\Big( \frac{\diffVar_1}{\diffVar_2} \Big) = \prj{\momentVar_1}{\diffVar_1/\diffVar_2} - \ode(\momentVar_1)
  \end{equation*}
  Scooting the expressions around, we get
  \begin{equation*}
    \prj{\momentVar_1}{\diffVar_1} =  \diffVar_2\ode_\affDist^*\Big( \frac{\diffVar_1}{\diffVar_2} \Big) + \diffVar_2\ode(\momentVar_1),
  \end{equation*}
  which when combined with (\ref{eq:compound-poisson-solve-1}), we get the following.
  \begin{equation*} 
    \prj{\momentVar_1}{\diffVar_1} - \momentVar_2\diffVar_2 - e^{\ode_\affDist(\momentVar_1) + \momentVar_2} + 1 
    = \diffVar_2 \log \diffVar_2 - \diffVar_2 + 1 + \ode_\affDist^*\Big(\frac{\diffVar_1}{\diffVar_2} \Big)
  \end{equation*}
  Observe that this results in the same integral as (\ref{eq:rf-compound-poisson}).
\end{example}
