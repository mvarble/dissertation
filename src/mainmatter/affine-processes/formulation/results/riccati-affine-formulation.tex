\begin{remark}
  \label{remark:riccati-affine-formulation}
  By Remark \ref{remark:affine-remarks}\ref{remark:affine-parts}, Theorem \ref{theorem:affine-regularity} is equivalent to the existence of maps
  \begin{equation}
    \begin{gathered}
      \affDrift^\trunc(\stateVar) = \affDriftPart_0^\trunc + \sum_{i=1}^d \stateVar^i \affDriftPart_i^\trunc, \qquad
      \affDiff(\stateVar) = \affDiffPart_0 + \sum_{i=1}^d \stateVar^i \affDiffPart_i, \\
      \affJump(\stateVar, \rmd\markVar) = \affJumpPart_0(\rmd\markVar) + \sum_{i=1}^d \stateVar^i \affJumpPart_i(\rmd\markVar)
    \end{gathered}
  \end{equation}
  which specify the dynamics of $\aff$.
  \begin{equation}
    \label{test}
  \end{equation}
  \begin{equation}
    \forall \stateVar \in \stateSpace, \qquad \left\{\begin{aligned}
      \dot\aff(t, \momentVar, \stateVar) &= \ode\big(\affPart(t,\momentVar), \stateVar\big) & t \geq 0 \\
      \aff(0,\momentVar, \stateVar) &= \prj{\momentVar}{\stateVar}
    \end{aligned}\right.
  \end{equation}
  Indeed, this is also because differentiability and our initial condition in (\ref{test}) are linear.
\end{remark}
