\begin{color}{gray}
  \begin{enumerate}
    \item
      Summarize how multiple \emph{frameworks} are utilized: Dembo, Feng, Puhalskii
    \item
      Summarize history of works treating DG+EM differently
  \end{enumerate}
\end{color}


\begin{enumerate}
  \item
    {\bfseries Familiar large deviation principles.}
    Purpose of section is to familiarize with rate functions we already have and set the stage for how we operate with our theorem.
    \begin{enumerate}
      \item
        Cite Mogulskii's theorem.
      \item
        Brownian motion of our regime: achieved with Mogulskii's theorem with Normal increments and exponential tightness (Schilder).
      \item
        Poisson process of our regime: achieved with Mogulskii's theorem with Poisson increments and use exponential tightness.
      \item
        Friedlin Wentzel (not so much our regime): use contraction mapping principles
      \item
        Birth rate process: Can we similarly contraction map Poisson to get this?
      \item
        Extension of Freidlin Wentzel to affine: KK
    \end{enumerate}
  \item
    {\bfseries Coupling states.}
    Indicate that when contraction mappings are not sufficient, we may \emph{couple correlated states}, in the sense of looking at LDPs of joint processes.
    \begin{enumerate}
      \item
        Compound Poisson: Two \emph{sources} of randomness; the arrivals and the jump sizes. Appeal to Duffy results for heuristical calculations.
      \item
        Compound linear Hawkes: Similarly two \emph{sources} of randomness. Duffy also gives us the calculations.
        Note on Zhu paper for \emph{sidestep}; less general jumps, more general nonlinear relationship of arrivals.
      \item
        The jumps of a general jump-diffusion do not have a well-posed notion of arrivals and jump sizes; we thus turn our focus to locally countable jump-diffusions, in which the three \emph{sources} of randomness are the continuous local martingale, the arrival times, and the jump sizes.
        Note how this is discussed in next section.
    \end{enumerate}
  \item
    {\bfseries Locally countable affine processes.}
    Perform the necessary calculus and proceed to show our general formulation.
    \begin{enumerate}
      \item
        State result in numerous flavors, depending on which \emph{base} quantities in which we choose to focus the large deviations.
          \begin{align*}
            \X &= \affDrift(\X)\shortInt \leb + \X^\rmc + \id_\vecSpace \ast \compensate{\jumpMeas^\X} \\
            \N^\X &= 1 \ast \jumpMeas^\X \\
            V^\X &= \id_\vecSpace \ast \jumpMeas^\X 
          \end{align*}
        \begin{enumerate}
          \item
            \textbf{overdetermined flavor.}
            $(\X, \X^\rmc, \N^\X, V^\X)$ produces an overdetermined system which requires another condition for $\rf(\pathVar, \omega, \eta, \gamma)$ to be finite.
            \[
              \dot\pathVar(t) = \affDrift\big(\pathVar(t)\big) + \dot\omega(t) + \dot\eta(t)\dot\gamma(t)
            \]
            However, the rate function is very simple to understand.
          \item
            \textbf{determine-continuous-noise flavor.}
            Normal term gets messy
          \item
            \textbf{determine-arrivals flavor.}
            Poisson and jump-term-denominator gets messy
          \item
            \textbf{determine-jumps flavor.}
            This is the one we have already presented; the jump-term-numerator gets messy.
        \end{enumerate}
      \item
        Discuss how the deviations of $\X$ from the dynamical system $\X = \affDrift(\X) \shortInt \leb$ are imposed from \emph{continuous deviations} $\X^\rmc$ and \emph{discontinuous deviations} $\id_\vecSpace \ast \compensate{\jumpMeas^\X}$.
      \item
        Discuss how four quantities $\X, \X^\rmc, \N^\X, V^\X$ heuristically relate in simple infinitesimal equality $(\X, \X^\rmc, \N^\X, V^\X) \approx (\pathVar, \omega, \eta, \gamma)$.
        \begin{align*}
          \dot\pathVar(t) = \affDrift\big(\pathVar(t)\big) + \dot\omega(t) + \dot\eta(t) \cdot \dot\gamma(t)
        \end{align*}
      \item
        Each of the primitive deviations have a simple analogy, when we think of infinitesimals.
        \begin{align*}
          &\dot\omega(t)  & \text{\emph{normal deviations of covariance }} \affDiff\big(\pathVar(t)\big) \\
          &\dot\eta(t) & \text{\emph{Poisson deviations of rate }} \affInt\big(\pathVar(t)\big) \\
          &\dot\gamma(t) & \text{\emph{jump deviations from }} \affDist\big(\pathVar(t), \rmd\markVar\big) \\
          \pathVar(t) &=\affDrift\big(\pathVar(t)\big) + \dot\omega(t) + \dot\eta(t)\cdot\dot\gamma(t) & \text{\emph{all combined deviations}}
        \end{align*}
      \item
        Think of results from first section in this regard.
        \begin{enumerate}
          \item
            birth is  $\dot\pathVar(t) = \dot\eta(t)$, so we only need $\pathVar \approx \X$.
          \item
            diffusion is $\dot\pathVar(t) = \affDrift\big(\pathVar(t)\big) + \dot\omega(t)$, and so we only need $\pathVar\approx\X$ and rate function includes $\pathVar(t) - \affDrift\big(\pathVar(t)\big)$ where $\dot\omega$ is.
          \item
            compound Poisson is $\dot\pathVar(t) = \dot\eta(t) \cdot \dot\gamma(t)$, so we choose one of the following pairs $(\pathVar, \eta) \approx (\X, \N^\X)$, $(\pathVar, \gamma) \approx (\X, V^\X)$, or $(\eta, \gamma) \approx (\N^\X, V^\X)$.
          \item
            compound linear Hawkes is $\dot\pathVar(t) = \affDrift\big(\pathVar(t)\big) + \dot\eta(t) \cdot \dot\gamma(t)$, so we can choose $(\pathVar, \eta) \approx (\X, \N^\X)$ or $(\pathVar, \dot\gamma) \approx (\X, V^\X)$.
        \end{enumerate}
    \end{enumerate}
\end{enumerate}
