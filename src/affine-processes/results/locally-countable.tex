\begin{proposition}
  \label{proposition:locally-countable}
  If the jump kernel satisfies $\affJump(\stateVar, \vecSpace) < \infty$ for all $\stateVar \in \stateSpace$, then $\X$ is $(\Prb_\stateVar, \scrF)$ locally countable for all $\stateVar \in \stateSpace$.
  In the resulting factorization,
  \begin{equation*}
    \affJump(\stateVar, \rmd\markVar) = \affInt(\stateVar) \affDist(\stateVar, \rmd\markVar),
  \end{equation*}
  the intensity $\affInt$ is an affine map and the jump distribution $\affDist$ is a convex mixture of probability distributions $\affDistPart_0, \ldots, \affDistPart_d$ whenever $\affInt(\stateVar) \neq 0$.
  \begin{equation*}
    \affInt(\stateVar) = \affIntPart_0 + \sum_{i=1}^d \stateVar^i \affIntPart_i, \qquad \affDist(\stateVar, \rmd\markVar) = \frac{\affIntPart_0}{\affInt(\stateVar)} \affDistPart_0(\rmd\markVar) + \sum_{i=1}^d \frac{\stateVar^i\affIntPart_i}{\affInt(\stateVar)} \affDistPart_i(\rmd\markVar), \quad 
  \end{equation*}
\end{proposition}
\begin{proof}
  \label{proof:proposition:locally-countable}
  By combining Lemmas \ref{lemma:affJump-regularity} and \ref{lemma:countable}, we get the desired local countability.
  Because $0 \in \stateSpace$ and $\operatorname{span}\stateSpace = \vecSpace$, we are able to take appropriate linear combinations to ensure finiteness of the quantities $\affIntPart_i \defeq \affJumpPart_i(\vecSpace)$ for each $i = 0, \ldots, d$.
  This allows us to define our intensity map.
  \begin{equation*}
    \affInt(\stateVar) \defeq \affIntPart_0 + \sum_{i=1}^d \stateVar^i \affIntPart_i = \affJumpPart_0(\vecSpace) + \sum_{i=1}^d \stateVar^i \affJumpPart_i(\vecSpace) = \affJump(\stateVar, \vecSpace)
  \end{equation*}
  Now, just as in Remark \ref{remark:countable}, each non-zero $\affIntPart_i$ will produce a probability distribution $\affDistPart_i(\rmd\markVar) \defeq \affJumpPart_i(\rmd\markVar) / \affIntPart_i$; otherwise, simply define $\affDistPart_i(\rmd\markVar) \defeq \delta_{\basisVec_1}$.
  This way, we have the factoring $\affJumpPart_i(\rmd\markVar) = \affIntPart_i\affDistPart_i(\rmd\markVar)$ for each $i = 0, \ldots, d$.
  If $\affInt(\stateVar) \neq 0$, we see our other desired identity.
  \begin{align*}
    \affDist(\stateVar, \rmd\markVar) 
    &\defeq \frac{1}{\affInt(\stateVar)} \affJump(\stateVar, \rmd\markVar) \\
    &= \frac{1}{\affInt(\stateVar)} \bigg( \affJumpPart_0(\rmd\markVar) + \sum_{i=1}^d \stateVar^i \affJumpPart_i(\rmd\markVar) \bigg) \\
    &= \frac{1}{\affInt(\stateVar)} \bigg( \affIntPart_0 \affDistPart_0(\rmd\markVar) + \sum_{i=1}^d \stateVar^i \affIntPart_i \affDistPart_i(\rmd\markVar) \bigg) \\
    &= \frac{\affIntPart_0}{\affInt(\stateVar)} \affDistPart_0(\rmd\markVar) + \sum_{i=1}^d \frac{\stateVar^i\affIntPart_i}{\affInt(\stateVar)} \affDistPart_i(\rmd\markVar) 
  \end{align*}
\end{proof}
