Our first chapter will get us familiar with affine processes.
Most of the results will be simple consequences of \cite{cuchiero2011} and \cite{keller2015}, which pretty much comprehensively prove everything there is to know about affine processes on convex state spaces.
It is still important for us to have this chapter, for we will need to have a careful understanding of how the spaces of finite moments change with respect to time, and these facts only seem to exist in the case of affine diffusions.
Furthermore, the magnificent papers above provide us so much information at the cost of being very formal.
Readers which have not found themselves familiar with massive texts like \cite{jacod2003} may not be able to decipher the results involving semimartingale characteristics in intuitive terms.
It is for this reason that we found it beneficial to provide some calculus-heavy proofs that somehow \emph{demystify} these objects.
In particular, we translate many of the results of \cite{jacod2003} to the special setting of \emph{jump-diffusions}---those semimartingales in which the characteristics are differentiable.
While certainly less powerful than their general semimartingale brethren, jump-diffusions involve deterministic functions $(\affDrift, \affDiff, \affJump)$ which make calculations far more intuitive.
Furthermore, they can always be understood as weak solutions to stochastic differential equations involving a Brownian motion and a Poisson random measure, which gives them a somewhat \emph{generative} flavor.
In the case of affine processes, we may remove one more layer of abstraction, insisting that the functions each take an affine form in their respective vector spaces.
\begin{align*}
  \affDrift(\stateVar^1, \ldots, \stateVar^d) &= \affDriftPart_0 + \sum_{i=1}^d \stateVar^i \affDriftPart_i \\
  \affDiff(\stateVar^1, \ldots, \stateVar^d) &= \affDiffPart_0 + \sum_{i=1}^d \stateVar^i \affDiffPart_i \\
  \affJump(\stateVar^1, \ldots, \stateVar^d, \rmd\markVar) &= \affJumpPart_0(\rmd\markVar) + \sum_{i=1}^d \stateVar^i \affJumpPart_i(\rmd\markVar) 
\end{align*}
Any result which does not require this affine property is in Appendix \ref{jump-diffusions}---a choice which was made less because the information is unrelated to affine processes, and more that the extra generality of non-affine differential characteristics comes at little-to-no cost.
In any case, we suggest the reader familiarize themselves with Appendix \ref{jump-diffusions} sooner than later, as we use this material very often throughout the chapters.
The remainder of this chapter is organized as follows.

\begin{enumerate}[leftmargin=24mm]
  \item[\,{\hyperref[affine-processes:formulation]{Section }}\ref{affine-processes:formulation}.]
    Defines an affine process and identifies some relevant notions from which we begin our investigation.
  \item[\,{\hyperref[affine-processes:real-moments]{Section }}\ref{affine-processes:real-moments}.]
    Explores consequences of the affine transform formula on real moments.
  \item[\,{\hyperref[affine-processes:fdds]{Section }}\ref{affine-processes:fdds}.]
    Lifts the notions of the preceding section from marginals to finite-dimensional distributions.
  \item[\,{\hyperref[affine-processes:jump-diffusions]{Section }}\ref{affine-processes:jump-diffusions}.]
    Explores more on the path-based properties---particularly the jump-diffusion nature---of affine processes.
\end{enumerate}
