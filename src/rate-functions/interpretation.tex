We finish our discussion by interpreting the last result.
As discussed in the introduction, this result gives us the following approximation,
\begin{equation*}
  \hat\Prb^\epsilon_{\hat\stateVar}\Big( \hat\Xe \in \ball{\hat\pathVar}{\delta} \Big) 
  \approx \exp\Big(-\frac1\epsilon \hat\rf_{\hat\stateVar}(\hat\pathVar) \Big),
\end{equation*}
so $\hat\rf_{\hat\stateVar}(\hat\pathVar)$ gives us an exponential rate of decay of probabilities of the family $\hat\Xe$ being near $\hat\pathVar$.
In terms of components $\hat\pathVar = (\pathVar, \omega, \gamma, \eta)$, the function $\hat\rf_{\hat\stateVar}$ thus reads properties of $\pathVar, \omega, \gamma, \eta$ and communicates to us how unlikely they are to be exhibited by $\Xe$, ${\Xe}^{,\rmc}$, ${\Xe}^{,\rmd}$, and $\epsilon\N^{\Xe}$, respectively.
Our semi-closed form in Theorem \ref{theorem:ldp-closed-form} tells us a lot.

Let us first elaborate on the conditions of $\hat\pathVar$ which allow $\hat\rf_{\hat\stateVar}(\hat\pathVar)$ to evaluate as a (possibly infinite) integral.
Should any of these properties not be satisfied, we would have $\hat\rf_{\hat\stateVar}(\hat\pathVar) = \infty$, which means super-exponential decay of $\hat\Xe$ being within a neighborhood of $\hat\pathVar$.
That said, we interpret these conditions as minimum requirements on $\hat\pathVar$ to even be considered \emph{possible}.
Throughout, notice we establish an intuition with $\hat\pathVar$ being a proxy for $\hat\Xe$.

\begin{itemize}
  \item
    $\hat\pathVar(0) = \hat\stateVar$.

    Seeing as we $\hat\Prb^\epsilon_{\hat\stateVar}$-almost-surely have $\epsilon\hat\X^\epsilon_0 = \hat\stateVar$, we insist $\hat\pathVar(0) = \hat\stateVar$.
  \item
    $\hat\pathVar \in \acpathSpace{[0,\infty)}{\hat\stateSpace}$.

    This tells us that the paths of $\hat\Xe$ become smoother as $\epsilon \rightarrow 0$.
    This should make sense when we look at Proposition \ref{proposition:sde-asymptotics}, where $\Xe$ is a dynamical system with perturbations on the scale of $\epsilon$.
  \item
    $\dot\pathVar(t) = \affDrift\big(\pathVar(t)\big) + \dot\omega(t) + \dot\gamma(t)$.

    Note that this is the same thing as saying,
    \[
      \pathVar(t) = \stateVar + \affDrift(\pathVar) \shortInt \leb_t + \omega(t) + \gamma(t),
    \]
    (when we include the initial condition above) which is the exact condition each $\hat\Xe$ satisfies.
    \[
      \Xe_t = \stateVar + \affDrift(\Xe) \shortInt \leb_t + {\Xe_t}^{,\rmc} + {\Xe_t}^{,\rmd}
    \]
  \item
    $\dot\omega(t) \in \operatorname{image}\big(\affDiff(\pathVar(t))\big)$.

    Seeing as $\epsilon \affDiff(\Xe) \shortInt \leb$ is the predictable quadratic covariation process associated with $\epsilon\X^{\epsilon,\rmc}$, we see the following moments of an increment $\epsilon\X^{\epsilon,\rmc}_{t+\tau}-\epsilon\X^{\epsilon,\rmc}_t$.
    \begin{gather*}
      \Exp_{\Prb_\stateVar}\big( \epsilon\X^{\epsilon,\rmc}_{t+\Delta t} - \epsilon\X^{\epsilon,\rmc}_t  | \scrF^\epsilon_t \big) = 0, \\
      \Exp_{\Prb_\stateVar}\Big(\big( \epsilon\X^{\epsilon,i,\rmc}_{t+\Delta t} -  \epsilon\X^{\epsilon,i,\rmc}_t \big) \big( \epsilon\X^{\epsilon,j,\rmc}_{t+\Delta t} -  \epsilon\X^{\epsilon,j,\rmc}_t \big) | \scrF^\epsilon_t \Big) = \Exp_{\Prb_{\Xe_t}} \int_t^{t+\Delta t} \epsilon\affDiff_{ij}\big(\Xe_s\big) \rmd s
    \end{gather*}
    Thus, on an infinitesimal level, the increment $\dot\omega(t)$ of our continuous noise $\omega$ can be thought of as distributing $\operatorname{Normal}(0, \affDiff(\pathVar(t)))$, which begs the question of $\dot\omega(t) \in \operatorname{image}\big(\affDiff(\pathVar(t))\big)$.
  \item
    $\affInt\big(\pathVar(t)\big) > 0$.
    
    Note that we have the following identity,
    \begin{align*}
      \N^{\Xe}_t - \int_0^t \frac1\epsilon \affInt(\Xe_s) \rmd s
      &= 1 \ast \jumpMeas^{\Xe}_t - \int_0^t \int_\vecSpace \frac1\epsilon\affJump(\Xe_s, \rmd\markVar) \rmd s \\
      &= 1 \ast \jumpMeas^{\Xe}_t - 1 \ast \predproj{\jumpMeas^{\Xe}}_t \\
      &= 1 \ast \compensate{\jumpMeas^{\Xe}}_t,
    \end{align*}
    which tells us $\N^{\Xe}$ has intensity $\frac1\epsilon\affInt(\Xe)$.
    On the infinitesimal level, the increment $\dot\eta(t)$ of our arrivals $\eta$ should arrive $\operatorname{Poisson}\big(\affInt(\pathVar(t))\big)$ with positive intensity $\affInt(\pathVar(t))$.
  \item
    $\dot\eta(t) \geq 0$.

    Seeing as $\N^{\Xe}$ is a counting process, its increments of $\epsilon\N^{\Xe}$ are non-negative, and so should those in the infinitesimal sense, $\dot\eta(t) \geq 0$.
\end{itemize}

Following this, for those $\hat\pathVar$ that are \emph{possible}, in the sense of satisfying the preceding properties, we may interpret the integral form of $\hat\rf_{\hat\stateVar}(\hat\pathVar)$ intuitively.
We mentioned in the points above that
\[
  \dot\pathVar(t) = \affDrift\big(\pathVar(t)\big) + \dot\omega(t) + \dot\gamma(t)
\]
is a proxy for our dynamics associated with $\Xe$, and that our derivatives can be interpreted as increments.
\begin{equation*}
  \dot\omega(t) ~\leftrightarrow~ \operatorname{Normal}\Big(0, \affDiff\big(\pathVar(t)\big)\Big), \quad
  \dot\eta(t) ~\leftrightarrow~ \operatorname{Poisson}\Big(\affInt\big(\pathVar(t)\big)\Big)
\end{equation*}
Meanwhile, since $\dot\gamma$ is a proxy for an increment of ${\Xe}^{,\rmd}$, we recognize that the following identity,
\begin{align*}
  {\Xe_t}^{,\rmd} 
  &= \id_\vecSpace \ast \compensate{\jumpMeas^{\Xe}}_t \\
  &= \id_\vecSpace \ast \jumpMeas^{\Xe}_t - \id_\vecSpace \ast \predproj{\jumpMeas^{\Xe}}_t \\
  &= \int_{[0,t]\times \vecSpace} \markVar \jumpMeas^{\Xe}(\rmd s, \rmd\markVar) - \int_0^t \int_\vecSpace \epsilon\markVar \frac1\epsilon\affJump(\Xe_s, \rmd\markVar) \rmd s \\
  &= \int_0^t \Delta(\Xe)_s \, \rmd\N^{\Xe}_s  - \int_0^t \affInt(\Xe_s) \overline{\affDist(\Xe_s, \cdot)},
\end{align*}
gives us a similar infinitesimal interpretation.
\begin{equation*}
  \frac{\dot\gamma(t) + \affInt\big(\pathVar(t)\big) \overline{\affDist(\pathVar(t), \cdot)}}{\dot\eta(t)} \text{ arrives at rate } \dot\eta(t) \text{ with distribution } \affDist\big(\pathVar(t), \cdot\big)
\end{equation*}
In total, we have the following categories of deviations.
\begin{align*}
  &\dot\omega(t) & \text{ continuous deviations, }&\operatorname{Normal}\Big(0, \affDiff\big(\pathVar(t)\big)\Big) \\
  &\dot\eta(t) & \text{ jump-arrival deviations, }& \operatorname{Poisson}\Big(\affInt\big(\pathVar(t)\big)\Big) \\
  &\frac{\dot\gamma(t)+\affInt\big(\pathVar(t)\big)\overline{\affDist(\pathVar(t), \cdot)}}{\dot\eta(t)} & \text{ jump-size deviations, }& \affDist\big(\pathVar(t), \cdot\big) \\
  &\dot\pathVar(t) & \text{ combined deviations, } & \affDrift\big(\pathVar(t)\big) + \dot\omega(t) + \dot\gamma(t)
\end{align*}

Each of these interpretations explains its respective integral term, as we explain below.
\begin{itemize}
  \item
    The continuous deviations $\dot\omega$ in our rate function take the integral,
    \begin{equation*}
      \int_0^\infty \frac12 \Bprj{\dot\omega(t)}{\affDiff\big(\pathVar(t)\big)^\dagger \dot\omega(t)} \rmd t,
    \end{equation*}
    which takes larger values whenever $\dot\omega(t)$ is a rare sample of $\operatorname{Normal}\big(0, \affDiff(\pathVar(t))\big)$; the likely path here is $\dot\omega(t) = 0$.
  \item
    The jump-arrival deviations $\dot\eta$ in our rate function take the integral,
    \begin{equation*}
      \int_0^\infty \bigg( \dot\eta(t)\log\Big(\frac{\dot\eta(t)}{\affInt\big(\pathVar(t)\big)} \Big) - \dot\eta(t) + \affInt\big(\pathVar(t)\big) \bigg) \rmd t,
    \end{equation*}
    which takes larger values whenever $\dot\eta(t)$ is a rare sample of $\operatorname{Poisson}\big(\affInt(\pathVar(t))\big)$; the likely path here is $\dot\eta(t) = \affInt(\pathVar(t))$.
  \item
    The jump-size deviations $\big(\dot\gamma(t)+\affInt(\pathVar(t))\overline{\affDist(\pathVar(t), \cdot)}\big)/\dot\eta(t)$ take the following integral.
    \begin{equation*}
      \int_0^\infty \dot\eta(t) \ode_{\affDist(\pathVar(t),\cdot)}^*\bigg( \frac{\dot\gamma(t) + \affInt\big(\pathVar(t)\big)\overline{\affDist(\pathVar(t), \cdot)}}{\dot\eta(t)} \bigg) \rmd t
    \end{equation*}
    This takes larger values when $\big(\dot\gamma(t)+\affInt(\pathVar(t))\overline{\affDist(\pathVar(t), \cdot)}\big)/\dot\eta(t)$ is a rare sample of $\affDist(\pathVar(t), \cdot)$, and the prefactor $\dot\eta(t)$ serves to express the increased rarity of unlikely jumps occurring with high-frequency.
\end{itemize}

We also would like to mention that Theorem \ref{theorem:ldp-closed-form} can take many equivalent forms, depending on which quantities we couple with $\Xe$ and which parameters $\affDiff, \affJump$ are non-zero..
For instance, as demonstrated by the redundancy,
\[
  \dot\pathVar(t) = \affDrift\big(\pathVar(t)\big) + \dot\omega(t) + \dot\gamma(t),
\]
we may have considered the smaller tuple $(\Xe, {\Xe}^{,\rmd}, \N^{\Xe})$ and the subsequent rate function would have the following substitution in the first integral.
\begin{equation}
  \label{eq:Xc-redundancy}
  \dot\omega(t) = \dot\pathVar(t) - \affDrift\big(\pathVar(t)\big) - \dot\gamma(t),
\end{equation}
Note that in the case of $\affJump=0$ (diffusions), each of ${\Xe}^{,\rmd}$ and $\N^{\Xe}$ constantly take zero-value, and so we need no coupling; the principle for $\Xe$ already has a closed form.
This is hinted at in (\ref{eq:Xc-redundancy}), which suggests the ultimate rate function.
\[
  \dot\omega(t) =\dot\pathVar(t) - \affDrift(\pathVar(t)) \quad\leadsto\quad \int_0^\infty \frac12 \Bprj{\dot\pathVar(t)-\affDrift\big(\pathVar(t)\big)}{\affDiff\big(\pathVar(t)\big)^\dagger \Big(\dot\pathVar(t)-\affDrift\big(\pathVar(t)\big)\Big)} \rmd t,
\]
Similarly, we may have considered another tuple $(\Xe, {\Xe}^{,\rmc}, \N^{\Xe})$ and the subsequent rate function would have the following substitution in the last integral.
\begin{equation}
  \label{eq:Xd-redundancy}
  \dot\gamma(t) = \dot\pathVar(t) - \affDrift\big(\pathVar(t)\big) - \dot\omega(t)
\end{equation}
Note that in the case of $\affDiff=0$ (pure-jump processes), ${\Xe}^{,\rmc}$ constantly takes zero-value, and so we only need coupling $(\Xe, \N^{\Xe})$, as hinted at in (\ref{eq:Xd-redundancy}).
This covers the case of our compund-Poisson and Hawkes processes.
Lastly, we could have alternatively coupled $(\Xe, {\Xe}^{,\rmc}, \id_\vecSpace \ast \jumpMeas^{\Xe}, \N^{\Xe})$ so that the final integral term will clean up a bit,
\begin{equation*}
  \int_0^\infty \dot\eta(t) \ode_{\affDist(\pathVar(t),\cdot)}^*\Big(\frac{\dot\wp(t)}{\dot\eta(t)}\Big) \rmd t
\end{equation*}
with our proxy $\hat\pathVar = (\pathVar, \omega, \wp, \eta)$ now satisfying the following.
\[
  \dot\pathVar(t) = \affDrift\big(\pathVar(t)\big) + \dot\omega(t) + \hat\wp(t) - \affInt\big(\pathVar(t)\big) \overline{\affDist(\pathVar(t), \cdot)}
\]
All of these different perspectives can be proven directly or explained immediately with the contraction principle.
