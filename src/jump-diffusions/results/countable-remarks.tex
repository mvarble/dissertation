\begin{remark}
  \label{remark:countable}
  \begin{enumerate}[label=(\alph*)]
    \item
      Such objects $\affInt, \affDist$ always exist with our assumption of the Lemma.
      Seeing as $\affJump$ is a transition kernel from $(\stateSpace, \scrB(\stateSpace))$ to $(\vecSpace, \scrB(\vecSpace))$, we have our desired measurability.
      \[
        \affInt \defeq \affJump(\cdot, \vecSpace) \in \scrB(\stateSpace)/\scrB(\bbR_+)
      \]
      Constructing $\affDist$ should be obvious algebra, so long as we have no zero measures; otherwise, we may define
      \[
        \affDist(\stateVar, \Gamma) \defeq \delta_{\basisVec_1}(\Gamma) \cdot 1_{\affInt^{-1}\{0\}}(\stateVar) + \frac{\affJump(\stateVar, \Gamma)}{\affInt(\stateVar)} 1_{\stateSpace - \affInt^{-1}\{0\}}(\stateVar),
      \]
      where $\delta_{\basisVec_1}$ is the degenerate measure at $\basisVec_1 \in \vecSpace$. 
      This ensures that any $\affDist(\cdot, \Gamma) \in \scrB(\stateSpace)/\scrB([0,1])$ and any $\affDist(\stateVar, \cdot)$ a probability measure on $\scrB(\vecSpace)$.
      Also, when $\affJump(\stateVar, \cdot)$ is the zero measure,
      \[
        \affJump(\stateVar, \rmd\markVar) = 0 = \affInt(\stateVar) \cdot \delta_{\basisVec_1}(\rmd\markVar) = \affInt(\stateVar) \affDist(\stateVar, \rmd\markVar),
      \]
      and otherwise,
      \[
        \affJump(\stateVar, \rmd\markVar) = \affJump(\stateVar,\vecSpace) \frac{\affJump(\stateVar, \rmd\markVar)}{\affJump(\stateVar, \vecSpace)} = \affInt(\stateVar) \affDist(\stateVar, \rmd\markVar).
      \]
    \item
      We call $\affInt$ the \emph{intensity map} and $\affDist$ the \emph{(conditional) jump distribution}
    \item
      As far as we know, there is no widely accepted source which explores jump-diffusions to the extent of declaring a notion like \emph{locally countable}, as we have.
      This means that there is likely some clash of terminology, should such a concept already exist.
  \end{enumerate}
\end{remark}
