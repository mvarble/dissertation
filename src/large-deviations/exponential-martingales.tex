% discuss how martingales will induce the alternative approach
Now that we have our principle, we discuss how various exponential martingales $\mgDensity^{\epsilon,\twistFunc}$ induce changes of measure alternative to those $\fddDensity^{\epsilon,\underline t,\underline\twistVar}$ which themselves inspire a different flavor of proof.
We start by defining the building blocks of these exponential martingales.
\begin{gather*}
  \twistSpace \defeq \big\{ \twistFunc \in \pathSpace{[0,\infty)}{\vecSpace} : \twistFunc \text{ has compact support and finite-variation} \big\} \\
  \begin{aligned}[t]
    &\mgRf: \pathSpace{[0,\infty)}{\stateSpace} \times \twistSpace \rightarrow \bbR,  \\
    &\hspace{1cm} \mgRf(\pathVar, \twistFunc) \defeq \bprj{\pathVar(0)}{\twistFunc(0)} -\int_0^\infty \pathVar(s-) \rmd\twistFunc(s) - \int_0^\infty \ode\big(\twistFunc(s), \pathVar(s)\big) \rmd s
  \end{aligned}
\end{gather*}

% construct the martingales
\begin{proposition}
  \label{proposition:martingale-measure}
  For each $\epsilon > 0$ and $\twistFunc \in \twistSpace$, we have the following identities, where $\calE(H)$ denotes the Dol\'eans-Dade exponential of $H$.
  \begin{align*}
    \mgDensity^{\epsilon,\twistFunc}
    &\defeq \exp\Big(\frac1\epsilon\mgRf(\Xe,\twistFunc) \Big) \\
    &= \exp\bigg( \bprj{\Xee_0}{\twistFunc(0)} - \int_0^\infty \Xee_{s-} \rmd\twistFunc(s) - \int_0^\infty \ode^\epsilon\big(\twistFunc(s), \Xee_s\big) \rmd s \bigg) \\
    &= \exp\Big( \twistFunc \shortInt \Xee_\infty - \ode^\epsilon\big(\twistFunc, \Xee\big) \shortInt \leb_\infty \Big) \\
    &= \calE\Big( \twistFunc \shortInt \X^{\epsilon,c} - \big( e^\prj{\twistFunc}{\id_\vecSpace} - 1 \big) \ast \compensate{\jumpMeas^{\Xee}} \Big)_\infty
  \end{align*}
  Furthermore, $\mgDensity^{\epsilon,\twistFunc}$ is integrable and $t \mapsto \Exp_{\Prb_\stateVar}(\mgDensity^{\epsilon,\twistFunc}| \scrF_t)$ is a martingale.
\end{proposition}
\begin{proof}
  We start by establishing the first two identities.
  Realizing $\twistFunc$ as a predictable, finite-variation process, we apply integration by parts (see \cite[Proposition I.4.49(b)]{jacod2003}) to get the following identity.
  \begin{align*}
    \frac1\epsilon \mgRf(\Xe, \twistFunc)
    &= \frac1\epsilon \Big( \bprj{\Xe_0}{\twistFunc(0)} - \int_0^\infty \Xe_{s-} \rmd\twistFunc(s) - \int_0^\infty \ode\big( \twistFunc(s), \Xe_s \big) \rmd s \Big) \\
    &= \bprj{\Xee_0}{\twistFunc(0)} - \int_0^\infty \Xee_{s-} \rmd\twistFunc(s) - \int_0^\infty \ode^\epsilon\big( \twistFunc(s), \Xee_s \big) \rmd s \\
    &= \bprj{\Xee_0}{\twistFunc(0)} - \Xee_- \shortInt \twistFunc_\infty - \ode^\epsilon(\twistFunc, \Xee_-) \shortInt \leb_\infty \\
    &= \twistFunc \shortInt \Xee_\infty - \ode^\epsilon(\twistVar, \Xee_-) \shortInt \leb_\infty
  \end{align*}
  The remaining identity is a special case of that from \cite[Theorem III.7.24]{jacod2003}, but we will perform the It\^o calculus here for completion's sake.
  Note that Theorem \ref{theorem:exponential-martingale} tells us that $\exp(\twistFunc \shortInt \Xee)$ is a $(\Prb_\stateVar, \scrF)$ jump-diffusion, and (\ref{eq:jump-diffusion-exponentially-special}) from Theorem \ref{theorem:LK-exponential-martingale} gives us its special semimartingale decomposition.
  \begin{equation*}
    \exp\big( \twistFunc \shortInt \Xee_t \big) 
    = \begin{aligned}[t]
      &\Big(\exp\big( \twistFunc \shortInt \Xee \big) \cdot \ode(\twistFunc, \Xee) \Big) \shortInt \leb_t + \Big( \exp\big( \twistFunc \shortInt \Xee_- \big) \cdot \twistFunc \Big) \shortInt {\Xee}^{,\rmc}_t \\
      &+ \exp\big(\twistFunc\shortInt\Xee_-\big) \Big( e^{\prj{\twistFunc}{\id_\vecSpace}} - 1 \Big) \ast \jumpMeas^{\Xee}_t
    \end{aligned} 
  \end{equation*}
  Note that this allows us to define the process $H$ inside the alleged Dol\'eans-Dade exponential.
  \begin{equation*}
    H = \twistFunc \shortInt {\Xee}^{,c} + \big( e^\prj{\twistFunc}{\id_\vecSpace} - 1 \big) \ast \compensate{\jumpMeas^{\Xee}}
  \end{equation*}
  Observe that, by definition of the Dol\'eans-Dade exponential and the special semimartingale decomposition of $\Xee$,
  \begin{gather*}
    \Xee_t = \frac1\epsilon \affDrift(\Xe) \shortInt \leb + {\Xee}^{,\rmc}_t + \id_\vecSpace \ast \compensate{\jumpMeas^{\Xee}}_t,  \\
    \bprj{{\Xee}^{,\rmc,i}}{{\Xee}^{,\rmc,j}} = \frac1\epsilon\affDiff(\Xe) \shortInt \leb, \\
    \predproj{\jumpMeas^{\Xee}}(\rmd s, \rmd\markVar) = \frac1\epsilon \affJump(\Xe, \rmd\markVar) \rmd s,
  \end{gather*}
  we have the following.
  \begin{align*}
    \log\calE(H)_t
    &= H_t - H_0 - \frac12 \prj{H^c}{H^c}_t + \big( \log(1+\id_\bbR) - \id_\bbR \big) \ast \jumpMeas^H_t \\
    &=\begin{aligned}[t]
      &\twistFunc \shortInt {\Xee_t}^{,c} + \big( e^\prj{\twistFunc}{\id_\vecSpace} - 1 \big) \ast \compensate{\jumpMeas^{\Xee}}_t \\
      &- \frac12 \prj{\twistFunc \shortInt {\Xee}^{,c}}{\twistFunc \shortInt {\Xee}^{,c}}_t - \big( e^\prj{\twistFunc}{\id_\vecSpace} - 1 -\prj{\twistFunc}{\id_\vecSpace}\big) \ast \jumpMeas^{\Xee}_t
    \end{aligned} \\
    &=\begin{aligned}[t]
      &\twistFunc \shortInt {\Xee_t}^{,c} - \big( e^\prj{\twistFunc}{\id_\vecSpace} - 1  - \prj{\twistFunc}{\id_\vecSpace}\big) \ast \predproj{\jumpMeas^{\Xee}}_t \\
      &- \frac12 \prj{\twistFunc}{\affDiff^\epsilon(\Xee)\twistFunc} \shortInt \leb_t + \prj{\twistFunc}{\id_\vecSpace} \ast \compensate{\jumpMeas^{\Xee}}_t
    \end{aligned} \\
    &=\begin{aligned}[t]
      &\twistFunc \shortInt \Xee_t - \affDrift^\epsilon(\Xee) \shortInt \leb_t - \prj{\twistFunc}{\id_\vecSpace} \ast \compensate{\jumpMeas^{\Xee}}_t - \big( e^\prj{\twistFunc}{\id_\vecSpace} - 1  - \prj{\twistFunc}{\id_\vecSpace}\big) \ast \predproj{\jumpMeas^{\Xee}}_t \\
      &- \frac12 \prj{\twistFunc}{\affDiff^\epsilon(\Xee)\twistFunc} \shortInt \leb_t + \prj{\twistFunc}{\id_\vecSpace} \ast \compensate{\jumpMeas^{\Xee}}_t
    \end{aligned} \\
    &= \twistFunc \shortInt \Xee_t - \ode^\epsilon(\twistFunc, \Xee) \shortInt \leb_t
  \end{align*}
  Note that this is one argument that $\mgDensity^{\epsilon,\twistFunc} = \calE(H)_\infty$ corresponds to a $(\Prb, \scrF)$ local martingale, since $H$ is (see \cite[Theorem I.4.61(b)]{jacod2003}).
  In any case, the martingale nature comes from Theorem \ref{theorem:exponential-martingale}.
\end{proof}


% show how the process distributes under the change of measure
The above proposition prescribes another change of measure.
For each $\epsilon > 0$, $\twistFunc \in \twistSpace$, and $\stateVar \in \stateSpace$, define
\begin{equation*}
  \mgPrb^{\epsilon,\twistFunc}_\stateVar(\rmd\omega) \defeq \mgDensity^{\epsilon,\twistFunc}(\omega) \cdot \Prb^\epsilon_\stateVar(\rmd\omega).
\end{equation*}
Let us explore the distribution of $\Xe$ over these spaces.

\begin{proposition}
  \label{proposition:mgPrb-dynamics}
  Fix $\epsilon > 0$, $\twistFunc \in \twistSpace$, and $\stateVar \in \stateSpace$.
  The process $\Xee$ is a $(\mgPrb^{\epsilon,\twistFunc}_\stateVar, \scrF^\epsilon)$ special semimartingale with the following decomposition.
  \begin{equation*}
    \Xee_t = \stateVar/\epsilon + \frac1\epsilon \affDrift^\twistFunc(\cdot, \Xe_-) \shortInt \leb_t + {\Xee_t}^{,c} + \id_\vecSpace \ast \compensate{\jumpMeas^{\twistFunc,\Xee}}_t,
  \end{equation*}
  where the drift $\affDrift^\twistFunc$, diffusion $\affDiff$, and jump predictable compensator $\predproj{\jumpMeas^{h,\Xee}}$ (above, we have $\compensate{\jumpMeas^{h,\Xee}} = \jumpMeas^{\Xee} - \predproj{\jumpMeas^{\twistFunc,\Xee}}$) are as follows.
  \begin{align*}
    \affDrift^\twistFunc(s, \stateVar) 
    &\defeq \affDrift(\stateVar) + \affDiff(\stateVar)\twistFunc(s) + \int_\vecSpace \markVar\big(e^\prj{\twistFunc(s)}{\markVar} - 1 \big) \affJump(\stateVar, \rmd\markVar) \\
    %
    \langle {\Xee}^{,c,i}, {\Xee}^{,c,j} \rangle &= \frac1\epsilon\affDiff_{i,j}(\Xe_-) \shortInt \leb \\
    \predproj{\jumpMeas^{\twistFunc,\Xee}}(\rmd s, \rmd\markVar)  
    &\defeq e^\prj{\twistFunc(s)}{\markVar} \frac1\epsilon \affJump(\Xe_{s-}, \rmd\markVar) \rmd s
  \end{align*}
  Moreover the distributions $\mgPrb^{\epsilon,\twistFunc}_\stateVar$ weakly converge to a degenerate measure $\delta_{\pathVar_\twistFunc}$ at the solution $\pathVar_\twistFunc$ to the following dynamical system.
  \begin{equation}
    \label{eq:limit-ode}
    \left\{\begin{aligned}
      \dot\pathVar_\twistFunc(t) &= \affDrift^\twistFunc\big(t, \pathVar_\twistFunc(t)\big) & t \geq 0 \\
      \pathVar_\twistFunc(0) &= \stateVar
    \end{aligned}\right.
  \end{equation}
\end{proposition}
\begin{proof}
  The $(\mgPrb^{\epsilon,\twistFunc}_\stateVar, \scrF^\epsilon)$ dynamics of $\Xee$ simply come from the Girsanov theorem (see \cite[Theorem III.3.24 or Theorem III.7.23]{jacod2003}).
  As far as weak convergence is concerned, this is immediate from the exponential tightness of our family and convergence of finite-dimensional distributions.
\end{proof}


% discuss high-level how exponential martingale method works
From here, a large deviation principle may be approached by concentrating probability on a ball and using Chebyshev-like bounds on $\mgDensity^{\epsilon,\twistFunc}$.
\begin{align*}
  \epsilon\log\Prb^\epsilon_\stateVar\big( \Xe \in \ball{\pathVar}{\delta} \big)
  &= \epsilon\log\Exp_{\mgPrb^{\epsilon,\twistFunc}_\stateVar}\big( (\mgDensity^{\epsilon,\twistFunc})^{-1} 1_{\ball{\pathVar}{\delta}}(\Xe) \big) \\
  &= \epsilon\log\Exp_{\mgPrb^{\epsilon,\twistFunc}_\stateVar}\Big( \exp\big(-\frac1\epsilon \mgRf(\Xe, \twistFunc) \big) 1_{\ball{\pathVar}{\delta}}(\Xe) \Big) \\
  &\leq -\inf_{\pathVar' \in \ball{\pathVar}{\delta}} \mgRf(\pathVar', \twistFunc)
\end{align*}
Showing lower semi-continuity of $\mgRf(\cdot,\twistFunc)$ would then result in us being able to say
\begin{equation*}
  \lim_{\delta \rightarrow 0} \epsilon\log\Prb^\epsilon_\stateVar\big( \Xe \in \ball{\pathVar}{\delta} \big) \leq -\mgRf(\pathVar, \twistFunc),
\end{equation*}
for all $\twistFunc \in \twistSpace$, and so we'd have
\begin{equation*}
  \lim_{\delta\rightarrow0} \epsilon\log\Prb^\epsilon_\stateVar\big( \Xe \in \ball{\pathVar}{\delta} \big) \leq - \sup_{\twistFunc \in \twistSpace} \mgRf(\pathVar, \twistFunc)
\end{equation*}
Meanwhile, the lower bound would be approached by showing $\twistSpace$ is suitably rich to have a dense family of limit functions $\pathVar_\twistFunc$ as in (\ref{eq:limit-ode}) of Proposition \ref{proposition:mgPrb-dynamics}, on which $\sup_{\twistFunc' \in \twistSpace} \mgRf(\pathVar_\twistFunc, \twistFunc') = \mgRf(\pathVar_\twistFunc, \twistFunc)$, and so a large deviations lower bound is attained from the following bound.
\begin{align*}
  \epsilon\log\Prb^\epsilon_\stateVar\big( \Xe \in \ball{\pathVar_\twistFunc}{\delta} \big)
  &= \epsilon\log\Exp_{\mgPrb^{\epsilon,\twistFunc}_\stateVar}\big( (\mgDensity^{\epsilon,\twistFunc})^{-1} 1_{\ball{\pathVar_\twistFunc}{\delta}}(\Xe) \big) \\
  &= \epsilon\log\Exp_{\mgPrb^{\epsilon,\twistFunc}_\stateVar}\Big( \exp\big(-\frac1\epsilon \mgRf(\Xe, \twistFunc) \big) 1_{\ball{\pathVar_\twistFunc}{\delta}}(\Xe) \Big) \\
  &\geq -\sup_{\pathVar' \in \ball{\pathVar_\twistFunc}{\delta}} \mgRf(\pathVar', \twistFunc) + \epsilon\log\mgPrb^{\epsilon,\twistFunc}_\stateVar\big( \Xe \in \ball{\pathVar_\twistFunc}{\delta} \big) \\
  &\geq -\mgRf(\pathVar_\twistFunc, \twistFunc) + \epsilon\log\mgPrb^{\epsilon,\twistFunc}_\stateVar\big( \Xe \in \ball{\pathVar_\twistFunc}{\delta} \big) \\
  &= -\sup_{\twistFunc' \in \twistSpace} \mgRf(\pathVar_\twistFunc, \twistFunc') + \epsilon\log\mgPrb^{\epsilon,\twistFunc}_\stateVar\big( \Xe \in \ball{\pathVar_\twistFunc}{\delta} \big)
\end{align*}

The benefit of this approach is that our rate function $\pathVar \mapsto \sup_{\twistFunc\in\twistSpace} \mgRf(\pathVar, \twistFunc)$ has an integral form.
Instead of proving the large deviation principle in this alternative fashion, we reconcile the measure changes that appear in each of the approaches.

% resolve the two approaches as one
It is readily evident that $\mgPrb^{\epsilon,\twistFunc}_\stateVar$ is a generalization of $\fddPrb^{\epsilon,\underline t,\underline\twistVar}_\stateVar$, as we are replacing summations in $\fddDensity^{\epsilon, \underline t, \underline\twistVar}$ with integrals in $\mgDensity^{\epsilon,\twistFunc}$.
\begin{equation}
  \label{eq:densities-side-by-side}
  \begin{aligned}
    \fddDensity^{\epsilon, \underline t, \underline\twistVar} 
    &= \exp \sum_{k=1}^{|\underline t|}\Big( \bprj{\twistVar_k}{\Xee_{t_k}-\Xee_{t_{k-1}}} - \affShift^\epsilon\big(\Delta t_k, \twistVar_k, \Xee_{t_{k-1}} \big) \Big) \\
    \mgDensity^{\epsilon, \twistFunc}
    &= \exp\Big( \int_0^\infty \twistFunc(s) \rmd \Xee_s - \int_0^\infty \ode^\epsilon\big(\twistFunc(s), \Xee_s\big) \rmd s \Big)
  \end{aligned}
\end{equation}
The summand $\prj{\twistVar_k}{\Xee_{t_k}-\Xee_{t_{k-1}}}$ relates to the integral term $\twistFunc(s) \rmd \Xee_s$, while $\affShift^\epsilon(\Delta t_k, \twistVar_k, \Xee_{t_{k-1}})$ relates to $\ode^\epsilon\big(\twistFunc(s), \Xee_s\big)\rmd s$.
To explicitly resolve these two expressions, we disambiguate the operations in $\fddDensity^{\epsilon,\underline t, \underline\twistVar}$ involving $\Xee$.
\begin{equation*}
  \begin{aligned}[t]
    &\fddRf(\cdot,\underline t, \underline\twistVar): \pathSpace{[0,\infty)}{\stateSpace} \rightarrow \bbR, \\
    &\quad \fddRf(\pathVar,\underline t, \underline\twistVar) \defeq \sum_{k=1}^{|\underline t|} \Big( \bprj{\twistVar_k}{\pathVar(t_k)-\pathVar(t_{k-1})} - \aff\big(\Delta t_k, \twistVar_k, \pathVar(t_{k-1})\big) \Big)
  \end{aligned}
\end{equation*}
We now have a common notation for factoring $\Xee$ through each density.
\begin{equation*}
  \fddDensity^{\epsilon, \underline t, \underline\twistVar} = \exp\Big( \frac1\epsilon \fddRf\big(\Xe, \underline t, \underline\twistVar\big) \Big), \quad
  \mgDensity^{\epsilon, \twistFunc} = \exp\Big( \frac1\epsilon \mgRf\big(\Xe, \twistFunc\big) \Big)
\end{equation*}
We now state the main theorem of this section, which resolves the \emph{twisting/tilting} approach of measure changes $\fddPrb^{\epsilon,\underline t,\underline\twistVar}$ in Proposition \ref{proposition:twists} with the \emph{exponential martingale} approach of measure changes $\mgPrb^{\epsilon,\twistFunc}$ in Proposition \ref{proposition:mgPrb-dynamics}.
It relies on the following parameterization of maps $\twistFunc(\cdot,\underline t, \underline\twistVar)$ over $\underline t \vdash [0,\infty)$ and $\underline\twistVar \in \momentSpace(\underline t)$.
\begin{equation}
  \label{eq:resolve-twists}
  \twistFunc(t, \underline t, \underline\twistVar) = \sum_{k=1}^{|\underline t|} 1_{[t_{k-1},t_k)}(t) \affPart(\Delta t_k, \twistVar_k) 
\end{equation}

\begin{theorem}
  \label{theorem:equivalence}
  For each $\underline t \vdash [0,\infty)$ and $\underline\twistVar \in \momentSpace(\underline t)$, we have $\twistFunc(\cdot,\underline t, \underline\twistVar) \in \twistSpace$, and for any semimartingale $H$,
  \begin{equation*}
    \fddRf(H, \underline t, \underline\twistVar) = \mgRf\big(H, \twistFunc(\cdot, \underline t, \underline\twistVar)\big).
  \end{equation*}
  Thus, for any $\epsilon > 0$ and $\stateVar \in \stateSpace$, we have the following identities.
  \begin{equation*}
    \fddDensity^{\epsilon, \underline t, \underline\twistVar} = \mgDensity^{\epsilon, \twistFunc(\cdot, \underline t, \underline\twistVar)}, \quad 
    \fddPrb^{\epsilon, \underline t, \underline\twistVar}_\stateVar = \mgPrb^{\epsilon, \twistFunc(\cdot, \underline t, \underline\twistVar)}_\stateVar
  \end{equation*}
\end{theorem}
\begin{proof}
  \label{proof:theorem:equivalence}
  It is clear that $\twistFunc(\cdot, \underline t, \underline\twistFunc)$ is compactly supported on the intervals of $\underline t$.
  Meanwhile, Proposition \ref{proposition:momentSpace-facts}\ref{proposition:momentSpace-facts:3} tells us that $\twistFunc(\cdot, \underline t, \underline\twistVar)$ is differentiable everywhere but potentially the nodes of $\underline t$, and so it is of finite variation.
  This concludes $\twistFunc \in \twistSpace$.

  Seeing as $\aff$ is $\stateSpace$-affine and stochastic integration is linear, we can use a simplified version of It\^o's formula below (again using Proposition \ref{proposition:momentSpace-facts}\ref{proposition:momentSpace-facts:3})
  \begin{align*}
    \fddRf(H, \underline t,\underline\twistVar)
    &= \sum_{k=1}^{|\underline t|} \Big( \bprj{\twistVar_k}{H_{t_k}-H_{t_{k-1}}} - \affShift\big(\Delta t_k, \twistVar_k, H_{t_{k-1}}\big) \Big) \\
    &= \sum_{k=1}^{|\underline t|} \Big( \aff\big(t_k-t_k, \twistVar_k, H_{t_k}\big) - \aff\big(t_k-t_{k-1}, \twistVar_k, H_{t_{k-1}}\big) \Big) \\
    &= \sum_{k=1}^{|\underline t|}  \int_{t_{k-1}}^{t_k} \rmd \aff(t_k-\cdot, \twistVar_k, H_\cdot) \\
    &= \sum_{k=1}^{|\underline t|} \bigg( -\int_{t_{k-1}}^{t_k} \dot\aff(t_k-t, \twistVar_k, H_t) \rmd t + \int_{t_{k-1}}^{t_k} \affPart(t_k-t,\twistVar_k) \rmd H_t \bigg) \\
    &= \sum_{k=1}^{|\underline t|} \bigg( \int_{t_{k-1}}^{t_k} \affPart(t_k-t, \twistVar_k) \rmd H_t - \int_{t_{k-1}}^{t_k} \ode\big(\affPart(t_k-t, \twistVar_k), H_t\big) \rmd t \bigg) \\
    &= \sum_{k=1}^{|\underline t|} \bigg( \int_{t_{k-1}}^{t_k} \twistFunc(t, \underline t, \underline\twistVar) \rmd H_t - \int_{t_{k-1}}^{t_k} \ode\big(\twistFunc(t, \underline t, \underline\twistVar), H_t\big) \rmd t \bigg) \\
    &= \twistFunc(\cdot, \underline t, \underline\twistVar) \shortInt H_\infty - \ode\big(\twistFunc(\cdot, \underline t, \underline\twistVar), H\big) \shortInt \leb_\infty
  \end{align*}
  From here, as we did in Proposition \ref{proposition:martingale-measure}, we apply integration by parts to complete the equality.
  \begin{align*}
    \fddRf(H, \underline t,\underline\twistVar)
    &=\twistFunc(\cdot, \underline t, \underline\twistVar) \shortInt H_\infty - \ode\big(\twistFunc(\cdot, \underline t, \underline\twistVar), H\big) \shortInt \leb_\infty \\
    &=\prj{\twistFunc(0,\underline t,\underline\twistVar)}{H_0} - \int_0^\infty H_{s-} \rmd\twistFunc(s,\underline t, \underline\twistVar) - \int_0^\infty \ode\big( \twistFunc(\cdot, \underline t, \underline\twistVar), H_s\big) \rmd s \\
    &= \mgRf\big(H, \twistFunc(\cdot,\underline t,\underline\twistVar)\big)
  \end{align*}
  Evaluating this identity at $\Xe$ now gives us the remaining equalities.
\end{proof}

\begin{remark}
  Note that a conditional cumulant $\affShift$ and L\'evy-Khintchine map $\ode$ may be defined for any jump-diffusion, despite not generally being $\stateSpace$-affine.
  \begin{align*}
    \affShift(t, \twistVar, \stateVar) &= \Exp_{\Prb_\stateVar}\exp\prj{\twistVar}{\X-\stateVar}, \\
    \ode(\momentVar, \stateVar) &= \bprj{\momentVar}{\affDrift(\stateVar)} + \frac12\bprj{\momentVar}{\affDrift(\stateVar)\momentVar} + \int_\vecSpace \Big( e^\prj{\momentVar}{\markVar} - 1 - \bprj{\momentVar}{\markVar} \Big) \affJump(\stateVar, \rmd\markVar)
  \end{align*}
  Furthermore, we may always construct (local) measure changes like $\fddDensity^{\underline t, \underline\twistVar}$ and $\mgDensity^{\twistFunc}$ in (\ref{eq:densities-side-by-side}) and postulate an equivalence like Theorem \ref{theorem:equivalence}.
  \begin{equation*}
    \exp \sum_{k=1}^{|\underline t|}\Big( \bprj{\twistVar_k}{\X_{t_k}-\X_{t_{k-1}}} - \affShift^\epsilon\big(\Delta t_k, \twistVar_k, \X_{t_{k-1}} \big) \Big) 
    = \exp\Big( \int_0^\infty \twistFunc(s) \rmd \X_s - \int_0^\infty \ode^\epsilon\big(\twistFunc(s), \X_s\big) \rmd s \Big)
  \end{equation*}
  This is the very result which would \emph{generally} prove that the approaches of twisting/tilting and martingales are in fact identical.
\end{remark}

\begin{corollary}
  \label{corollary:equivalence}
  For each $\underline t \vdash [0,\infty)$ and $\underline\twistVar \in \momentSpace(\underline t)$, the distributions $\fddPrb^{\epsilon,\underline t,\underline\twistVar}_\stateVar$ converge weakly to the degenerate measure $\delta_{\pathVar_{\underline t, \underline\twistVar}}$ at the solution $\pathVar_{\underline t, \underline\twistVar}$ to the dynamical system in which $\pathVar_{\underline t, \underline\twistVar}(0) = \stateVar$ and for each $k = 1, \ldots, |\underline t|$, we have the following equation.
  \begin{align*}
    &\dot\pathVar_{\underline t, \underline\twistVar}(t) = \begin{aligned}[t]
      &\affDrift\big(\pathVar_{\underline t, \underline\twistVar}(t) \big) + \affDiff\big(\pathVar_{\underline t, \underline\twistVar}(t)\big) \affPart(t_k - t, \twistVar_k) \\
      &\hspace{25mm}+ \int_\vecSpace \markVar \big( e^\prj{\affPart(t_k-t,\twistVar_k)}{\markVar} - 1 \big) \affJump\big(\pathVar_{\underline t, \underline\twistVar}(t), \rmd\markVar\big),
    \end{aligned} 
    &  \quad t \in [t_{k-1}, t_k)
  \end{align*}
\end{corollary}
\begin{proof}
  \label{proof:corollary:equivalence}
  This is simply substituting $\twistFunc(\cdot, \underline t, \underline\twistVar)$ for $\twistFunc$ in Proposition \ref{proposition:mgPrb-dynamics}.
\end{proof}

