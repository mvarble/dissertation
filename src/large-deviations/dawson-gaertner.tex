This section proves our large deviation principle from a perspective similar to that of Dawson-G\"artner (see \cite[Theorem 4.6.1]{dembo2010}), in which we prove the principle for the finite-dimensional projections, so that we get a principle on the projective space associated with $\pathSpace{[0,\infty)}{\stateSpace}$ that we may tighten to the Skorokhod space through exponential tightness.
Though we attribute this approach to the names of Dawson and G\"artner, we specifically use results in \cite{feng2006} which instead use results which appeal more to weak convergence arguments that are comparable to Prokhorov on an exponential scale.

\begin{theorem}
  \label{theorem:fdds}
  For each $\stateVar \in \stateSpace^\circ$ and $\underline t \vdash [0,\infty)$, the family $(\push{\Prb^\epsilon_\stateVar}{\pi_{\underline t}})_{\epsilon>0}$ satisfies a large deviation principle on $\vecSpace^{|\underline t|}$ with good convex rate function $\aff^*(\underline t, \cdot, \stateVar)$, the Fenchel-Legendre transform of $\aff(\underline t, \cdot, \stateVar)$.
  \begin{equation*}
    \aff^*(\underline t, \underline\stateVar, \stateVar) \defeq \sup_{\underline\momentVar \in \vecSpace^{|\underline t|}} \Big( \bprj{\underline\momentVar}{\underline\stateVar} - \aff(\underline t, \underline\momentVar, \stateVar) \Big)
  \end{equation*}
\end{theorem}
\begin{proof}
  We first prove a principle on the discrete family $(\push{\Prb^{\epsilon_m}_\stateVar}{\pi_{\underline t}})_{m\in\bbN}$ for $\epsilon_m \defeq 1/m$.
  Note that Proposition \ref{proposition:mean-field-asymptotics} allows us to consider a space $(\Omega, \Sigma, \Prb)$ equipped with an i.i.d.\ sequence $(\X^{(j)})_{j\in\bbN}$ of elements distributing from $\Prb_\stateVar$ and realize each $\epsilon_m \X^{\epsilon_m}$.
  \begin{equation*}
    \epsilon_m \X^{\epsilon_m} = \frac1m\sum_{j=1}^m \X^{(j)}
  \end{equation*}
  We now use a specific instance of Cram\'er's theorem \cite[Corollary 6.1.6]{dembo2010} to conclude that if $\underline 0$ is an interior point in the finite domain of $\aff(\underline t, \cdot, \stateVar)$, then our principle is satisfied with good rate function $\aff^*(\underline t, \cdot, \stateVar)$.
  Note that Proposition \ref{proposition:real-moments-revisit}\ref{proposition:real-moments-revisit:big-time-small-moments} tells us that $\underline0 \in \momentSpace(\underline t)$, an open set.
  Denoting some ball $\ball{\underline0}{\delta} \subseteq \momentSpace(\underline t)$, Theorem \ref{theorem:mgf-fdds} indicates that $\twistToMoment_{\underline t}\ball{\underline 0}{\delta}$ is an open set containing $\underline0$ in the finite domain of $\aff(\underline t, \cdot, \stateVar)$.

  Now that we have established a large deviation principle for $(\push{\Prb^{\epsilon_m}_\stateVar}{\pi_{\underline t}})_{m\in\bbN}$, we seek to establish one for $(\push{\Prb^\epsilon_\stateVar}{\pi_{\underline t}})_{\epsilon>0}$.
  We start by defining a map $\epsilon \mapsto \tilde\epsilon$ which discretizes the nature of $\epsilon > 0$; denoting $[r] \in \bbZ$ the integer part of $r \in \bbR$, define $\tilde\epsilon \defeq [\epsilon^{-1}]^{-1}$.
  The following quick inequalities relating $\epsilon$ and $\tilde\epsilon$,
  \begin{equation*}
    \tilde\epsilon - \tilde\epsilon^2 < \epsilon \leq \tilde\epsilon
  \end{equation*}
  make it easy to directly show $(\push{\Prb^{\tilde\epsilon}_\stateVar}{\pi_{\underline t}})_{\epsilon>0}$ satisfies a large deviation principle; for each $\Gamma \in \scrB(\vecSpace^{|\underline t|})$,
  \begin{align*}
    -\inf_{\underline\stateVar \in \Gamma^\circ} \aff^*(\underline t, \underline\stateVar, \stateVar) 
    &\leq \liminf_{m\rightarrow\infty} \epsilon_m \log \push{\Prb^{\epsilon_m}_\stateVar}{\pi_{\underline t}} \Gamma \\
    &= \liminf_{\epsilon\rightarrow0} \tilde\epsilon \log \push{\Prb^{\tilde\epsilon}_\stateVar}{\pi_{\underline t}} \Gamma \\
    &\leq \liminf_{\epsilon\rightarrow0} (\epsilon + \tilde\epsilon^2) \log \push{\Prb^{\tilde\epsilon}_\stateVar}{\pi_{\underline t}} \Gamma \\ 
    &= \liminf_{\epsilon\rightarrow0} \epsilon \log \push{\Prb^{\tilde\epsilon}_\stateVar}{\pi_{\underline t}} \Gamma \\
    &\leq \limsup_{\epsilon\rightarrow0} \epsilon \log \push{\Prb^{\tilde\epsilon}_\stateVar}{\pi_{\underline t}} \Gamma \\
    &\leq \limsup_{\epsilon\rightarrow0} \tilde\epsilon \log \push{\Prb^{\tilde\epsilon}_\stateVar}{\pi_{\underline t}} \Gamma \\
    &= \limsup_{m\rightarrow\infty} \epsilon_m \log \push{\Prb^{\epsilon_m}_\stateVar}{\pi_{\underline t}} \Gamma 
    \leq -\inf_{\underline\stateVar \in \overline\Gamma} \aff^*(\underline t, \underline\stateVar, \stateVar)
  \end{align*}

  To obtain a large deviation principle for our family $(\push{\Prb^\epsilon_\stateVar}{\pi_{\underline t}})_{\epsilon>0}$, we show regularity $\epsilon \rightarrow \push{\Prb^\epsilon_\stateVar}{\pi_{\underline t}}$ to lift the principle for the discretized family $(\push{\Prb^{\tilde\epsilon}_\stateVar}{\pi_{\underline t}})_{\epsilon>0}$.
  This notion in the literature is known as an \emph{exponential approximation}, which is explored in \cite[Section 4.2.2]{dembo2010}.
  From \cite[Theorem 4.2.13]{dembo2010}, it suffices to construct a probability space $(\Omega, \Sigma, \Prb)$ such that elements $(\Xe)_{\epsilon>0}$ on this space with distributions $(\Prb^\epsilon_\stateVar)_{\epsilon>0}$ satisfy the following exponential equivalence property.
  \begin{equation*}
    \limsup_{\epsilon\rightarrow0} \epsilon\log\Prb\Big( |\epsilon\X^\epsilon_{\underline t} - \tilde\epsilon \X^{\tilde\epsilon}_{\underline t}| > \delta \Big) = -\infty
  \end{equation*}
  The \emph{scaled-dynamics} realization from Proposition \ref{proposition:sde-asymptotics} will do just this.
\end{proof}


Cram\'er's theorem---the tool we leveraged to prove the above principle---is proven by using measure changes induced by densities of the following form.
\begin{equation*}
  \exp\Big( \prj{\underline\momentVar}{\Xee_{\underline t}} - \aff^\epsilon(\underline t, \underline\momentVar, \Xee_0) \Big)
\end{equation*}
Observe that Theorem \ref{theorem:mgf-fdds} gives us a perspective of how this measure depends on the increments; a valid moment $\underline\momentVar \in \vecSpace^{|\underline t|}$ for the above expression must satisfy $\underline\momentVar = \twistToMoment_{\underline t}(\underline\twistVar)$ for some $\underline\twistVar \in \momentSpace(\underline t)$, and the following.
\begin{equation*}
  \exp\Big( \prj{\underline\momentVar}{\Xee_{\underline t}} - \aff^\epsilon(\underline t, \underline\momentVar, \Xee_0) \Big) = 
  \exp\sum_{k=1}^{|\underline t|} \Big( \prj{\twistVar_k}{\Xee_{t_k}-\Xee_{t_{k-1}}} - \affShift^\epsilon\big(\Delta t_k, \twistVar_k, \Xee_{t_{k-1}} \big) \Big) 
\end{equation*}
We denote the above quantity $\fddDensity^{\epsilon,\underline t, \underline\twistVar}$, its associated measure change by
\begin{equation*}
  \fddPrb^{\epsilon,\underline t,\underline\twistVar}_\stateVar(\rmd\omega) \defeq \fddDensity^{\epsilon,\underline t,\underline\twistVar}(\omega) \cdot \Prb(\rmd\omega),
\end{equation*}
and observe the nature of how they make our increments distribute.

\begin{proposition}
  \label{proposition:twists}
  Fix $\epsilon > 0$, $\stateVar \in \stateSpace$, $\underline t \vdash [0,\infty)$, and $\underline\twistVar \in \momentSpace(\underline t)$. 
  For each $\ell = 1, \ldots, |\underline t|$ and $\twistVar \in \vecSpace$ with $\twistVar_\ell + \twistVar \in \momentSpace(\Delta t_\ell)$, we have the following moments.
  \begin{equation*}
    \begin{aligned}[t]
      &\Exp_{\fddPrb^{\epsilon, \underline t, \underline\twistVar}_\stateVar}\Big( \exp\prj{\twistVar}{\Xee_{t_\ell}-\Xee_{t_{\ell-1}}} | \scrF^\epsilon_{t_{\ell-1}} \Big)  \\
      &\hspace{1cm} = \exp\Big(\affShift^\epsilon\big(\Delta t_\ell, \twistVar_\ell + \twistVar, \Xee_{t_{\ell-1}} \big)  - \affShift^\epsilon\big(\Delta t_\ell, \twistVar_\ell, \Xee_{t_{\ell-1}} \big)\Big)
    \end{aligned}
  \end{equation*}
  This furthermore means we have the following conditional expectations.
  \begin{equation*}
    \Exp_{\fddPrb^{\epsilon, \underline t,\underline\twistVar}_\stateVar} \Big( \Xe_{t_\ell} - \Xe_{t_{\ell-1}} | \Xe_{\underline t_{1:\ell-1}} = \underline\stateVar_{1:\ell-1} \Big) = \nabla_{\twistVar_\ell}\affShift(\Delta t_\ell, \twistVar_\ell, \stateVar_{\ell-1})
  \end{equation*}
\end{proposition}
\begin{proof}
  \label{proof:proposition:twists}
  Denote $n \defeq |\underline t|$ for brevity.
  We first show the following conditional expectation for any $m = n-1,\ldots, 1$.
  \begin{equation}
    \label{eq:fddDensity-conditional-expectation}
    \Exp_{\Prb^\epsilon_\stateVar}\big(\fddDensity^{\epsilon,\underline t, \underline\twistVar} | \scrF^\epsilon_{t_m} \big) = \prod_{k=1}^m \big( \prj{\twistVar_k}{\Xee_{t_k}-\Xee_{t_{k-1}}} - \affShift^\epsilon(\Delta t_k, \twistVar_k, \Xee_{t_{k-1}})\big),
  \end{equation}
  by iteratively projecting onto $\scrF^\epsilon_{t_n}, \ldots, \scrF^\epsilon_{t_{m+1}}$.
  For any quantity $H \in \scrF_{t_m}/\scrB(\bbR_+)$, we have the following.
  \begin{align*}
    &\Exp_{\Prb^\epsilon_\stateVar}\big(H \fddDensity^{\epsilon,\underline t, \underline\twistVar} \big) \\
    &= \Exp_{\Prb^\epsilon_\stateVar}\Big( H \prod_{k=1}^n \exp\big(  \prj{\twistVar_k}{\Xee_{t_k}-\Xee_{t_{k-1}}} - \affShift^\epsilon(\Delta t_k, \twistVar_k, \Xee_{t_{k-1}})\big) \Big)  \\
    &= \Exp_{\Prb^\epsilon_\stateVar}\Big( \begin{aligned}[t]
      &H \prod_{k=1}^{n-1} \exp\big(  \prj{\twistVar_k}{\Xee_{t_k}-\Xee_{t_{k-1}}} - \affShift^\epsilon(\Delta t_k, \twistVar_k, \Xee_{t_{k-1}})\big)  \\
      &\quad\Exp_{\Prb^\epsilon_\stateVar}\big( \exp\prj{\twistVar_n}{\Xee_{t_n}-\Xee_{t_{n-1}}} | \scrF^\epsilon_{t_{n-1}}\big)\exp\big(-\affShift^\epsilon(\Delta t_n, \twistVar_n, \Xee_{t_{n-1}})\big) \Big)  
    \end{aligned}\\
    &= \Exp_{\Prb^\epsilon_\stateVar}\Big( H \prod_{k=1}^{n-1} \exp\big(  \prj{\twistVar_k}{\Xee_{t_k}-\Xee_{t_{k-1}}} - \affShift^\epsilon(\Delta t_k, \twistVar_k, \Xee_{t_{k-1}})\big) \Big) \\
    &\quad\vdots \\
    &= \Exp_{\Prb^\epsilon_\stateVar}\Big( H \prod_{k=1}^m \exp\big(  \prj{\twistVar_k}{\Xee_{t_k}-\Xee_{t_{k-1}}} - \affShift^\epsilon(\Delta t_k, \twistVar_k, \Xee_{t_{k-1}})\big) \Big)
  \end{align*}
  which indicates that (\ref{eq:fddDensity-conditional-expectation}) is true.
  Now choosing $H \in \scrF^\epsilon_{t_{\ell-1}}/\scrB(\bbR_+)$, we apply (\ref{eq:fddDensity-conditional-expectation}) for $m=\ell$ and $m=\ell-1$ to see that the following holds.
  \begin{align*}
    &\Exp_{\fddPrb^{\epsilon, \underline t, \underline\twistVar}_\stateVar}\Big( \exp\prj{\twistVar}{\Xee_{t_\ell}-\Xee_{t_{\ell-1}}} H \Big) \\
    &=\Exp_{\Prb^\epsilon_\stateVar}\Big( \exp\prj{\twistVar}{\Xee_{t_\ell}-\Xee_{t_{\ell-1}}} H \fddDensity^{\epsilon,\underline t,\underline\twistVar} \Big)  \\
    &=\Exp_{\Prb^\epsilon_\stateVar}\Big( \exp\prj{\twistVar}{\Xee_{t_\ell}-\Xee_{t_{\ell-1}}} H \prod_{k=1}^\ell \exp\big(\prj{\twistVar_k}{\Xee_{t_k}-\Xee_{t_{k-1}}} - \affShift^\epsilon(\Delta t_k, \twistVar_k, \Xee_{t_{k-1}})\big) \Big)  \\
    &=\Exp_{\Prb^\epsilon_\stateVar}\Big( \begin{aligned}[t]
      &H  \prod_{k=1}^{\ell-1} \exp\big(\prj{\twistVar_k}{\Xee_{t_k}-\Xee_{t_{k-1}}} - \affShift^\epsilon(\Delta t_k, \twistVar_k, \Xee_{t_{k-1}})\big) \\
      &\hspace{16mm} \Exp_{\Prb^\epsilon_\stateVar}\big(\exp\prj{\twistVar_\ell + \twistVar}{\Xee_{t_\ell}-\Xee_{t_{\ell-1}}}|\scrF_{t_{\ell-1}}\big) \exp\big(-\affShift^\epsilon(\Delta t_\ell, \twistVar_\ell, \Xee_{t_{\ell-1}})\big) \Big)
    \end{aligned} \\
    &=\Exp_{\Prb^\epsilon_\stateVar}\Big( \begin{aligned}[t]
      &H  \prod_{k=1}^{\ell-1} \exp\big(\prj{\twistVar_k}{\Xee_{t_k}-\Xee_{t_{k-1}}} - \affShift^\epsilon(\Delta t_k, \twistVar_k, \Xee_{t_{k-1}})\big) \\
      &\hspace{16mm} \exp\big(\affShift^\epsilon(\Delta t_\ell, \twistVar_\ell + \twistVar, \Xee_{t_{\ell-1}})-\affShift^\epsilon(\Delta t_\ell, \twistVar_\ell, \Xee_{t_{\ell-1}})\big) \Big)
    \end{aligned} \\
    &=\Exp_{\Prb^\epsilon_\stateVar}\Big( \exp\big(\affShift^\epsilon(\Delta t_\ell, \twistVar_\ell + \twistVar, \Xee_{t_{\ell-1}})-\affShift^\epsilon(\Delta t_\ell, \twistVar_\ell, \Xee_{t_{\ell-1}})\big) H \fddDensity^{\epsilon,\underline t,\underline\twistVar} \Big) \\
    &=\Exp_{\fddPrb^{\epsilon, \underline t, \underline\twistVar}_\stateVar}\Big( \exp\big(\affShift^\epsilon(\Delta t_\ell, \twistVar_\ell + \twistVar, \Xee_{t_{\ell-1}})-\affShift^\epsilon(\Delta t_\ell, \twistVar_\ell, \Xee_{t_{\ell-1}})\big) H \Big) 
  \end{align*}
  This gives us our first desired identity.
  \begin{equation*}
    \begin{aligned}[t]
      &\Exp_{\fddPrb^{\epsilon, \underline t, \underline\twistVar}_\stateVar}\Big( \exp\prj{\twistVar}{\Xee_{t_\ell}-\Xee_{t_{\ell-1}}} | \scrF^\epsilon_{t_{\ell-1}} \Big) \\
      &\hspace{1cm} = \exp\big(\affShift^\epsilon(\Delta t_\ell, \twistVar_\ell + \twistVar, \Xee_{t_{\ell-1}})-\affShift^\epsilon(\Delta t_\ell, \twistVar_\ell, \Xee_{t_{\ell-1}})\big)
    \end{aligned}
  \end{equation*}
  Now that this identity is established, we appeal to Propositions \ref{proposition:real-moments-revisit}\ref{proposition:real-moments-revisit:big-time-small-moments}, \ref{proposition:aff-differentiable}, and \ref{proposition:aff-asymptotics} in specifying an open ball $\ball{\twistVar_k}{\delta} \subseteq \momentSpace(\Delta t_k)$ on which we may apply derivatives to get the following identity.
  \begin{align*}
     \Exp_{\fddPrb^{\epsilon, \underline t,\underline\twistVar}_\stateVar} \Big( \Xe_{t_\ell} - \Xe_{t_{\ell-1}} | \scrF^\epsilon_{t_{\ell-1}} \Big) 
    &= \epsilon \Exp_{\fddPrb^{\epsilon, \underline t,\underline\twistVar}_\stateVar} \Big( \Xee_{t_\ell} - \Xee_{t_{\ell-1}} | \scrF^\epsilon_{t_{\ell-1}} \Big)  \\
    &= \epsilon \Exp_{\fddPrb^{\epsilon, \underline t,\underline\twistVar}_\stateVar} \Big( \nabla_{\twistVar} \exp\prj{\twistVar}{\Xee_{t_\ell} - \Xee_{t_{\ell-1}}}|_{\twistVar=0} | \scrF^\epsilon_{t_{\ell-1}} \Big) \\
    &=\epsilon \nabla_{\twistVar} \Exp_{\fddPrb^{\epsilon, \underline t,\underline\twistVar}_\stateVar} \Big( \exp\prj{\twistVar}{\Xee_{t_\ell} - \Xee_{t_{\ell-1}}} | \scrF^\epsilon_{t_{\ell-1}} \Big) \Big|_{\twistVar=0}\\
    &= \epsilon \nabla_{\twistVar} \exp\Big(\affShift^\epsilon\big(\Delta t_\ell, \twistVar_\ell + \twistVar, \Xee_{t_{\ell-1}}\big)-\affShift^\epsilon\big(\Delta t_\ell, \twistVar_\ell, \Xee_{t_{\ell-1}}\big)\Big) \Big|_{\twistVar=0} \\
    &= \nabla_{\twistVar} \exp\Big(\affShift\big(\Delta t_\ell, \twistVar_\ell + \twistVar, \Xe_{t_{\ell-1}}\big)-\affShift\big(\Delta t_\ell, \twistVar_\ell, \Xe_{t_{\ell-1}}\big)\Big) \Big|_{\twistVar=0} \\
    &= \nabla_{\twistVar_\ell}\affShift\big(\Delta t_\ell, \twistVar_\ell, \Xe_{t_{\ell-1}}\big)
  \end{align*}
  Seeing as the above quantity is measurable with respect to the $\sigma$-algebra generated by $\X_{\underline t_{1:\ell}}$, we have our desired identity.
  \begin{equation*}
    \Exp_{\fddPrb^{\epsilon, \underline t,\underline\twistVar}_\stateVar} \Big( \Xe_{t_\ell} - \Xe_{t_{\ell-1}} | \Xe_{\underline t_{1:\ell-1}} = \underline\stateVar_{1:\ell-1} \Big)  = \nabla_{\twistVar_\ell} \affShift\big(\Delta t_\ell, \twistVar_\ell, \Xe_{t_{\ell-1}}\big)
  \end{equation*}
\end{proof}


Understanding these measures $\fddPrb^{\epsilon,\underline t,\underline\twistVar}_\stateVar$ will be important for our next section, where we relate the seemingly alternative exponential martingale method to studying large deviations.
We now proceed from that short tangent back to our large deviation principle.

\begin{proposition}
  \label{proposition:exponential-tightness}
  For each $\stateVar \in \stateSpace$, the family $(\Prb^\epsilon_\stateVar)_{\epsilon>0}$ is exponentially tight on $\pathSpace{[0,\infty)}{\stateSpace}$.
\end{proposition}
\begin{proof}
  \label{proof:proposition:exponential-tightness}
  First note that each family $(\push{\Prb^\epsilon_\stateVar}{\pi_{\underline t}})_{\epsilon>0}$ of finite-dimensional distributions is exponentially tight by Theorem \ref{theorem:fdds}.
  This is because each $\rho > 0$ produces a compact set $K_\rho \defeq \aff^*(\underline t, \cdot, \stateVar)^{-1}[0,\rho] \subseteq \vecSpace^{|\underline t|}$ (by goodness of the rate function) clearly satisfying the following.
  \begin{equation*}
    \limsup_{\epsilon\rightarrow0} \epsilon\log\push{\Prb^\epsilon_\stateVar}{\pi_{\underline t}} (K_\rho^c) \leq -\inf_{\underline\stateVar\in \overline{K_\rho^c}} \aff^*(\underline t, \underline\stateVar, \stateVar) \leq -\rho
  \end{equation*}
  By \cite[Theorem 4.1]{feng2006} (rather, an adaptation of it involving a continuously parameterized family), our desired exponential tightness is obtained if we produce to each $\epsilon, \delta, \lambda, T > 0$ a random variable $\gamma_\epsilon(\delta,\lambda,T)$ with the following dominating property over all $t \in [0,T]$, $s \in [0,\delta]$,
  \begin{gather}
    \label{eq:gamma-dominating}
    \Exp_{\Prb_\stateVar}\Big(
      \exp\big( \epsilon^{-1} \lambda \big( |\Xe_{t+s}-\Xe_t| \wedge 1 \big) 
      | \scrF^\epsilon_t \Big) 
    \leq \Exp_{\Prb_\stateVar}\Big( 
      \exp\gamma_\epsilon(\delta, \lambda, T) 
      | \scrF^\epsilon_t \Big),
  \end{gather}
  such that the following equalities are true for all $\lambda > 0$.
  \begin{gather}
    \label{eq:no-gamma-bound}
    \lim_{\delta\rightarrow0} \limsup_{\epsilon\rightarrow0} 
      \epsilon\log\Exp_{\Prb_\stateVar}\exp\big( 
        \epsilon^{-1}\lambda \big(|\Xe_\delta - \Xe_0| \wedge 1\big) 
      \big) = 0 \\
    \label{eq:gamma-bound}
    \lim_{\delta\rightarrow0} \limsup_{\epsilon\rightarrow0} \epsilon\log\Exp_{\Prb_\stateVar}\exp\gamma_\epsilon(\delta,\lambda, T) = 0 
  \end{gather}
  To show this fact, first note that, by Proposition \ref{proposition:real-moments-revisit}\ref{proposition:real-moments-revisit:aff-uniform-bound}, to each $\lambda > 0$ there exist $\delta_\lambda > 0$ and $C_\lambda > 0$ such that the following inequality holds.
  \begin{equation*}
    \big| \aff(t, \momentVar, \stateVar) - \aff(0, \momentVar, \stateVar) \big| \leq C_\lambda \cdot t \cdot \big( 1 + |\stateVar| \big), \quad t \in [0,\delta_\lambda], ~ \momentVar \in  \closedBall{0}{\lambda\sqrt d}, ~ \stateVar \in  \stateSpace
  \end{equation*}
  We then define the following function $f_\epsilon(\cdot,\delta,\lambda): \stateSpace \rightarrow \bbR$ for each $\epsilon > 0$, $\lambda > 0$, and $\delta \in [0,\delta_\lambda]$.
  \begin{equation*}
    f_\epsilon(\stateVar',\delta,\lambda) \defeq \log 2d + \frac{1}{\epsilon} \cdot C_\lambda \cdot \delta \cdot \big( 1 + |\stateVar| \big)
  \end{equation*}
  We now define $\gamma_\epsilon(\delta, \lambda, T) \defeq f_\epsilon\big( \Xe_t, \delta, \lambda \big)$.

  Note that for all $\lambda, \epsilon, t > 0$, $\delta \in [0,\delta_\lambda]$, and $s \in [0,\delta]$, we use Proposition \ref{proposition:aff-asymptotics} to get the following.
  \begin{align}
    \notag
    &\Exp_{\Prb_\stateVar}\Big(
      \exp\big( \epsilon^{-1} \lambda \big( |\Xe_{t+s}-\Xe_t| \wedge 1 \big) 
      | \scrF^\epsilon_t \Big)  \tag*{} \\
    \notag
    &\leq \Exp_{\Prb_\stateVar}\Big(
      \exp\big( \epsilon^{-1} \lambda \big| \Xe_{t+s}-\Xe_t \big|  \big)
      | \scrF^\epsilon_t \Big) \\
    \notag
    &\leq \sum_{\ell=0}^1 \sum_{i=1}^d \Exp_{\Prb_\stateVar}\bigg(
      \exp\Big( \Bprj{(-1)^\ell\epsilon^{-1}\lambda\sqrt d  \basisVec^i}{\Xe_{t+s}-\Xe_t} \Big)
      | \scrF^\epsilon_t \bigg) \\
    \notag
    &= \sum_{\ell=0}^1 \sum_{i=1}^d \exp\bigg( 
      \epsilon^{-1}\aff\Big(s, (-1)^\ell\lambda\sqrt d \basisVec^i, \Xe_t \Big) 
      - \epsilon^{-1}\Bprj{(-1)^\ell\lambda\sqrt d \basisVec^i}{\Xe_t} \bigg) \\
    \notag
    &\leq 2d \cdot \exp\bigg( \frac1\epsilon \cdot C_\lambda \cdot \delta \cdot \big( 1 + |\Xe_t| \big) \bigg) \\
    &= \exp f_\epsilon(\Xe_t, \delta, \lambda)
    \label{eq:verify-gamma-dominating}
  \end{align}
  Note that (\ref{eq:verify-gamma-dominating}) makes (\ref{eq:gamma-dominating}) true.
  \begin{align*}
    &\Exp_{\Prb_\stateVar}\Big( \exp\big( \epsilon^{-1} \lambda \big( |\Xe_{t+s}-\Xe_t| \wedge 1 \big) | \scrF^\epsilon_t \Big) \\
    &\leq \exp f_\epsilon(\Xe_t, \delta, \lambda) \\
    &= \exp \gamma_\epsilon(\delta, \lambda,T) \\
    &= \Exp_{\Prb_\stateVar}\Big( \exp\gamma_\epsilon(\delta,\lambda,T) | \scrF^\epsilon_t \Big)
  \end{align*}
  For (\ref{eq:no-gamma-bound}), we also use (\ref{eq:verify-gamma-dominating}).
  \begin{align*}
    &\hspace{-1em}\lim_{\delta\rightarrow0} \limsup_{\epsilon\rightarrow0} 
      \epsilon\log\Exp_{\Prb_\stateVar}\exp\big( 
        \epsilon^{-1}\lambda \big(|\Xe_\delta - \Xe_0| \wedge 1\big) 
      \big)  \tag*{} \\
    &=\lim_{\delta\rightarrow0} \limsup_{\epsilon\rightarrow0} 
      \epsilon\log\Exp_{\Prb_\stateVar}
      \Exp_{\Prb_\stateVar}\Big(
      \exp\big(\epsilon^{-1}\lambda \big(|\Xe_\delta - \Xe_0| \wedge 1\big) \big)
      |\scrF^\epsilon_0 \Big) \\
    &\leq \lim_{\delta\rightarrow0}\limsup_{\epsilon\rightarrow0} \epsilon\log\Exp_{\Prb_\stateVar} \exp f_\epsilon(\Xe_0, \delta, \lambda) \\
    &= \lim_{\delta\rightarrow0}\limsup_{\epsilon\rightarrow0} \Big( \epsilon\log 2d + C_\lambda \cdot \delta \cdot \big( 1 + |\stateVar| \big) \Big) \\
    &= 0
  \end{align*}

  Using Proposition \ref{proposition:real-moments-revisit}\ref{proposition:real-moments-revisit:big-time-small-moments}, to each $\lambda > 0$, there exists $\delta_\lambda' > 0$ such that $B(0,\delta_\lambda) \subseteq \momentSpace(\delta_\lambda')$.
  Now, for any $\delta < \delta_\lambda \wedge \delta_\lambda'/(2C_\lambda\sqrt d)$, we again use Proposition \ref{proposition:aff-asymptotics} to get the following.
  \begin{align*}
    &\hspace{-1em}\epsilon\log\Exp_{\Prb_\stateVar}\Big( \exp \gamma_\epsilon(\delta, \lambda,T) \Big) \tag*{} \\
    &=\epsilon\log\Exp_{\Prb_\stateVar}\Big( \exp f_\epsilon(\Xe_t, \delta, \lambda) \Big) \tag*{} \\
    &= \epsilon\log 2d + C_\lambda \cdot\delta + \epsilon\log \Exp_{\Prb_\stateVar}\exp\big(\epsilon^{-1}C_\lambda\delta |\Xe_t| \big) \\
    &\leq \epsilon\log 2d + C_\lambda \cdot\delta + \epsilon\log \sum_{\ell=0}^1\sum_{i=1}^d \Exp_{\Prb_\stateVar}\exp\bprj{(-1)^\ell\epsilon^{-1}C_\lambda \delta\sqrt d \basisVec^i}{\Xe_t} \\
    &\leq 2\epsilon\log2d + C_\lambda \cdot \delta + \max_{\substack{\ell=0,1\\i=1,\ldots,d\\k=1,\ldots,|\underline t|}} \epsilon\log\Exp_{\Prb_\stateVar} \exp\Bprj{ (-1)^\ell \epsilon^{-1} C_\lambda\delta \sqrt d \basisVec^i}{\Xe_t} \\
    &\leq 2\epsilon\log2d + C_\lambda \cdot \delta + \max_{\substack{\ell=0,1\\i=1,\ldots,d\\k=1,\ldots,|\underline t|}} \aff\Big( t, (-1)^\ell C_\lambda \delta\sqrt d \basisVec^i, \stateVar\Big)
  \end{align*}
  This gives us (\ref{eq:gamma-bound}).
\end{proof}

\begin{theorem}
  \label{theorem:ldp}
  For each $\stateVar \in \stateSpace^\circ$, the family $(\Prb^\epsilon_\stateVar)_{\epsilon>0}$ satisfies a large deviation principle on $\pathSpace{[0,\infty)}{\stateSpace}$ with good rate function $\rf_\stateVar: \pathSpace{[0,\infty)}{\stateSpace} \rightarrow [0,\infty]$ as follows.
  \begin{equation}
    \label{eq:rf}
    \rf_\stateVar(\pathVar) = \left\{\begin{array}{ll}
      \displaystyle\sup_{\underline t\vdash \Delta_\pathVar^c} \aff^*\big(\underline t, \pathVar(\underline t), \pathVar(0)\big) & \pathVar(0) = \stateVar \\
      \infty & \text{otherwise}
    \end{array}\right.
  \end{equation}
  Above, $\Delta_\pathVar \subseteq [0,\infty)$ denotes the points of discontinuity of $\pathVar$.
\end{theorem}

\begin{proof}
  \label{proof:theorem:ldp}
  Provided we have some $\underline t \vdash [0,\infty)$ the vector $\hat{\underline t}$ associated with prepending $0$ to $\underline t$, 
  \begin{equation*}
    \hat{\underline t} = (0, t_1, \ldots, t_{|\underline t|}),
  \end{equation*}
  induces the following finite-dimensional distributions.
  \begin{equation*}
    \push{\Prb^\epsilon_\stateVar}{\pi_{\hat{\underline t}}} = \delta_\stateVar \otimes \push{\Prb^\epsilon_\stateVar}{\pi_{\underline t}}
  \end{equation*}
  For these partitions, it is easy to see from Theorem \ref{theorem:fdds} that the large deviations principle for $(\push{\Prb^\epsilon_\stateVar}{\pi_{\hat{\underline t}}})_{\epsilon>0}$ has good rate function as below.
  \begin{equation*}
    (\stateVar_0, \stateVar_1, \ldots, \stateVar_{|\underline t|}) \mapsto \left\{\begin{array}{ll}
      \aff^*(\underline t, \underline\stateVar, \stateVar_0), & \stateVar_0 = \stateVar \\
      \infty & \text{otherwise}
    \end{array}\right.
  \end{equation*}
  By \cite[Theorem 4.28]{feng2006}, we now use Theorem \ref{theorem:fdds} and Proposition \ref{proposition:exponential-tightness} to get a large deviations principle for the family $(\Prb^\epsilon_\stateVar)_{\epsilon>0}$ with good rate function $\rf_\stateVar$.
\end{proof}

