% make Mogulskii understandable on our path space
A surprisingly powerful theorem in the theory of large deviations of stochastic processes is that of Mogulskii (see \cite[Theorems 5.1.2 and 5.1.19 and Exercise 5.122]{dembo2010}).
Fixing a family $(\V_j)_{j\in\bbN}$ of independent quantities distributing with common distribution $\affDist$ with light tails,
\begin{equation}
  \label{eq:cumulant-generating-function}
  \ode_\affDist(\momentVar) \defeq \log\int_\vecSpace e^\prj\momentVar\markVar \affDist(\rmd\markVar) < \infty, \quad \momentVar \in \vecSpace
\end{equation}
this theorem provides a large deviation principle for the laws associated to quantities $Y^\epsilon$ as below.
\begin{equation*}
  Y^\epsilon_t = \epsilon \sum_{j=1}^{[t/\epsilon]} \V_j, \quad t \in [0,\tau]
\end{equation*}
It states that the associated laws $(\Prb^\epsilon)_{\epsilon>0}$ satisfy a large deviation principle on the space $\bbL^\infty[0,\tau]$ of bounded functions $[0,\tau] \rightarrow \vecSpace$, equipped with the supremum norm.
The rate function, like ours, is an integral of the Fenchel-Legendre transform of $\ode_\affDist$.
\begin{equation*}
  \pathVar \mapsto \left\{\begin{array}{ll}
    \displaystyle\int_0^\tau \ode_\affDist^*\big(\dot\pathVar(t)\big) \rmd t & \pathVar(0) = 0, ~ \pathVar \in \acpathSpace{[0,\tau]}{\vecSpace}\\
    \infty & \text{otherwise}
  \end{array}\right.
\end{equation*}
Very minor adjustments can actually make this theorem similar to the context of our principle.
Firstly, the principle may be lifted to the space $\bbL^\infty_\loc[0,\infty)$ of locally bounded functions $[0,\infty) \rightarrow \vecSpace$, equipped with the weighted supremum norm,
\begin{equation*}
  (\pathVar, \pathVar') \mapsto \sup_{t\in [0,\infty)} e^{-t} |\pathVar(t) - \pathVar'(t)|,
\end{equation*}
for this metric is consistent with $\pathVar_n \rightarrow \pathVar$ if and only if $\pathVar_n|_{[0,\tau]} \rightarrow \pathVar|_{[0,\tau]}$ uniformly for all $\tau \geq 0$, which is the same as the projective limit space induced by the restriction maps.
\[
  (\pathVar_\tau)_{\tau > 0} \in \lim\limits_{\leftarrow\tau} \bbL^\infty[0,\tau] \quad\xleftrightarrow{\pathVar_\tau = \pathVar|_{[0,\tau]}}\quad \pathVar \in \bbL^\infty_\loc[0,\infty)
\]
Applying Dawson-G\"artner \cite[Theorem 4.6.1]{dembo2010}, the rate function over this space is as follows.
\[
  \pathVar \mapsto \left\{ \begin{array}{ll}
    \displaystyle\sup_{\tau>0} \int_0^\tau \ode_\affDist^*\big(\dot\pathVar(t)\big) \rmd t & \pathVar(0) = 0, ~ \pathVar \in \acpathSpace{[0,\tau)}{\vecSpace} \text{ for all } \tau > 0 \\
    \infty & \text{otherwise}
  \end{array}\right.
\]
From here, we recognize that each process $Y^\epsilon$ is c\`adl\`ag; if $\nu$ is supported on $\stateSpace$, the process takes values in $\pathSpace{[0,\infty)}{\stateSpace}$, and so we may restrict our principle (see \cite[Lemma 4.1.5(b)]{dembo2010}).
Our rate function then takes the same form (recall the local definition of absolute continuity $\acpathSpace{[0,\infty)}{\stateSpace}$).
\begin{equation}
  \label{eq:rf-mogulskii}
  \pathVar \mapsto \left\{\begin{array}{ll}
    \displaystyle\int_0^\infty \ode_\affDist^*\big(\dot\pathVar(t)\big) \rmd t & \pathVar(0) = 0, ~\pathVar \in \acpathSpace{[0,\infty)}{\stateSpace} \\
    \infty & \text{otherwise}
  \end{array}\right.
\end{equation}

% apply Mogulskii to get Brownian motion
\begin{example}[Brownian motion]
  \label{example:brownian}
  Applying Mogulskii's theorem when our increment distribution $\affDist$ is $\operatorname{Normal}(0,\id_\vecSpace)$, the integral in our rate function in (\ref{eq:rf-mogulskii}) becomes the following.
  \begin{equation}
    \label{eq:rf-brownian}
    \int_0^\infty \frac12\big|\dot\pathVar(t)\big|^2 \rmd t
  \end{equation}
  Furthermore, for a Brownian motion $\W$, the process $\sqrt\epsilon\W$ ends up being exponentially equivalent to $Y^\epsilon$,
  \begin{equation*}
    \limsup_{\epsilon\rightarrow0} \epsilon\log \Prb\big( |\sqrt\epsilon\W - Y^\epsilon| \geq \delta \big) = -\infty,
  \end{equation*}
  which makes the family $\sqrt\epsilon\W$ satisfy the large deviation principle with rate function (\ref{eq:rf-brownian}); this result is known as Schilder's theorem (see \cite[Theorem 5.2.3]{dembo2010}).

  Note that $(\sqrt\epsilon\W)_{\epsilon>0}$ is a family of affine processes covered Theorem \ref{theorem:ldp-integral}.
  We have $\Xe = \sqrt\epsilon\W$, where the base process $\X$ has special differential characteristics $(0, \id_\vecSpace, 0)$.
  The easiest way to see this is by considering Proposition \ref{proposition:sde-asymptotics} with initial state $\stateVar = 0$.
  Our theorem also immediately resolves (\ref{eq:rf-integral}) the same rate function.
  \begin{equation*}
    \ode^*(\diffVar, \stateVar)
    = \sup_{\momentVar \in \vecSpace} \bigg( \prj{\momentVar}{\diffVar} - \frac12\prj{\momentVar}{\id_\vecSpace\cdot \momentVar} \bigg) = \frac12 |\diffVar|
  \end{equation*}
\end{example}


% apply Mogulskii to get Poisson
\begin{example}[Poisson]
  \label{example:poisson}
  One may apply a very similar argument for when our increment distribution $\affDist$ is $\operatorname{Poisson}(1)$.
  In this case, the integral in the rate function in (\ref{eq:rf-mogulskii}) evaluates to
  \begin{equation}
    \label{eq:rf-poisson}
    \int_0^\infty \Big( \dot\pathVar(t)  \log\big(\dot\pathVar(t)\big) - \dot\pathVar(t) + 1 \Big) \rmd t,
  \end{equation}
  so long as $\pathVar(t) \geq 0$ for Lebesgue-almost-every $t \geq 0$ (otherwise, it is infinite).
  In the case that $\pathVar(t) = 0$, we are taking the continuous extension of the integrand, i.e.\ $0\log(0) \defeq 0$.
  Similar to the work of Schilder's theorem, we may show, for a standard Poisson process $\N$, $\epsilon\N_{\cdot/\epsilon}$ is exponentially equivalent to this $Y^\epsilon$, which makes the family satisfy a large deviation principle with rate function (\ref{eq:rf-poisson}).
  In fact and exercise of our reference text, \cite[Exercise 5.2.12]{dembo2010}, suggests the reader to show just this.

  Again, such a family $(\epsilon\N_{\cdot/\epsilon})_{\epsilon>0}$ is covered by Theorem \ref{theorem:ldp-integral}.
  To see this, consider a base affine process $\X$ on $(\bbR, \scrB(\bbR))$ with special differential characteristics as below, where $\delta_1$ denotes the degenerate distribution at $1 \in \bbR$.
  \begin{equation*}
    \affDrift(\stateVar) = 1, \quad \affDiff(\stateVar) = 0, \quad \affJump(\stateVar, \rmd\markVar) = \delta_1(\rmd\markVar)
  \end{equation*}
  Setting the initial state $\stateVar = 0$ and looking at Proposition \ref{proposition:sde-asymptotics}, we may say that $\Xe$ can be realized as follows.
  \begin{align*}
    \Xe_t 
    &= t + \epsilon 1_{[0,1]}(\id_\bbR) \ast \compensate{\poisRM^\epsilon}_t \\
    &= t + \epsilon 1_{[0,1]}(\id_\bbR) \ast \poisRM^\epsilon_t - \epsilon 1_{[0,1]}(\id_\bbR) \ast \predproj{\poisRM^\epsilon}_t \\
    &= t + \epsilon \poisRM([0,t/\epsilon] \times [0,1]) - \int_0^{t/\epsilon} \int_\bbR \epsilon 1_{[0,1]}(\markVar) \rmd\markVar \rmd s \\
    &= \epsilon \poisRM([0,t/\epsilon] \times [0,1])
  \end{align*}
  As stated in \cite[Theorem II.4.8]{jacod2003}, this Poisson random measure $\poisRM$ is a Poisson point process with Lebesgue intensity.
  This means that, for each $t \geq 0$, $\N_t \defeq \poisRM([0,t] \times [0,1])$ distributes $\operatorname{Poisson}(t)$, and $\N_t - \N_s = \poisRM((s,t] \times [0,1])$ is independent of $\N_s = \poisRM([0,s], [0,1])$ for each $0 \leq s < t$.
  In other words, $\N$ is a standard Poisson process and
  \begin{equation*}
    \Xe_t = \epsilon \poisRM([0,t/\epsilon] \times [0,1]) = \epsilon\N_{t/\epsilon}.
  \end{equation*}
  As with the normal increments, our rate function (\ref{eq:rf-integral}) resolves this immediately.
  \begin{align*}
    \ode^*(\diffVar, \stateVar) 
    &= \sup_{\momentVar\in\vecSpace} \bigg( \momentVar\diffVar - \momentVar - \int_\bbR \big( e^{\momentVar\markVar} - 1 - \momentVar\markVar \big) \delta_1(\rmd\markVar) \bigg)  \\
    &= \sup_{\momentVar\in\vecSpace} \bigg( \momentVar\diffVar - e^\momentVar + 1 \bigg) \\
    &= \left\{\begin{array}{ll}
      \diffVar \log\diffVar - \diffVar + 1, & \diffVar \geq 0 \\
      \infty, & \text{otherwise}
    \end{array}\right.
  \end{align*}
\end{example}

