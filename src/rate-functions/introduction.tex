We dedicate this final chapter to interpreting the rate function (\ref{eq:rf-integral}) from Theorem \ref{theorem:ldp-integral} and deriving a semi-closed form for $\ode^*$ which illustrates more on the nature of our distributions $(\Prb^\epsilon_\stateVar)_{\epsilon>0}$.
Before doing this, let us establish some intuition with what our integral form \emph{already} tells us.

Firstly, we recognize that a large deviation principle introduces a first-order exponential asymptotic;
the principle shows us that the following limit holds,
\begin{equation*}
  \lim_{\delta\rightarrow0} \lim_{\epsilon\rightarrow0} \epsilon\log\Prb^\epsilon_{\stateVar}\Big( \Xe \in \ball{\pathVar}{\delta} \Big) = -\rf_{\stateVar}(\pathVar),
\end{equation*}
and so we may introduce a correction term $o(\epsilon, \delta)$,
\begin{equation*}
  \Prb^\epsilon_{\stateVar}\Big( \Xe \in \ball{\pathVar}{\delta} \Big)
  = \exp\Big( -\frac1\epsilon\rf_{\stateVar}(\pathVar) + o(\epsilon, \delta) \Big), 
  \qquad
\end{equation*}
which satisfies growth smaller than $1/\epsilon$, in the explicit sense of the following limit.
\begin{equation*}
  \lim_{\delta\rightarrow0} \lim_{\epsilon\rightarrow0} \epsilon o(\epsilon,\delta) = 0
\end{equation*}
By ignoring this expression $o(\epsilon, \delta)$, we are concerning ourselves with a first-order exponential approximation of the following probabilities.
\begin{equation}
  \label{eq:asymptotic-intuition}
  \Prb^\epsilon_{\stateVar}\Big( \Xe \in \ball{\pathVar}{\delta} \Big) \approx 
  \exp\Big( -\frac1\epsilon\rf_{\stateVar}(\pathVar) \Big)
\end{equation}
To this end, $\rf_\stateVar$ interprets properties of $\pathVar$ that the family $(\Xe)_{\epsilon>0}$ is systematically trending towards/against.
Should $\rf_\stateVar(\pathVar)$ be large, the decay of (\ref{eq:asymptotic-intuition}) is faster, demonstrating that $\pathVar$ exhibits some property that the family $(\Xe)_{\epsilon>0}$ does not tend to satisfy.
Thus is the notion of $\rf_\stateVar$ governing a \emph{principle} for what behaviors make a path $\pathVar$ \emph{largely deviate} from the typical behavior of $(\Xe)_{\epsilon>0}$.

The integral nature of $\rf_\stateVar$ indicates to us that there are local properties that the family $(\Xe)_{\epsilon>0}$ systematically exhibits which lends to faster/slower decay from some $\pathVar$.
Take, for instance, if we perturbed some $\pathVar$ only on some interval $[t,\tau]$, coercing the values of $\ode^*(\dot\pathVar(s), \pathVar(s))$ to be larger on this interval $s \in [t,\tau]$.
This will increase the value of $\rf_\stateVar(\pathVar)$, indicating that it was this behavior on the interval $[t,\tau]$ that produced a deviation from the behavior of our family $(\Xe)_{\epsilon>0}$.
Knowing this, it is imperative that we derive some form for $\ode^*$.
Doing so will allow us to see local properties that the family $(\Xe)_{\epsilon>0}$ exhibits.

Unfortunately, the $\stateSpace$-affine nature of $\ode$ provides us no simplification in evaluating $\ode^*$.
\begin{align*}
  \ode^*(\diffVar, \stateVar) 
  &= \prj{\momentVar}{\diffVar} - \ode(\momentVar, \stateVar)  \\
  &= \prj{\momentVar}{\diffVar} -  \bprj{\momentVar}{\affDrift^\trunc(\stateVar)} - \frac12\bprj{\momentVar}{\affDiff(\stateVar)} - \int_\vecSpace \big( e^\prj{\momentVar}{\markVar} - 1 - \prj{\momentVar}{\trunc(\markVar)} \big) \affJump(\stateVar, \rmd\markVar),
\end{align*}
However, by looking to existing rate functions from the literature \cite{duffy2004,dembo2010,kang2014,gao2018b}, we will be able to build an intuition and a toolbox for developing a semi-closed form for $\ode*$.
We find this result remarkable for multiple reasons.
\begin{enumerate}
  \item
    The representation of $\ode^*$ will feel familiar to already-known results
  \item
    The representation of $\ode^*$ will generalize the results we list
  \item
    The representation (though not the large deviation principle) of $\ode^*$ and the corresponding rate function $\rf_\stateVar$ exist independently of the affine assumption.
\end{enumerate}
This chapter is very example heavy, and so we find it to be the most intuitive of the chapters.
By the end, we hope rate functions, as well as large deviations, will be better understood from a general perspective.
All this said, let us now proceed to our final chapter, which is organized as follows.

\begin{enumerate}[leftmargin=24mm]
  \item[\,{\hyperref[rate-functions:mogulskii]{Section }}\ref{rate-functions:mogulskii}.]
    Introduces an important large deviation principle for stochastic processes with independent increments.
  \item[\,{\hyperref[rate-functions:transformations]{Section }}\ref{rate-functions:transformations}.]
    Introduces tools which allow us to lift the preceding principles to stochastic processes with state-dependence.
  \item[\,{\hyperref[rate-functions:coupling]{Section }}\ref{rate-functions:coupling}.]
    Introduces a trick which allows one to evaluate rate functions in the case of there being \emph{multiple sources} of randomness.
  \item[\,{\hyperref[rate-functions:main]{Section }}\ref{rate-functions:main}.]
    Establishes our main result of developing a semi-closed form for $\ode^*$.
  \item[\,{\hyperref[rate-functions:final]{Section }}\ref{rate-functions:final}.]
    Discusses the intuition and potential extensions of our result.
\end{enumerate}
