This chapter concerns proving, for a fixed $\stateVar \in \stateSpace^\circ$, a large deviation principle for a family $(\Prb^\epsilon_\stateVar)_{\epsilon>0}$ of distributions $\Prb^\epsilon_\stateVar$ of affine processes $\Xe$, where each $(\Prb^\epsilon_\stateVar)_{\stateVar\in\stateSpace}$ is the family of kernels respectively associated with $\Xe$.
The parameterization that establishes these distributions $(\Prb^\epsilon_\stateVar)_{\epsilon>0}$ is stated very simply on the level of parameterizing fixed special differential characteristics $(\affDrift, \affDiff, \affJump)$ which already associate with an affine process $\X \defeq \X^1$.
To this end, $(\Xe)_{\epsilon>0}$ is a family of affine processes in which we are effectively regularizing $\X$ as $\epsilon \rightarrow 0$, and our large deviation principle explains this regularization.

The principle is established with regarding $(\Prb^\epsilon_\stateVar)_{\epsilon>0}$ as a family of measures $\Prb^\epsilon_\stateVar$ on the Borel space associated to our Skorokhod topology on $\pathSpace{[0,\infty)}{\stateSpace}$.
This means that there exists a lower-semi-continuous function $\rf_\stateVar: \pathSpace{[0,\infty)}{\stateSpace} \rightarrow [0,\infty]$, known as a \emph{rate function}, with the characterizing property that any $\Gamma \in \scrB(\pathSpace{[0,\infty)}{\stateSpace})$ is such that the probabilities $\big(\Prb^\epsilon_\stateVar( \Xe \in \Gamma)\big)_{\epsilon>0}$
have a first-order exponential asymptotic that corresponds to variational problems on $\rf_\stateVar$.
\begin{equation*}
  -\inf_{\pathVar \in \Gamma^\circ} \rf_\stateVar(\pathVar) 
  \leq \liminf_{\epsilon\rightarrow0} \epsilon\log\Prb^\epsilon_\stateVar\big( \Xe \in \Gamma \big)
  \leq \limsup_{\epsilon\rightarrow0} \epsilon\log\Prb^\epsilon_\stateVar\big( \Xe \in \Gamma \big)
  \leq -\inf_{\pathVar \in \overline\Gamma} \rf_\stateVar(\pathVar) 
\end{equation*}
Having the full strength of the Borel space on $\pathSpace{[0,\infty)}{\stateSpace}$, our admissible sets $\Gamma$ can include a variety of tests on $\Xe$.
For instance, suitable selection of $\Gamma$ allows us to derive asymptotics of probabilities associated with stopping times, path-integrals, and finite-dimensional distributions of $(\Xe)_{\epsilon>0}$.

A great resource for a systematic approach to the theory of large deviations is \cite{dembo2010}.
This text offers comprehension on the subject with both intuitive and technically abstract perspectives, along with providing a descriptive history of the subject.
While we only use this text to cite Cram\'er's theorem, its perspective paves the path of our argument.
In particular, Cram\'er is credited with the \emph{measure-change argument} we use throughout, which generally could be explained as defining exponential measure changes,
\begin{gather}
  \label{eq:vague-measure-change}
  \mgDensity^{\epsilon,\twistVar} \defeq \exp\Big( A(\Xee, \twistVar) - B(\Xee, \twistVar) \Big) \\
  \notag
  \mgPrb^{\epsilon,\twistVar}(\rmd\omega) \defeq \mgDensity^{\epsilon,\twistVar}(\omega) \cdot \Prb^\epsilon_\stateVar(\rmd\omega),
\end{gather}
where $A$ is some linear form and $B$ is appropriately compensating it to make $\Exp_{\Prb^\epsilon_\stateVar}\mgDensity^{\epsilon,\twistVar} = 1$.
The object $\twistVar$ in the above expression is a parameter suitably chosen to make the measure $\mgPrb^{\epsilon,\twistVar}(\Xe \in \Gamma^c)$ effectively \emph{small enough} to neglect.
We will explain this more in the chapter, but we find it important to state here that the nature of our linear form $A$ can be associated with a \emph{finite-dimensional} projection of $\Xee$ or be a linear operator on the \emph{infinite-dimensional} space of paths.

The expression (\ref{eq:vague-measure-change}) looks familiar to our local martingale in Theorem \ref{theorem:LK-exponential-martingale}.
Such a measure change is of the \emph{infinite-dimensional} nature, and is a commonly used tool in the literature of large deviations which we refer to as the \emph{exponential martingale approach}.
Note that the dynamics in Theorem \ref{theorem:LK-girsanov} allow for easy calibration our density in (\ref{eq:vague-measure-change}).
A particular case in which this tool is used is \cite{kang2014}, where they prove a large deviation principle for continuous affine (jump-)diffusions with special differential characteristics $(\affDrift, \affDiff, 0)$.
This approach lends itself to integral expressions which ultimately find their way in the rate function $\rf_\stateVar$.
Having a rate $\rf_\stateVar(\pathVar)$ involve an integral of $\pathVar$ is very useful for investigating local properties of the family $(\Xe)_{\epsilon>0}$ and will be explained more in depth in the next chapter.

For the \emph{finite-dimensional} approach, we may simply use inner-products on $\vecSpace^n$ as our operators $A$ in (\ref{eq:vague-measure-change}).
\begin{gather}
  \label{eq:vague-measure-change-fdd}
  \exp\Big( \sum_{k=1}^n \prj{\momentVar_k}{\X_{t_k}} - B(t_1, \ldots, t_n, \momentVar_1, \ldots, \momentVar_n) \Big),\\
  \notag
   \Exp_\Prb\exp\sum_{k=1}^n \prj{\momentVar_k}{\X_{t_k}} \eqdef B(t_1, \ldots, t_n, \momentVar_1, \ldots, \momentVar_n) 
\end{gather}
We refer to this approach as \emph{twisting} or \emph{tilting}, as each parameter $\momentVar_k$ decides the distribution of the increment $\X_{t_k} - \X_{t_{k-1}}$ for calibration of our density.
The Dawson-G\"artner theorem \cite[Theorem 4.6.1]{dembo2010} provides a great abstraction on how to achieve from here a full principle on the paths.
This is because a stochastic process $\X=(\X_t)_{t\geq0}$---at its weakest---corresponds to a measure on the projective space $(\stateSpace^{[0,\infty)}, \bigotimes_{t\in[0,\infty)}\scrB(\stateSpace))$, and the theorem concerns itself with large deviations of abstract projective limit spaces.
One issue with this approach is that the resulting rate function $\rf_\stateVar$ does not have an integral form like in the exponential martingale approach.
In fact, \cite{kang2014} specifically mentioned this issue when proving their principle.
Another issue is that the projective limit space corresponds to a product topology on $\stateSpace^{[0,\infty)}$, which means we cannot get asymptotics for $\Gamma$ described above.
We need to leverage properties of the Skorokhod topology to then tighten the principle.

We remark in this chapter that, for affine processes, the exponential martingale and twisting approaches are nearly identical.
The affine map $\aff$ which characterizes an affine process provides us with a way of resolving (\ref{eq:vague-measure-change-fdd}) as a specific selection of parameter $\twistVar$ in the exponential-martingale approach.
In this way, our proof is different from \cite{kang2014}, in that we use this resolution to derive an integral form for our Dawson-G\"artner rate function.
A summary of the proof is as simple as:
\begin{center}
  \itshape
  prove large deviation principles for finite-dimensional distributions, prove exponential tightness, conclude large deviation principle on Skorokhod space
\end{center}
which can be compared to Prokhorov's theorem for weak convergence of measures on a separable metric space,
\begin{center}
  \itshape
  prove weak convergence of finite-dimensional distributions, prove tightness, conclude weak convergence
\end{center}
which was the original intent behind defining the Skorokhod topology (see in \cite{skorokhod1956}).
This comparison is not due to us, for researchers are actively studying large deviations theory from a weak convergence perspective.
Puhalskii has actively developed such approaches, and we manipulate a proof from \cite{puhalskii2001} to get the integral representation of our rate function.
For resources on large deviations on the Skorokhod space, see \cite{feng2006}, a resource which we use to get some of our results.

Now that we have introduced the key ideas, the remainder of this chapter is organized as follows.
\begin{enumerate}[leftmargin=20mm]
  \item[\,{\hyperref[large-deviations:asymptotics]{Section }}\ref{large-deviations:asymptotics}.]
    Describes our parameterization $(\Prb^\epsilon_\stateVar)_{\epsilon>0}$, 
  \item[\,{\hyperref[large-deviations:assumptions]{Section }}\ref{large-deviations:assumptions}.]
    Delineates our assumptions for the result,
  \item[\,{\hyperref[large-deviations:dawson-gaertner]{Section }}\ref{large-deviations:dawson-gaertner}.]
    Proves our principle via twisting,
  \item[\,{\hyperref[large-deviations:exponential-martingales]{Section }}\ref{large-deviations:exponential-martingales}.]
    Resolves the approach of twisting with that of exponential martingales,
  \item[\,{\hyperref[large-deviations:rate-function]{Section }}\ref{large-deviations:rate-function}.]
    Simplifies our rate function to an integral form.
\end{enumerate}
