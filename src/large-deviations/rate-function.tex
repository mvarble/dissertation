% discuss how we intend to get integral form of rate function
So far, for each $\stateVar \in \stateSpace^\circ$, Theorem \ref{theorem:ldp} provides us a large deviation principle for $(\Prb^\epsilon_\stateVar)_{\epsilon>0}$ with good rate function $\rf_\stateVar$ as in (\ref{eq:rf}).
This section is concerned with simplifying the nature of $\rf_\stateVar$ to take a more explicit integral form, comparable to existing principles for other families of stochastic processes.
\begin{equation*}
  \rf_\stateVar(\pathVar) = \left\{\begin{array}{ll}
    \displaystyle \int_0^\infty \ode^*\big(\dot\pathVar(t), \pathVar(t)\big) \rmd t & \pathVar(0) = \stateVar, ~ \pathVar \in \acpathSpace{[0,\infty)}{\stateSpace} \\
    \infty & \text{otherwise}
  \end{array}\right.
\end{equation*}
To show this, we start by defining a map $\rf$ which composes the initial state through our rate functions $(\rf_\stateVar)_{\stateVar\in\stateSpace}$, to remove the finiteness condition of each $\rf_\stateVar$ in (\ref{eq:rf}).
\begin{equation*}
  \rf(\pathVar) \defeq \rf_{\pathVar(0)}(\pathVar) = \sup_{\underline t \vdash \Delta_\pathVar^c} \aff^*\big(\underline t, \pathVar(\underline t), \pathVar(0) \big)
\end{equation*}
In the following results, we will--without mention--assume evaluations of $\rf(\pathVar)$ for $\pathVar(0) \in \stateSpace^\circ$, so that we can use Theorem \ref{theorem:mgf-fdds}.
Note that this is at no loss of generality, since we are resolving our rate function $\rf_\stateVar$ for a large deviation principle that already requires $\stateVar \in \stateSpace^\circ$.

% Show that we need \pathVar to be absolutely continuous for it to have finite rate.
\begin{lemma}
  \label{lemma:need-ac}
  If $\pathVar \not\in \acpathSpace{[0,\infty)}{\stateSpace}$, then $\rf(\pathVar) = \infty$.
\end{lemma}
\begin{proof}
  \label{proof:lemma:need-ac}
  Fix some $\pathVar \in \pathSpace{[0,\infty)}{\stateSpace}$ with $\pathVar \not\in \acpathSpace{[0,\infty)}{\stateSpace}$.
  That is to say, there exists some $\tau > 0$ for which $\pathVar \not\in \acpathSpace{[0,\tau]}{\stateSpace}$.
  For any $\rho > 0$, we now use Proposition \ref{proposition:real-moments-revisit}\ref{proposition:real-moments-revisit:aff-uniform-bound} to produce some $\delta > 0$ and $C_{\delta,\rho} > 0$ such that the following bound holds.
  \begin{equation*}
    \big|\aff(t, \momentVar, \stateVar) - \aff(0, \momentVar, \stateVar)\big| \leq C_{\delta,\rho} \cdot t \cdot \big( 1 + |\stateVar| \big), \quad t \in [0,\delta], ~ \momentVar \in \closedBall{0}{\rho}, ~ \stateVar \in \stateSpace
  \end{equation*}
  Because $\pathVar \not\in \acpathSpace{[0,\tau]}{\stateSpace}$, there exists $\epsilon > 0$ and a partition $\underline t^\rho \vdash [0,\tau]$ such that
  \begin{gather*}
    \sum_{k=1}^{|\underline t^\rho|} \Delta t^\rho_k < \delta \wedge \Big(C_{\delta,\rho} \big( 1 + \sup_{t\in[0,\tau]}|\pathVar(t)| \big) \Big)^{-1} \\
    \sum_{k=1}^{|\underline t^\rho|} \big|\pathVar(t^\rho_k) - \pathVar(t^\rho_{k-1}) \big| \geq \epsilon
  \end{gather*}
  The countable nature of $\Delta_\pathVar$ allows us to further impose that $\underline t^\rho \vdash \Delta_\pathVar^c$.
  This, along with Theorem \ref{theorem:mgf-fdds}, results in the following inequality.
  \begin{align*}
    &\sup_{\underline t \vdash [0,\infty)} \aff^*\big( \underline t, \pathVar(\underline t), \pathVar(0) \big) \\
    &\quad\geq \sup_{\rho > 0} \aff^*\big( \underline t^\rho, \pathVar(\underline t^\rho), \pathVar(0) \big) \\
    &\quad\geq \sup_{\rho > 0} \sup_{\underline\twistVar \in \momentSpace(\underline t)}\sum_{k=1}^{|\underline t^\rho|}  \Big( \prj{\twistVar_k}{\pathVar(t^\rho_k)-\pathVar(t^\rho_{k-1})} -  \affShift\big( \Delta t^\rho_k, \twistVar_k, \pathVar(t^\rho_{k-1}) \big) \Big) \\
    &\quad= \sup_{\rho > 0} \sum_{k=1}^{|\underline t^\rho|} \sup_{\twistVar_k\in\momentSpace(\Delta t^\rho_k)} \begin{aligned}[t]
      &\bigg( \bprj{\twistVar_k}{\pathVar(t^\rho_k)-\pathVar(t^\rho_{k-1})} \\
      &\hspace{1em} -\Big(\aff\big( \Delta t^\rho_k, \twistVar_k, \pathVar(t^\rho_{k-1}) \big) - \aff^*\big(0, \twistVar_k, \pathVar(t^\rho_{k-1})\big) \Big) \bigg)\\
    \end{aligned} \\
    &\quad\geq \sup_{\rho>0} \sum_{k=1}^{|\underline t^\rho|} \bigg( \rho\big|\pathVar(t^\rho_k)-\pathVar(t^\rho_{k-1})\big| - C_{\delta,\rho} \cdot \Delta t^\rho_k \cdot \big( 1 + |\pathVar(t^\rho_{k-1})| \big) \bigg) \\
    &\quad\geq \epsilon \cdot \sup_{\rho>0} \rho - 1 \\
    &\quad= \infty
  \end{align*}
\end{proof}



% Write \affShift as an integral operator.
\begin{lemma}
  \label{lemma:fdd-rf-as-integral}
  For each $\pathVar \in \acpathSpace{[0,\infty)}{\stateSpace}$, $\underline t \vdash [0,\infty)$ and $\underline\twistVar \in \prod_{k=1}^{|\underline t|} \momentSpace(\Delta t_k)$, we have an identity similar to that of Theorem \ref{theorem:equivalence}.
  \begin{equation*}
    \fddRf(\pathVar, \underline t, \underline\twistVar) = \mgRf\big(\pathVar, \twistFunc(\cdot, \underline t, \underline\twistVar)\big) = \int_0^\infty \Big( \bprj{\twistFunc(t, \underline t, \underline\twistVar)}{\dot\pathVar(t)} - \ode\big(\twistFunc(t, \underline t, \underline\twistVar), \pathVar(t)\big) \Big) \rmd t.
  \end{equation*}
\end{lemma}

\begin{proof}
  Similar to as in Theorem \ref{theorem:equivalence}, we use Propositions \ref{proposition:momentSpace-facts}\ref{proposition:momentSpace-facts:3} and \ref{proposition:real-moments-revisit}\ref{proposition:real-moments-revisit:big-time-small-moments} and apply the fundamental theorem of calculus and integration by parts.
  \begin{align*}
    \fddRf(\pathVar, \underline t, \underline\twistVar)
    &= \sum_{k=1}^{|\underline t|} \Big( \bprj{\twistVar_k}{\pathVar(t_k)-\pathVar(t_{k-1})} - \affShift\big(\Delta t_k, \twistVar_k, \pathVar(t_{k-1})\big) \Big) \\
    &= \sum_{k=1}^{|\underline t|} \Big( \aff\big(t_k-t_k, \twistVar_k, \pathVar(t_k) \big) - \aff\big(t_k-t_{k-1}, \twistVar_k, \pathVar(t_{k-1})\big) \Big) \\
    &= \sum_{k=1}^{|\underline t|} \int_{t_{k-1}}^{t_k} \frac{\partial}{\partial t} \aff\big(t_k-t, \twistVar_k, \pathVar(t)\big) \rmd t \\
    &= \sum_{k=1}^{|\underline t|} \int_{t_{k-1}}^{t_k} \Big( -\dot\aff\big(t_k-t, \twistVar_k, \pathVar(t)\big) + \bprj{\affPart(t_k-t, \twistVar_k)}{\dot\pathVar(t)} \Big) \rmd t \\
    &= \sum_{k=1}^{|\underline t|} \int_{t_{k-1}}^{t_k} \Big( \bprj{\affPart(t_k-t, \twistVar_k)}{\dot\pathVar(t)} - \ode\big( \affPart(t_k-t,\twistVar_k), \pathVar(t) \big)\Big) \rmd t \\
    &= \int_0^\infty \Big( \bprj{\twistFunc(t,\underline t,\underline\twistVar)}{\dot\pathVar(t)} - \ode\big(\twistFunc(t, \underline t, \underline\twistVar), \pathVar(t)\big) \Big) \rmd t \\
    &= \bprj{\twistFunc(0, \underline t, \underline\twistVar)}{\pathVar(0)} - \int_0^\infty \pathVar(t-) \rmd\twistFunc(t,\underline t,\underline\twistVar)- \int_0^\infty \ode\big(\twistFunc(t, \underline t, \underline\twistVar), \pathVar(t)\big) \rmd t \\
    &= \mgRf\big(\pathVar, \twistFunc(\cdot, \underline t, \underline\twistVar) \big)
  \end{align*}
\end{proof}



% "piecewise functions" show rate function sum < rate function integral
\begin{proposition}
  \label{proposition:rf-upper-bound}
  For each $\pathVar \in \acpathSpace{[0,\infty)}{\stateSpace}$, we have the following upper bound.
  \begin{equation*}
    \rf(\pathVar) \leq \int_0^\infty \ode^*\big(\dot\pathVar(t), \pathVar(t)\big) \rmd t
  \end{equation*}
\end{proposition}
\begin{proof}
  Fix $\underline t \vdash [0,\infty)$ and $\underline\momentVar \in \vecSpace^{|\underline t|}$.
  Observe that if $\aff\big(\underline t, \underline\momentVar, \pathVar(0)\big) = \infty$, we immediately have the following inequality.
  \begin{equation*}
    \bprj{\underline\momentVar}{\pathVar(\underline t)} - \aff\big(\underline t, \underline\momentVar, \pathVar(0)\big) = -\infty \leq 0 = \int_0^\infty \ode^*\big(\dot\pathVar(t), \pathVar(t)\big) \rmd t
  \end{equation*}
  Otherwise, Theorem \ref{theorem:mgf-fdds} tells us that $\underline\momentVar = \twistToMoment_{\underline t}(\underline\twistVar)$ for some $\underline\twistVar \in \momentSpace(\underline t)$.
  By Lemma \ref{lemma:fdd-rf-as-integral}, we now see the same inequality.
  \begin{align*}
    \bprj{\underline\momentVar}{\pathVar(\underline t)} - \aff\big(\underline t, \underline\momentVar, \pathVar(0)\big) 
    &= \sum_{k=1}^{|\underline t|} \Big(\bprj{\twistVar_k}{\pathVar(t_k)-\pathVar(t_{k-1})} - \affShift(\Delta t_k, \twistVar_k, \pathVar(t_{k-1}) \big)\Big) \\
    &= \fddRf(\pathVar, \underline t, \underline\twistVar) \\
    &= \int_0^\infty \Big( \bprj{\twistFunc(t,\underline t,\underline\twistVar)}{\dot\pathVar(t)} - \ode\big(\twistFunc(t, \underline t, \underline\twistVar), \pathVar(t)\big) \Big) \rmd t \\
    &\leq \int_0^\infty \ode^*\big(\dot\pathVar(t), \pathVar(t)\big) \rmd t
  \end{align*}
  Thus, we have the following upper bound.
  \begin{align*}
    \rf(\pathVar) 
    = \sup_{\underline t \vdash \Delta_\pathVar^c} \aff^*\big(\underline t, \pathVar(\underline t), \pathVar(0)\big)
    &= \sup_{\underline t \vdash [0,\infty)} \sup_{\underline\momentVar \in \vecSpace^{|\underline t|}} \Big( \bprj{\underline\momentVar}{\pathVar(\underline t)} - \aff\big(\underline t, \underline\momentVar, \pathVar(0)\big) \Big) \\
    &\leq \int_0^\infty \ode^*\big(\dot\pathVar(t), \pathVar(t)\big) \rmd t
  \end{align*}
\end{proof}



% "piecewise functions" are dense in continuous functions
\begin{lemma}
  \label{lemma:span-affPart-dense}
  Fix $\tau > 0$ and $\twistFunc \in \cpathSpace{[0,\tau]}{\stateSpace}$.
  For each $\epsilon > 0$, there exists a partition $\underline t \vdash [0,\tau]$ and $\underline\twistVar \in \momentSpace(\underline t)$ such that we have the following approximation.
  \begin{equation*}
    \sup_{t \in [0,\tau)} \big| \twistFunc(t, \underline t, \underline\momentVar) - \twistFunc(t) \big| < \epsilon
  \end{equation*}
\end{lemma}

\begin{proof}
  \label{proof:lemma:span-affPart-dense}
  Denote $M \defeq \sup_{t\in[0,\tau]} |\twistFunc(t)|$ and fix $\epsilon > 0$.
  We start by using Proposition \ref{proposition:real-moments-revisit}\ref{proposition:real-moments-revisit:aff-uniform-bound} to guarantee $\delta_0 > 0$ and $C_M > 0$ such that the following bound holds.
  \begin{equation*}
    \big|\aff(t,\momentVar,\stateVar)-\aff(0,\momentVar,\stateVar)\big| \leq C_M \cdot t \cdot \big(1 + |\stateVar|\big), \quad  t \in [0,\delta_0], ~ \momentVar \in  \closedBall{0}{M}, ~\stateVar \in \stateSpace
  \end{equation*}
  From here, we use the affine structure of $\aff$ to see the following inequality.
  \begin{equation*}
    \big|\affPart(t,\momentVar)-\momentVar\big| = 3C_M\sqrt d \cdot t, \quad (t,\momentVar) \in [0,\delta_0]\times\overline B(0,M)
  \end{equation*}
  Seeing as $\twistFunc \in \cpathSpace{[0,\tau]}{\vecSpace}$, it is uniformly continuous.
  Fix $\delta_1 > 0$ such that all $s, t \in [0,\tau]$ with $|t-s| < \delta_1$, we have the following inequality.
  \begin{equation*}
    \big|\twistFunc(t) - \twistFunc(s)\big| < \epsilon/2
  \end{equation*}
  Fix an integer $N \in \bbN$ large enough to impose the following inequality.
  \begin{equation*}
    \frac{\tau}{N} < \frac{\epsilon}{6C_M \sqrt d} \wedge \delta_0 \wedge \delta_1
  \end{equation*}
  Now define partition $\underline t$ by $t_k = k\tau/N$ for $k = 1, \ldots, N$.
  This way, for each $k = 1, \ldots, N$, and $t \in [t_{k-1},t_k)$, we have
  \begin{align*}
    \Big| \twistFunc(t, \underline t, \twistFunc(\underline t)\big) - \twistFunc(t) \Big| 
    &= \Big| \affPart\big(t_k - t, \twistFunc(t_k)\big) - \twistFunc(t) \Big| \\
    &\leq \Big| \affPart\big(t_k - t, \twistFunc(t_k)\big) - \twistFunc(t_k) \Big| + \big| \twistFunc(t_k) - \twistFunc(t) \big| \\
    &< 3 C_M \sqrt d \cdot (t_k-t) + \epsilon/2 \\
    &< \epsilon.
  \end{align*}
\end{proof}



% rate function integral < rate function sum
\begin{proposition}
  \label{proposition:rf-lower-bound}
  For each $\pathVar \in \acpathSpace{[0,\infty)}{\stateSpace}$, we have the following inequality.
  \begin{equation*}
    \rf(\pathVar) \geq \int_0^\infty \ode^*\big( \dot\pathVar(t), \pathVar(t)\big) \rmd t
  \end{equation*}
\end{proposition}

\begin{proof}
  \label{proof:proposition:rf-lower-bound}
  We proceed in a way similar to that of \cite{puhalskii2001}.
  Define a map $f: [0,\infty) \times \vecSpace \rightarrow \bbR$ as below.
  \begin{equation*}
    f(t, \twistVar) \defeq \bprj{\twistVar}{\dot\pathVar(t)} - \ode\big(\twistVar, \pathVar(t)\big)
  \end{equation*}
  For a fixed $\epsilon > 0$, $t \in [0,\infty)$, we define the following set.
  \begin{equation*}
    \Gamma^\epsilon_t \defeq \left\{ \twistVar \in \vecSpace : \Big( \sup_{\twistVar' \in \vecSpace} f(t, \twistVar') - \epsilon \Big)_+ \wedge \frac1\epsilon \leq f(t, \twistVar) \leq \frac1\epsilon \right\}
  \end{equation*}
  Continuity of $f(t,\cdot)$ and the least upper bound property guarantees $\Gamma^\epsilon_t$ is nonempty and measurable.
  Thus, we may construct a measurable selection $\twistFunc: [0,\infty) \rightarrow \vecSpace$ for $\Gamma^\epsilon$, which is to say $\twistFunc$ is Lebesgue measurable and the following holds.
  \begin{equation}
    \label{eq:measurable-in-gamma}
    f\big(t, \twistFunc(t)\big) \in \Gamma^\epsilon_t, \quad t \in [0,\infty)
  \end{equation}
  We now use Luzin's theorem to approximate $\twistFunc|_{[0,1/\epsilon]}$ with $\tilde\twistFunc \in \cpathSpace{[0,1/\epsilon]}{\vecSpace}$ to the following extent.
  \begin{equation}
    \label{eq:luzin}
    \int_{\tilde\twistFunc\neq\twistFunc}\rmd t < \epsilon^2
  \end{equation}
  Now, we combine our inequalities from (\ref{eq:measurable-in-gamma}) and (\ref{eq:luzin}) to see that
  \begin{align}
    \notag
    &\hspace{-1em}\int_0^{1/\epsilon} \Big( f\big(t, \tilde\twistFunc(t)\big) \vee 0 \Big) \rmd t \\
    \notag
    &= \int_0^{1/\epsilon}  f\big(t, \twistFunc(t)\big) \rmd t + \int_{\tilde\twistFunc\neq\twistFunc} \bigg(\Big( f\big(t, \tilde\twistFunc(t)\big) \vee 0 \Big) - f(t, \twistFunc(t)\big) \bigg)\rmd t \\
    \notag
    &\geq \int_0^{1/\epsilon}  \bigg( \Big( \sup_{\twistVar\in\vecSpace} f(t, \twistVar) - \epsilon \Big) \wedge \frac1\epsilon \bigg) \rmd t - \int_{\tilde\twistFunc\neq\twistFunc} \frac1\epsilon \rmd t \\
    &= \int_0^{1/\epsilon} \bigg( \Big( \ode^*\big(\dot\pathVar(t),\pathVar(t)\big) - \epsilon \Big) \wedge \frac1\epsilon \bigg) \rmd t - \epsilon
    \label{eq:reduce-to-continuous}
  \end{align}
  By Lemma \ref{lemma:span-affPart-dense}, the fact that each $f(t,0) = 0$, and continuity of each $f(t,\cdot)$, we may now use Fatou's lemma to guarantee some $\underline t \vdash [0,\infty)$ and $\underline\twistVar \in \momentSpace(\underline t)$ such that
  \begin{equation}
    \label{eq:reduce-to-affPart-span}
    \int_0^{1/\epsilon} f\big(t, \twistFunc(t, \underline t, \underline\twistVar)\big) \rmd t
    \geq \int_0^{1/\epsilon} \Big( f\big(t, \tilde\twistFunc(t)\big) \vee 0 \Big) \rmd t - \epsilon
  \end{equation}
  Combining (\ref{eq:reduce-to-continuous}) and (\ref{eq:reduce-to-affPart-span}), we now see that
  \begin{equation*}
    \int_0^{1/\epsilon} f\big(t, \twistFunc(t, \underline t, \underline\twistVar)\big) \rmd t 
    \geq \int_0^{1/\epsilon} \bigg( \Big( \ode^*\big(\dot\pathVar(t),\pathVar(t)\big) - \epsilon \Big) \wedge \frac1\epsilon \bigg) \rmd t - 2\epsilon.
  \end{equation*}
  By Theorem \ref{theorem:mgf-fdds} and Lemmas \ref{lemma:need-ac} and \ref{lemma:fdd-rf-as-integral}, the above inequality gives us the following.
  \begin{align*}
    \rf_\stateVar(\pathVar)
    &\geq \aff^*\big(\underline t, \pathVar(\underline t), \pathVar(0)\big) \\
    &\geq\sum_{k=1}^{|\underline t|} \Big( \bprj{\twistVar_k}{\pathVar(t_k)} - \aff\big(\Delta t_k, \twistVar_k, \pathVar(t_{k-1})\big) \Big) \\
    &=\sum_{k=1}^{|\underline t|} \int_{t_{k-1}}^{t_k} \Big( \bprj{\affPart(t_k-t,\twistVar_k)}{\dot\pathVar(t)} - \ode\big(\affPart(t_k-t, \twistVar_k), \pathVar(t)\big) \Big) \rmd t \\
    &= \int_0^{1/\epsilon} f\big( t, \twistFunc(t, \underline t, \underline\twistVar)\big) \rmd t \\
    &\geq \int_0^{1/\epsilon} \bigg( \Big( \ode^*\big(\dot\pathVar(t),\pathVar(t)\big) - \epsilon \Big) \wedge \frac1\epsilon \bigg) \rmd t - 2\epsilon
  \end{align*}
  Taking $\epsilon \rightarrow 0$ now yields our desired result.
  \begin{equation*}
    \rf_\stateVar(\pathVar) \geq \int_0^\infty \ode^*\big(\dot\pathVar(t), \pathVar(t)\big) \rmd t
  \end{equation*}
\end{proof}



% LDP with integral
\begin{theorem}
  \label{theorem:ldp-integral}
  For each $\stateVar \in \stateSpace^\circ$, the family $(\Prb^\epsilon_\stateVar)_{\epsilon>0}$ satisfies a large deviation principle on $\pathSpace{[0,\infty)}{\stateSpace}$ with good rate function $\rf_\stateVar: \pathSpace{[0,\infty)}{\stateSpace} \rightarrow [0,\infty]$ as follows.
  \begin{equation}
    \label{eq:rf-integral}
    \rf_\stateVar(\pathVar) = \left\{\begin{array}{ll}
      \displaystyle\int_0^\infty \ode^*\big(\dot\pathVar(t), \pathVar(t)\big) \rmd t & \pathVar \in \acpathSpace{[0,\infty)}{\stateSpace}, ~ \pathVar(0) = \stateVar \\
      \infty & \text{otherwise}
    \end{array}\right.
  \end{equation}
\end{theorem}
\begin{proof}
  Theorem \ref{theorem:ldp} gives us our large deviation principle with rate function $\rf_\stateVar$ as in (\ref{eq:rf}).
  Fix $\pathVar \in \pathSpace{[0,\infty)}{\stateSpace}$.
  If $\pathVar(0) \neq \stateVar$, then we already have $\rf_\stateVar(\pathVar)=\infty$.
  Otherwise, $\rf_\stateVar(\pathVar) = \rf_{\pathVar(0)}(\pathVar) = \rf(\pathVar)$, and so Lemma \ref{lemma:need-ac} tells us $\rf_\stateVar(\pathVar) = \infty$ if $\pathVar \not\in \acpathSpace{[0,\infty)}{\stateSpace}$.
  If $\pathVar \in \acpathSpace{[0,\infty)}{\stateSpace}$, then Propositions \ref{proposition:rf-upper-bound} and \ref{proposition:rf-lower-bound} tell us that
  \begin{equation*}
    \rf_\stateVar(\pathVar) = \int_0^\infty \ode^*\big( \dot\pathVar(t), \pathVar(t) \big) \rmd t.
  \end{equation*}
  This concludes that $\rf$ can be written as in (\ref{eq:rf-integral}).
\end{proof}


