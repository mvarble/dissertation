\begin{theorem}
  \label{eq:ldp-closed-form}
  Let $\X$ be an affine process as introduced above with Assumption \ref{assumption:joint-affine} satisfied.
  Then $\hat\X$ is a base affine process with which we may parameterize $\hat\Xe$ with distribution as in Section \ref{large-deviations:asymptotics}.
  Fixing $\hat\stateVar \in \hat\stateSpace^\circ$ and denoting $\hat\Prb^\epsilon_{\hat\stateVar}$ the distribution associated with $\hat\Xe$ starting at $\hat\stateVar$, we have a large deviation principle for $(\hat\Prb^\epsilon_{\hat\stateVar})_{\epsilon>0}$.
  The rate function $\rf_{\hat\stateVar}$ simplifies to the following semi-closed form, where we denote the components of an arbitrary function $\hat\pathVar$ by $\hat\pathVar \defeq (\pathVar, \omega, \gamma, \eta)$.
  \begin{equation*}
    \rf_{\hat\stateVar}(\hat\pathVar) = 
    \begin{aligned}[t]
      &\int_0^\infty \frac12 \Bprj{\dot\omega(t)}{\affDiff\big(\pathVar(t)\big)^\dagger \dot\omega(t)} \rmd t 
      + \int_0^\infty \bigg( \dot\eta(t)\log\Big(\frac{\dot\eta(t)}{\affInt\big(\pathVar(t)\big)} \Big) - \dot\eta(t) + \affInt\big(\pathVar(t)\big) \bigg) \rmd t  \\
      &\qquad+ \int_0^\infty \dot\eta(t) \ode_{\affDist(\pathVar(t),\cdot)}\bigg( \frac{\dot\gamma(t) + \affInt\big(\pathVar(t)\big)\overline{\affDist(\pathVar(t), \cdot)}}{\dot\eta(t)} \bigg) \rmd t
    \end{aligned}
  \end{equation*}
  In the evaluation above, we are insisting that $\hat\pathVar$ satisfies the following properties below, where statements involving $t$ are taken Lebesgue-almost-everywhere; otherwise $\rf_{\hat\stateVar}(\hat\pathVar) = \infty$.
  \begin{itemize}
    \item
      $\hat\pathVar(0) = \hat\stateVar$,
    \item
      $\pathVar \in \acpathSpace{[0,\infty)}{\stateSpace}$, $\omega \in \acpathSpace{[0,\infty)}{\vecSpace}$, $\gamma \in \acpathSpace{[0,\infty)}{\vecSpace}$, and $\eta \in \acpathSpace{[0,\infty)}{\bbR_+}$,
    \item
      $\dot\pathVar(t) = \affDrift\big(\pathVar(t)\big) + \dot\omega(t) + \dot\gamma(t)$,
    \item
      $\dot\omega(t) \in \operatorname{range}\big(\affDiff(\pathVar(t))\big)$,
    \item
      $\dot\eta(t) \geq 0$,
    \item
      $\affInt\big(\pathVar(t)\big) \geq 0$.
  \end{itemize}
\end{theorem}
\begin{proof}
  Assumption \ref{assumption:joint-affine} is precisely the Riccati equation with which the following process
  \[
    t \mapsto \exp\Big( \hat\affPart_0(\tau-t, \hat\momentVar) + \Bprj{\hat\affPart(\tau-t, \hat\momentVar)}{\hat\X_t} \Big)
  \]
  is a martingale which gives us our desired transform formula.
  This makes $\hat\X$ an affine process, so Lemma \ref{lemma:joint-cone} and Proposition \ref{proposition:joint-characteristics} tell us that $\hat\X$ satisfies the assumptions of Section \ref{large-deviations:assumptions}.
  Thus, the parameterization $\hat\Xe$ is a family described by Theorem \ref{theorem:ldp-integral}, and so we have a large deviation principle for $(\hat\Prb^\epsilon_{\hat\stateVar})_{\epsilon>0}$.
  Again using Proposition \ref{proposition:joint-characteristics}, we see that the integrand evaluates the following function.
  \begin{align*}
    \hat\ode^*(\hat{\diffVar}, \hat\stateVar) 
    &= \sup_{\hat\momentVar \in \hat\vecSpace} \bigg( \bprj{\hat\momentVar}{\hat\diffVar} - \hat\ode\big(\hat\momentVar, \hat\stateVar\big) \bigg) \\
    &= \sup_{\hat\momentVar \in \hat\vecSpace} \bigg(\begin{aligned}[t]
      &\prj{\momentVar_1}{\diffVar_1} 
      + \prj{\momentVar_2}{\diffVar_2} 
      + \prj{\momentVar_3}{\diffVar_3} 
      + \momentVar_4\diffVar_4
      - \prj{\momentVar_1}{\affDrift(\stateVar_1)}  \\
      &- \frac12\bprj{\momentVar_1+\momentVar_2}{\affDiff(\stateVar)(\momentVar_1+\momentVar_2)} 
      - \affInt(\stateVar_1)\exp\Big(\ode_{\affDist(\stateVar_1,\cdot)}(\momentVar_1+\momentVar_3) + \momentVar_4\Big) \\
      &+ \affInt(\stateVar_1) 
      + \bprj{\momentVar_1+\momentVar_3}{\affInt(\stateVar_1)\overline{\affDist(\stateVar_1,\cdot)}}  \bigg)
    \end{aligned}
  \end{align*}
  We now break the problem into cases.
  \begin{enumerate}
    \item
      First, assume $\diffVar_1 = \affDrift(\stateVar_1) + \diffVar_2 + \diffVar_3$, $\diffVar_2 \in \operatorname{range}(\affDiff(\stateVar_1))$, $\diffVar_4 \geq 0$, and $\affInt(\stateVar_1) \geq 0$.
      Using Lemma \ref{lemma:ode-differentiable}, we are able to take the gradients of the above expression to get the following system of equations.
      \begin{align*}
        0 &= \begin{aligned}[t]
          &\diffVar_1 - \affDrift(\stateVar_1) - \affDiff(\stateVar_1) (\momentVar_1 + \momentVar_2) \\
          &- \affInt(\stateVar_1) \exp\Big( \ode_{\affDist(\stateVar_1,\cdot)}(\momentVar_1+\momentVar_3) + \momentVar_4\Big) \nabla\ode_{\affDist(\stateVar_1,\cdot)}(\momentVar_1+\momentVar_3) + \affInt(\stateVar_1) \overline{\affDist(\stateVar_1, \cdot)}
        \end{aligned} \\
        0 &= \diffVar_2 - \affDiff(\stateVar_1) ( \momentVar_1 + \momentVar_2 ) \\
        0 &= \diffVar_3 - \affInt(\stateVar_1) \exp\Big( \ode_{\affDist(\stateVar_1,\cdot)}(\momentVar_1+\momentVar_3) + \momentVar_4\Big) \nabla\ode_{\affDist(\stateVar_1,\cdot)}(\momentVar_1+\momentVar_3) + \affInt(\stateVar_1) \overline{\affDist(\stateVar_1, \cdot)} \\
        0 &= \diffVar_4 - \affInt(\stateVar_1) \exp\Big( \ode_{\affDist(\stateVar_1,\cdot)}(\momentVar_1+\momentVar_3) + \momentVar_4\Big) 
      \end{align*}
      These equations yield the following via simple algebra.
      \begin{equation}
        \label{eq:semi-closed-algebra}
        \begin{aligned}
          \momentVar_1 + \momentVar_2 &= \affDiff(\stateVar_1)^\dagger \diffVar_2 \\
          \diffVar_4 &= \affInt(\stateVar_1) \exp\Big( \ode_{\affDist(\stateVar_1,\cdot)}(\momentVar_1+\momentVar_3) + \momentVar_4\Big)  \\
          \diffVar_4\log\Big(\frac{\diffVar_4}{\affInt(\stateVar_1)}\Big) &=   \diffVar_4\ode_{\affDist(\stateVar_1,\cdot)}(\momentVar_1+\momentVar_3) + \momentVar_4\diffVar_4 \\
          \frac{\diffVar_3 + \affInt(\stateVar_1)\overline{\affDist(\stateVar_1,\cdot)}}{\diffVar_4} &= \nabla\ode_{\affDist(\stateVar_1,\cdot)}(\momentVar_1+\momentVar_3)
        \end{aligned}
      \end{equation}
      Seeing as $\ode_{\affDist(\stateVar_1,\cdot)}$ is a convex function, the last equality of (\ref{eq:semi-closed-algebra}) corresponds to the unique extreme point of the Fenchel-Legendre transform.
      \begin{equation*}
        \Bprj{\momentVar_1 + \momentVar_3}{\frac{\diffVar_3+\affInt(\stateVar_1)\overline{\affDist(\stateVar_1,\cdot)}}{\diffVar_4}} - \ode_{\affDist(\stateVar_1,\cdot)}(\momentVar_1+\momentVar_3) = \ode_{\affDist(\stateVar_1,\cdot)}^*\Big(\frac{\diffVar_3+\affInt(\stateVar_1)\overline{\affDist(\stateVar_1,\cdot)}}{\diffVar_4}\Big)
      \end{equation*}
      Combining this equality with those of (\ref{eq:semi-closed-algebra}) now gives us the following identity.
      \begin{align*}
        \bprj{\hat\momentVar}{\hat\diffVar} - \hat\ode\big(\hat\momentVar, \hat\stateVar\big) 
        &=\begin{aligned}[t]
          &\prj{\momentVar_1}{\diffVar_1} 
          + \prj{\momentVar_2}{\diffVar_2} 
          + \prj{\momentVar_3}{\diffVar_3} 
          + \momentVar_4\diffVar_4
          - \prj{\momentVar_1}{\affDrift(\stateVar_1)}  \\
          &\quad- \frac12\bprj{\momentVar_1+\momentVar_2}{\affDiff(\stateVar)(\momentVar_1+\momentVar_2)} \\
          &\quad- \affInt(\stateVar_1)\exp\Big(\ode_{\affDist(\stateVar_1,\cdot)}(\momentVar_1+\momentVar_3) + \momentVar_4\Big) \\
          &\quad+ \affInt(\stateVar_1) 
          + \bprj{\momentVar_1+\momentVar_3}{\affInt(\stateVar_1)\overline{\affDist(\stateVar_1,\cdot)}}
        \end{aligned} \\
        &= \begin{aligned}[t]
          &\frac12\Bprj{\diffVar_2}{\affDiff(\stateVar_1)^\dagger\diffVar_2} + \diffVar_4 \log\Big(\frac{\diffVar_4}{\affInt(\stateVar_1)}\Big) - \diffVar_4 + \affInt(\stateVar_1) \\
          &+ \ode_{\affDist(\stateVar_1,\cdot)}^*\Big(\frac{\diffVar_3+\affInt(\stateVar_1)\overline{\affDist(\stateVar_1,\cdot)}}{\diffVar_4}\Big)
        \end{aligned}
      \end{align*}
    \item
  \end{enumerate}
\end{proof}
