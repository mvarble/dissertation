\begin{proposition}
  \label{proposition:aff-asymptotics}
  Assume $\stateSpace$ is a cone satisfying $\operatorname{span}\stateSpace = \vecSpace$.
  For each $\epsilon > 0$, we have $\momentSpace = \momentSpacee$ and the following identities.
  \begin{equation*}
    \aff^\epsilon(t, \momentVar, \stateVar) = \frac1\epsilon \aff(t, \momentVar, \epsilon\stateVar), \quad 
    \affPart^\epsilon_0(t,\momentVar) = \frac1\epsilon \affPart_0(t,\momentVar), \quad
    \affPart^\epsilon(t,\momentVar) = \affPart(t,\momentVar), 
  \end{equation*}
\end{proposition}
\begin{proof}
  \label{proof:proposition:aff-asymptotics}
  Start by selecting $(\tau, \momentVar) \in \momentSpace$.
  This means that $\momentVar \in \momentSpace(\tau)$ and $\aff(\cdot, \momentVar, \cdot)$ is a solution to $\system(\ode, \momentVar, \tau)$.
  Observe that this implies the following identity for all $\stateVar \in \stateSpace$.
  \begin{gather*}
    \label{eq:aff-dominated-1}
    \frac{\partial}{\partial t} \frac1\epsilon \aff(t, \momentVar, \epsilon\stateVar) = \frac1\epsilon \dot\aff(t, \momentVar, \epsilon\stateVar) = \frac1\epsilon \ode\big(\affPart(t,\momentVar), \epsilon\stateVar\big) = \ode^\epsilon(\affPart(t, \momentVar), \stateVar), \quad t \in [0,\tau] \\
    \frac1\epsilon\aff(0,\momentVar, \epsilon\stateVar) = \frac1\epsilon\prj{\momentVar}{\epsilon\stateVar} = \prj{\momentVar}{\stateVar}
  \end{gather*}
  This means that $\frac1\epsilon\aff(\cdot, \momentVar, \epsilon\cdot)$ is a solution to $\system(\ode^\epsilon, \tau, \momentVar)$.
  By definition, existence of a solution means that $\momentVar \in \momentSpacee(\tau)$, and so $(\tau, \momentVar) \in \momentSpacee$.
  Theorem \ref{theorem:KM} then tells us $\aff^\epsilon(\cdot, \momentVar, \cdot)$ exists and is dominated by the other solution.
  \begin{equation*}
    \aff^\epsilon(t, \momentVar, \stateVar) \leq \frac1\epsilon \aff(t, \momentVar, \epsilon\stateVar), \quad t \in [0,\tau], ~ \stateVar \in \stateSpace
  \end{equation*}

  On the other hand, if we have $(\tau, \momentVar) \in \momentSpacee$, then $\momentVar \in \momentSpacee(\tau)$, and so $\aff^\epsilon(\cdot, \momentVar, \cdot)$ is a solution to $\system(\ode^\epsilon, \tau, \momentVar)$.
  Now, we have the following identity for all $\stateVar \in \stateSpace$,
  \begin{gather*}
    \frac{\partial}{\partial t} \epsilon \aff^\epsilon(t, \momentVar, \stateVar/\epsilon) = \epsilon \dot\aff^\epsilon(t, \momentVar, \stateVar/\epsilon) = \epsilon \ode^\epsilon\big(\affPart^\epsilon(t,\momentVar), \epsilon\stateVar\big) = \ode(\affPart(t, \momentVar), \stateVar), \quad t \in [0,\tau] \\
    \epsilon\aff^\epsilon(0,\momentVar, \stateVar/\epsilon) = \epsilon\prj{\momentVar}{\stateVar/\epsilon} = \prj{\momentVar}{\stateVar},
  \end{gather*}
  and so $\epsilon\aff^\epsilon(\cdot,\momentVar,\cdot)$ is a solution to $\system(\ode, \tau, \momentVar)$.
  Again, we may conclude from this that $(\tau,\momentVar) \in \momentSpace$ and that $\aff^\epsilon(\cdot, \momentVar, \cdot)$ exists and is dominated by the other solution.
  \begin{equation*}
    \label{eq:aff-dominated-2}
    \aff(t, \momentVar, \stateVar) \leq \epsilon \aff^\epsilon(t, \momentVar, \stateVar/\epsilon), \quad t \in [0,\tau], ~ \stateVar \in \stateSpace
  \end{equation*}

  In total, we have now shown that $\momentSpace = \momentSpacee$, and inequalities (\ref{eq:aff-dominated-1}) and (\ref{eq:aff-dominated-2}) indicate to us that the following functions agree.
  \begin{equation*}
    \aff^\epsilon(t, \momentVar, \stateVar) = \frac1\epsilon \aff(t, \momentVar, \stateVar), \quad (t, \momentVar) \in \momentSpace, ~ \stateVar \in \stateSpace
  \end{equation*}
  This means equality of the following affine expressions.
  \begin{align*}
    \affPart^\epsilon_0(t, \momentVar) + \bprj{\affPart^\epsilon(t,\momentVar)}{\stateVar} 
    &= \aff^\epsilon(t, \momentVar, \stateVar) \\
    &= \frac1\epsilon \aff(t, \momentVar, \epsilon\stateVar) \\
    &= \frac1\epsilon \affPart_0(t, \momentVar) + \frac1\epsilon \bprj{\affPart(t, \momentVar)}{\epsilon\stateVar}
    = \frac1\epsilon \affPart_0(t, \momentVar) + \bprj{\affPart(t, \momentVar)}{\stateVar}
  \end{align*}
  Seeing as $\operatorname{span}\stateSpace = \vecSpace$, we may take appropriate linear combinations to show the remaining identities.
  \begin{equation*}
    \affPart^\epsilon_0(t, \momentVar) = \frac1\epsilon\affPart_0(t,\momentVar), \quad
    \affPart^\epsilon_i(t, \momentVar) = \affPart_i(t,\momentVar), \quad i = 1, \ldots, d
  \end{equation*}
\end{proof}
