% specify the state space
We start by specifying a state space on which our stochastic processes live.
Let $\vecSpace$ be a finite-dimensional real vectorspace with inner-product $\prj\cdot\cdot$.
Equip $\vecSpace$ with the canonical topology and Borel algebra from $\prj\cdot\cdot$.
Denote the dimension $d \defeq \dim\vecSpace$ and establish the canonical isometric isomorphism $\vecSpace \equiv \bbR^d$ by specifying an orthonormal basis $\basisVec_1, \ldots, \basisVec_d \in \vecSpace$, so that we may identify components of vectors in $\vecSpace$.
\begin{equation*}
  \vecVar \in \vecSpace \quad\longleftrightarrow\quad \vecVar^i \defeq \prj{\vecVar}{\basisVec_i}, \quad i = 1, \ldots, d
\end{equation*}
Similarly identify any map $f: \bbA \rightarrow \vecSpace$ with component maps $f_1, \ldots, f_d: \bbA \rightarrow \bbR$.
\begin{equation*}
  f: \bbA \rightarrow \vecSpace \quad\longleftrightarrow\quad (f_1,\ldots, f_d): \bbA \rightarrow \bbR^d; ~ f_i(a) = \prj{f(a)}{\basisVec_i}
\end{equation*}
Extend the inner-product symmetrically to a bilinear form on $\vecSpace\oplus\im\vecSpace$.
\begin{equation*}
  \bprj{\vecVar_1+\im w_1}{\vecVar_2 + \im w_2} = \big(\prj{\vecVar_1}{\vecVar_2} - \prj{w_1}{w_2}\big) + \im\big(\prj{\vecVar_1}{w_2} + \prj{w_1}{\vecVar_2}\big)
\end{equation*}
Fix a convex and closed $\stateSpace \subseteq \vecSpace$ satisfying $0 \in \stateSpace$ and $\operatorname{span}\stateSpace = \vecSpace$.
Associate this space with the finite exponentials.
\begin{equation*}
  \boundedSpace \defeq \Big\{ \momentVar \in \vecSpace \oplus \im\vecSpace : \sup_{\stateVar \in \stateSpace} \exp\prj{\Re(\momentVar)}{\stateVar} < \infty \Big\}
\end{equation*}
We may now define the notion of an affine process on $\stateSpace$.

% define an affine process
\begin{definition}
  \label{definition:affine-process}
  For a probability space $(\Omega, \Sigma, \Prb)$ with filtration $\scrF = (\scrF_t)_{t\geq0}$, an affine process $\X$ on $\stateSpace$ is a stochastically continuous, time-homogeneous $(\Prb, \scrF)$-Markov process on $\stateSpace$ in which the bounded complex moments have the following log-affine dependence on the initial state.
  \begin{align}
    \label{eq:affine-definition}
    &\begin{aligned}
      \Exp_{\Prb_\stateVar}\exp\prj{\momentVar}{\X_t} &= \exp\aff(t, \momentVar, \stateVar) \\
      \aff(t, \momentVar, \stateVar) &= \affPart_0(t, \momentVar) + \bprj{\affPart(t, \momentVar)}{\stateVar},  \\
    \end{aligned}
    & t \geq 0, ~ \momentVar \in \boundedSpace
  \end{align}
  Above, we are denoting $(\Prb_\stateVar)_{\stateVar\in\stateSpace}$ the conditional distributions of $\X$ factored through the initial state (see Appendix \ref{appendix:markov} for further specification and notation).
\end{definition}


% clear up our definition with the literature
\begin{remark}
  \label{remark:affine-remarks}
  \begin{enumerate}[label=(\alph*)]
    \item
      See \cite[Remark 2.3]{keller2015} for an argument on how our assumptions on $\stateSpace$ are at no loss of generality; $\stateSpace$ may as well be any nonempty convex set.
    \item
      Note how (\ref{eq:affine-definition}) specifies the characteristic function of each transition kernel of the Markov process $\X$; thus, should an affine process exist for choice of $\aff$, \emph{only} one will exist, up to distribution.
    \item
      See how our notation $(\affPart_0, \affPart)$ differs from that of \cite{keller2015} and other papers, which typically use $(\phi, \affPart)$.
      We choose to do this because affine functions prevail throughout our investigation of affine processes, and we saw this an opportunity to have more cohesive notation of all such affine functions.
      \begin{equation*}
        \affDiff(\stateVar) = \affDiffPart_0 + \sum_{i=1}^d \stateVar^i \affDiffPart_i
      \end{equation*}
    \item
      \label{remark:affine-parts}
      If we have a vector space $\bbA$ and affine map $\alpha: \stateSpace \rightarrow \bbA$ determined by $a_0, \ldots, a_d \in \bbA$ via $\alpha(\stateVar) = a_0 + \sum_{i=1}^d \stateVar^i a_i$, then our linear assumptions $0 \in \stateSpace$ and $\operatorname{span}\stateSpace = \vecSpace$ uniquely determine $a_0, \ldots, a_d \in \bbA$.
      In particular, the map $\aff$ uniquely identifies its parts $\affPart_i: \bbR_+ \times \boundedSpace \rightarrow \bbC$ for $i=0,\ldots,d$.
  \end{enumerate}
\end{remark}


% motivate the importance of \ode, \affDrift, \affDiff, \affJump
While the tuple $(\affPart_0, \affPart)$ in (\ref{eq:affine-definition}) is a simpler object than the distributions $(\Prb_\stateVar)_{\stateVar\in\stateSpace}$, selecting an \emph{admissible} tuple|one $(\affPart_0, \affPart)$ which actually can appear in (\ref{eq:affine-definition})|is seemingly prohibitive.
The following result is incredibly useful at demonstrating primitive affine objects which determine the time-dynamics of $\big(\affPart_0(\cdot,\momentVar), \affPart(\cdot,\momentVar)\big)$ for each $\momentVar \in \momentSpace$.
These implicitly depend on a \emph{truncation function} $\trunc$, which we select as follows.
\begin{equation}
  \label{eq:truncation}
  \trunc: \vecSpace \rightarrow \vecSpace, \quad \trunc(\vecVar) \defeq \left\{\begin{aligned}
    &\vecVar, & |\vecVar| \leq 1 \\
    &0, & |\vecVar| > 1 
  \end{aligned}\right.
\end{equation}
The importance of this function is for local integrability properties|the details of which we defer until Section \ref{affine-processes:jump-diffusions}.

% state the existence of \ode, \affDrift, \affDiff, \affJump
\begin{theorem}
  \label{theorem:affine-regularity}
  Fix an affine process $\X$ on $\stateSpace$.
  There exist $\affDriftPart^\trunc_0, \ldots, \affDriftPart^\trunc_d \in \vecSpace$, $\affDiffPart_0, \ldots, \affDiffPart_d \in \linSpace(\vecSpace)$, and $\affJumpPart_0, \ldots, \affJumpPart_d \in \signedMeasSpace{\scrB(\vecSpace)}$ such that the following maps $L_0,\ldots,L_d: \boundedSpace \rightarrow \bbR$,
  \begin{equation}
    \odePart_i(\momentVar)
    \defeq \bprj{\momentVar}{\affDriftPart_i^\trunc(\stateVar)} + \frac12\bprj{\momentVar}{\affDiffPart_i(\stateVar)\momentVar} + \int_\vecSpace \Big( e^\prj{\momentVar}{\markVar} - 1 - \bprj{\momentVar}{\trunc(\markVar)} \Big) \affJumpPart_i(\stateVar, \rmd\markVar),
  \end{equation}
  determine the dynamics of $(\affPart_0, \affPart)$.
  \begin{equation}
    \left\{\begin{aligned}
      \affPart_0(t,\momentVar) &= \odePart_0\big(\affPart(t, \momentVar)\big) & t \geq 0 \\
      \affPart(t,\momentVar) &= \odePart\big(\affPart(t,\momentVar)\big) & t \geq 0 \\
      \affPart_0(0,\momentVar) &= 0 \\
      \affPart(0, \momentVar) &= \momentVar
    \end{aligned}\right.
  \end{equation}
\end{theorem}

\begin{proof}
  \label{proof:affine-regularity}
  This is simply a restatement of \cite[Theorem 1.5.4]{cuchiero2011}.
\end{proof}

\begin{remark}
  \label{remark:riccati-affine-formulation}
  By Remark \ref{remark:affine-remarks}\ref{remark:affine-parts}, Theorem \ref{theorem:affine-regularity} is equivalent to the existence of maps
  \begin{equation}
    \label{eq:affine-maps}
    \begin{aligned}
      % drift
      \affDrift^\trunc(\stateVar) 
      &\defeq \affDriftPart_0^\trunc + \sum_{i=1}^d \stateVar^i \affDriftPart_i^\trunc, \\
      % diffusion
      \affDiff(\stateVar) 
      &\defeq \affDiffPart_0 + \sum_{i=1}^d \stateVar^i \affDiffPart_i, \\
      % jumps
      \affJump(\stateVar, \rmd\markVar) 
      &\defeq \affJumpPart_0(\rmd\markVar) + \sum_{i=1}^d \stateVar^i \affJumpPart_i(\rmd\markVar) \\
      % ode
      \ode(\momentVar, \stateVar) 
      &= \odePart_0(\momentVar) + \sum_{i=1}^d \stateVar^i \odePart_i(\momentVar) \\
      &\defeq \bprj{\momentVar}{\affDrift^\trunc(\stateVar)} + \frac12 \bprj{\momentVar}{\affDiff(\stateVar)\momentVar} + \int_\vecSpace \Big( e^\prj{\momentVar}{\markVar} - 1 - \bprj{\momentVar}{\trunc(\markVar)} \Big) \affJump(\stateVar, \rmd\markVar) \\
    \end{aligned}
  \end{equation}
  which specify the dynamics of $\aff$.
  \begin{equation}
    \label{eq:riccati-system-bounded}
    \forall \stateVar \in \stateSpace, \qquad \left\{\begin{aligned}
      \dot\aff(t, \momentVar, \stateVar) &= \ode\big(\affPart(t,\momentVar), \stateVar\big) & t \geq 0 \\
      \aff(0,\momentVar, \stateVar) &= \prj{\momentVar}{\stateVar}
    \end{aligned}\right.
  \end{equation}
  Indeed, this is also because differentiability and our initial condition in (\ref{eq:riccati-system-bounded}) are linear.
\end{remark}

\begin{remark}
  In \cite{cuchiero2011} there are immediate results on our functions $\affDrift^\trunc, \affDiff, \affJump$ which are readily apparent in Section \ref{affine-processes:jump-diffusions}, such as, for all $\stateVar \in \stateSpace$, the following are true.
  \begin{equation}
    \begin{gathered}
      \affDiff(\stateVar) \text{ is positive semidefinite} \\
      \int_\vecSpace \big(1 \wedge |\markVar|^2 \big) \affJump(\stateVar, \rmd\markVar) < \infty \\
      \affJump(\stateVar, \{0\}) = 0
    \end{gathered}
  \end{equation}
  \color{gray}
  discuss how other papers try to clearly specify which other conditions on $\affDrift^\trunc, \affDiff, \affJump$ parameterize all admissible $(\affPart_0, \affPart)$, depending on the definition of $\stateSpace$.
\end{remark}


% fix our "base" affine process
Henceforth, we fix $\X$ a c\`adl\`ag affine process with conditional distributions $(\Prb_\stateVar)_{\stateVar\in\stateSpace}$, induced filtration $\scrF=(\scrF_t)_{t\geq0}$, and moment function $\aff$ as in Definition \ref{definition:affine-process}.
Also fix the parameters $(\affDriftPart^\trunc_i, \affDiffPart_i, \affJumpPart_i, \odePart_i)_{i=0}^d$ from Theorem \ref{theorem:affine-regularity} and the associated functions $\affDrift^\trunc, \affDiff, \affJump, \ode$ from Remark \ref{remark:riccati-affine-formulation}.
