% specify the state space
We start by specifying a state space on which our stochastic processes live.
Let $\vecSpace$ be a finite-dimensional real vectorspace with inner-product $\prj\cdot\cdot$.
Equip $\vecSpace$ with the canonical topology and Borel algebra from $\prj\cdot\cdot$.
Denote the dimension $d \defeq \dim\vecSpace$ and establish the canonical isometric isomorphism $\vecSpace \equiv \bbR^d$ by specifying an orthonormal basis $\basisVec_1, \ldots, \basisVec_d \in \vecSpace$, so that we may identify components of vectors in $\vecSpace$.
\begin{equation}
  \vecVar \in \vecSpace \quad\longleftrightarrow\quad \vecVar^i \defeq \prj{\vecVar}{\basisVec_i}, \quad i = 1, \ldots, d
\end{equation}
Similarly identify any map $f: \bbA \rightarrow \vecSpace$ with component functions $f_1,\ldots,f_d: \bbA \rightarrow \bbR$.
Extend the inner-product to a complex bilinear form on $\vecSpace\oplus\im\vecSpace$, linearly and symmetrically.
\begin{equation}
  \bprj{\vecVar_1+\im w_1}{\vecVar_2 + \im w_2} = \big(\prj{\vecVar_1}{\vecVar_2} - \prj{w_1}{w_2}\big) + \im\big(\prj{\vecVar_1}{w_2} + \prj{w_1}{\vecVar_2}\big)
\end{equation}
Fix a convex and closed $\stateSpace \subseteq \vecSpace$ satisfying $0 \in \stateSpace$ and $\operatorname{span}\stateSpace = \vecSpace$.
Associate this space with the finite exponentials.
\begin{equation}
  \boundedSpace \defeq \Big\{ \momentVar \in \vecSpace \oplus \im\vecSpace : \sup_{\stateVar \in \stateSpace} \exp\prj{\Re(\momentVar)}{\stateVar} < \infty \Big\}
\end{equation}
We may now define the notion of an affine process on $\stateSpace$.

% define an affine process
\begin{definition}
  \label{definition:affine-process}
  For a probability space $(\Omega, \Sigma, \Prb)$ with filtration $\scrF = (\scrF_t)_{t\geq0}$, an affine process $\X$ on $\stateSpace$ is a stochastically continuous, time-homogeneous $(\Prb, \scrF)$-Markov process on $\stateSpace$ in which the bounded complex moments have the following log-affine dependence on the initial state.
  \begin{align}
    \label{eq:affine-definition}
    &\begin{aligned}
      \Exp_{\Prb_\stateVar}\exp\prj{\momentVar}{\X_t} &= \exp\aff(t, \momentVar, \stateVar) \\
      \aff(t, \momentVar, \stateVar) &= \affPart_0(t, \momentVar) + \bprj{\affPart(t, \momentVar)}{\stateVar},  \\
    \end{aligned}
    & t \geq 0, ~ \momentVar \in \boundedSpace
  \end{align}
  Above, we are denoting $(\Prb_\stateVar)_{\stateVar\in\stateSpace}$ the conditional distributions of $\X$ factored through the initial state (see Appendix \ref{appendix:markov} for further specification and notation).
\end{definition}


% note that (\affPart_0, \affPart) are uniquely decided from \aff.
\begin{remark}
  \label{remark:affine-remarks}
  \begin{enumerate}[label=(\alph*)]
    \item
      See \cite[Remark 2.3]{keller2015} for an argument on how our assumptions on $\stateSpace$ are at no loss of generality; $\stateSpace$ may as well be any nonempty convex set.
    \item
      Note how (\ref{eq:affine-definition}) specifies the characteristic function of each transition kernel of the Markov process $\X$; thus, should an affine process exist for choice of $\aff$, \emph{only} one will exist, up to distribution.
    \item
      See how our notation $(\affPart_0, \affPart)$ differs from that of \cite{keller2015} and other papers, which typically use $(\phi, \affPart)$.
      We choose to do this because affine functions prevail throughout our investigation of affine processes, and we saw this an opportunity to have more cohesive notation of all such affine functions.
      \begin{equation*}
        \affDiff(\stateVar) = \affDiffPart_0 + \sum_{i=1}^d \stateVar^i \affDiffPart_i
      \end{equation*}
    \item
      \label{remark:affine-parts}
      If we have a vector space $\bbA$ and affine map $\alpha: \stateSpace \rightarrow \bbA$ determined by $a_0, \ldots, a_d \in \bbA$ via $\alpha(\stateVar) = a_0 + \sum_{i=1}^d \stateVar^i a_i$, then our linear assumptions $0 \in \stateSpace$ and $\operatorname{span}\stateSpace = \vecSpace$ uniquely determine $a_0, \ldots, a_d \in \bbA$.
      In particular, the map $\aff$ uniquely identifies its parts $\affPart_i: \bbR_+ \times \boundedSpace \rightarrow \bbC$ for $i=0,\ldots,d$.
  \end{enumerate}
\end{remark}


While the tuple $(\affPart_0, \affPart)$ in (\ref{eq:affine-definition}) is a simpler object than the distributions $(\Prb_\stateVar)_{\stateVar\in\stateSpace}$, selecting an \emph{admissible} tuple|one $(\affPart_0, \affPart)$ which actually can appear in (\ref{eq:affine-definition})|is seemingly prohibitive.
The following result is incredibly useful at demonstrating \emph{parameters} $\affDrift^\trunc, \affDiff, \affJump$ of an affine process, that easily specify admissible pairs $(\affPart_0, \affPart)$.

% Show how there is a map $\ode$
\begin{theorem}
  Fix an affine process $\X$ on $\stateSpace$.
  There exists affine functions $\affDrift^\trunc, \affDiff, \affJump$ of the form,
  \begin{align}
    % drift
    \affDrift^\trunc(\stateVar) 
    &\defeq \affDriftPart^\trunc_0 + \sum_{i=1}^d \stateVar^i \affDriftPart^\trunc_i,
    & \affDriftPart^\trunc_0, \ldots, \affDriftPart^\trunc_d \in \vecSpace \\
    % diffusion
    \affDiff(\stateVar) 
    &\defeq \affDiffPart_0 + \sum_{i=1}^d \stateVar^i \affDiffPart_i,
    & \affDiffPart_0, \ldots, \affDiffPart_d \in \linSpace(\vecSpace) \\
    % jumps
    \affJump(\stateVar, \rmd\markVar) 
    &\defeq \affJumpPart_0(\rmd\markVar) + \sum_{i=1}^d \affJumpPart_i(\rmd\markVar),
    & \affJumpPart_0, \ldots, \affJumpPart_d \in \probSpace{\scrB(\vecSpace)}
  \end{align}
  which induce $\aff$ as follows.
  For each $\momentVar \in \boundedSpace$, we have affine function $\ode(\momentVar, \cdot): \stateSpace \rightarrow \bbR$, defined by
  \begin{gather}
    \ode(\momentVar, \stateVar) 
    \defeq \bprj{\momentVar}{\affDrift^\trunc(\stateVar)} + \frac12\bprj{\momentVar}{\affDiff(\stateVar)\momentVar} + \int_\vecSpace \big( e^\prj{\momentVar}{\markVar} - 1 - \prj{\momentVar}{\trunc(\markVar)} \big) \affJump(\stateVar, \rmd\markVar) \\
    \trunc(\markVar) 
    \defeq \markVar1_{|\markVar|\leq 1}
  \end{gather}
  \begin{gather}
    \forall\stateVar \in \stateSpace \qquad \left\{\begin{aligned}
      \dot\aff(t, \momentVar, \stateVar) &= \ode\big(\affPart(t, \momentVar), \stateVar \big) & t \geq 0 \\
      \aff(0, \momentVar, \stateVar) &= \prj{\momentVar}{\stateVar}
    \end{aligned}\right. \\
  \end{gather}
\end{theorem}


