We now turn our focus to the real moments of $(\Prb, \scrF)$ jump-diffusions and the extension of our L\'evy-Khintchine map $\ode$ to real moments.
\begin{equation*}
  \ode(\momentVar, \stateVar) = \bprj{\momentVar}{\affDrift^\trunc(\stateVar)} + \frac12\bprj{\momentVar}{\affDiff(\stateVar)} + \int_\vecSpace \big( e^\prj{\momentVar}{\markVar} - 1 - \prj{\momentVar}{\trunc(\markVar)} \big) \affJump(\stateVar, \rmd\markVar), \quad \momentVar \in \vecSpace, ~ \stateVar \in \stateSpace
\end{equation*}
The above expression may be infinite, as the final term includes an unbounded integral over a possibly infinite measure.
That said, we find it imperative to denote the following sets of finiteness.
\begin{equation}
  \label{eq:odeSpace-general}
  \odeSpace(\stateVar) \defeq \Big\{ \momentVar \in \vecSpace : \ode(\momentVar, \stateVar) < \infty \Big\}, \qquad
  \odeSpace \defeq \bigcap_{\stateVar \in \stateSpace} \odeSpace(\stateVar)
\end{equation}
The following results will explore the nature of the maps $\ode(\cdot,\stateVar): \odeSpace(\stateVar) \rightarrow \bbR$ for fixed differentiable $\trunc$-characteristics $(\affDrift^\trunc, \affDiff, \affJump)$, where our truncation function $\trunc$ is defined by $\trunc(\markVar) = \markVar 1_{|\markVar|\leq 1}$.
Note that there is no loss of generality in selecting this truncation function, since they all evaluate $\ode$ identically.

\begin{lemma}
  \label{lemma:odeSpace-rationale}
  We have $\momentVar \in \odeSpace$ if and only if each $\stateVar \in \stateSpace$ is such that
  \begin{equation*}
    \int_{|\vecVar|>1} e^\prj{\momentVar}{\vecVar} \affJump(\stateVar, \rmd\markVar) < \infty
  \end{equation*}
\end{lemma}

\begin{lemma}
  \label{lemma:odeSpace-convex}
  $\odeSpace$ is convex.
\end{lemma}

\begin{lemma}
  \label{lemma:ode-differentiable}
  For each $\stateVar \in \stateSpace$, the map $\ode(\cdot,\stateVar)$ is continuously differentiable on $\odeSpace(\stateVar)^\circ$, with derivative $\Der \ode(\cdot,\stateVar): \odeSpace(\stateVar)^\circ \rightarrow \bbL(\vecSpace, \bbR)$ as follows.
  \begin{equation}
    \label{eq:ode-derivative-truncation}
    \Der \ode(\momentVar,\stateVar) w = \Bprj{\affDrift^\trunc(\stateVar) + \affDiff(\stateVar)\momentVar + \int_\vecSpace \big( e^\prj{\momentVar}{\markVar} \markVar - \trunc(\markVar)\big) \affJump(\stateVar,\rmd\markVar)}{w}, \quad \momentVar \in \odeSpace(\stateVar)^\circ
  \end{equation}
\end{lemma}
\begin{proof}
  \label{proof:lemma:ode-differentiable}
  Fix $\stateVar \in \stateSpace$, $\momentVar \in \odeSpace(\stateVar)^\circ$.
  Let $\epsilon > 0$ such that $B(\momentVar, \epsilon) \subseteq \odeSpace(\stateVar)$.
  For all $0 < \delta < \epsilon$ and $i = 1, \ldots, d$, we now have the following identity
  \begin{equation}
    \label{eq:difference-quotient}
    \frac{\ode(\momentVar+\delta \basisVec_i,\stateVar) - \ode(\momentVar,\stateVar)}{\delta}
    = \begin{aligned}[t]
      &\prj{\basisVec_i}{\affDrift^\trunc(\stateVar)} + \prj{\basisVec_i}{\affDiff(\stateVar)\momentVar} + \frac12\prj{\delta \basisVec_i}{\affDiff(\stateVar)\momentVar} \\
      &+ \int_{|\markVar|\leq 1} \frac1\delta\Big( e^\prj{\momentVar+\delta \basisVec_i}{\markVar} - e^\prj{\momentVar}{\markVar} - \prj{\delta \basisVec_i}{\markVar} \Big) \affJump(\stateVar,\rmd\markVar) \\
      &+ \int_{|\markVar| > 1}  \frac1\delta\Big( e^\prj{\momentVar+\delta \basisVec_i}{\markVar} - e^\prj{\momentVar}{\markVar} \Big) \affJump(\stateVar,\rmd\markVar) \\
    \end{aligned}
  \end{equation}
  Evaluating the limit of (\ref{eq:difference-quotient}) as $\delta \rightarrow 0$ is now a matter of exchanging the limit with integration; we will do this by using the dominated convergence theorem.

  For the first integral, Taylor's theorem provides us $\gamma_0, \gamma_1 \in [0,1]$ such that the following hold.
  \begin{gather*}
    e^\prj{\momentVar+\delta \basisVec_i}{\markVar} = 1 + \prj{\momentVar+\delta \basisVec_i}{\markVar} + \frac12 \prj{\momentVar+\delta \basisVec_i}{\markVar}^2 e^{\gamma_0\prj{\momentVar+\delta \basisVec_i}{\markVar}} \\
    e^\prj{\momentVar}{\markVar} = 1 + \prj{\momentVar}{\markVar} + \frac12 \prj{\momentVar}{\markVar}^2 e^{\gamma_1\prj{\momentVar}{\markVar}}
  \end{gather*}
  This shows us that, for all $0 < \delta < \epsilon$ and $|\markVar| \leq 1$,
  \begin{align*}
    \Big| \frac1\delta \Big( e^\prj{\momentVar+\delta \basisVec_i}{\markVar} - e^\prj{\momentVar}{\markVar} - \prj{\delta \basisVec_i}{\markVar} \Big) \Big|
    &= \Big| \frac{1}{2} \prj{\momentVar+\delta \basisVec_i}{\markVar}^2 e^{\gamma_0\prj{\momentVar+\delta \basisVec_i}{\markVar}} + \frac12\prj{\momentVar}{\markVar}^2 e^{\gamma_1\prj{\momentVar}{\markVar}} \Big| \\
    &\leq \Big(\big(|\momentVar| + \epsilon\big)^2 e^{|\momentVar|+\epsilon} \Big) |\markVar|^2.
  \end{align*}
  This dominating function is integrable,
  \begin{equation*}
    \int_{|\markVar|\leq 1} \Big( \big(|\momentVar|+\epsilon\big)^2 e^{|\momentVar|+\epsilon} \Big) |\markVar|^2 \affJump(\stateVar,\rmd\markVar)
    \leq  \Big( \big(|\momentVar|+\epsilon\big)^2 e^{|\momentVar|+\epsilon} \Big) \int_\vecSpace (1 \wedge |\markVar|^2) \affJump(\stateVar,\rmd\markVar) < \infty,
  \end{equation*}
  so we may apply the dominated convergence theorem.
  \begin{align}
    &\lim_{\delta \rightarrow 0} \int_{|\markVar|\leq 1} \frac1\delta \Big( e^\prj{\momentVar+\delta \basisVec_i}{\markVar} - e^\prj{\momentVar}{\markVar} - \prj{\delta \basisVec_i}{\markVar} \Big) \affJump(\stateVar,\rmd\markVar) \notag \\
    &\quad= \int_{|\markVar|\leq 1} \lim_{\delta\rightarrow0} \frac{1}{\delta} \Big( e^\prj{\momentVar+\delta \basisVec_i}{\markVar} - e^\prj{\momentVar}{\markVar} - \prj{\delta \basisVec_i}{\markVar} \Big) \affJump(\stateVar,\rmd\markVar) \notag \\
    \label{eq:dct1}
    &\quad= \int_{|\markVar|\leq1} \Big( e^\prj{\momentVar}{\markVar} \markVar_i - \markVar_i \Big) \affJump(\stateVar,\rmd\markVar)
  \end{align}

  For the second integral, we again use Taylor's theorem to establish for each $0 < \delta < \epsilon/2$, some $\gamma_\delta \in [0,\delta]$ such that
  \begin{equation*}
    e^\prj{\momentVar+\delta \basisVec_i}{\markVar} = e^\prj{\momentVar}{\markVar} + \prj{\delta \basisVec_i}{\markVar} e^\prj{\momentVar+\gamma_\delta \basisVec_i}{\markVar}
  \end{equation*}
  This way, we have the following dominating function.
  \begin{equation*}
    \Big| \frac1\delta \Big( e^\prj{\momentVar+\delta \basisVec_i}{\markVar} - e^\prj{\momentVar}{\markVar} \Big) \Big| 
    %
    \leq \Big|\prj{\basisVec_i}{\markVar} e^\prj{\momentVar+\gamma_\delta \basisVec_i}{\markVar} \Big|
    %
    \leq |\markVar_i| e^{\prj{\momentVar}{\markVar} + \epsilon|\markVar_i|/2}
  \end{equation*}
  The claim is that this dominating function is integrable.
  To see this, first note that because we have the following limit,
  \begin{equation*}
    \lim_{|\markVar|\rightarrow\infty} \frac{|\markVar_i|e^{\prj{\momentVar}{\markVar}+\epsilon|\markVar_i|/2}}{e^{\prj{\momentVar}{\markVar}+2\epsilon|\markVar_i|/3}} = \lim_{|\markVar|\rightarrow\infty} \frac{|\markVar_i|}{e^{\epsilon|\markVar_i|/6}} =  0
  \end{equation*}
  There exists $M > 0$ such that for all $|\markVar| > M$, 
  \begin{equation*}
    |\markVar_i| e^{\prj{\momentVar}{\markVar} + \epsilon|\markVar_i|/2} < e^{\prj{\momentVar}{\markVar} + 2\epsilon|\markVar_i|/3}.
  \end{equation*}
  We now see that
  \begin{align*}
    &\int_{|\markVar| > 1} |\markVar_i| e^{\prj{\momentVar}{\markVar} + \epsilon|\markVar_i|/2} \affJump(\stateVar,\rmd\markVar) \\
    &\quad=\int_{1 < |\markVar|\leq M} |\markVar_i| e^{\prj{\momentVar}{\markVar} + \epsilon|\markVar_i|/2} \affJump(\stateVar,\rmd\markVar) +  \int_{|\markVar| > M} |\markVar_i| e^{\prj{\momentVar}{\markVar} + \epsilon|\markVar_i|/2} \affJump(\stateVar,\rmd\markVar) \\
    &\quad\leq \int_{1 < |\markVar| \leq M} M e^{(|\momentVar|+\epsilon/2)M} \affJump(\stateVar,\rmd\markVar) + \int_{|\markVar| > M} e^{\prj{\momentVar}{\markVar} + 2\epsilon|\markVar_i|/3} \affJump(\stateVar,\rmd\markVar) \\
    &\quad\leq Me^{(|\momentVar|+\epsilon/2)M} \int_\vecSpace (1 \wedge |\markVar|^2) \affJump(\stateVar,\rmd\markVar) + \sum_{\ell=0}^1 \int_{|\markVar|>1} e^\prj{\momentVar+2\epsilon \basisVec_i/3}{\markVar} \affJump(\stateVar,\rmd\markVar) \\
    &\quad < \infty.
  \end{align*}
  We again use the dominated convergence theorem to deduce the following.
  \begin{align}
    &\lim_{\delta \rightarrow 0} \int_{|\markVar|> 1} \frac1\delta \Big( e^\prj{\momentVar+\delta \basisVec_i}{\markVar} - e^\prj{\momentVar}{\markVar} \Big) \affJump(\stateVar,\rmd\markVar) \notag \\
    &\quad= \int_{|\markVar|> 1} \lim_{\delta\rightarrow0} \frac{1}{\delta} \Big( e^\prj{\momentVar+\delta \basisVec_i}{\markVar} - e^\prj{\momentVar}{\markVar}  \Big) \affJump(\stateVar,\rmd\markVar) \notag \\
    \label{eq:dct2}
    &\quad= \int_{|\markVar|>1} e^\prj{\momentVar}{\markVar} \markVar_i  \affJump(\stateVar,\rmd\markVar)
  \end{align}
  Combining equations (\ref{eq:difference-quotient}), (\ref{eq:dct1}), and (\ref{eq:dct2}) now yields our desired identity.
  \begin{equation*}
    \Der_i \ode(\momentVar,\stateVar)
    = \Bprj{\basisVec_i}{\affDrift^\trunc(\stateVar) + \affDiff(\stateVar)\momentVar + \int_\vecSpace \Big( e^\prj{\momentVar}{\markVar} \markVar - \trunc(\markVar) \Big) \affJump(\stateVar,\rmd\markVar)}
  \end{equation*}
  Continuity of $\Der_i \ode(\momentVar, \stateVar)$ for $\momentVar \in \odeSpace(\stateVar)^\circ$ involves very similar dominated convergence theorem arguments as above.
  From here, it is clear that $\ode$ is continuously differentiable with the form in (\ref{eq:ode-derivative-truncation}).
\end{proof}


As we have seen in Lemmas \ref{lemma:special} and \ref{lemma:countable}, if we have local boundedness of certain integrals of a jump kernel $\affJump$, we can leverage these to $(\Prb, \scrF)$ local conditions of the associated jump-diffusion $\X$.
Throughout the remainder of this section, we impose the following uniform-boundedness principle for the kernel $\affJump$.
\begin{equation}
  \label{eq:affJump-regularity}
  \begin{aligned}
    &\int_\vecSpace f(\markVar) \affJump(\stateVar, \rmd\markVar) < \infty \text{ for all } \stateVar \in \stateSpace \\
    &\quad\Longrightarrow\quad \stateVar \mapsto \int_\vecSpace f(\markVar) \affJump(\stateVar, \rmd\markVar) \text{ bounded on compact sets}
  \end{aligned}
\end{equation}
With this assumption, we get some nice results on finite exponential moments of $\X$.

\begin{proposition}
  \label{proposition:LK-special}
  Fix a $(\Prb, \scrF)$ jump-diffusion $\X$ with differential $\trunc$-characteristics $(\affDrift^\trunc, \affDiff, \affJump)$.
  Suppose we have the regularity condition (\ref{eq:affJump-regularity}) above.
  If $0 \in \odeSpace^\circ$, then $\X$ is special.
\end{proposition}
\begin{proof}
  \label{proof:proposition:LK-special}
  If $0 \in \odeSpace^\circ$, then there exists some $\delta > 0$ such that $\closedBall{0}{\delta} \subseteq \odeSpace$.
  Observe the following implication of this fact, for each $\stateVar \in \stateSpace$.
  \begin{align*}
    \int_{\vecSpace} \big|\markVar - \trunc(\markVar)\big| \affJump(\stateVar, \rmd\markVar)
    &= \int_{|\markVar|>1} |\markVar| \affJump(\stateVar, \rmd\markVar) \\
    &\leq \int_{|\markVar|>1} \frac{\sqrt{d}}{\delta} \exp\Big(\frac{\delta|\markVar|}{\sqrt d} \Big) \affJump(\stateVar, \rmd\markVar) \\
    &\leq \frac{\sqrt d}{\delta} \int_{|\markVar|>1} \exp\Big(\max_{i=1}^d \max_{\ell=0}^1 \bprj{(-1)^\ell \delta \basisVec^i}{\markVar} \Big) \affJump(\stateVar,\rmd\markVar) \\
    &\leq \frac{\sqrt d}{\delta} \sum_{i=1}^d \sum_{\ell=0}^1 \int_{|\markVar|>1} \exp\bprj{(-1)^\ell \delta \basisVec^i}{\markVar} \affJump(\stateVar,\rmd\markVar) \\
    &< \infty
  \end{align*}
  Our regularity condition (\ref{eq:affJump-regularity}) now allows us to apply Lemma \ref{lemma:special} to conclude $\X$ is special.
\end{proof}

\begin{proposition}
  \label{proposition:LK-real}
  Fix a $(\Prb, \scrF)$ special jump-diffusion $\X$ with special differential characteristics $(\affDrift, \affDiff, \affJump)$.
  Suppose we have the regularity condition (\ref{eq:affJump-regularity}) above.
  If $\momentVar \in \odeSpace$, then $\exp\prj\momentVar\X$ is special, and 
  \[
    \exp\big( \prj\momentVar\X - \ode(\momentVar, \X) \shortInt \leb \big)
  \]
  is a $(\Prb, \scrF)$ local martingale.
\end{proposition}
\begin{proof}
  \label{proposition:LK-real}
  Using Lemma \ref{lemma:ito} for the function $f_\momentVar(\vecVar) = \exp\prj\momentVar\vecVar$ and its derivative identities as in (\ref{eq:exponential-derivatives}), we get the following.
  \begin{equation}
    \label{eq:LK-ito-pause}
    \exp\prj{\momentVar}{\X_t}
    = \begin{aligned}[t]
      &\exp\prj{\momentVar}{\X_0} + \exp\prj{\momentVar}{\X_t} \Big( \prj{\momentVar}{\affDrift(\X)} + \frac12 \prj{\momentVar}{\affDiff(\X)\momentVar} \Big) \shortInt \leb_t \\
      &+ \Der f_\momentVar(\X_-) \shortInt \X^\rmc + \Big( \exp\prj{\momentVar}{\X_-} \prj{\momentVar}{\id_\vecSpace} \Big) \ast \compensate{\jumpMeas^\X}_t \\
      &+ \exp\prj{\momentVar}{\X_-} \cdot \Big( \exp\prj{\momentVar}{\id_\vecSpace} - 1 - \prj{\momentVar}{\id_\vecSpace} \Big) \ast \jumpMeas^\X
    \end{aligned}
  \end{equation}
  Note that localizing our final term on the sequence $(T_n)_{n\in\bbN}$ of stopping times in (\ref{eq:stopping-times}), we get the following.
  \begin{align*}
    &\Exp_\Prb  \Big| \exp\prj{\momentVar}{\X_-} \Big( \exp\prj{\momentVar}{\id_\vecSpace} - 1 - \prj{\momentVar}{\id_\vecSpace} \Big) \Big| \ast \predproj{\jumpMeas^\X}_{T_n} \\
    &= \Exp_\Prb \int_0^{T_n} \int_\vecSpace \Big| \exp\prj{\momentVar}{\X_s} \Big( \exp\prj{\momentVar}{\markVar} - 1 - \prj{\momentVar}{\markVar} \Big) \Big| \affJump(\X_s, \rmd\markVar) \rmd s \\
    &\leq n \cdot \sup_{|\stateVar|\leq n} \bigg( e^\prj\momentVar\stateVar \int_\vecSpace \big| e^\prj{\momentVar}{\markVar} - 1 - \prj{\momentVar}{\markVar} \big| \affJump(\stateVar, \rmd\markVar) \bigg)
  \end{align*}
  Seeing as $\momentVar \in \odeSpace$, the integral in the above quantity is finite, and so (\ref{eq:affJump-regularity}) gives us finiteness of the supremum.
  Using \cite[Proposition II.1.28]{jacod2003} now allows us to compensate the jump term in (\ref{eq:LK-ito-pause}).
  \begin{equation*}
    \exp\prj{\momentVar}{\X_t}
    = \begin{aligned}[t]
      &\exp\prj{\momentVar}{\X_0} 
      + \Big( \exp\prj{\momentVar}{\X} \cdot \ode(\momentVar, \X) \Big) \shortInt \leb_t 
      + \Der f_\momentVar(\X_-) \shortInt \X^\rmc \\
      &+ \Big( \exp\prj{\momentVar}{\X_-} \prj{\momentVar}{\id_\vecSpace} \Big) \ast \compensate{\jumpMeas^\X}_t 
    \end{aligned}
  \end{equation*}
  This is a representation of $\exp\prj{\momentVar}{\X}$ as an initial term, predictable term of finite variation, and a local martingale.
  Thus, it is a special semimartingale.
  From here, we may perform the product rule on $\exp\big(\prj{\momentVar}{\X} - \ode(\momentVar, \X) \shortInt \leb\big)$ as we did in Proposition \ref{proposition:LK} to show that the process is a local martingale.
\end{proof}

\begin{theorem}
  \label{theorem:LK-exponential-martingale}
  Fix a $(\Prb, \scrF)$ jump-diffusion $\X$ with differential $\trunc$-characteristics $(\affDrift^\trunc, \affDiff, \affJump)$.
  Suppose we have the regularity condition (\ref{eq:affJump-regularity}) above and that $0 \in \odeSpace^\circ$.
  For each $(\Prb, \scrF)$ predictable $H$ of finite-variation with image contained in $\odeSpace^\circ$, the process $\exp(H \shortInt \X)$ is special and 
  \[
    \exp\Big(H \shortInt \X - \ode(H, \X) \shortInt \leb\Big)
  \]
  is a $(\Prb, \scrF)$ local martingale.
\end{theorem}
\begin{proof}
  \label{proof:theorem:LK-exponential-martingale}
  We first note that Proposition \ref{proposition:LK-special} allows us to conclude $\X$ is special.
  Perform It\^o's formula \cite[Theoerem I.4.57]{jacod2003} in addition to its jump-diffusion variant in Lemma \ref{lemma:ito} and various stochastic integral identities \cite[Remarks I.4.36, I.4.37, Theorem I.4.40(d), Proposition II.1.30(b)]{jacod2003}.
  \begin{align}
    \notag
    &\exp\big( H \shortInt \X_t \big) \\
    \notag
    &= \begin{aligned}[t]
      &\exp\big( H \shortInt \X_- \big) \shortInt \big( H \shortInt \X \big)_t + \frac12 \exp\big( H \shortInt \X_- \big) \shortInt \bprj{(H\shortInt \X)^\rmc}{(H\shortInt\X)^\rmc}_t \\
      &+ \sum_{0<s\leq t} \Big( \exp\big(H \shortInt \X_{s-} + \Delta\big(H \shortInt \X\big)_s \big) - \exp\big(H\shortInt \X_{s-}\big) - \exp\big(H \shortInt\X_{s-}\big) \Delta \big(H \shortInt \X\big)_s\Big)
    \end{aligned} \\
    \notag
    &= \begin{aligned}[t]
      &\Big(\exp\big( H \shortInt \X_- \big) \cdot H \Big) \shortInt \X_t + \frac12 \exp\big( H \shortInt \X \big) \bprj{H}{\affDiff(\X)H} \shortInt \leb_t \\
      &+ \exp\big(H\shortInt\X_-\big) \Big( e^{\prj{H}{\id_\vecSpace}} - 1 - \prj{H}{\id_\vecSpace} \big) \ast \jumpMeas^\X_t
    \end{aligned} \\
    \label{eq:LK-exp-ito-pause}
    &= \begin{aligned}[t]
      &\Big(\exp\big( H \shortInt \X \big) \cdot \prj{H}{\affDrift} + \frac12 \exp\big( H \shortInt \X \big) \bprj{H}{\affDiff(\X)H} \Big) \shortInt \leb_t + \Big( \exp\big( H \shortInt \X_- \big) \cdot H \Big) \shortInt \X^\rmc_t \\
      &+ \exp\big( H \shortInt \X_- \big) \prj{H}{\id_\vecSpace} \ast \compensate{\jumpMeas^\X}_t \\
      &+ \exp\big(H\shortInt\X_-\big) \Big( e^{\prj{H}{\id_\vecSpace}} - 1 - \prj{H}{\id_\vecSpace} \Big) \ast \jumpMeas^\X_t
    \end{aligned} 
  \end{align}
  Now, choosing our $(\Prb, \scrF)$ localizing sequence $(T_n)_{n\in\bbN}$ as in \ref{eq:stopping-times}, we have the following bound.
  \begin{align*}
    &\Exp_\Prb \Big| \exp\big(H \shortInt \X_- \big) \Big(e^\prj{H}{\id_\vecSpace} - 1 - \prj{H}{\id_\vecSpace} \Big) \ast \predproj{\jumpMeas^\X}_{T_n} \Big| \\
    &= \Exp_\Prb \int_0^{T_n} \int_\vecSpace \Big| \exp\big(H \shortInt \X_s \big) \Big( e^\prj{H(s)}{\markVar} - 1 - \prj{H(s)}{\markVar} \Big) \Big| \affJump(\X_s, \rmd\markVar) \rmd s \\
    &\leq n \cdot \sup_{|\stateVar| \leq n} \sup_{s \in [0,n]} e^{|\stateVar|\cdot|H(s)|} \int_\vecSpace \big| e^\prj{H(s)}{\markVar} - 1 - \prj{H(s)}{\markVar} \big| \affJump(\stateVar, \rmd\markVar) 
  \end{align*}
  Seeing as $\ode(\cdot, \stateVar)$ is continuously differentiable, it is uniformly bounded on $\odeSpace^\circ$.
  This, along with the fact that $H$ is bounded (it has finite variation) and assumption (\ref{eq:affJump-regularity}) allow us to conclude that the preceding expression is finite.
  Thus, we may compensate the final jump integral in (\ref{eq:LK-exp-ito-pause}).
  \begin{equation}
    \label{eq:jump-diffusion-exponentially-special}
    \exp\big( H \shortInt \X_t \big) 
    = \begin{aligned}[t]
      &\Big(\exp\big( H \shortInt \X \big) \cdot \ode(H, \X) \Big) \shortInt \leb_t + \Big( \exp\big( H \shortInt \X_- \big) \cdot H \Big) \shortInt \X^\rmc_t \\
      &+ \exp\big(H\shortInt\X_-\big) \Big( e^{\prj{H}{\id_\vecSpace}} - 1 \Big) \ast \compensate{\jumpMeas^\X}_t
    \end{aligned} 
  \end{equation}
  The decomposition of $\exp(H \shortInt \X)$ into a predictable finite-variation process and a  local martingale implies that it is special.
  Now, we write $M$ as the local martingale term above, $A = \exp(H \shortInt \X)$, and $B = \exp(-\ode(H, \X) \shortInt \leb)$.
  We now recognize that $B$ is predictable and finite-variation and use \cite[Proposition I.4.49(b)]{jacod2003} to conclude our proof.
  \begin{align*}
    \exp\big( H \shortInt \X_t - \ode(H, \X) \shortInt \leb_t \big) 
    &= A_t B_t \\
    &= A_- \shortInt B_t + B \shortInt A_t \\
    &= \big( A \cdot B \cdot -\ode(H, \X) \big) \shortInt \leb_t + B \shortInt \Big( \big( \exp( H \shortInt \X ) \cdot \ode(H, \X) \big) \shortInt \leb + M \Big)_t \\
    &= \big( A \cdot B \cdot -\ode(H, \X) \big) \shortInt \leb_t + \big( B \cdot A \cdot \ode(H, \X) \big) \shortInt \leb_t + B \shortInt M_t \\
    &= B \shortInt M_t
  \end{align*}
\end{proof}


