We now clearly spell out the assumptions we will need to prove our large deviation principle.
The first of which concerns existence of our affine processes.

As mentioned in the previous section, the base parameters $(\affDrift^\trunc, \affDiff, \affJump)$ parameterize those $(\affDrift^{\trunc,\epsilon}, \affDiff^\epsilon, \affJump^\epsilon)$ for each $\epsilon > 0$.
This means that selecting the base affine process $\X$ immediately imposes the laws $\Xe$ for all other $\epsilon > 0$, \emph{should they exist}.
Generally speaking, there exist functions $(\affDrift^\trunc, \affDiff, \affJump)$ for which there is no jump-diffusion that makes them differential characteristics.
Proposition \ref{proposition:mean-field-asymptotics} indicated that, should distributions $(\Prb_\stateVar)_{\stateVar\in\stateSpace}$ exist for an affine process $\X$ with differential characteristics $(\affDrift^\trunc, \affDiff, \affJump)$, then we can construct distributions $\Prb^{\epsilon_m}_\stateVar$ for each $m \in \bbN$ and $\stateVar \in \stateSpace$ by taking convolutions (recall, $\epsilon_m \defeq 1/m$).
However, we find it important to establish our large deviation principle over a continuously defined family $\epsilon > 0$.
This now motivates the following assumption, which was already implicitly assumed in the previous section.
\begin{assumption}
  The affine parameters $(\affDrift^\trunc, \affDiff, \affJump)$ are chosen so that each $\epsilon \in (0,1]$, affine processes $\Xee$ exist which exhibit the parameters $(\affDrift^{\trunc,\epsilon}, \affDiff^\epsilon, \affJump^\epsilon)$ as in (\ref{eq:characteristic-asymptotics}).
\end{assumption}
\begin{remark}
  \begin{enumerate}[label=(\alph*)]
    \item
      If we wanted to show a large deviation principle over only the family $(\epsilon_m \X^{\epsilon_m})_{m\in\bbN}$, then this assumption is unnecessary.
    \item
      Selecting $(\affDrift^\trunc, \affDiff, \affJump)$ parameterizes each $\ode^\epsilon$, which does specify the systems (\ref{eq:complex-riccati-system}) and (\ref{eq:riccati-system}) associated with complex and real moments, respectively.
      There are already many results which allow one to say that $\Xee$ exists in this scenario.
      For instance, if our state space is $\stateSpace = \bbR^m_+ \times \bbR^n$, then \cite[Proposition 7.4]{duffie2003} provides us the existence we require, so long as the parameters are chosen to be \emph{admissible}.
      Otherwise, one can specify weaker assumptions on $\stateSpace$ at the expense of choosing diffusions ($\affJump(\cdot, \rmd\markVar) = 0$) or pure-jump processes ($\affDiff = 0$) on certain factors of the space; see \cite[Section 2.5, Section 3.4]{cuchiero2011}.
  \end{enumerate}
\end{remark}

Our proof ultimately comes from the parameterizations of our moment functions $\aff^\epsilon$ in the previous section.
These relied on $\stateSpace$ being a cone, so we specifically mention that here.
\begin{assumption}
  The space $\stateSpace$ is a cone and $\operatorname{span}\stateSpace = \vecSpace$.
\end{assumption}
\begin{remark}
  Note that $\operatorname{span}\stateSpace$ is at no loss of generality, per Remark \ref{remark:affine-remarks}.
\end{remark}

Our last assumption is of key importance.
Large deviations are best understood through moment generating functions, which serve no purpose if they are not finite.
For an initial point $\stateVar_0 \in \stateSpace^\circ$, a large deviation principle holds for a family $(\Prb^\epsilon_{\stateVar_0})_{\epsilon>0}$ on the projective limit space (i.e.\ the product topology), so long as $0 \in \odeSpace^\circ$.
However, in order to strengthen this result to the Skorokhod topology---a step which is necessary for interesting asymptotics and an integral-form of our rate function---we need the full strength of $\odeSpace = \vecSpace$.
\begin{assumption}
  We have $\odeSpace = \vecSpace$; equivalently, by Lemma \ref{lemma:odeSpace-rationale},
  \begin{equation*}
    \int_{|\markVar| > 1} e^\prj{\momentVar}{\markVar} \affJump(\stateVar, \rmd\markVar) < \infty, \quad \stateVar \in \stateSpace, ~ \momentVar \in \momentSpace
  \end{equation*}
\end{assumption}
\begin{remark}
  \begin{enumerate}[label=(\alph*)]
    \item
      It is easy to see in the proofs where we assume $\odeSpace = \vecSpace$ versus simply using $0 \in \odeSpace^\circ$.
    \item
      By imposing even the simpler of the two assumptions that $0 \in \odeSpace^\circ$ immediately tells us that all $\Xee$ are special, and so we proceed the rest of the chapter with a presentation in terms of special differential characteristics $(\affDrift, \affDiff, \affJump)$.
  \end{enumerate}
\end{remark}

With this assumption in mind, we state versions of the propositions at the end of Section \ref{affine-processes:real-moments} in our scenario.

\begin{proposition}
  \label{proposition:real-moments-revisit}
  \begin{enumerate}[label=(\alph*)]
    \item
      \label{proposition:real-moments-revisit:big-time-small-moments}
      For each $\tau > 0$, $\momentSpace(\tau)$ is open; in particular, there exists $\delta > 0$ such that $\ball{0}{\delta} \subseteq \momentSpace(\tau)$.
    \item
      \label{proposition:real-moments-revisit:aff-unique}
      For each $\tau > 0$ and $\momentVar \in \momentSpace(\tau)$, $\aff(\cdot,\momentVar,\cdot)$ from Theorem \ref{theorem:KM} is the unique solution to $\system(\ode, \tau, \momentVar)$.
    \item
      \label{proposition:real-moments-revisit:big-moments-small-time}
      For each $M > 0$, there exists $\delta > 0$ such that $\closedBall{0}{M} \subseteq \momentSpace(\delta)$.
    \item
      \label{proposition:real-moments-revisit:aff-uniform-bound}
      For each $M > 0$, there exist $\delta > 0$ and $C_M$ such that the following holds.
      \begin{equation*}
        \big| \aff(t, \momentVar, \stateVar) - \aff(0, \momentVar, \stateVar) \big| \leq C_M \cdot t \cdot \big( 1 + |\stateVar| \big), \quad t \in [0,\delta], ~ \momentVar \in \closedBall{0}{M}, ~ \stateVar \in \stateSpace
      \end{equation*}
  \end{enumerate}
\end{proposition}
\begin{proof}
  \begin{enumerate}[label=(\alph*)]
    \item
      Proposition \ref{proposition:momentSpace-facts}\ref{proposition:momentSpace-facts:1} tells us that $\momentSpace(\tau)$ is open in $\odeSpace^\circ$, which is now $\vecSpace$ by our assumption.
      Also, seeing as $\Exp_{\Prb_\stateVar}\exp\prj{0}{\X_\tau} = 1 < \infty$ for all $\stateVar \in \stateSpace$, $0 \in \momentSpace(\tau)$.
      Openness of $\momentSpace(\tau)$ now grants some $\ball{0}{\delta} \subseteq \momentSpace(\tau)$.
    \item
      Proposition \ref{proposition:momentSpace-facts}\ref{proposition:momentSpace-facts:2} gives us uniqueness of $\aff(\cdot, \momentVar, \cdot)$ as a solution to $\system(\ode, \tau, \momentVar)$ for any $\momentVar \in \momentSpace(\tau) \cap \odeSpace^\circ$, which is now the same thing as $\momentVar \in \momentSpace(\tau)$.
    \item
      Seeing as $\odeSpace = \vecSpace$, we now have $\closedBall{0}{M} \subseteq \odeSpace^\circ$, so Proposition \ref{proposition:compact-ode-existence} gives us $\delta > 0$ such that $\closedBall{0}{M} \subseteq \momentSpace(\delta)$.
    \item
      Fix $M > 0$.
      By part \ref{proposition:real-moments-revisit:big-moments-small-time}, there exists $\delta > 0$ such that $\closedBall{0}{2M} \subseteq \momentSpace(2\delta)$.
      Now, $[0,\delta] \times \closedBall{0}{M}$ is a compact subset of $\momentSpace^\circ$, and so Proposition \ref{proposition:aff-differentiable} gives us our desired $C_M$.
  \end{enumerate}
\end{proof}

