% formulate jump-diffusions in terms of characteristics
As in \cite[Definition III.2.18]{jacod2003}, a $(\Prb, \scrF)$ jump-diffusion $\X$ on state space $(\stateSpace, \scrB(\stateSpace))$ is a $(\Prb, \scrF)$ semimartingale in which the $\trunc$-characteristics $(B^\trunc, A, \predproj{\jumpMeas^\X})$ have the following decompositions,
\begin{equation}
  \label{eq:jump-diffusion-characteristics}
  B^\trunc_t = \int_0^t \affDrift^\trunc(\X_s) \rmd s, \quad 
  A_t = \int_0^t \affDiff(\X_s) \rmd s, \quad
  \predproj{\jumpMeas^\X}(\rmd s,\rmd\markVar) = \affJump(\X_s, \rmd\markVar) \rmd s,
\end{equation}
where the functions have the following properties.
\begin{itemize}
  \item
    $\affDrift^\trunc: \stateSpace \rightarrow \vecSpace$ is Borel measurable, $\affDrift^\trunc \in \scrB(\stateSpace)/\scrB(\vecSpace)$.
  \item
    $\affDiff: \stateSpace \rightarrow \bbL(\vecSpace)$ is Borel measurable, $\affDiff \in \scrB(\stateSpace)/\scrB(\bbL(\vecSpace))$, and $\affDiff(\stateVar)$ is self-adjoint and nonnegative for each $\stateVar \in \stateSpace$.
  \item
    $\affJump: \stateSpace \times \scrB(\vecSpace) \rightarrow [0,\infty]$ is a transition kernel from $\stateSpace$ to $\vecSpace$, and it satisfies the following properties for each $\stateVar \in \stateSpace$.
    \[
      \affJump\big(\stateVar, \{0\}\big) = 0, \quad \int_\vecSpace \big( 1 \wedge |\markVar|^2\big) \affJump(\stateVar, \rmd\markVar) < \infty
    \]
\end{itemize}
In other words, our jump-diffusion $\X$ has the following canonical semimartingale representation (see \cite[Theorem II.2.34]{jacod2003} for definition).
\begin{equation}
  \label{eq:jump-diffusion-canonical}
  \begin{gathered}
    \X = \X_0 + \affDrift^\trunc(\X) \shortInt \leb + \X^\rmc +  \trunc \ast \compensate{\jumpMeas^\X} + (\id_\vecSpace - \trunc) \ast \jumpMeas^\X \\
    \prj{\X^{\rmc,i}}{\X^{\rmc,j}} = \affDiff_{ij}(\X) \shortInt \leb \\
    \predproj{\jumpMeas^\X}(\rmd s, \rmd\markVar) = \affJump(\X_s, \rmd\markVar) \rmd s
  \end{gathered}
\end{equation}

% a quick remark on differences and terminology.
\begin{remark}
  \label{remark:jump-diffusions}
  \begin{enumerate}[label=(\alph*)]
    \item
      Note that we differ slightly from the definition we reference by imposing a time-homogenity formulation.
      There is no loss of generality in doing so, because we may always extend the state to $\bbR_+ \times \stateSpace$ via $\hat\X_t = (t, \X_t)$.
    \item
      Note that (\ref{eq:jump-diffusion-characteristics}) can be written concisely by using the identity $\lambda$ on $\bbR_+$.
      \[
        B^\trunc_t = \affDrift^\trunc(\X) \shortInt \leb_t, \quad
        A_t = \affDiff(\X) \shortInt \leb_t, \quad
        \predproj{\jumpMeas^\X}([0,t],\rmd\markVar) = \affJump(\X, \rmd\markVar) \shortInt \leb_t
      \]
    \item
      \label{remark:characteristic-switch}
      If we have a jump-diffusion with $\trunc$-characteristics in (\ref{eq:jump-diffusion-characteristics}), we call $(\affDrift^\trunc, \affDiff, \affJump)$ the \emph{differential $\trunc$-characteristics}.
      Just as with usual characteristics, there are simple expressions which relate $\affDrift^\trunc$ and $\affDrift^{\hat\trunc}$ between different truncation functions $\trunc, \hat\trunc$.
      \begin{equation}
        \label{eq:affDrift-relationship}
        \affDrift^{\hat\trunc}(\stateVar) = \affDrift^\trunc(\stateVar) + \int_\vecSpace \big(\hat\trunc(\markVar) - \trunc(\markVar) \big) \affJump(\stateVar, \rmd\markVar)
      \end{equation}
    \item
      The conditions on $\affDiff(\stateVar)$ and $\affJump(\stateVar, \rmd\markVar)$ are immediate consequences of (\ref{eq:jump-diffusion-characteristics}).
      For the most general setting, see the corresponding result for \emph{any} semimartingale, in \cite[Proposition II.2.9]{jacod2003}.
  \end{enumerate}
\end{remark}


% a quick example to demonstrate Brownian motion and Poisson.
\begin{example}
  \label{example:jump-diffusion-drivers}
  Fix a probability space $(\Omega, \Sigma, \Prb)$ and filtration $\scrF=(\scrF)_{t\geq0}$. 

  Just as with $(\bbR^d, \scrB(\vecSpace))$, we say that $\W$ is a standard $(\Prb, \scrF)$ Brownian motion on $(\vecSpace, \scrB(\vecSpace))$ if it is a continuous $(\Prb, \scrF)$ martingale with predictable quadratic covariation as follows.
  \begin{equation*}
    \prj{\W^i}{\W^j}_t = \left\{\begin{array}{ll}
      t & i = j \\
      0 & \text{otherwise}
    \end{array}\right.
  \end{equation*}
  It is clear that $\W$ is a $(\Prb, \scrF)$ jump-diffusion with differential $\trunc$-characteristics $(0, \affDiff, 0)$, where $\affDiff(\stateVar) = \id_\vecSpace$ for all $\stateVar \in \stateSpace$.

  Similarly, we say that $\poisRM$ is a standard $(\Prb, \scrF)$ Poisson random measure on $\scrB(\bbR_+ \times \vecSpace)$ if its $(\Prb, \scrF)$ predictable projection is the Lebesgue measure $\predproj{\poisRM}(\rmd s, \rmd\markVar) = \rmd s \otimes \rmd\markVar$ (identifying measures on $\scrB(\bbR^d)$ as those on $\scrB(\vecSpace)$).
  By \cite[Theorem II.4.8]{jacod2003}, this $\poisRM$ is the same as a Poisson point process with Lebesgue intensity, which has infinitely many jumps on any nonempty interval of time.
  The accumulated jumps $\id_\vecSpace \ast \poisRM$ form a $(\Prb, \scrF)$ jump-diffusion with parameters as follows.
  \[
    \affDrift^\trunc(\stateVar) = \int_\vecSpace \trunc(\markVar) \rmd\markVar, \quad
    \affDiff(\stateVar) = 0, \quad
    \affJump(\stateVar, \rmd\markVar) = \rmd \markVar,
  \]
  because we have the following decomposition.
  \begin{align*}
    \id_\vecSpace \ast \poisRM 
    &= \trunc \ast \poisRM + (\id_\vecSpace - \trunc) \ast \poisRM  \\
    &= \trunc \ast \predproj{\poisRM} + \trunc\ast\compensate{\poisRM} + (\id_\vecSpace - \trunc) \ast \poisRM  \\
    &= \affDrift^\trunc \shortInt \leb + \trunc \ast \compensate{\poisRM} + (\id_\vecSpace - \trunc) \ast \poisRM
  \end{align*}
  Note that the infinite activity of $\poisRM$ means that the last term cannot be compensated.

  We will see at the end of this section that these two objects $\W$ and $\poisRM$ are the fundamental building blocks of all jump-diffusions.
\end{example}


The following Lemma will be repeatedly used as a shortcut of It\^o's formula and various identities that always apply with jump-diffusions.

% restate Ito in terms of jump-diffusions
\begin{lemma}
  \label{lemma:ito}
  Let $\X$ be a jump-diffusion with differential $\trunc$-characteristics $(\affDrift^\trunc, \affDiff, \affJump)$ and $f \in \ccpathSpace{2}{\vecSpace}{\bbR}$.
  The composition $f(\X)$ has the following semimartingale representation.
  \begin{equation*}
    f(\X_t)
    = \begin{aligned}[t]
      &f(\X_0) + \Big( \Der f(\X) \cdot \affDrift^\trunc(\X) \Big) \shortInt \leb_t + \frac12 \tr\Big( \Hess f(\X) \circ \affDiff(\X) \Big)  \shortInt \leb_t + \Der f(\X_-) \shortInt \X^\rmc \\
      &+ \Big( \Der f(\X_-) \cdot \trunc \Big) \ast \compensate{\jumpMeas^\X}_t + \Big( f(\X_-+\id_\vecSpace) - f(\X_-) - \Der f(\X_-) \cdot \trunc \Big) \ast \jumpMeas^\X_t
    \end{aligned}
  \end{equation*}
\end{lemma}
\begin{proof}
  \label{proof:lemma:ito}
  Apply It\^o's formula \cite[Theorem I.4.57]{jacod2003} and use the predictable covariation identity in (\ref{eq:jump-diffusion-canonical}) to get the following.
  \begin{align*}
    f(\X_t) 
    &= \begin{aligned}[t]
      &f(\X_0) + \sum_{i=1}^d \Der_i f(\X_-) \shortInt \X^i_t + \frac12\sum_{i,j=1}^d \Der_{ij} f(\X_-) \shortInt \prj{\X^{\rmc,i}}{\X^{\rmc,j}}_t \\
      &+ \sum_{0<s\leq t} \Big( f(\X_s) - f(\X_{s-}) - \sum_{i=1}^d \Der f_i(\X_{s-}) \Delta\X_s \Big)
    \end{aligned} \\
    &= \begin{aligned}[t]
      &f(\X_0) + \Der f(\X_-) \shortInt \X_t + \frac12\sum_{i,j=1}^d \Der_{ij} f(\X_-) \shortInt \big(\affDiff_{ij}(\X) \shortInt \leb\big)_t \\
      &+ \Big( f(\X_-+\id_\vecSpace) - f(\X_-) - \Der f(\X_-) \cdot \id_\vecSpace \Big) \ast \jumpMeas^\X_t
    \end{aligned}
  \end{align*}
  Using the iterated stochastic integral formula \cite[Remark I.4.37]{jacod2003}, we may simplify the above equation to the following.
  \begin{equation*}
    f(\X_t) = \begin{aligned}[t]
      &f(\X_0) + \Der f(\X_-) \shortInt \X_t + \frac12 \tr \Big( \Der_{ij} f(\X_-) \circ \affDiff(\X) \Big) \shortInt \leb_t \\
      &+ \Big( f(\X_-+\id_\vecSpace) - f(\X_-) - \Der f(\X_-) \cdot \id_\vecSpace \Big) \ast \jumpMeas^\X_t
    \end{aligned}
  \end{equation*}
  Now substitute our representation of (\ref{eq:jump-diffusion-canonical}) and repeat the iterated stochastic integral to get the following.
  \begin{align*}
    f(\X_t)
    &= \begin{aligned}[t]
      &f(\X_0) + \Der f(\X_-) \shortInt \big( \X_0 + \affDrift^\trunc(\X) \shortInt \leb + \X^\rmc + \trunc \ast \compensate{\jumpMeas^\X} + (\id_\vecSpace-\trunc) \ast \jumpMeas^\X \big)_t \\
      &+ \frac12 \tr\Big( \Hess f(\X_-) \circ \affDiff(\X) \Big)  \shortInt \leb_t  + \Big( f(\X_-+\id_\vecSpace) - f(\X_-) - \Der f(\X_-) \cdot \id_\vecSpace \Big) \ast \jumpMeas^\X_t
    \end{aligned} \\[1em]
    &= \begin{aligned}[t]
      &f(\X_0) + \Big( \Der f(\X_-) \cdot \affDrift^\trunc(\X) \Big) \shortInt \leb_t + \frac12 \tr\Big( \Hess f(\X_-) \circ \affDiff(\X) \Big)  \shortInt \leb_t + \Der f(\X_-) \shortInt \X^\rmc \\
      &+ \Der f(\X_-) \shortInt \big( \trunc \ast \compensate{\jumpMeas^\X} \big)_t + \Big( f(\X_-+\id_\vecSpace) - f(\X_-) - \Der f(\X_-) \cdot \trunc \Big) \ast \jumpMeas^\X_t
    \end{aligned}
  \end{align*}
  Furthermore, since $\X_- = \X$ on all but a countable amount of jumps, we may rewrite the Lebesgue integrals.
  \begin{equation}
    \label{eq:ito-pause}
    f(\X_t)
    = \begin{aligned}[t]
      &f(\X_0) + \Big( \Der f(\X) \cdot \affDrift^\trunc(\X) \Big) \shortInt \leb_t + \frac12 \tr\Big( \Hess f(\X) \circ \affDiff(\X) \Big)  \shortInt \leb_t + \Der f(\X_-) \shortInt \X^\rmc \\
      &+ \Der f(\X_-) \shortInt \big( \trunc \ast \compensate{\jumpMeas^\X} \big)_t + \Big( f(\X_-+\id_\vecSpace) - f(\X_-) - \Der f(\X_-) \cdot \trunc \Big) \ast \jumpMeas^\X_t
    \end{aligned}
  \end{equation}
  For the remaining equality, we construct localizing sequence $(T_n)_{n\in\bbN}$ of $\scrF$ stopping times,
  \begin{equation}
    \label{eq:stopping-times}
    T_n(\omega) \defeq \inf \big\{ t > 0 : \X_t(\omega) > n \big\} \wedge n, \quad \omega \in \Omega, ~ n \in \bbN,
  \end{equation}
  to see that $\Der f(\X_-)$ is $(\Prb, \scrF)$ locally bounded.
  \begin{equation*}
    \big|\Der f(\X^{T_n}_{s-})\big| \leq \sup_{|\stateVar| \leq n} \big| \Der f(\stateVar) \big|
  \end{equation*}
  Thus, by \cite[Proposition II.1.30]{jacod2003}, we may rewrite the following.
  \begin{equation*}
    \Der f(\X_-) \shortInt \big( \trunc \ast \compensate{\jumpMeas^\X} \big)_t 
    = \big( \Der f(\X_-) \cdot \trunc \big) \ast \compensate{\jumpMeas^\X}_t ,
  \end{equation*}
  which when substituted into (\ref{eq:ito-pause}) gives us our desired identity.
\end{proof}



% express the notion of the generator of X
In the above lemma, the final term in the semimartingale decomposition of $f(\X)$ is typically not able to be compensated into a local martingale.
If we did have local integrability of the following quantity,
\begin{equation*}
  \Big| f(\X_-+\id_\vecSpace) - f(\X_-) + \Der f(\X_-) \cdot \trunc \Big| \ast \predproj{\jumpMeas^\X},
\end{equation*}
then by \cite[Proposition II.1.28]{jacod2003} we could rewrite $f(\X)$ into a canonical special semimartingale decomposition.
\begin{equation}
  \label{eq:jump-diffusion-generator}
  \begin{aligned}
    f(\X_t) &= f(\X_0) + \generator f(\X) \shortInt \leb_t + \Der f(\X_-) \shortInt \X^\rmc + \big( f(\X_- + \id_\vecSpace) - f(\X_-) \big) \ast \compensate{\jumpMeas^\X}_t \\
    \generator f(\stateVar) &\defeq \begin{aligned}[t]
      &\Der f(\stateVar) \cdot \affDrift^\trunc(\stateVar) + \frac12 \tr\Big( \Hess f(\stateVar) \circ \affDiff(\stateVar) \Big) \\
      &\hspace{25mm}+ \int_\vecSpace \Big( f(\stateVar + \markVar) - f(\stateVar) - \Der f(\stateVar) \cdot \trunc(\markVar) \Big) \affJump(\stateVar, \rmd\markVar)
    \end{aligned}
  \end{aligned}
\end{equation}
So long as $f$ is bounded, we can guarantee this special semimartingale property.

% bounded
\begin{proposition}
  \label{proposition:ito-bounded}
  Let $\X$ and $f$ as in Lemma \ref{lemma:ito}, and further impose $f$ is bounded.
  Then the composition $f(\X)$ is a special semimartingale with the decomposition as in (\ref{eq:jump-diffusion-generator}).
\end{proposition}
\begin{proof}
  \label{proof:proposition:ito-bounded}
  Seeing as $f$ is bounded, \cite[Lemma I.4.24]{jacod2003} tells us that $f(\X)$ is a special semimartingale.
  By \cite[Proposition I.4.23]{jacod2003}, it is then the case that the following term is locally integrable.
  \[
    \Big( f(\X_-+\id_\vecSpace) - f(\X_-) - \Der f(\X_-) \cdot \trunc \Big) \ast \jumpMeas^\X_t
  \]
  By our discussion above, this suffices to conclude (\ref{eq:jump-diffusion-generator}).
\end{proof}


This operator $\generator$ in (\ref{eq:jump-diffusion-generator}) gives a nice closed from for suitable $f(\X)$, and so we reserve it the term of \emph{generator} associated with $\X$.
Note that we do not mark dependence on $\trunc$, as any other truncation function $\hat\trunc$ will produce the same operator; see Remark \ref{remark:jump-diffusions}\ref{remark:characteristic-switch} and note that the displacement from $\affDrift^\trunc$ and $\affDrift^{\hat\trunc}$ would be the same as that in the integral term.
One particular setting in which this result is useful is establishing a L\'evy-Khintchine formula for jump-diffusions.

% Levy-Khintchine
\begin{proposition}
  \label{proposition:LK}
  Fix a jump-diffusion $\X$ with differential $\trunc$-characteristics $(\affDrift^\trunc, \affDiff, \affJump)$.
  Then, for each $\momentVar \in \im\vecSpace$, the process $\exp\big(\prj{\momentVar}{\X} - \ode(\momentVar, \X) \shortInt \leb\big)$ is a complex-valued $(\Prb, \scrF)$ local martingale, where $\ode: \im\vecSpace \times \stateSpace \rightarrow \bbR$ is the associaed \emph{L\'evy-Khintchine map}.
  \begin{equation*}
    \ode(\momentVar, \stateVar) = \bprj{\momentVar}{\affDrift^\trunc(\stateVar)} + \frac12\bprj{\momentVar}{\affDiff(\stateVar)} + \int_\vecSpace \big( e^\prj{\momentVar}{\markVar} - 1 - \prj{\momentVar}{\trunc(\markVar)} \big) \affJump(\stateVar, \rmd\markVar),
  \end{equation*}
\end{proposition}
\begin{proof}
  \label{proof:proposition:LK}
  For a fixed $\momentVar \in \im\vecSpace$, note that the map $f_\momentVar$, defined by $f_\momentVar(\vecVar) = \exp\prj{\momentVar}{\vecVar}$ is bounded.
  Thus, by Proposition \ref{proposition:ito-bounded}, we have 
  \[
    f_\momentVar(\X_t) = f_\momentVar(\X_0) +  \generator f_\momentVar(\X) \shortInt \leb_t + M_t,
  \]
  where $M$ is a $(\Prb, \scrF)$ local martingale.
  Observe that the partial derivatives of $f$ are as follows,
  \begin{equation}
    \label{eq:exponential-derivatives}
    \Der_i f_\momentVar(\stateVar) = f_\momentVar(\stateVar) \momentVar_i, \qquad \Der_{ij} f_\momentVar(\stateVar) = f_\momentVar(\stateVar) \momentVar_i\momentVar_j,
  \end{equation}
  so we have the following equation.
  \begin{align*}
    \generator f_\momentVar(\stateVar)
    &=\begin{aligned}[t]
      &\Der f_\momentVar(\stateVar) \cdot \affDrift^\trunc(\stateVar) + \frac12 \tr\Big( \Hess f_\momentVar(\stateVar) \circ \affDiff(\stateVar) \Big) \\
      &+ \int_\vecSpace \Big( f_\momentVar(\stateVar + \markVar) - f_\momentVar(\stateVar) - \Der f(\stateVar) \cdot \trunc(\markVar) \Big) \affJump(\stateVar, \rmd\markVar)
    \end{aligned} \\
    &= f_\momentVar(\stateVar) \bprj{\momentVar}{\affDrift^\trunc(\stateVar)} + \frac12 f_\momentVar(\stateVar) \bprj{\momentVar}{\affDiff(\stateVar)\momentVar} +  f_\momentVar(\stateVar) \int_\vecSpace \Big( f_\momentVar(\markVar) - 1 - \bprj{\momentVar}{\trunc(\markVar)} \Big) \affJump(\stateVar, \rmd\markVar) \\
    &= f_\momentVar(\stateVar) \cdot \ode(\momentVar, \stateVar)
  \end{align*}
  Denoting $A = f_\momentVar(\X) = \exp\prj\momentVar\X$ and $B = \exp\big(-\ode(\momentVar, \X) \shortInt \leb\big)$, we now use the fact that $B$ is $\scrF$ predictable and of finite-variation, so \cite[Proposition I.4.49(b)]{jacod2003} gives us the following.
  \begin{align*}
    &\exp\Big(\prj{\momentVar}{\X} - \ode(\momentVar, \X) \shortInt \leb \Big) \\
    &= A_t B_t \\
    &= A_0 B_0 + A_- \shortInt B_t + B \shortInt A_t \\
    &= \exp\prj{\momentVar}{\X_0} + A_- \shortInt \Big(\big( -B \cdot \ode(\momentVar, \X) \big)\shortInt \leb \Big)_t  + B \shortInt \Big( f_\momentVar(\X_0) + \generator f_\momentVar(\X) \shortInt \leb + M \Big)_t \\
    &= \exp\prj{\momentVar}{\X_0} - \Big( A \cdot B \cdot \ode(\momentVar, \X) \Big) \shortInt \leb_t + \Big( B \cdot f_\momentVar(\X) \cdot \ode(\momentVar, \X) \Big) \shortInt \leb_t + B \shortInt M_t \\
    &= \exp\prj{\momentVar}{\X_0} + B\shortInt M_t
  \end{align*}
  This identity and \cite[Remark I.4.34(b)]{jacod2003} concludes the proof.
\end{proof}


% manifest equivalence
It turns out that each of the preceding results is sufficient in characterizing a semimartingale $\X$ as a jump-diffusion.

\begin{theorem}
  \label{theorem:jump-diffusion-characterizations}
  The following statements are equivalent for a stochastic process $\X$ on state space $(\stateSpace, \scrB(\stateSpace))$.
  \begin{enumerate}[label=(\alph*)]
    \item
      $\X$ is a $(\Prb, \scrF)$ jump-diffusion with differential $\trunc$-characteristics $(\affDrift^\trunc, \affDiff, \affJump)$.
    \item
      For each bounded $f \in \ccpathSpace{2}{\vecSpace}{\bbR}$, the process $f(\X_t) - \generator f(\X_t) \shortInt \leb_t$ is a $(\Prb, \scrF)$ local martingale, where
      \[
        \generator f(\stateVar) \defeq \Der f(\stateVar) \cdot \affDrift^\trunc(\stateVar) + \frac12 \tr\Big( \Hess f(\stateVar) \circ \affDiff(\stateVar) \Big) + \int_\vecSpace \Big( f(\stateVar + \markVar) - f(\stateVar) - \Der f(\stateVar) \cdot \trunc(\markVar) \Big) \affJump(\stateVar, \rmd\markVar)
      \]
    \item
      For each $\momentVar \in \im\vecSpace$, the process $\exp\Big(\prj{\momentVar}{\X} - \ode(\momentVar, \X) \shortInt \leb \Big)$ is a $(\Prb, \scrF)$ local martingale, where $\ode$ is our L\'evy-Khintchine map.
      \[
        \ode(\momentVar, \stateVar) = \bprj{\momentVar}{\affDrift^\trunc(\stateVar)} + \frac12\bprj{\momentVar}{\affDiff(\stateVar)} + \int_\vecSpace \big( e^\prj{\momentVar}{\markVar} - 1 - \prj{\momentVar}{\trunc(\markVar)} \big) \affJump(\stateVar, \rmd\markVar),
      \]
    \item
      Denoting $(\Prb_\stateVar)_{\stateVar\in\stateSpace}$ the $\Prb$-conditional distributions of $\X$ factored through the initial state $\X_0$ and selecting Borel functions $\affDiffSqrt, \thinner$ to satisfy,
      \begin{equation*}
        \label{eq:sde-terms}
        \begin{aligned}
          \affDiffSqrt: \stateSpace \rightarrow \bbL(\vecSpace) && \affDiffSqrt\affDiffSqrt^*(\stateVar) &= \affDiff(\stateVar) \\
          \thinner: \stateSpace \times \vecSpace \rightarrow \vecSpace && \affJump(\stateVar, \Gamma) &= \int 1_\Gamma\big( \thinner(\stateVar, \markVar) \big) \rmd\markVar
        \end{aligned}
      \end{equation*}
      each $\Prb_\stateVar$ is a solution to the equation associated with a standard Brownian motion $\W$ and Poisson random measure $\poisRM$, where $\trunc' = \id_\vecSpace - \trunc$.
      \begin{equation*}
        \X_t = \stateVar + \affDrift^\trunc(\X) \shortInt \leb_t + \sigma(\X_-) \shortInt \W_t + \big( \trunc \circ \thinner(\X_-, \id_\vecSpace) \big) \ast \compensate{\poisRM}_t + \big( \trunc' \circ \thinner(\X_-, \id_\vecSpace) \big) \ast \poisRM_t
      \end{equation*}
  \end{enumerate}
\end{theorem}
\begin{proof}
  \label{proof:theorem:jump-diffusion-characterizations}
  This is simply restating \cite[Theorems II.2.42, II.2.49, and III.2.26]{jacod2003} in terms of our identities from the previous propositions and lemmas.
  The choice of standard intensity $\rmd t \otimes \rmd\markVar$ for the Poisson random measure is such that the jump factor $\rmd\markVar$ satisfies the atomless and infinite properties in \cite[Remark III.2.28(3)]{jacod2003}.
\end{proof}

