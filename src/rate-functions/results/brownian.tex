\begin{example}[Brownian motion]
  \label{example:brownian}
  Applying Mogulskii's theorem when our increment distribution $\affDist$ is $\operatorname{Normal}(0,\id_\vecSpace)$, the integral in our rate function in (\ref{eq:rf-mogulskii}) becomes the following.
  \begin{equation}
    \label{eq:rf-brownian}
    \int_0^\infty \frac12\big|\dot\pathVar(t)\big|^2 \rmd t
  \end{equation}
  Furthermore, for a Brownian motion $\W$, the process $\sqrt\epsilon\W$ ends up being exponentially equivalent to $Y^\epsilon$,
  \begin{equation*}
    \limsup_{\epsilon\rightarrow0} \epsilon\log \Prb\big( |\sqrt\epsilon\W - Y^\epsilon| \geq \delta \big) = -\infty,
  \end{equation*}
  which makes the family $\sqrt\epsilon\W$ satisfy the large deviation principle with rate function (\ref{eq:rf-brownian}); this result is known as Schilder's theorem (see \cite[Theorem 5.2.3]{dembo2010}).

  Note that $(\sqrt\epsilon\W)_{\epsilon>0}$ is a family of affine processes covered Theorem \ref{theorem:ldp-integral}.
  We have $\Xe = \sqrt\epsilon\W$, where the base process $\X$ has special differential characteristics $(0, \id_\vecSpace, 0)$.
  The easiest way to see this is by considering Theorem \ref{theorem:regime-dynamics} with initial state $\stateVar = 0$.
  Our theorem also immediately resolves (\ref{eq:rf-integral}) the same rate function.
  \begin{equation*}
    \ode^*(\dot\stateVar, \stateVar)
    = \sup_{\momentVar \in \vecSpace} \bigg( \prj{\momentVar}{\dot\stateVar} - \frac12\prj{\momentVar}{\id_\vecSpace\cdot \momentVar} \bigg) = \frac12 |\dot\stateVar|
  \end{equation*}
\end{example}
