\begin{color}{gray}
  \begin{enumerate}
    \item
      Discuss how contraction maps or an alternative proof can deal with the \textbf{overdetermined} nature, at the expense of a nastier rate function.
      \begin{enumerate}
        \item
          \textbf{determine-continuous-noise flavor.}
          Normal term gets messy
        \item
          \textbf{determine-arrivals flavor.}
          Poisson and jump-term-denominator gets messy
        \item
          \textbf{determine-jumps flavor.}
          This is the one we have already presented; the jump-term-numerator gets messy.
      \end{enumerate}
    \item
      Discuss how the deviations of $\X$ from the dynamical system $\X = \affDrift(\X) \shortInt \leb$ are imposed from \emph{continuous deviations} $\X^\rmc$ and \emph{discontinuous deviations} $\id_\vecSpace \ast \compensate{\jumpMeas^\X}$.
    \item
      Discuss how four quantities $\X, \X^\rmc, \N^\X, V^\X$ heuristically relate in simple infinitesimal equality $(\X, \X^\rmc, \N^\X, V^\X) \approx (\pathVar, \omega, \eta, \gamma)$.
      \begin{align*}
        \dot\pathVar(t) = \affDrift\big(\pathVar(t)\big) + \dot\omega(t) + \dot\eta(t) \cdot \dot\gamma(t)
      \end{align*}
    \item
      Each of the primitive deviations have a simple analogy, when we think of infinitesimals.
      \begin{align*}
        &\dot\omega(t)  & \text{\emph{normal deviations of covariance }} \affDiff\big(\pathVar(t)\big) \\
        &\dot\eta(t) & \text{\emph{Poisson deviations of rate }} \affInt\big(\pathVar(t)\big) \\
        &\dot\gamma(t) & \text{\emph{jump deviations from }} \affDist\big(\pathVar(t), \rmd\markVar\big) \\
        \pathVar(t) &=\affDrift\big(\pathVar(t)\big) + \dot\omega(t) + \dot\eta(t)\cdot\dot\gamma(t) & \text{\emph{all combined deviations}}
      \end{align*}
    \item
      Think of results from first section in this regard.
      \begin{enumerate}
        \item
          birth is  $\dot\pathVar(t) = \dot\eta(t)$, so we only need $\pathVar \approx \X$.
        \item
          diffusion is $\dot\pathVar(t) = \affDrift\big(\pathVar(t)\big) + \dot\omega(t)$, and so we only need $\pathVar\approx\X$ and rate function includes $\pathVar(t) - \affDrift\big(\pathVar(t)\big)$ where $\dot\omega$ is.
        \item
          compound Poisson is $\dot\pathVar(t) = \dot\eta(t) \cdot \dot\gamma(t)$, so we choose one of the following pairs $(\pathVar, \eta) \approx (\X, \N^\X)$, $(\pathVar, \gamma) \approx (\X, V^\X)$, or $(\eta, \gamma) \approx (\N^\X, V^\X)$.
        \item
          compound linear Hawkes is $\dot\pathVar(t) = \affDrift\big(\pathVar(t)\big) + \dot\eta(t) \cdot \dot\gamma(t)$, so we can choose $(\pathVar, \eta) \approx (\X, \N^\X)$ or $(\pathVar, \dot\gamma) \approx (\X, V^\X)$.
      \end{enumerate}
  \end{enumerate}
\end{color}
