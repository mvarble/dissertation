% intuitively describe "multiple sources" of randomness
The previous result did not include any examples in which the jump distribution was non-degenerate.
Intuitively speaking, there is no way to naturally transform a Poisson process to a \emph{compound-}Poisson process, for the distribution of the jumps introduces a new source of randomness.
This intuition coincides with our difficulties in evaluating our rate function (\ref{eq:rf-integral}) for such processes.

% show technical difficulties in rate function
Consider the simple example of a compound-Poisson process $\X$ driven by standard Poisson process $\N$ and independent jumps $(V_k)_{k\in\bbN}$ distributing with common distribution $\affDist$ satisfying the light-tails condition (\ref{eq:cumulant-generating-function}).
\[
  \X_t = \sum_{k=1}^{\N_t} \V_k
\]
This process has special differential characteristics $(\affDrift, \affDiff, \affJump)$ as below,
\[
  \affDrift(\stateVar) = \overline\affDist, \quad
  \affDiff(\stateVar) = 0, \quad
  \affJump(\stateVar, \rmd\markVar) = \affDist(\rmd\markVar),
\]
where $\overline\affDist \defeq \int_\vecSpace \markVar \affDist(\rmd\markVar)$ denotes the mean of $\affDist$.
Associating an asymptotic family $(\Xe)_{\epsilon>0}$ with this base process $\X$ will result in the large deviation principle of Theorem \ref{theorem:ldp-integral}, where the rate function (\ref{eq:rf-integral}) will involve the following expression,
\begin{align*}
  \ode^*(\diffVar, \stateVar) 
  &= \sup_{\momentVar\in\vecSpace} \bigg( \prj{\momentVar}{\diffVar} - \prj{\momentVar}{\overline\affDist} - \int_\vecSpace \big( e^\prj{\momentVar}{\markVar} - 1 - \prj{\momentVar}{\markVar} \big) \affDist(\rmd\markVar) \bigg) \\
  &= \sup_{\momentVar\in\vecSpace} \bigg( \prj{\momentVar}{\diffVar} - e^{\ode_\affDist(\momentVar)} + 1 \bigg),
\end{align*}
where we recall $\ode_\affDist$ is the cumulant generating function associated with $\affDist$.
The arbitrary nature of this function means that resolving even a semi-closed form for the above expression is a difficult task.

% discuss how problem was "underdetermined" and can be resolved with coupling
Our expression $\ode^*$ is determined by the special differential characteristics $(\affDrift, \affDiff, \affJump)$ associated with $\X$, which---if we are familiar with the theory of semimartingales---serve as the predictable projections of a semimartingale.
This somehow suggests to us that $\ode^*$ is insufficient in understanding the deviations of $\X$, since the jump times of $\N$ are totally inaccessible.
This raises the question of if somehow coupling $(\X, \N)$ will provide us \emph{more} information.
From the technical perspective of $\sigma$-algebras, the answer is \emph{no}, since $\X$ determines $\N$.
However, as moot of a discussion as this is from a technical perspective, it turns out to head us in the right direction.

We will see in the below examples that coupling $(\X, \N)$ will give us semi-closed forms for our rate function.
For illustrative purposes, these examples will again include heuristics on how to prove a principle and derive the rate function without Theorem \ref{theorem:ldp-integral}.
However, these arguments are no longer backed by results in literature, for we are now entering uncharted territory.
We reiterate that these results need not extend past heuristics, for Theorem \ref{theorem:ldp-integral} already provides us the principle.

% compound-poisson example
\begin{example}[Compound-Poisson]
\end{example}


% compound-hawkes example
\begin{example}[Compound-Hawkes]
\end{example}

