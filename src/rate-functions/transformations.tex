% discuss how transforms + measure changes allow for state-dependent processes
While Mogulskii's theorem specifies that the processes $(Y^\epsilon)_{\epsilon>0}$---by design---have independent increments, we may use transformation arguments to produce large deviation principles for families of processes with state-dependent increments.
The two ways of leveraging this are via the contraction principle or measure-change arguments.

% discuss transforms
The contraction principle states that mapping the quantities $F(Y^\epsilon)$ via a continuous map $F$ produces a large deviation principle for the family $(\push{\Prb^\epsilon}{F})_{\epsilon>0}$ of measures $\push{\Prb^\epsilon}{F}$ associated with these respective quantities $F(Y^\epsilon)$ (see \cite[Theorem 4.2.1]{dembo2010}).
Seeing as this section serves as a survey for intuition on rate functions, we will digress from discussing the specifics of continuity of $F$ on restricted spaces and/or exponentially equivalent families in our example below.

% introduce drift to Brownian motion
\begin{example}[Diffusions]
  \label{example:diffusion}
  We can leverage Example \ref{example:brownian} to a family of processes $(\Xe)_{\epsilon>0}$,
  \begin{equation*}
    \Xe = \stateVar + \affDrift(\Xe) \shortInt \leb + \sqrt\epsilon \shortInt \W,
  \end{equation*}
  where the drift $\affDrift: \vecSpace \rightarrow \vecSpace$ is bounded and Lipschitz.
  Having a map $F_\affDrift$ which implicitly solves the equation,
  \begin{equation}
    \label{eq:F-brownian-drift}
    F_\affDrift(\omega) = \pathVar, \quad \pathVar(t) = \stateVar + \affDrift(\pathVar) \shortInt \leb_t + \omega_t,
  \end{equation}
  will make $F_\affDrift(\sqrt\epsilon\W) = \Xe$ for each $\epsilon > 0$, so the contraction principle states that the distributions of $(\Xe)_{\epsilon>0}$ satisfy a large deviation principle in which the rate function $\rf_\affDrift$ is derived from that $\rf_\W$ from Example \ref{example:brownian}.
  \begin{align*}
    \rf_\affDrift(\pathVar) &\defeq \inf\Big\{ \rf_\W(\omega) : F_\affDrift(\omega) = \pathVar \Big\},\\
    \rf_\W(\omega) &\defeq \left\{\begin{array}{ll}
      \displaystyle\int_0^\infty \frac12\big|\dot\omega(t)\big|^2 \rmd t, & \omega(0) = 0, ~ \omega \in \acpathSpace{[0,\infty)}{\vecSpace}, \\
      \infty, & \text{otherwise}
    \end{array}\right.
  \end{align*}
  When $F_\affDrift(\omega) = \pathVar$, equation (\ref{eq:F-brownian-drift}) tells us $\pathVar(0) = \stateVar$ and $\dot\omega = \dot\pathVar - \affDrift(\pathVar)$, and so we have the following.
  \begin{equation*}
    \rf_\affDrift(\pathVar) = \left\{\begin{array}{ll}
      \displaystyle \int_0^\infty \frac12 \Big|\dot\pathVar - \affDrift\big(\pathVar(t)\big)\Big| \rmd t, & \pathVar(0) = \stateVar, ~ \pathVar \in \acpathSpace{[0,\infty)}{\vecSpace}, \\
      \infty, & \text{otherwise}
    \end{array}\right.
  \end{equation*}
  Similarly, we may introduce a bounded, Lipschitz diffusion $\affDiff = \affDiffSqrt\affDiffSqrt^*: \vecSpace \rightarrow \bbL(\vecSpace)$ in which each $\affDiff(\stateVar)$ is invertible, so that the dynamics become as follows.
  \begin{equation}
    \label{eq:F-brownian-drift-diff}
    \Xe = \stateVar + \affDrift(\Xe) \shortInt \leb + \sqrt\epsilon \affDiffSqrt(\Xe) \shortInt \W,
  \end{equation}
  Having a map $F_{\affDrift,\affDiff}$ which implicitly solves the equation,
  \begin{equation*}
    F_{\affDrift,\affDiff}(\omega) = \pathVar, \quad \pathVar(t) = \affDrift(\pathVar) \shortInt \leb_t + \affDiffSqrt(\pathVar) \shortInt \omega_t,
  \end{equation*}
  will allow us to repeat the above argument to get a large deviation principle for $(\Xe)_{\epsilon>0}$ with rate function $\rf_{\affDrift,\affDiff}$; when $\pathVar(0) = \stateVar$, $\pathVar \in \acpathSpace{[0,\infty)}{\vecSpace}$, we get the following.
  \begin{equation*}
    \rf_{\affDrift,\affDiff}(\pathVar) = \int_0^\infty \frac12 \bprj{\Big(\dot\pathVar(t)-\affDrift\big(\pathVar(t)\big)\Big)}{\affDiff\big(\pathVar(t)\big)^{-1}\Big(\dot\pathVar(t)-\affDrift\big(\pathVar(t)\big)\Big)} \rmd t
  \end{equation*}

  The true details of this result, attributed to Freidlin-Wentzel \cite[Theorems 5.6.3 and 5.6.7]{dembo2010}, are rather complicated, and the above argument is just a heuristic.
  Also, note that this result does not apply to the general class of affine diffusions, as $\affDrift, \affDiff$ are generally not bounded or Lipschitz, and $\affDiff$ need not be invertible.
  However, \cite{kang2014}---a paper which inspires parts of our proof---first proved that affine (jump-)diffusions with special differential characteristics $(\affDrift, \affDiff, 0)$ satisfy a large deviation principle with rate function similar to that above.
  Our rate function (\ref{eq:rf-integral}) from Theorem \ref{theorem:ldp-integral} immediately resolves an identical representation.
  \begin{align*}
    \ode^*(\diffVar, \stateVar)
    &= \sup_{\momentVar \in \vecSpace} \bigg( \prj{\momentVar}{\diffVar} - \bprj{\momentVar}{\affDrift(\stateVar)} - \frac12\bprj{\momentVar}{\affDiff(\stateVar)\momentVar} \bigg) \\
    &= \left\{\begin{array}{ll}
      \displaystyle \frac12\Bprj{\big(\diffVar-\affDrift(\stateVar)\big)}{\affDiff(\stateVar)^\dagger\big(\diffVar-\affDrift(\stateVar)\big)}, & \diffVar - \affDrift(\stateVar) \in \operatorname{image}\affDiff(\stateVar), \\
      \infty, & \text{otherwise}
    \end{array}\right.
  \end{align*}
  Above, $a^\dagger \in \bbL(\vecSpace)$ denotes the pseudoinverse of $a \in \bbL(\vecSpace)$.
\end{example}


% discuss measure changes
The above result leveraged a large deviation principle for Brownian motions $(\sqrt\epsilon\W)_{\epsilon>0}$ to one on state-dependent diffusions $(\Xe)_{\epsilon>0}$ via mappings $\Xe = F(\sqrt\epsilon\W)$.
The analogue of the Brownian motion $\W$ for jump processes---in the sense of homogeneous independent-increments---is the Poisson process $\N$.
To introduce state-dependence to our sequence $(\epsilon\N_{\cdot/\epsilon})_{\epsilon>0}$ from Example \ref{example:poisson}, we may perform a measure-change argument.

% introduce intensity to Poisson
\begin{example}[Continuous-branching/Hawkes]
  Consider our sequence $(\epsilon\N_{\cdot/\epsilon})_{\epsilon>0}$ derived from a Poisson process $\N$, as in Example \ref{example:poisson}.
  Denoting $\Prb^\epsilon$ the distribution of each $J^\epsilon \defeq \epsilon\N_{\cdot/\epsilon}$, we may construct a measure $\mgPrb^\epsilon \sim \Prb^\epsilon$ in which $\frac1\epsilon J^\epsilon$ has $\mgPrb^\epsilon$ intensity $\frac1\epsilon\affInt(J^\epsilon)$ for affine function $\affInt$.
  \begin{equation*}
    \affInt: \bbR_+ \rightarrow \bbR_+, \quad \affInt(\stateVar) = \affIntPart_0 + \affIntPart_1 \stateVar
  \end{equation*}
  The martingale that induces this change of measure is familiar from Theorem \ref{theorem:LK-exponential-martingale}, selecting $H = \log\affInt(J^\epsilon)$.
  The L\'evy-Khintchine map associated with $\frac1\epsilon J^\epsilon$ is $\ode(\momentVar, \stateVar) = \frac1\epsilon (e^\momentVar - 1)$, and it resolves to the following.
  \begin{equation*}
    \mgDensity^\epsilon \defeq \exp\bigg( \frac1\epsilon \log\affInt(J^\epsilon_-) \shortInt J^\epsilon + \frac1\epsilon\big( 1 - \affInt(J^\epsilon) \big) \shortInt \leb \bigg)
  \end{equation*}
  The associated measure $\mgPrb^\epsilon(\rmd\omega) \defeq \mgDensity^\epsilon(\omega) \cdot \Prb^\epsilon(\rmd\omega)$ makes $\frac1\epsilon J^\epsilon$ have the desired intensity $\frac1\epsilon\affInt(J^\epsilon)$.
  The large deviation principle associated with $(\mgPrb^\epsilon)_{\epsilon>0}$ then comes from that of $(\Prb^\epsilon)_{\epsilon>0}$.
  The change in rate function comes from exponential corrections of our martingale term.
  Indeed, we observe that if $J^\epsilon$ is uniformly within $\delta > 0$ of some absolutely continuous increasing $\pathVar$ on some compact interval $[0,\tau]$, then we have the following inequalities.
  \begin{align*}
    \bigg| \int_0^\tau \Big( 1 - \affInt\big(J^\epsilon_t\big) \Big) \rmd t - \int_0^\tau \Big( 1 - \affInt\big( \pathVar(t) \big) \Big) \rmd t \bigg|
    &\leq \ell_1 \delta \tau \\
    %
    \bigg| \int_0^\tau \log\affInt\big(J^\epsilon_{t-}\big) \rmd J^\epsilon_t - \int_0^\tau \log\affInt\big(\pathVar(t-)\big) \rmd J^\epsilon_t \bigg| 
    &\leq \sup_{t\in[0,\tau]} \Big| \log\affInt\big(J^\epsilon_t\big) - \log\affInt\big(\pathVar(t)\big) \Big| \cdot J^\epsilon_T \\
    &\leq \frac{\affIntPart_1}{\affIntPart_0} \delta \Big( \pathVar(\tau) + \delta \Big), \\
    %
    \bigg| \int_0^\tau \log\affInt\big(\pathVar(t-)\big) \rmd J^\epsilon_t - \int_0^\tau \log\affInt\big(\pathVar(t-)\big) \rmd \pathVar(t)  \bigg| 
    &\leq \log\affInt\big(\pathVar(\tau)\big) \delta \\
    %
    \bigg| \int_0^\tau \log\affInt\big(J^\epsilon_{t-}\big) \rmd J^\epsilon_t - \int_0^\tau \log\affInt\big(\pathVar(t-)\big) \rmd \pathVar(t) \bigg| 
    &\leq \frac{\affIntPart_1}{\affIntPart_0} \delta \Big( \pathVar(\tau) + \delta \Big) + \log\affInt\big(\pathVar(\tau)\big) \delta 
  \end{align*}
  Then these give us the following identity, revealing the rate function.
  \begin{align*}
    &\lim_{\delta\rightarrow0}\lim_{\epsilon\rightarrow0}\epsilon\log\mgPrb^\epsilon\big( J^\epsilon \in \ball{\pathVar}{\delta} \big) \\
    &= \lim_{\delta\rightarrow0}\lim_{\epsilon\rightarrow0} \epsilon\log\Exp_{\Prb^\epsilon}\bigg( \exp\Big(\frac1\epsilon\int_0^\tau \log\affInt(J^\epsilon_{s-}) \rmd J^\epsilon_s + \frac1\epsilon\int_0^\tau \big(1 - \affInt(J^\epsilon_s) \big) \rmd s  \Big) 1_{J^\epsilon \in \ball{\pathVar}{\delta}} \bigg) \\
    &= \begin{aligned}[t]
      & \int_0^\tau \log\affInt\big(\pathVar(s-)\big) \rmd\pathVar(s) + \int_0^\tau \Big(1 - \affInt\big(\pathVar(s)\big) \Big) \rmd s \\
      &+\lim_{\delta\rightarrow0} \Big( \affIntPart_1 \delta \tau + \frac{\affIntPart_1}{\affIntPart_0} \delta \big( \pathVar(\tau) + \delta \big) + \log\affInt\big(\pathVar(\tau)\big) \delta \Big) 
      +\lim_{\delta\rightarrow0}\lim_{\epsilon\rightarrow0} \epsilon \log \Prb\Big( \epsilon N_{\cdot/\epsilon} \in \ball{\pathVar}{\delta} \Big)
    \end{aligned} \\
    &= \int_0^\tau \Big( \dot\pathVar(s) \log \affInt\big(\pathVar(s)\big) - \affInt\big(\pathVar(s)\big) + 1 \Big) \rmd s - \int_0^\tau \Big(\dot\pathVar(t) \log \dot\pathVar(t) - \dot\pathVar(t) + 1 \Big) \rmd s \\
    &= -\int_0^\tau \bigg( \dot\pathVar(s) \log\Big(\frac{\dot\pathVar(s)}{\affInt\big(\pathVar(s)\big)} \Big) - \dot\pathVar(s) + \affInt\big(\pathVar(s)\big) \bigg) \rmd s
  \end{align*}

  Of course, many technical details are missing in the above argument, but we again reiterate that this is merely to gather intuition on rate functions.
  For a rigorous proof involving such analysis, we refer the reader to \cite{Gao2018b}, where they prove a large deviation principle for an asymptotic family of nonlinear Hawkes processes.
  In any case, we again note that Theorem \ref{theorem:ldp-integral} covers the large deviation principle and rate function mentioned above.
  A continuous-branching process $\X$ with intensity $\affInt(\X)$ has special semimartingale decomposition as below,
  \begin{equation*}
    \X = 1 \ast \jumpMeas^{\X} = 1 \ast \predproj{\jumpMeas^{\X}} + 1 \ast \compensate{\jumpMeas^{\X}} = \affInt(\X) \shortInt \leb + 1 \ast \compensate{\jumpMeas^{\X}}
  \end{equation*}
  which lends itself to the following special differential characteristics $(\affDrift, \affDiff, \affJump)$,
  \begin{equation*}
    \affDrift(\stateVar) = \affInt(\stateVar), \quad
    \affDiff(\stateVar) = 0, \quad
    \affJump(\stateVar, \rmd\markVar) = \affInt(\stateVar) \delta_1(\rmd\markVar)
  \end{equation*}
  and so the corresponding integrand in (\ref{eq:rf-integral}) evaluates the following function.
  \begin{align*}
    \ode^*\big(\diffVar, \stateVar \big) 
    &= \sup_{\momentVar\in\bbR}\bigg( \momentVar \diffVar - \momentVar\affInt(\stateVar) - \int_\bbR \big( e^{\momentVar\markVar} - 1 - \momentVar\markVar \big) \affInt(\stateVar)\delta_1(\rmd\markVar) \bigg) \\
    &= \sup_{\momentVar\in\bbR}\bigg( \momentVar \diffVar - \affInt(\stateVar) e^\momentVar + \affInt(\stateVar) \bigg) \\
    &= \left\{\begin{array}{ll}
      \displaystyle \diffVar \log\Big(\frac{\diffVar}{\affInt(\stateVar)}\Big) - \diffVar + \affInt(\stateVar), & \diffVar \geq 0, ~ \affInt(\stateVar) \geq 0 \\
      \infty, & \text{otherwise}
    \end{array}\right.
  \end{align*}

  We may even extend this to Hawkes processes $(\Xe)_{\epsilon>0}$ induced by base process,
  \begin{equation*}
    \X = \kappa\big(\mu- \X_t \big) \shortInt \leb + \N, \quad \N \text{ intensity } \affInt(\X),
  \end{equation*}
  for this process $\X$ has affine special differential characteristics.
  \begin{equation*}
    \affDrift(\stateVar) = \kappa(\mu - \stateVar) + \affInt(\stateVar), \quad
    \affDiff(\stateVar) = 0, \quad
    \affJump(\stateVar, \rmd\markVar) = \affInt(\stateVar) \delta_1(\rmd\markVar)
  \end{equation*}
  The rate function then involves the following expression,
  \begin{align*}
    \ode^*(\diffVar, \stateVar) 
    &= \sup_{\momentVar\in\bbR} \bigg( \momentVar\diffVar - \momentVar\Big(\kappa(\mu - \stateVar) - \affInt(\stateVar)\Big) - \int_\bbR \big( e^{\momentVar\markVar} - 1 - \momentVar\markVar \big) \affInt(\stateVar) \delta_1(\rmd\markVar) \bigg) \\
    &= \sup_{\momentVar\in\bbR} \bigg( \momentVar\Big(\diffVar - \kappa(\mu-\stateVar)\Big) - \affInt(\stateVar)e^\momentVar + \affInt(\stateVar) \bigg) \\
    &= \left\{\begin{array}{ll}
      \displaystyle \diffVar \log\Big(\frac{\diffVar-\kappa(\mu-\stateVar)}{\affInt(\stateVar)}\Big) - \diffVar + \affInt(\stateVar), & \diffVar - \kappa(\mu-\stateVar) \geq 0, ~ \affInt(\stateVar) \geq 0, \\
      \infty, & \text{otherwise}
    \end{array}\right.
  \end{align*}
  which is similar to the linear case of \cite{Gao2018b}.
\end{example}

