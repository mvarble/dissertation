\begin{theorem}
  \label{theorem:KM}
  \begin{enumerate}[label=(\alph*)]
    \item
      \label{theorem:KM:1}
      There exists a map $\aff: \momentSpace \times \stateSpace \rightarrow \bbR$ of the form
      \begin{equation*}
        \aff(t,\momentVar,\stateVar) = \affPart_0(t,\momentVar) + \bprj{\affPart(t,\momentVar)}{\stateVar}
      \end{equation*}
      such that, for each $(\tau,\momentVar) \in \momentSpace$, $\aff(\cdot,\momentVar,\cdot)$ is a solution to $\system(\ode,\tau,\momentVar)$ dominated by all other such solutions.
      Moreover, this map satisfies the following for each $(\tau, \momentVar) \in \momentSpace$ and $\stateVar \in \stateSpace$.
      \begin{equation}
        \label{theorem:KM-moments}
        \Exp_{\Prb_\stateVar}\exp\prj{\momentVar}{\X_t} = \exp\aff(t, \momentVar, \stateVar), \quad t \in [0,\tau]
      \end{equation}
    \item
      \label{theorem:KM:2}
      If $\tau \geq 0$, $\momentVar \in \vecSpace$, and $\stateVar \in \stateSpace^\circ$ are such that  $\Exp_{\Prb_\stateVar}\exp\prj{\momentVar}{\X_\tau} < \infty$, then $(\tau, \momentVar) \in \momentSpace$.
  \end{enumerate}
\end{theorem}
\begin{proof}
  \label{proof:theorem:KM}
  With Remark \ref{remark:riccati-affine-formulation}, this is the same as \cite[Theorem 2.14]{keller2015}.
\end{proof}
