\begin{remark}
  \label{remark:affine-remarks}
  \begin{enumerate}[label=(\alph*)]
    \item
      See \cite[Remark 2.3]{keller2015} for an argument on how our assumptions on $\stateSpace$ are at no loss of generality; $\stateSpace$ may as well be any nonempty convex set.
    \item
      Note how (\ref{eq:affine-definition}) specifies the characteristic function of each transition kernel of the Markov process $\X$; thus, should an affine process exist for choice of $\aff$, \emph{only} one will exist, up to distribution.
    \item
      See how our notation $(\affPart_0, \affPart)$ differs from that of \cite{keller2015} and other papers, which typically use $(\phi, \affPart)$.
      We choose to do this because affine functions prevail throughout our investigation of affine processes, and we saw this an opportunity to have more cohesive notation of all such affine functions.
      \begin{equation*}
        \affDiff(\stateVar) = \affDiffPart_0 + \sum_{i=1}^d \stateVar^i \affDiffPart_i
      \end{equation*}
    \item
      \label{remark:affine-parts}
      If we have a vector space $\bbA$ and affine map $\alpha: \stateSpace \rightarrow \bbA$ determined by $a_0, \ldots, a_d \in \bbA$ via $\alpha(\stateVar) = a_0 + \sum_{i=1}^d \stateVar^i a_i$, then our linear assumptions $0 \in \stateSpace$ and $\operatorname{span}\stateSpace = \vecSpace$ uniquely determine $a_0, \ldots, a_d \in \bbA$.
      In particular, the map $\aff$ uniquely identifies its parts $\affPart_i: \bbR_+ \times \boundedSpace \rightarrow \bbC$ for $i=0,\ldots,d$.
  \end{enumerate}
\end{remark}
