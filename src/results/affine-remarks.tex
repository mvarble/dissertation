\begin{remark}
  \label{remark:affine-remarks}
  Our definitions of $\stateSpace$ and $\aff$ include the following conventions and motivations.
  \begin{enumerate}[label=(\alph*)]
    \item
      {\color{gray} why we denote $\affPart_0, \affPart$ instead of KM $\varphi, \affPart$ or Cuchiero $\Phi, \Psi$}
    \item
      {\color{gray} how assumptions $0 \in \stateSpace$, $\operatorname{span}\stateSpace = \vecSpace$ are nonrestrictive}
    \item
      {\color{gray} how (\ref{eq:affine-definition}) decides the distribution of $\X$ and how the distribution of the affine process decides $\aff$}
    \item
      \label{remark:affine-parts}
      If we have a vectorspace $\bbA$ and affine map $\alpha: \stateSpace \rightarrow \bbA$ determined by $a_0, \ldots, a_d \in \bbA$ via $\alpha(\stateVar) = a_0 + \sum_{i=1}^d \stateVar^i a_i$, then our linear assumptions $0 \in \stateSpace$ and $\operatorname{span}\stateSpace = \vecSpace$ uniquely determine $a_0, \ldots, a_d \in \bbA$.
      In particular, the map $\aff$ uniquely identifies its parts $\affPart_i: \bbR_+ \times \boundedSpace \rightarrow \bbC$ for $i=0,\ldots,d$.
    \item
      In \cite[Theorem 1.2.7]{cuchiero2011}, it is shown that, without loss of generality on conditional distributions $(\Prb_\stateVar)_{\stateVar\in\stateSpace}$, an affine process $\X$ can be chosen to have c\`adl\`ag paths.
      Thus, each distribution $\Prb_\stateVar$ may (and will) be recognized as a measure on the Borel algebra associated with the space $\pathSpace{[0,\infty)}{\stateSpace}$ of c\`adl\`ag functions equipped with the Skorokhod topology (see Appendix \ref{appendix:skorokhod}).
  \end{enumerate}
\end{remark}
