% discuss that we are coming up with semi-closed form
We have now gathered enough familiarity with rate functions that appear in the large deviations literature and the various tools we may use to lift existing principles to new ones.
This will allow us to more easily understand the nature and derivation of our semi-closed form for (\ref{eq:rf-integral}) in Theorem \ref{theorem:ldp-integral}.
Let us now elaborate on the family on which we derive our semi-closed form; it is assumed that $\X$ is selected with our assumptions of Section \ref{large-deviations:assumptions} that we needed for Theorem \ref{theorem:ldp-integral}.

% introduce the factors of our new process from the original one
Firstly, in order for us to perform the tricks of coupling we explored in the previous section, we need to couple our arrivals and/or accumulated jumps.
We will need these objects to be locally integrable, so we impose that our base process $\X$ is \emph{locally countable} in the sense of our definition in Appendix \ref{jump-diffusions:countable}.
In particular, suppose that our base process $\X$ has special differential characteristics $(\affDrift, \affDiff, \affJump)$, where $\affJump$ has the finiteness condition below.
\begin{equation}
  \label{eq:assumption-affJump-finite}
  \affJump(\stateVar, \vecSpace) < \infty, \quad \stateVar \in \stateSpace
\end{equation}
Note that Proposition \ref{proposition:locally-countable} thus provides us with a factoring 
\begin{equation}
  \label{eq:affJump-factoring}
  \affJump(\stateVar,\rmd\markVar) = \affInt(\stateVar) \affDist(\stateVar, \rmd\markVar)
\end{equation}
into an intensity function $\affInt \in \scrB(\stateSpace)/\scrB(\bbR_+)$ and conditional jump distribution $\affDist$ from $(\stateSpace, \scrB(\stateSpace))$ to $(\vecSpace, \scrB(\vecSpace))$.
Denote to each $\affJump(\stateVar, \cdot)$ the notation from Sections \ref{rate-functions:mogulskii} and \ref{rate-functions:coupling}.
\begin{equation*}
  \overline{\affDist(\stateVar, \cdot)} \defeq \int_\vecSpace \markVar \affDist(\stateVar, \rmd\markVar), \quad
  \ode_{\affDist(\stateVar,\cdot)}(\momentVar) \defeq \int_\vecSpace e^\prj{\momentVar}{\markVar} \affDist(\stateVar, \rmd\markVar)
\end{equation*}
Note that our finiteness assumption (\ref{eq:assumption-affJump-finite}) and our light-tails assumption $\odeSpace = \vecSpace$ (along with Lemma \ref{lemma:odeSpace-rationale}) give us the following properties.
\begin{equation*}
  \overline{\affDist(\stateVar, \cdot)} < \infty, \quad \ode_{\affDist(\stateVar, \cdot)}(\momentVar) < \infty, \quad \stateVar \in \stateSpace
\end{equation*}
We also have local integrability of the following process by our finiteness assumption (\ref{eq:assumption-affJump-finite}).
\begin{equation}
  \label{eq:arrivals}
  \N^\X \defeq 1 \ast \jumpMeas^\X
\end{equation}
Note that this is a stronger restriction than local integrability of the compensated jumps
\begin{equation}
  \label{eq:jumps}
  \V^\X \defeq \id_\vecSpace \ast \compensate{\jumpMeas^\X},
\end{equation}
which we already have from the special property of $\X$.

% technical measure space stuff for joint process
The remainder of this section will assume $\hat\X \defeq (\X, \X^\rmc, \V^\X, \N^\X)$ is specified from an affine process $\X$ with the properties discussed above.
We will also factor our kernel $(\Prb_\stateVar)_{\stateVar\in\stateSpace}$ even further to one $(\hat\Prb_{\hat\stateVar})_{\hat\stateVar\in\hat\stateSpace}$ derived from the regular conditional $\Prb_\stateVar$-distribution of $(\Prb_\stateVar, \scrF)$ time-homogeneous Markov process $(\X^\rmc, \V^\X, \N^\X)$.
We will see in the coming results that $\hat\X$ and its associated kernel $(\hat\Prb_{\hat\stateVar})_{\hat\stateVar\in\hat\stateSpace}$ and filtration $\scrF$ still satisfy the assumptions of Section \ref{large-deviations:assumptions} needed to apply Theorem \ref{theorem:ldp-integral}; furthermore, its coupling structure will allow us to derive a semi-closed form for (\ref{eq:rf-integral}).
This is a process on 
\begin{equation*}
  \hat\stateSpace \defeq \stateSpace \times \vecSpace \times \vecSpace \times \bbR_+,
\end{equation*}
which is a subset of vector space
\begin{equation*}
  \hat\vecSpace \defeq \vecSpace \times \vecSpace \times \vecSpace \times \bbR,
\end{equation*}
when equipped with componentwise operations.
We henceforth denote elements $\hat\vecVar \in \hat\vecSpace$ with components,
\begin{equation*}
  \hat\vecVar \defeq (\vecVar_1, \vecVar_2, \vecVar_3, \vecVar_4),
\end{equation*}
and equip $\hat\vecSpace$ with the inner-product below.
\begin{equation*}
  \bprj{\hat\markVar}{\hat w} \defeq \prj{\markVar_1}{w_1} + \prj{\markVar_2}{w_2} + \prj{\markVar_3}{w_3} + \markVar_4 w_4
\end{equation*}
We will also indifferently treat such tuples as column vectors, so that operators on $\bbL(\hat\vecSpace)$ can be seen as $4\times4$ block-matrices of operators on $\bbL(\vecSpace)$ and $\bbL(\bbR)$ (with the natural blocking).
If we want to go a level deeper within coordinates, we will use the following notation.
\begin{equation*}
  \hat\vecVar \defeq \big(\vecVar^{11}, \ldots, \vecVar^{1d}, \vecVar^{21}, \ldots, \vecVar^{2d}, \vecVar^{31}, \ldots, \vecVar^{3d}, \vecVar^{41} \big)
\end{equation*}

% preliminary results to ensure theorem is satsified
\begin{lemma}
  \label{lemma:joint-cone}
  The space $\hat\stateSpace$ is a cone in $\hat\vecSpace$ with operations taken componentwise.
\end{lemma}
\begin{proof}
  \label{proof:lemma:joint-cone}
  This is obvious, as each of the factors $\stateSpace$, $\vecSpace$, $\bbR_+$ that appear in our product is also a cone.
\end{proof}

\begin{proposition}
  \label{proposition:joint-characteristics}
  For each $\hat\stateVar \in \hat\stateSpace$, the process $\hat\X$ is a $(\hat\Prb_{\hat\stateVar}, \scrF)$ special jump-diffusion with affine special differential characteristics $(\hat\affDrift, \hat\affDiff, \hat\affJump)$ as below,
  \begin{equation}
    \label{eq:joint-characteristics}
    \begin{gathered}
      \hat\affDrift(\hat\stateVar) = \begin{pmatrix} \affDrift(\stateVar_1) \\ 0 \\ 0 \\ \affInt(\stateVar_1) \end{pmatrix}, \quad
      \hat\affDiff(\hat\stateVar) = \begin{pmatrix} \affDiff(\stateVar_1) & \affDiff(\stateVar_1) & 0 & 0 \\ \affDiff(\stateVar_1) & \affDiff(\stateVar_1) & 0 & 0 \\ 0 & 0 & 0 & 0 \\ 0 & 0 & 0 & 0 \end{pmatrix}, \\
      \hat\affJump(\hat\stateVar, \rmd\hat\markVar) = \affInt(\stateVar_1) \affDist(\stateVar_1, \rmd\markVar_1) \zeta(\rmd\markVar_2) \delta_{\markVar_1}(\rmd\markVar_3) \delta_1(\rmd\markVar_4),
    \end{gathered}
  \end{equation}
  where $\zeta$ is the zero measure on $\scrB(\vecSpace)$.
  Moreover, denoting $\hat\ode$ the associated L\'evy-Khintchine map, we have $\odeSpaceJoint = \hat\vecSpace$ and the following.
  \begin{equation*}
    \hat\ode(\hat\momentVar, \hat\stateVar) = \begin{aligned}[t]
      &\prj{\momentVar_1}{\affDrift(\stateVar_1)} + \frac12\bprj{\momentVar_1+\momentVar_2}{\affDiff(\stateVar)(\momentVar_1+\momentVar_2)} \\
      &+ \affInt(\stateVar_1)\exp\Big(\ode_{\affDist(\stateVar_1,\cdot)}(\momentVar_1+\momentVar_3) + \momentVar_4\Big) - \affInt(\stateVar_1) -  \bprj{\momentVar_1+\momentVar_3}{\affInt(\stateVar_1)\overline{\affDist(\stateVar_1,\cdot)}} 
    \end{aligned}
  \end{equation*}
\end{proposition}
\begin{proof}
  \label{proof:proposition:joint-characteristics}
  The fact that $\hat\X$ is special simply comes from the following decomposition.
  \begin{align*}
    \hat\X 
    = \begin{pmatrix}
      \X \\ \X^\rmc \\ \V^\X \\ \N^\X 
    \end{pmatrix}  
    &= \begin{pmatrix} 
      \stateVar_1 + \affDrift(\X) \shortInt \leb_t + \X^\rmc + \id_\vecSpace \ast \compensate{\jumpMeas^\X}_t \\ \stateVar_2 + \X^\rmc \\ \stateVar_3 + \id_\vecSpace \ast \compensate{\jumpMeas^\X}_t  \\ \stateVar_4 + 1 \ast \jumpMeas^\X_t
    \end{pmatrix}  \\
    &= \begin{pmatrix} 
      \stateVar_1 + \affDrift(\X) \shortInt \leb_t + \X^\rmc + \id_\vecSpace \ast \compensate{\jumpMeas^\X}_t \\ \stateVar_2 + \X^\rmc \\ \stateVar_3 + \id_\vecSpace \ast \compensate{\jumpMeas^\X}_t  \\ \stateVar_4 + 1 \ast \predproj{\jumpMeas^\X}_t + 1 \ast \compensate{\jumpMeas^\X}_t
    \end{pmatrix}  \\
    &= \hat\stateVar 
    + \begin{pmatrix} 
      \affDrift(\X) \\
      0 \\
      0 \\
      \affInt(\X)
    \end{pmatrix} \shortInt \leb_t
    + \begin{pmatrix}
      \X^\rmc \\
      \X^\rmc \\
      0 \\
      0
    \end{pmatrix}
    + \begin{pmatrix}
      \id_\vecSpace \\
      0 \\
      \id_\vecSpace \\
      1 
    \end{pmatrix} \ast \compensate{\jumpMeas^\X}_t
  \end{align*}
  Observe that the first term establishes the drift $\hat\affDrift$ as in (\ref{eq:joint-characteristics}).
  The diffusion $\hat\affDiff$ in (\ref{eq:joint-characteristics}) is verified by the following identities.
  \begin{align*}
    \Bprj{\hat\X^{\rmc,1i}}{\hat\X^{\rmc,1j}} &= \Bprj{\X^{\rmc,i}}{\X^{\rmc,j}} = \affDiff_{ij}(\X) \shortInt \leb, &
    \Bprj{\hat\X^{\rmc,1i}}{\hat\X^{\rmc,2j}} &= \Bprj{\X^{\rmc,i}}{\X^{\rmc,j}} = \affDiff_{ij}(\X) \shortInt \leb, \\
    \Bprj{\hat\X^{\rmc,1i}}{\hat\X^{\rmc,3j}} &= \Bprj{\X^{\rmc,i}}{0} = 0, &
    \Bprj{\hat\X^{\rmc,1i}}{\hat\X^{\rmc,4j}} &= \Bprj{\X^{\rmc,i}}{0} = 0, \\
    \Bprj{\hat\X^{\rmc,2i}}{\hat\X^{\rmc,2j}} &= \Bprj{\X^{\rmc,i}}{\X^{\rmc,j}} = \affDiff_{ij}(\X) \shortInt \leb, &
    \Bprj{\hat\X^{\rmc,2i}}{\hat\X^{\rmc,3j}} &= \Bprj{\X^{\rmc,i}}{0} = 0,  \\
    \Bprj{\hat\X^{\rmc,2i}}{\hat\X^{\rmc,4j}} &= \Bprj{\X^{\rmc,i}}{0} = 0, &
    \Bprj{\hat\X^{\rmc,3i}}{\hat\X^{\rmc,4j}} &= \Bprj{0}{0} = 0, 
  \end{align*}
  Lastly, for any predictable $H: \Omega \times \bbR_+ \times \hat\vecSpace \rightarrow \bbR_+$, we have the following.
  \begin{align*}
    \Exp_{\Prb_\stateVar} \big( H \ast \jumpMeas^{\hat\X}_\infty \big)
    &= \Exp_{\Prb_\stateVar} \int_{\bbR_+\times\vecSpace} H(\cdot, s, \markVar_1, 0, \markVar_1, 1) \jumpMeas^\X(\rmd s, \rmd\hat\markVar) \\
    &= \Exp_{\Prb_\stateVar} \int_0^\infty \int_\vecSpace H(\cdot, s, \markVar_1, 0, \markVar_1, 1) \affJump(\X_s, \rmd\markVar_1) \rmd s \\
    &= \Exp_{\Prb_\stateVar} \int_0^\infty \int_\vecSpace H(\cdot, s, \markVar_1, 0, \markVar_1, 1) \affInt(\X_s) \affDist(\X_s, \rmd\markVar_1)  \rmd s \\
    &= \Exp_{\Prb_\stateVar} \int_0^\infty \int_{\hat\vecSpace} H(\cdot, s, \hat\markVar) \affInt(\X_s) \affDist(\X_s, \rmd\markVar_1) \zeta(\rmd\markVar_2) \delta_{\markVar_1}(\rmd\markVar_3) \delta_1(\rmd\markVar_4)  \rmd s
  \end{align*}
  This concludes that, $\predproj{\jumpMeas^{\hat\X}}(\rmd s, \rmd\hat\markVar) = \hat\affJump(\hat\X_s, \rmd\hat\markVar) \rmd s$, where $\hat\affJump$ is as in (\ref{eq:joint-characteristics}).
  This work shows that $\hat\X$ is a $(\hat\Prb_{\hat\stateVar\in\hat\stateSpace}, \scrF)$ special jump-diffusion; its special differential characteristics are also affine.
  \begin{align*}
    \hat\affDrift(\hat\stateVar) &= \begin{pmatrix} \affDriftPart_0 \\ 0 \\ 0 \\ \affIntPart_0 \end{pmatrix} + \sum_{i=1}^d \stateVar^{1i} \begin{pmatrix} \affDriftPart_i \\ 0 \\ 0 \\ \affIntPart_i \end{pmatrix} \\
    \hat\affDiff(\hat\stateVar) &=  \begin{pmatrix} \affDiffPart_0 & \affDiffPart_0 & 0 & 0 \\ \affDiffPart_0 & \affDiffPart_0 & 0 & 0 \\ 0 & 0 & 0 & 0 \\ 0 & 0 & 0 & 0 \end{pmatrix} + \sum_{i=1}^d \stateVar^{1i} \begin{pmatrix} \affDiffPart_0 & \affDiffPart_0 & 0 & 0 \\ \affDiffPart_0 & \affDiffPart_0 & 0 & 0 \\ 0 & 0 & 0 & 0 \\ 0 & 0 & 0 & 0 \end{pmatrix} \\
    \hat\affJump(\hat\stateVar, \rmd\hat\markVar) &= \affJumpPart_0(\rmd\markVar_1)\zeta(\rmd\markVar_2)\delta_{\markVar_1}(\rmd\markVar_3)\delta_1(\rmd\markVar_4) + \sum_{i=1}^d \stateVar^{1i}  \affJumpPart_i(\rmd\markVar_1)\zeta(\rmd\markVar_2)\delta_{\markVar_1}(\rmd\markVar_3)\delta_1(\rmd\markVar_4)
  \end{align*}
  The associated L\'evy-Khintchine map $\hat\ode$ associated with it satisfies $\odeSpaceJoint = \hat\vecSpace$ by Lemma \ref{lemma:odeSpace-rationale}.
  \begin{equation*}
    \int_{|\hat\markVar|>1} e^\prj{\hat\momentVar}{\hat\markVar} \hat\affJump(\hat\stateVar, \rmd\hat\markVar)
    = \int_{|\hat\markVar|>1} e^{\prj{\momentVar_1+\momentVar_3}{\markVar_1} + \momentVar_4\markVar_4} \affInt(\stateVar_1) \affDist(\rmd\markVar_1)\delta_1(\rmd\markVar_4) < \infty
  \end{equation*}
  We may also simplify it to the desired expression.
  \begin{align*}
    \hat\ode(\hat\momentVar, \hat\stateVar) 
    &= \bprj{\hat\momentVar}{\hat\affDrift(\hat\stateVar)} + \frac12\bprj{\hat\momentVar}{\hat\affDiff(\hat\stateVar)\hat\momentVar} + \int_{\hat\vecSpace} \Big( e^\prj{\hat\momentVar}{\hat\markVar} - 1 - \bprj{\hat\momentVar}{\hat\markVar} \Big) \hat\affJump(\hat\stateVar, \rmd\hat\markVar) \\
    &= \begin{aligned}[t]
      &\prj{\momentVar_1}{\affDrift(\stateVar_1)} + \momentVar_4\affInt(\stateVar_1) \\
      &+ \frac12\bprj{\momentVar_1}{\affDiff(\stateVar_1)\momentVar_1}  + \frac12\bprj{\momentVar_1}{\affDiff(\stateVar_1)\momentVar_2}  + \frac12\bprj{\momentVar_2}{\affDiff(\stateVar_1)\momentVar_1}+ \frac12\bprj{\momentVar_2}{\affDiff(\stateVar_1)\momentVar_2}\\
      &+ \int_\vecSpace \Big( e^{\prj{\momentVar_1}{\markVar} + \prj{\momentVar_3}{\markVar} + \momentVar_4} - 1 - \prj{\momentVar_1}{\markVar_1} - \prj{\momentVar_3}{\markVar} - \momentVar_4  \Big) \affInt(\stateVar_1) \affDist(\stateVar_1, \rmd\markVar)
    \end{aligned} \\
    &= \begin{aligned}[t]
      &\prj{\momentVar_1}{\affDrift(\stateVar_1)} + \frac12\bprj{\momentVar_1+\momentVar_2}{\affDiff(\stateVar_1)(\momentVar_1+\momentVar_2)}  \\
      &+ \affInt(\stateVar_1)\Big(\int_\vecSpace e^\prj{\momentVar_1+\momentVar_3}{\markVar} \affDist(\stateVar_1, \rmd\markVar) \Big) e^{\momentVar_4} - \affInt(\stateVar_1) - \bprj{\momentVar_1 + \momentVar_3}{\affInt(\stateVar_1)\overline{\affDist(\stateVar_1,\cdot)}}
    \end{aligned} \\
    &= \begin{aligned}[t]
      &\prj{\momentVar_1}{\affDrift(\stateVar_1)} + \frac12\bprj{\momentVar_1+\momentVar_2}{\affDiff(\stateVar)(\momentVar_1+\momentVar_2)} \\
      &+ \affInt(\stateVar_1)\exp\Big(\ode_{\affDist(\stateVar_1,\cdot)}(\momentVar_1+\momentVar_3) + \momentVar_4\Big) - \affInt(\stateVar_1) -  \bprj{\momentVar_1+\momentVar_3}{\affInt(\stateVar_1)\overline{\affDist(\stateVar_1,\cdot)}} 
    \end{aligned}
  \end{align*}
\end{proof}


% annoying assumption I cannot work out
Unfortunately, despite $\hat\X$ having affine characteristics, this is not enough to ensure it is an affine process.
We technically need existence of a solution $\hat\aff$ to its associated system $\partial_t\hat\aff(t, \hat\momentVar, \hat\stateVar) = \ode\big(\hat\affPart(t, \hat\momentVar), \hat\stateVar)$ as in Proposition \ref{proposition:affine-resolution}.
It is partially solved by the state-independence of $\ode$ in the coordinates $\stateVar_2, \stateVar_3, \stateVar_4$, but we cannot find a way to leverage the equation associated with $\ode$.
Thus is the following assumption.
\begin{assumption}
  \label{assumption:joint-affine}
  Assume that $\X$ is such that the associated special differential characteristics $(\affDrift, \affDiff, \affJump)$ and the associated parts $\affDriftPart_i, \affDiffPart_i, \affJumpPart_i, \affIntPart_i$ for $i = 0, \ldots, d$ induce a solution to the following equation, for each $\hat\momentVar\in\boundedSpaceJoint$.
  \begin{equation*}
    \begin{aligned}[t]
      &i = 0, \ldots, d \\
      & \partial_t\hat{\affPart_i}(t, \hat\momentVar) = \begin{aligned}[t]
        &\Bprj{\hat\affPart(t, \hat\momentVar)}{\affDriftPart_i} +  \frac12\Bprj{\hat\affPart(t, \hat\momentVar)+\momentVar_2}{\affDiffPart_i\big(\hat\affPart(t,\hat\momentVar)+\momentVar_2\big)} \\
        &+ \int_\vecSpace \Big( e^{\prj{\hat\affPart(t,\hat\momentVar) + \momentVar_3}{\markVar} + \momentVar_4} - 1 - \Bprj{\hat\affPart(t,\hat\momentVar) + \momentVar_3}{\markVar} \Big) \affJumpPart_i(\rmd\markVar) 
      \end{aligned} \\
      & \affPart_0(t, \momentVar) = 0, ~ \affPart_i(t, \momentVar) = \momentVar_i
    \end{aligned}
  \end{equation*}
\end{assumption}
\begin{remark}
  \begin{enumerate}
    \item
      When we are dealing with the canonical space $\stateSpace = \bbR^m_+ \times \bbR^n$, then $\hat\stateSpace = \bbR^{\hat m}_+ \times \bbR^{\hat n}$, up to reordering the factors.
      In this scenario, we have admissibility conditions on the special differential characteristics which impose existence of the solution, for each $\hat\momentVar \in \boundedSpaceJoint$.
    \item
      Note that we already have a solution to the following equation for each $\momentVar \in \boundedSpace$, by existence of $\X$ as an affine process and Theorem \ref{theorem:affine-regularity}.
      \begin{equation*}
        \begin{aligned}[t]
          &i = 0, \ldots, d \\
          & \dot\affPart_i(t, \momentVar) = \begin{aligned}[t]
            &\prj{\affPart(t, \momentVar)}{\affDriftPart_i} + \frac12\prj{\affPart(t, \momentVar)}{\affDiffPart_i\affPart(t,\momentVar)} \\
            &+ \int_\vecSpace \big( e^\prj{\affPart(t,\momentVar)}{\markVar} - 1 - \prj{\affPart(t,\momentVar)}{\markVar} \big) \affJumpPart_i(\rmd\markVar) 
          \end{aligned} \\
          & \affPart_0(t, \momentVar) = 0, ~ \affPart_i(t, \momentVar) = \momentVar_i
        \end{aligned}
      \end{equation*}
  \end{enumerate}
\end{remark}

% restatement of main theorem
\begin{theorem}
  \label{eq:ldp-closed-form}
  Let $\X$ be an affine process with special differential characteristics $(\affDrift, \affDiff, \affJump)$ satisfying the assumptions specified in Section \ref{large-deviations:assumptions} and the local countability condition (\ref{eq:assumption-affJump-finite}) and subsequent factoring denoted by (\ref{eq:affJump-factoring}).
  Consider the family $(\Xe)_{\epsilon>0}$ parameterized by $\X$ through (\ref{eq:characteristic-asymptotics}).
  Then, for each $\epsilon > 0$, the tuple $\hat\Xe = (\Xe, {\Xe}^{,\rmc}, \V^{\Xe}, \N^{\Xe})$ is an affine process on $\vecSpace \times \vecSpace \times \vecSpace \times \bbR_+$ satisfying the conditions of Theorem \ref{theorem:ldp-integral}.
  Denoting $\hat\stateVar\in \stateSpace^\circ$ its starting point, the rate function $\rf_{\hat\stateVar}$ in (\ref{eq:rf-integral}) simplifies to the following semi-closed form, where we denote the components of an arbitrary function $\hat\pathVar$ by $\hat\pathVar \defeq (\pathVar, \omega, \gamma, \eta)$.
  \begin{equation*}
    \rf_{\hat\stateVar}(\hat\pathVar) = 
    \begin{aligned}[t]
      &\int_0^\infty \frac12 \Bprj{\dot\omega(t)}{\affDiff\big(\pathVar(t)\big)^\dagger \dot\omega(t)} \Big) \rmd t 
      + \int_0^\infty \bigg( \dot\eta(t)\log\Big(\frac{\dot\eta(t)}{\affInt\big(\pathVar(t)\big)} \Big) - \dot\eta(t) + \affInt\big(\pathVar(t)\big) \bigg) \rmd t  \\
      &\qquad+ \int_0^\infty \dot\eta(t) \ode_{\affDist(\pathVar(t),\cdot)}\bigg( \frac{\dot\gamma(t) + \affInt\big(\pathVar(t)\big)\overline{\affDist(\pathVar(t), \cdot)}}{\dot\eta(t)} \bigg) \rmd t
    \end{aligned}
  \end{equation*}
  In the evaluation above, we are insisting that $\hat\pathVar$ satisfies the following properties below, where statements involving $t$ are taken Lebesgue-almost-everywhere; otherwise $\rf_{\hat\stateVar}(\hat\pathVar) = \infty$.
  \begin{itemize}
    \item
      $\hat\pathVar(0) = \hat\stateVar$,
    \item
      $\pathVar \in \acpathSpace{[0,\infty)}{\stateSpace}$, $\omega \in \acpathSpace{[0,\infty)}{\vecSpace}$, $\gamma \in \acpathSpace{[0,\infty)}{\vecSpace}$, and $\eta \in \acpathSpace{[0,\infty)}{\bbR_+}$,
    \item
      $\dot\pathVar(t) = \affDrift\big(\pathVar(t)\big) + \dot\omega(t) + \dot\gamma(t)$,
    \item
      $\dot\omega(t) \in \operatorname{range}\big(\affDiff(\pathVar(t))\big)$,
    \item
      $\dot\eta(t) \geq 0$,
    \item
      $\affInt\big(\pathVar(t)\big) \geq 0$.
  \end{itemize}
\end{theorem}
\begin{proof}
\end{proof}

