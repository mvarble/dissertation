% mention how we are effectively repeating 
With a good grasp of the finite real moments associated with our affine process $\X$ and their correspondence with $\aff$, we now leverage these results to the finite-dimensional distributions.
In other words, this section serves to lift Theorem \ref{theorem:KM-moments} on marginals $\X_t$ to one on finite-dimensional distributions $(\X_{t_1}, \ldots, \X_{t_n})$.
Let us establish some notation.

For any space $\bbA$, positive integer $n\in\bbN$, and $\underline a \in \bbA^n$, adopt the convention of denoting $\underline a = (a_1, \ldots, a_n)$ and 
\begin{equation*}
  \underline a_{\ell:m} = (a_\ell,\ldots,a_m) \in \bbA^{m-\ell+1}, \quad 1 \leq \ell \leq m \leq n.
\end{equation*}
For each $n\in \bbN$ and $\underline t \in [0,\infty)^n$, define the projection map $\pi_{\underline t}: \stateSpace^{[0,\infty)} \rightarrow \stateSpace^n$ by
\begin{equation*}
  \pi_{\underline t}(\pathVar) \defeq \pathVar(\underline t) \defeq \big( \pathVar(t_1), \ldots, \pathVar(t_n) \big).
\end{equation*}
Denote $\underline t \vdash [0,\infty)$ to mean that $\underline t$ is additionally a partition of the following form.
\begin{equation*}
  0 < t_1 < \cdots < t_n
\end{equation*}
For each such partition $\underline t \vdash [0,\infty)$, associate the following notation.
\begin{align*}
  t_0 &\defeq 0 \\
  \Delta t_k &\defeq t_k - t_{k-1}, & 1 \leq k \leq n \\
  |\underline t| &\defeq n
\end{align*}
Lastly, for any $A \subseteq [0,\infty)$, denote $\underline t \vdash A$ to mean $\underline t \vdash [0,\infty)$ and $t_1,\ldots,t_{|\underline t|} \in A$.
For each $n \in \bbN$, extend the linear operations of $\vecSpace$ to $\vecSpace^n$, componentwise.
Similarly, extend the definition of our inner-product on $\vecSpace \oplus \im\vecSpace$ to one on $(\vecSpace \oplus \im\vecSpace)^n$, like so.
\begin{equation*}
  \prj{\underline\momentVar}{\underline\vecVar} \defeq \sum_{k=1}^n \prj{\momentVar_k}{\vecVar_k}
\end{equation*}

We now clearly specify the extension of $\aff$ to finite-dimensional projections from the perspective of Theorem \ref{theorem:KM} and equation (\ref{theorem:KM-moments}).
Note that this specifically \emph{permits} infinite values.

\begin{definition}
  \label{definition:aff-fdds}
  To each $\underline t \vdash [0,\infty)$, define $\aff(\underline t, \cdot, \cdot): (\vecSpace \oplus \im\vecSpace)^{|\underline t|} \times \stateSpace \rightarrow (-\infty,\infty]$ as the cumulant generating function of $\X_{\underline t}$.
  \begin{equation*}
    \Exp_{\Prb_\stateVar} \exp\prj{\underline\momentVar}{\X_{\underline t}} \eqdef \exp\aff(\underline t, \underline\momentVar, \stateVar)
  \end{equation*}
  Note that this extends the definition of $\aff$ in that we may always consider some time $t > 0$ as a partition $t \vdash [0,\infty)$.
\end{definition}


Before we investigate real moments, let us establish the easier result on purely complex moments.
This will give us intuition for the objects we create in the sequel.

\begin{proposition}
  \label{proposition:aff-fdds-complex}
  For any $\underline t \vdash [0,\infty)$, $\underline\momentVar \in \im\vecSpace^{|\underline t|}$, and $\stateVar \in \stateSpace$, we have the following identity, where we denote $n \defeq |\underline t|$ for brevity.
  \begin{gather*}
    \begin{aligned}
      \twistVar_n &\defeq \momentVar_n \\
      \twistVar_k &\defeq \momentVar_k + \affPart(\Delta t_{k+1}, \twistVar_{k+1}), & k = n-1, \ldots, 1 
    \end{aligned} \\
    \aff(\underline t, \underline\momentVar, \stateVar) = \sum_{k=1}^{|\underline t|} \affPart_0(\Delta t_k, \twistVar_k) + \bprj{\affPart(\Delta t_k, \twistVar_k)}{\stateVar}
  \end{gather*}
\end{proposition}
\begin{proof}
  \label{proof:proposition:aff-fdds-complex}
  We start by recognizing that $\underline\momentVar \in \im\vecSpace$ means the following identity.
  \begin{equation*}
    \big| e^\prj{\momentVar_k}{\stateVar} \big| = \exp\Re\prj{\momentVar_k}{\stateVar} = 1, \quad k = 1, \ldots, n
  \end{equation*}
  In particular, we have $\twistVar_n = \momentVar_n \in \boundedSpace$; we show $\twistVar_k \in \boundedSpace$ for $k = n-1,\ldots, 1$ by induction and Lemma \ref{lemma:affPart-bounded}.
  \begin{align*}
    \sup_{\stateVar\in\stateSpace}\exp\Re\prj{\twistVar_k}{\stateVar} 
    &= \sup_{\stateVar\in\stateSpace}\big| e^\prj{\twistVar_k}{\stateVar} \big|  \\
    &= \sup_{\stateVar\in\stateSpace}\big| e^\prj{\momentVar_k + \affPart(\Delta t_{k+1}, \twistVar_{k+1})}{\stateVar} \big| \\
    &= \sup_{\stateVar\in\stateSpace}\big| e^\prj{\momentVar_k}{\stateVar} \big| \cdot \big| e^\prj{\affPart(\Delta t_{k+1}, \twistVar_{k+1})}{\stateVar} \big| \\
    &= \sup_{\stateVar\in\stateSpace} \exp\Re\prj{\affPart(\Delta t_{k+1}, \twistVar_{k+1})}{\stateVar} < \infty
  \end{align*}
  Now observe the following identity.
  \begin{align}
    \notag
    &\aff(\underline t, \underline\momentVar, \stateVar) \\
    \notag
    &=\log\Exp_{\Prb_\stateVar} \exp\prj{\underline\momentVar}{\X_{\underline t}} \\
    \label{eq:complex-tower-start}
    &= \log\Exp_{\Prb_\stateVar}\left( \exp\sum_{k=1}^{n-1} \prj{\momentVar_k}{\X_{t_k}} \cdot \exp\prj{\twistVar_n}{\X_{t_n}} \right) \\
    \notag
    &= \log\Exp_{\Prb_\stateVar}\left( \exp\sum_{k=1}^{n-1} \prj{\momentVar_k}{\X_{t_k}} \cdot \Exp_{\Prb_\stateVar}\Big(\exp\prj{\twistVar_n}{\X_{t_n}} |\scrF_{t_{n-1}} \Big) \right) \\
    \notag
    &= \log\Exp_{\Prb_\stateVar}\left( \exp\sum_{k=1}^{n-1} \prj{\momentVar_k}{\X_{t_k}} \cdot \exp\aff(\Delta t_n, \twistVar_n, \X_{t_{n-1}})\right) \\
    \notag
    &= \begin{aligned}[t]
      &\affPart_0(\Delta t_n, \twistVar_n) \\
      &+ \log\Exp_{\Prb_\stateVar}\left( \exp\sum_{k=1}^{n-1} \prj{\momentVar_k}{\X_{t_k}} \cdot \exp\Big( \bprj{\affPart(\Delta t_n, \twistVar_n)}{\X_{t_{n-1}}} \Big)   \right) 
    \end{aligned} \\
    \notag
    &= \begin{aligned}[t]
      &\affPart_0(\Delta t_n, \twistVar_n) \\
      &+ \log\Exp_{\Prb_\stateVar}\left( \exp\sum_{k=1}^{n-2} \prj{\momentVar_k}{\X_{t_k}} \cdot \exp\Big( \bprj{\momentVar_{n-1} + \affPart(\Delta t_n, \twistVar_n)}{\X_{t_{n-1}}} \Big)   \right) 
    \end{aligned} \\
    \label{eq:complex-tower-end}
    &= \begin{aligned}[t]
      &\affPart_0(\Delta t_n, \twistVar_n) \\
      &+ \log\Exp_{\Prb_\stateVar}\left( \exp\sum_{k=1}^{n-2} \prj{\twistVar_k - \affPart(\Delta t_{k+1}, \twistVar_{k+1})}{\X_{t_k}} \cdot \exp\Big( \bprj{\twistVar_{n-1}}{\X_{t_{n-1}}} \Big)   \right) 
    \end{aligned}
  \end{align}
  The final term of (\ref{eq:complex-tower-start}) is identical to that of (\ref{eq:complex-tower-end}) where we have reduced $k = n$ to $k = n-1$.
  Repeating these equalities inductively $k = n -1, \ldots, 1$ will result in the desired identity.
  \begin{align*}
    \aff(\underline t, \underline\momentVar, \stateVar) 
    &= \sum_{k=2}^n \affPart_0(\Delta t_k, \twistVar_k) + \log\Exp_{\Prb_\stateVar}\exp\prj{\twistVar_1}{\X_{t_1}} \\
    &= \sum_{k=1}^n \affPart_0(\Delta t_k, \twistVar_k) + \bprj{\affPart(\Delta t_1, \twistVar_1)}{\stateVar}
  \end{align*}
\end{proof}


As the preceding result shows, the $\stateSpace$-affine structure of $\aff$ allows us to iteratively factor the exponentials in our expectation.
The problem with extending this to real moments like in Theorem \ref{theorem:KM} is that each of the produced quantities $\twistVar_k$ need not produce an integrable exponential on which we apply the transform formula.
The next result is our way of coercing such a property to occur; the map $\twistToMoment_{\underline t}$ serves to parameterize those moments $\underline\momentVar \in \vecSpace^{|\underline t|}$ which the resulting $\underline\twistVar$ is in $\prod_{k=1}^{|\underline t|} \momentSpace(\Delta t_k)$, since this is precisely the set on which we may perform the calculations between (\ref{eq:complex-tower-start}) and (\ref{eq:complex-tower-end}).
This set turns out to be important in our discussion, so we will reserve it special notation.
\begin{equation*}
  \momentSpace(\underline t) \defeq \prod_{k=1}^{|\underline t|} \momentSpace(\Delta t_k), \quad \underline t \vdash [0,\infty)
\end{equation*}

\begin{proposition}
  \label{proposition:twist-to-moment}
  For each $\underline t \vdash [0,\infty)$, the following map $\twistToMoment_{\underline t}$ is a continuous injection, where we denote $n \defeq |\underline t|$ for brevity.
  \begin{equation*}
    \twistToMoment_{\underline t}: \momentSpace(\underline t) \rightarrow \vecSpace^{|\underline t|}, \quad
    \twistToMoment_{\underline t}(\underline\twistVar) \defeq \Big( \twistVar_1 - \affPart(\Delta t_2, \twistVar_2), \ldots, \twistVar_{n-1} - \affPart(\Delta t_n, \twistVar_n), \twistVar_n \Big)
  \end{equation*}
  Moreover, for each $\stateVar \in \stateSpace$ and $\underline\twistVar \in \momentSpace(\underline t)$, we have the following (finite) identity.
  \begin{equation}
    \label{eq:aff-extension}
    \aff\big(\underline t, \twistToMoment_{\underline t}(\underline\twistVar), \stateVar\big) 
    = \sum_{k=1}^{|\underline t|} \affPart_0(\Delta t_k, \twistVar_k) + \bprj{\affPart(\Delta t_1, \twistVar_1)}{\stateVar}
  \end{equation}
\end{proposition}
\begin{proof}
  \label{proof:proposition:twist-to-moment}
  Fix $\underline\twistVar \in \momentSpace(\underline t)$.
  By definition, this means that to each $k = 1, \ldots, |\underline t|$, we have $\twistVar_k \in \momentSpace(\Delta t_k)$, and so $\affPart(\Delta t_k, \twistVar_k)$ is well-defined.
  This ensures that $\twistToMoment_{\underline t}$ is well-defined.
  Now select another point $\underline\twistVar' \in \momentSpace(\underline t)$ such that $\twistToMoment_{\underline t}(\underline\twistVar) = \twistToMoment_{\underline t}(\underline\twistVar')$.
  The final component of $\twistToMoment_{\underline t}$ ensures that $\twistVar_n = \twistVar_n'$; by means of induction, we then get $\twistVar_{k-1} = \twistVar_{k-1}'$ for $k = n, \ldots, 2$, via the equality on the respective component map.
  \begin{equation*}
    \twistVar_{k-1} - \affPart(\Delta t_k, \twistVar_k) = \twistToMoment_{\underline t,k-1}(\underline\twistVar) = \twistToMoment_{\underline t,k-1}(\underline\twistVar')   = \twistVar_{k-1}' - \affPart(\Delta t_k, \twistVar_k')
  \end{equation*}
  This indicates to us that $\twistToMoment_{\underline t}$ is an injection, and continuity comes simply from continuity of each $\affPart(\Delta t_k, \cdot)$ via Proposition \ref{proposition:momentSpace-facts}\ref{proposition:momentSpace-facts:3}.

  It now remains to show the identity in (\ref{eq:aff-extension}).
  This reduces down to repeatedly applying iterated expectations; fix $\stateVar \in \stateSpace$ and observe the following.
  \begin{align}
    \notag
    &\aff\big(\underline t, \twistToMoment_{\underline t}(\underline\twistVar), \stateVar\big) \\
    \notag
    &=\log\Exp_{\Prb_\stateVar} \exp\bprj{\twistToMoment_{\underline t}(\underline\twistVar)}{\X_{\underline t}} \\
    \label{eq:tower-start}
    &= \log\Exp_{\Prb_\stateVar}\left( \exp\sum_{k=1}^{n-1} \bprj{\twistVar_k - \affPart(\Delta t_{k+1}, \twistVar_{k+1})}{\X_{t_k}} \cdot \exp\bprj{\twistVar_n}{\X_{t_n}} \right) \\
    \notag
    &= \log\Exp_{\Prb_\stateVar}\left( \exp\sum_{k=1}^{n-1} \bprj{\twistVar_k - \affPart(\Delta t_{k+1}, \twistVar_{k+1})}{\X_{t_k}} \cdot \Exp_{\Prb_\stateVar}\Big(\exp\bprj{\twistVar_n}{\X_{t_n}} |\scrF_{t_{n-1}} \Big) \right) \\
    \notag
    &= \log\Exp_{\Prb_\stateVar}\left( \exp\sum_{k=1}^{n-1} \bprj{\twistVar_k - \affPart(\Delta t_{k+1}, \twistVar_{k+1})}{\X_{t_k}} \cdot \exp\aff(\Delta t_n, \twistVar_n, \X_{t_{n-1}})\right) \\
    \notag
    &= \begin{aligned}[t]
      &\affPart_0(\Delta t_n, \twistVar_n) \\
      &+ \log\Exp_{\Prb_\stateVar}\left( \exp\sum_{k=1}^{n-1} \bprj{\twistVar_k - \affPart(\Delta t_{k+1}, \twistVar_{k+1})}{\X_{t_k}} \cdot \exp\Big( \bprj{\affPart(\Delta t_n, \twistVar_n)}{\X_{t_{n-1}}} \Big)   \right) 
    \end{aligned} \\
    \label{eq:tower-end}
    &= \begin{aligned}[t]
      &\affPart_0(\Delta t_n, \twistVar_n) \\
      &+ \log\Exp_{\Prb_\stateVar}\left( \exp\sum_{k=1}^{n-2} \prj{\twistVar_k - \affPart(\Delta t_{k+1}, \twistVar_{k+1})}{\X_{t_k}} \cdot \exp\Big( \bprj{\twistVar_{n-1}}{\X_{t_{n-1}}} \Big)   \right) 
    \end{aligned}
  \end{align}
  The final term of (\ref{eq:tower-end}) is identical to that of (\ref{eq:tower-start}), where we have reduced $k = n$ to $k = n-1$.
  Repeating these equalities inductively $k = n -1, \ldots, 1$ will result in the desired identity.
  \begin{equation*}
    \aff\big(\underline t, \twistToMoment_{\underline t}(\underline\twistVar), \stateVar\big) =  \sum_{k=2}^n \affPart_0(\Delta t_k, \twistVar_k)  + \log\Exp_{\Prb_\stateVar} \exp\prj{\twistVar_1}{\X_{t_1}} = \sum_{k=1}^n \affPart_0(\Delta t_k, \twistVar_k) + \bprj{\affPart(\Delta t_1, \twistVar_1)}{\stateVar}
  \end{equation*}
\end{proof}


We now turn to the analogue of Theorem \ref{theorem:KM}\ref{theorem:KM:2}, in which $\Prb_\stateVar$-finite moments $\underline\momentVar \in \vecSpace^{|\underline t|}$ for initial points $\stateVar \in \stateSpace^\circ$ are precisely those $\underline\momentVar \in \momentSpace(\underline t)$.

\begin{proposition}
  \label{proposition:moment-to-twist}
  Fix $\underline t \vdash [0,\infty)$ and denote $n \defeq |\underline t|$ for brevity.
  If $\underline\momentVar \in \vecSpace^{|\underline t|}$ is such that $\aff(\underline t, \underline\momentVar, \stateVar) < \infty$ for some $\stateVar \in \stateSpace^\circ$, then the following recursion holds.
  \begin{equation}
    \label{eq:moment-to-twist}
    \begin{aligned}
      \twistVar_n &\defeq \momentVar_n \in \momentSpace(\Delta t_n) \\
      \twistVar_k &\defeq \momentVar_k + \affPart\big(\Delta t_{k+1}, \twistVar_{k+1}\big) \in \momentSpace(\Delta t_k), & k = n-1, \ldots, 1
    \end{aligned}
  \end{equation}
\end{proposition}
\begin{proof}
  \label{proof:proposition:moment-to-twist}
  Consider $\underline\momentVar \in \vecSpace^{|\underline t|}$ from which we may not construct the recursion in (\ref{eq:moment-to-twist}).
  In other words, there exists maximal $j \in \{1, \ldots, n\}$ in the recursion which fails; i.e.\ $\twistVar_k \in \momentSpace(\Delta t_k)$ for all $k = n, \ldots, j+1$ and $\twistVar_j \not\in\momentSpace(\Delta t_j)$.
  We now repeat the work as in (\ref{eq:tower-start})-(\ref{eq:tower-end}) for a fixed $\stateVar \in \stateSpace^\circ$ to get the following identity.
  \begin{align*}
    &\log\Exp_{\Prb_\stateVar}\exp\prj{\underline\momentVar}{\X_{\underline t}} \\
    &= \log\Exp_{\Prb_\stateVar}\bigg( \exp\Big(\sum_{k=1}^{n-1} \prj{\momentVar_k}{\X_{t_k}}\Big) \cdot \Exp_{\Prb_\stateVar}\Big( \exp\bprj{\momentVar_n}{\X_{t_n}} |\scrF_{t_{n-1}} \Big) \bigg) \\
    &= \begin{aligned}[t]
      &\affPart_0(\Delta t_n, \momentVar_n) \\
      &+ \log\Exp_{\Prb_\stateVar}\bigg( \exp\Big(\sum_{k=1}^{n-2} \prj{\momentVar_k}{\X_{t_k}}\Big) \cdot \Exp_{\Prb_\stateVar}\Big( \exp\bprj{\momentVar_{n-1} + \affPart(\Delta t_n, \momentVar_n)}{\X_{t_{n-1}}} | \scrF_{t_{n-2}} \Big)  \bigg) 
    \end{aligned} \\
    &= \begin{aligned}[t]
      &\affPart_0\big(\Delta t_n, \twistVar_n) \\
      &+ \log\Exp_{\Prb_\stateVar}\bigg( \exp\Big(\sum_{k=1}^{n-2} \prj{\momentVar_k}{\X_{t_k}}\Big) \cdot \Exp_{\Prb_\stateVar}\Big( \exp\bprj{\twistVar_{n-1}}{\X_{t_{n-1}}} | \scrF_{t_{n-2}} \Big)  \bigg) 
    \end{aligned} \\
    & \quad\vdots \\
    &= \sum_{k=j+1}^n \affPart_0(\Delta t_k, \twistVar_k) + \log\Exp_{\Prb_\stateVar}\bigg( \exp\Big(\sum_{k=1}^{j-1} \prj{\momentVar_k}{\X_{t_k}}\Big) \cdot \Exp_{\Prb_\stateVar}\Big( \exp\bprj{\twistVar_j}{\X_{t_j}} |\scrF_{t_{j-1}} \Big) \bigg) \\
    &= \sum_{k=j+1}^n \affPart_0(\Delta t_k, \twistVar_k) + \log\Exp_{\Prb_\stateVar}\bigg( \exp\Big(\sum_{k=1}^{j-1} \prj{\momentVar_k}{\X_{t_k}}\Big) \cdot \Exp_{\Prb_{\X_{t_{j-1}}}} \exp\bprj{\twistVar_j}{\X_{\Delta t_j}} \bigg)
  \end{align*}
  By Theorem \ref{theorem:KM}, since $\twistVar_j \not\in \momentSpace(\Delta t_j)$, we have $\Exp_{\Prb_{\stateVar'}} \exp\prj{\twistVar_j}{\X_{\Delta t_j}} = \infty$ for all $\stateVar' \in \stateSpace^\circ$, so the above integrand is infinite on the set $\X_{t_{j-1}} \in \stateSpace^\circ$.
  Seeing as $\stateVar \in \stateSpace^\circ$, this set is $\Prb_\stateVar$ non-negligible, and so the quantity is infinite.
  We conclude that $\underline\momentVar \not\in \momentSpace(\underline t)$, which finishes the proof by contrapositive.
\end{proof}


Our final result of this section explores more on how finite moments $\underline\momentVar$ of $\X_{\underline t}$ relate to those $\underline\twistVar$ of the increments $\X_{t_1}-\X_{t_0}, \X_{t_2} - \X_{t_1}, \ldots, \X_{t_n} - \X_{t_{n-1}}$.
To see this, we define the following increment cumulant generating function,
\begin{equation*}
  \affShift(t, \twistVar, \stateVar) \defeq \log\Exp_{\Prb_\stateVar}\exp\prj{\twistVar}{\X_t-\stateVar} = \aff(t, \twistVar, \stateVar) - \prj{\twistVar}{\stateVar}
\end{equation*}

\begin{theorem}
  \label{theorem:mgf-fdds}
  Fix $\underline t \vdash [0,\infty)$ and $\stateVar_0 \in \stateSpace^\circ$.
  The map $\twistToMoment_{\underline t}$ is a homeomorphism from $\momentSpace(\underline t)$ to the collection of $\underline\momentVar \in \vecSpace^{|\underline t|}$ for which $\aff(\underline t, \underline\momentVar, \stateVar_0) < \infty$.
  In particular, this means $\underline\momentVar \in \vecSpace^{|\underline t|}$ satisfies $\aff(\underline t, \underline\momentVar, \stateVar_0) < \infty$ if and only if $\underline\momentVar = \twistToMoment_{\underline t}(\underline\twistVar)$.
  Moreover, we have the following identity for all $\underline\stateVar \in \stateSpace^{|\underline t|}$.
  \begin{equation*}
    \prj{\underline\momentVar}{\underline\stateVar} - \aff(\underline t, \underline\momentVar, \stateVar_0) = \sum_{k=1}^{|\underline t|} \Big( \prj{\twistVar_k}{\stateVar_k-\stateVar_{k-1}} - \affShift(\Delta t_k, \twistVar_k, \stateVar_{k-1}) \Big), \quad \underline\momentVar = \twistToMoment_{\underline t}(\underline\twistVar)
  \end{equation*}
\end{theorem}
\begin{proof}
  \label{proof:theorem:mgf-fdds}
  By Proposition \ref{proposition:twist-to-moment}, we have that $\twistToMoment$ is indeed a continuous map from $\momentSpace(\underline t)$ into the finite domain of $\aff(\underline t, \cdot, \stateVar_0)$.
  Conversely, Proposition \ref{proposition:moment-to-twist} indicates to us that, on the finite domain of $\aff(\underline t, \cdot, \stateVar_0)$, a recursively-defined map $\momentToTwist_{\underline t}$ from (\ref{eq:moment-to-twist}) exists.
  Denoting $n \defeq |\underline t|$, we see that this map is continuous by induction and continuity of compositions.
  \begin{equation*}
    \momentToTwist_{\underline t}(\underline\momentVar) = \Big( \momentToTwist_{\underline t, 1}(\underline\momentVar_{1:n}), \ldots, \momentToTwist_{\underline t, n}(\underline\momentVar_{n:n}) \Big),  \quad
    \begin{aligned}[t]
      \momentToTwist_{\underline t,n}(\underline\momentVar_{n:n}) &= \momentVar_n \\
      \momentToTwist_{\underline t,k}(\underline\momentVar_{k:n}) &= \momentVar_k + \affPart\big(\Delta t_{k+1}, \momentToTwist_{\underline t}(\underline\momentVar_{k+1:n}) \big)
    \end{aligned}
  \end{equation*}

  Observe that $\momentToTwist_{\underline t}$ is the inverse of $\twistToMoment_{\underline t}$.
  To see this, fix $\underline\twistVar \in \momentSpace(\underline t)$ and $\underline\momentVar \defeq \twistToMoment_{\underline t}(\underline\twistVar)$.
  The final coordinate is obvious,
  \begin{equation*}
    \momentToTwist_{\underline t, n}(\underline\momentVar_{n:n}) = \momentVar_n = \twistToMoment_{\underline t,n}(\underline\twistVar) = \twistVar_n,
  \end{equation*}
  while an inductive hypothesis $\momentToTwist_{\underline t, k}(\underline\momentVar_{k:n}) = \twistVar_k$ gives us the next step.
  \begin{align*}
    \momentToTwist_{\underline t, k-1}(\underline\momentVar_{k-1:n}) 
    &= \twistToMoment_{\underline t, k-1}(\underline\twistVar) + \affPart\big(\Delta t_k, \momentToTwist_{\underline t, k}(\underline\momentVar_{k:n}) \big) \\
    &= \twistVar_{k-1} - \affPart(\Delta t_k, \twistVar_k) + \affPart(\Delta t_k, \twistVar_k )\\
    &= \twistVar_{k-1}
  \end{align*}
  Dual to this, fix $\underline\momentVar \in \vecSpace^{|\underline t|}$ for which $\aff(\underline t, \underline\momentVar, \stateVar_0) < \infty$ and define $\underline\twistVar \defeq \momentToTwist_{\underline t}(\underline\momentVar)$.
  Again, we immediately have
  \begin{equation*}
    \twistToMoment_{\underline t, n}(\underline\twistVar) = \twistVar_n = \momentToTwist_{\underline t, n}(\underline\momentVar_{n:n}) = \momentVar_n,
  \end{equation*}
  and an inductive hypothesis of $\twistToMoment_{\underline t, k}(\underline\twistVar) = \momentVar_k$ results in the next step.
  \begin{equation*}
    \twistToMoment_{\underline t, k-1}(\underline\twistVar) = \twistVar_{k-1} - \affPart(\Delta t_k, \twistVar_k) = \momentToTwist_{\underline t,k-1}(\underline\momentVar_{k-1:n}) - \affPart\big(\Delta t_k, \momentToTwist_{\underline t, k}(\underline\momentVar_{k:n})\big) = \momentVar_{k-1}
  \end{equation*}

  We have now showed that $\twistToMoment_{\underline t}$ is a homeomorphism with inverse $\momentToTwist_{\underline t}$.
  It remains to show our identity for a pairing $\underline\momentVar = \twistToMoment_{\underline t}(\underline\twistVar)$.
  \begin{align*}
    &\prj{\underline\momentVar}{\underline\stateVar} - \aff(\underline t, \underline\momentVar, \stateVar_0) \\
    &= \bprj{\twistToMoment_{\underline t}(\underline\twistVar)}{\underline\stateVar} - \aff\big(\underline t, \twistToMoment_{\underline t}(\underline\twistVar), \stateVar_0\big) \\
    &= \sum_{i=1}^{n-1} \bprj{\twistVar_k - \affPart(\Delta t_{k+1}, \twistVar_{k+1})}{\stateVar_k} + \prj{\twistVar_n}{\stateVar_n} - \sum_{i=1}^n \affPart_0(\Delta t_k, \twistVar_k) - \bprj{\affPart(\Delta t_1, \twistVar_1)}{\stateVar_0}  \\
    &= \sum_{i=1}^n \Big( \prj{\twistVar_k}{\stateVar_k} - \affPart_0(\Delta t_k, \twistVar_k) - \bprj{\affPart(\Delta t_k, \twistVar_k)}{\stateVar_k} \Big) \\
    &= \sum_{i=1}^n \Big( \prj{\twistVar_k}{\stateVar_k} - \aff(\Delta t_k, \twistVar_k, \stateVar_k) \Big) \\
    &= \sum_{i=1}^n \Big( \prj{\twistVar_k}{\stateVar_k-\stateVar_{k-1}} - \affShift(\Delta t_k, \twistVar_k, \stateVar_k) \Big)
  \end{align*}
\end{proof}

