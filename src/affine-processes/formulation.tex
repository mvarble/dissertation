% specify the state space
We start by specifying our affine processes as in \cite{keller2015}.
That is to say, we fix a finite-dimensional real vector space $\vecSpace$ with inner-product $\prj\cdot\cdot$ and select a convex, closed $\stateSpace \subseteq \vecSpace$ satisfying $0 \in \stateSpace$ and $\operatorname{span}\stateSpace = \vecSpace$.
Associate this space with the finite exponentials.
\begin{equation*}
  \boundedSpace \defeq \Big\{ \momentVar \in \vecSpace \oplus \im\vecSpace : \sup_{\stateVar \in \stateSpace} \exp\prj{\Re(\momentVar)}{\stateVar} < \infty \Big\}
\end{equation*}
We may now define the notion of an affine process on $\stateSpace$.

% define an affine process
\begin{definition}
  \label{definition:affine-process}
  For a probability space $(\Omega, \Sigma, \Prb)$ with filtration $\scrF = (\scrF_t)_{t\geq0}$, an affine process $\X$ on $\stateSpace$ is a stochastically continuous, time-homogeneous $(\Prb, \scrF)$-Markov process on $\stateSpace$ in which the bounded moments have the following log-affine dependence on the initial state.
  \begin{align}
    \label{eq:affine-definition}
    &\begin{aligned}
      \Exp_{\Prb_\stateVar}\exp\prj{\momentVar}{\X_t} &= \exp\aff(t, \momentVar, \stateVar) \\
      \aff(t, \momentVar, \stateVar) &= \affPart_0(t, \momentVar) + \bprj{\affPart(t, \momentVar)}{\stateVar},  \\
    \end{aligned}
    & t \geq 0, ~ \momentVar \in \boundedSpace
  \end{align}
  Above, we are denoting $(\Prb_\stateVar)_{\stateVar\in\stateSpace}$ the conditional distributions of $\X$ factored through the initial state (see Appendix \ref{appendix:markov} for further specification and notation).
\end{definition}


% clear up our definition with the literature
\begin{remark}
  \label{remark:affine-remarks}
  \begin{enumerate}[label=(\alph*)]
    \item
      See \cite[Remark 2.3]{keller2015} for an argument on how our assumptions on $\stateSpace$ are at no loss of generality; $\stateSpace$ may as well be any nonempty convex set.
    \item
      Note how (\ref{eq:affine-definition}) specifies the characteristic function of each transition kernel of the Markov process $\X$; thus, should an affine process exist for choice of $\aff$, \emph{only} one will exist, up to distribution.
    \item
      See how our notation $(\affPart_0, \affPart)$ differs from that of \cite{keller2015} and other papers, which typically use $(\phi, \affPart)$.
      We choose to do this because affine functions prevail throughout our investigation of affine processes, and we saw this an opportunity to have more cohesive notation of all such affine functions.
      \begin{equation*}
        \affDiff(\stateVar) = \affDiffPart_0 + \sum_{i=1}^d \stateVar^i \affDiffPart_i
      \end{equation*}
    \item
      \label{remark:affine-parts}
      If we have a vector space $\bbA$ and affine map $\alpha: \stateSpace \rightarrow \bbA$ determined by $a_0, \ldots, a_d \in \bbA$ via $\alpha(\stateVar) = a_0 + \sum_{i=1}^d \stateVar^i a_i$, then our linear assumptions $0 \in \stateSpace$ and $\operatorname{span}\stateSpace = \vecSpace$ uniquely determine $a_0, \ldots, a_d \in \bbA$.
      In particular, the map $\aff$ uniquely identifies its parts $\affPart_i: \bbR_+ \times \boundedSpace \rightarrow \bbC$ for $i=0,\ldots,d$.
  \end{enumerate}
\end{remark}


In \cite[Theorem 1.2.7]{cuchiero2011}, it is shown that, without loss of generality on conditional distributions $(\Prb_\stateVar)_{\stateVar\in\stateSpace}$, an affine process $\X$ can be chosen to have c\`adl\`ag paths.
Thus, each distribution $\Prb_\stateVar$ may be recognized as a measure on the Borel algebra associated with the space $\pathSpace{[0,\infty)}{\stateSpace}$ of c\`adl\`ag functions equipped with the Skorokhod topology.
By imposing this regularity, the following theorem tells us that an affine process $\X$ as in Definition \ref{definition:affine-process} is a $(\Prb_\stateVar, \scrF)$ jump-diffusion for each $\stateVar \in \stateSpace$.
For relevant definitions and results pertaining to jump-diffusions, we refer the reader to Appendix \ref{jump-diffusions}.

% state the existence of \ode, \affDrift, \affDiff, \affJump
\begin{theorem}
  \label{theorem:affine-regularity}
  Fix an affine process $\X$ on $\stateSpace$.
  There exist $\affDriftPart^\trunc_0, \ldots, \affDriftPart^\trunc_d \in \vecSpace$, $\affDiffPart_0, \ldots, \affDiffPart_d \in \linSpace(\vecSpace)$, and $\affJumpPart_0, \ldots, \affJumpPart_d \in \signedMeasSpace{\scrB(\vecSpace)}$ such that the following maps $L_0,\ldots,L_d: \boundedSpace \rightarrow \bbR$,
  \begin{equation*}
    \odePart_i(\momentVar)
    \defeq \bprj{\momentVar}{\affDriftPart_i^\trunc(\stateVar)} + \frac12\bprj{\momentVar}{\affDiffPart_i(\stateVar)\momentVar} + \int_\vecSpace \Big( e^\prj{\momentVar}{\markVar} - 1 - \bprj{\momentVar}{\trunc(\markVar)} \Big) \affJumpPart_i(\stateVar, \rmd\markVar),
  \end{equation*}
  determine the dynamics of each $\big(\affPart_0(\cdot,\momentVar), \affPart(\cdot,\momentVar))$.
  \begin{equation*}
    \left\{\begin{aligned}
      \affPart_0(t,\momentVar) &= \odePart_0\big(\affPart(t, \momentVar)\big) & t \geq 0 \\
      \affPart(t,\momentVar) &= \odePart\big(\affPart(t,\momentVar)\big) & t \geq 0 \\
      \affPart_0(0,\momentVar) &= 0 \\
      \affPart(0, \momentVar) &= \momentVar
    \end{aligned}\right.
  \end{equation*}
\end{theorem}

\begin{remark}
  \label{remark:riccati-affine-formulation}
  By Remark \ref{remark:affine-remarks}\ref{remark:affine-parts}, the equation in (\ref{eq:complex-riccati-charts}) is equivalent to the following system of equations.
  \begin{equation}
    \label{eq:complex-riccati-system}
    \forall ~ \stateVar \in \stateSpace, \qquad \left\{\begin{aligned}
      \dot\aff(t, \momentVar, \stateVar) &= \ode\big(\affPart(t,\momentVar), \stateVar\big) & t \geq 0 \\
      \aff(0,\momentVar, \stateVar) &= \prj{\momentVar}{\stateVar}
    \end{aligned}\right.
  \end{equation}
\end{remark}


% fix our "base" affine process
Henceforth, we fix $\X$ a c\`adl\`ag affine process with conditional distributions $(\Prb_\stateVar)_{\stateVar\in\stateSpace}$ on $\pathSpace{[0,\infty)}{\stateSpace}$, induced filtration $\scrF=(\scrF_t)_{t\geq0}$, and moment function $\aff$ as in Definition \ref{definition:affine-process}.
We will use the truncation function $\trunc(\markVar) = \markVar 1_{|\markVar|\leq 1}$ and fix the differential $\trunc$-characteristics $(\affDrift^\trunc, \affDiff, \affJump)$ and L\'evy-Khintchine map $\ode$ as in Theorem \ref{theorem:affine-regularity}.
