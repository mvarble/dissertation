In order to discuss jump-diffusions on a finite-dimensional real vector space, one must have a decent understanding of semimartingales.
A great text for a comprehensive study of this is \cite{jacod2003}, which we will refer to in our proofs.
In terms of notational differences, we choose our probability space $(\Omega, \Sigma, \Prb)$ and filtration $\scrF=(\scrF_t)_{t\geq0}$, where $\scrF_\infty \subseteq \Sigma$ denotes the joined space.
Furthermore, we do not explicitly write processes to take values in $\bbR^d$, but rather some vector space $\vecSpace$ with dimension $d \defeq \operatorname{dim}\vecSpace$ and inner-product $\prj\cdot\cdot$.
Surely---due to our isometric isomorphism $\vecSpace \equiv \bbR^d$---any componentwise or linear notion, such as integration or differentiation may be taken as equivalent.
Furthermore, we sometimes specify that a stochastic process $\X$ has a Borel state space $\stateSpace\subseteq \vecSpace$, as this is the case when studying affine processes.
We find it important to highlight the following important notation of objects introduced in \cite[Chapters I-II]{jacod2003}.
\begin{itemize}
  \item
    Given $(\Prb,\scrF)$ locally square-integrable martingales $M, N: \Omega \times \bbR_+ \rightarrow \bbR$, denote $\prj{M}{N}$ the predictable quadratic covariation.
  \item
    Given $H,\X: \Omega \times \bbR_+ \rightarrow \bbR$ with $H$ being $\scrF$ predictable and $(\Prb, \scrF)$ locally bounded and $\X$ a $(\Prb, \scrF)$ semimartingale, denote the stochastic integral process as follows.
    \[
      H\shortInt \X_t = \int_0^t H_s \rmd\X_s
    \]
    We may lift this concept componentwise and linearly.
    This allows us to choose the codomains of $H, \X$ to various combinations of $\vecSpace$ and $\bbL(\vecSpace, \bbW)$ when evaluating $H \shortInt \X$, so long as such a combination allows for $H_t \cdot \X_t$ to make sense.
  \item
    Denote $\leb: \bbR_+ \rightarrow \bbR_+$ the identity map to allow a concise notation for Lebesgue integration.
    \[
      H \shortInt \leb_t = \int_0^t H_s \rmd s
    \]
    Throughout, we will often use the following fact without mention; any c\`adl\`ag process $H$ is such that $H_t=H_{t-}$ at all but a countable amount of times $t \in \bbR$, and so 
    \[
      H_- \shortInt \leb_t = \int_0^t H_{s-} \rmd s = \int_0^t H_s \rmd s = H \shortInt \leb_t
    \]
  \item
    Given a random measure $\jumpMeas: \Omega \times \scrB(\bbR_+ \times \vecSpace) \rightarrow [0,\infty]$, denote the stochastic integral process against some suitably integrable process $H: \Omega \times \bbR_+ \times \vecSpace \rightarrow \bbR$ as follows.
    \[
      H \ast \jumpMeas_t = \int_{[0,t] \times \vecSpace} H_s(\markVar) \jumpMeas(\rmd s, \rmd\markVar)
    \]
    Denote its $(\Prb, \scrF)$ predictable projection by $\predproj{\jumpMeas}$ and the compensated measure $\compensate{\jumpMeas} = \jumpMeas - \predproj{\jumpMeas}$.
    We will frequently use, without mention, that $\predproj{\jumpMeas}$ is characterized by the property that any $\scrF$ predictable $H: \Omega \times \bbR_+ \times \vecSpace \rightarrow \bbR_+$ is such that
    \[
      \Exp_\Prb\big( H \ast \jumpMeas_\infty \big)
      = \Exp_\Prb\big( H \ast \predproj{\jumpMeas}_\infty \big)
    \]
    Also denote $H \ast \compensate{\jumpMeas}$ the compensated local martingale process for suitable $H \in G_\loc(\jumpMeas)$, as constructed in \cite[Definition II.1.27]{jacod2003}.
    Lift these integration notions to vector-valued $H$ componentwise.
    Instead of choosing a canonical variable for integrating expressions in this form, we use the identity maps $\id_\vecSpace$ or $\leb$.
    \[
      f(\leb, \id_\vecSpace) \ast \jumpMeas_t = \int_{[0,t] \times \vecSpace} f(s, \markVar) \jumpMeas(\rmd s, \rmd\markVar)
    \]
  \item
    Given $(\Prb, \scrF)$ semimartingales $\X, Y: \Omega \times \bbR_+ \rightarrow \bbR$, denote $[\X,Y]$ the quadratic covariation.
  \item
    Given a semimartingale $\X$, denote $\X^\rmc$ its continuous local martingale component and $\jumpMeas^\X$ its jump measure.
\end{itemize}
