\begin{example}[Compound-Hawkes]
  \label{example:compound-hawkes}
  We could similarly come up with a family $(\Xe)_{\epsilon>0}$ of compound-Hawkes processes with affine intensity $\affInt(\stateVar) = \affIntPart_0 + \affIntPart_1 \stateVar$,
  \begin{gather*}
    \Xe_t = \stateVar + \hawkesRate\big(\mu - \Xe\big) \shortInt \leb_t + \epsilon \sum_{k=1}^{N^\epsilon_t} \V_k, \\
    \N^\epsilon \text{ intensity } \frac1\epsilon\affInt(\Xe), \\
    (\V_k)_{k\in\bbN} \text{ independent and commonly distributed by } \affDist \text{ on } \scrB(\bbR_+)
  \end{gather*}
  A proof for a large deviation principle of $(\Xe, \epsilon\N^\epsilon)_{\epsilon>0}$ could use the steps of Examples \ref{example:hawkes} and \ref{example:compound-poisson}, summarized as below.
  \begin{enumerate}
    \item
      Use Example \ref{example:compound-poisson} to establish a large deviation principle for the family $(\epsilon K^\epsilon, \epsilon \N^\epsilon)_{\epsilon>0}$ where $\N^\epsilon$ is a Poisson process with rate $1/\epsilon$ and $K^\epsilon = \sum_{k=1}^{\N^\epsilon} V_k$ is the accumulated jump process.
      The integral in the rate function will be as (\ref{eq:rf-compound-poisson}).
      \begin{equation*}
        \rf_{K\N}^\text{pois}(\gamma, \eta) = \int_0^\infty \bigg( \dot\eta(t) \log \dot\eta(t) - \dot\eta(t) + 1 + \dot\eta(t) \ode_\affDist^*\Big(\frac{\dot\gamma(t)}{\dot\eta(t)} \Big) \bigg) \rmd t
      \end{equation*}
    \item
      We observe that defining the map $F$,
      \begin{equation*}
        F\gamma(t) = \mu + (\stateVar - \mu)e^{-\hawkesRate t} + \int_0^t e^{-\hawkesRate(t-s)} \rmd\gamma(s),
      \end{equation*}
      will be such that $\Xe = F(\epsilon K^\epsilon)$ gives us our path-properties as above.
      To see this, fix an arbitrary finite-variation $\gamma$ with $\gamma(0) = 0$ and define $\pathVar = F(\gamma)$.
      Observe that applying the fundamental theorem of calculus and Fubini's theorem provides the following equality.
      \begin{equation}
        \label{eq:most-annoying-thing-ever}
        \begin{aligned}[t]
          \pathVar(t)
          &= \mu + (\stateVar - \mu) e^{-\hawkesRate t} + \int_0^t e^{-\hawkesRate(t-s)} \rmd \gamma(s) \\
          &= \stateVar + \big( (\stateVar-\mu)e^{-\hawkesRate t} - (\stateVar - \mu) \big) + \int_0^t \Big( 1 + \big( e^{-\hawkesRate(t-s)} - e^{-\hawkesRate(s-s)} \big) \big) \rmd\gamma(s)  \\
          &= \stateVar - \hawkesRate (\stateVar - \mu) e^{-\hawkesRate \leb} \shortInt \leb_t + \gamma(t) - \int_0^t \int_s^t \hawkesRate e^{-\hawkesRate(\tau-s)} \rmd\tau \rmd\gamma(s) \\
          &= \stateVar + \hawkesRate \Big( \mu - \big( \mu + (\stateVar - \mu) e^{-\hawkesRate\leb} \big) \Big) \shortInt\leb_t + \gamma(t) - \int_0^t \int_0^\tau \hawkesRate e^{-\hawkesRate(\tau-s)} \rmd\gamma(s) \rmd\tau \\
          &= \stateVar + \int_0^t \hawkesRate \Big( \mu - \big( \mu + (\stateVar - \mu) e^{-\hawkesRate\tau} - \int_0^\tau e^{-\hawkesRate(\tau-s)} \rmd\gamma(s) \big) \Big) \rmd \tau + \gamma(t) \\
          &= \stateVar + \int_0^t \hawkesRate \big( \mu - \pathVar(\tau) \big) \rmd \tau + \gamma(t) \\
          &= \stateVar + \hawkesRate( \mu - \pathVar ) \shortInt \leb_t + \gamma(t)
        \end{aligned}
      \end{equation}
      Thus, if we use the contraction map $(\gamma, \eta) \mapsto (F\gamma, \eta)$, we will get a large deviation principle for $(\Xe, \epsilon\N^\epsilon)$.
      Note that if an absolutely continuous $\gamma$ is such that $F\gamma = \pathVar$, then (\ref{eq:most-annoying-thing-ever}) tells us
      \begin{equation*}
        \dot\pathVar(t) = \hawkesRate\big(\mu - \pathVar(t)\big) + \dot\gamma(t),
      \end{equation*}
      and so the integral in our rate function now evaluates as follows.
      \begin{equation*}
        \rf_{\X\N}^\text{pois}(\pathVar, \eta) = \int_0^\infty \bigg( \dot\eta(t) \log \dot\eta(t) - \dot\eta(t) + 1 + \dot\eta(t) \ode_\affDist^*\Big(\frac{\dot\pathVar(t) - \hawkesRate(\mu-\pathVar(t))}{\dot\eta(t)} \Big) \bigg) \rmd t
      \end{equation*}
    \item
      At last, we apply our measure-change argument as in Example \ref{example:hawkes} to change the intensity of $\N^\epsilon$ to $\frac1\epsilon\affInt(\Xe)$.
      The work is the exact same, and so our rate function associated with $(\Xe, \epsilon\N^\epsilon)_{\epsilon>0}$ will become the following.
      \begin{equation}
        \label{eq:rf-compound-hawkes}
        \rf_{\X\N}(\pathVar, \eta) = \int_0^\infty \bigg( \dot\eta(t) \log\Big( \frac{\dot\eta(t)}{\affInt\big(\pathVar(t)\big)} \Big) - \dot\eta(t) + \affInt\big(\pathVar(t)\big) + \dot\eta(t) \ode_\affDist^*\Big(\frac{\dot\pathVar(t) - \hawkesRate(\mu-\pathVar(t))}{\dot\eta(t)} \Big) \bigg) \rmd t
      \end{equation}
  \end{enumerate}
  Instead of working through the numerous details of the lengthy argument above, we can again use Theorem \ref{theorem:ldp-integral} and its rate function (\ref{eq:rf-integral}).
  By work nearly identical to the end of Example \ref{example:compound-poisson}, we can see that the special differential characteristics of $\hat\Xee = (\Xee, \N^\epsilon)$ are as below.
  \begin{equation*}
    \affDrift(\hat\stateVar) = \begin{pmatrix} \hawkesRate(\kappa - \stateVar_1) + \affInt(\stateVar_1)\overline{\affDist} \\ \affInt(\stateVar_1) \end{pmatrix}, \quad
    \affDrift(\hat\stateVar) = 0, \quad
    \affJump(\hat\stateVar, \hat\markVar) = \affInt(\stateVar_1) \affDist(\rmd\markVar_1) \delta_1(\rmd\markVar_2)
  \end{equation*}
  Our associated L\'evy-Khintchine map is then the following $\ode$.
  \begin{align*}
    \ode(\hat\momentVar, \hat\stateVar) 
    &= \momentVar_1\big(\hawkesRate(\mu - \stateVar_1) + \overline\affDist\big) + \momentVar_2 \affInt(\stateVar_1) + \int_{\bbR_+^2} \Big( e^{\momentVar_1\markVar_1 + \momentVar_2} - 1 - \momentVar_1\markVar_1 - \momentVar_2\markVar_2 \Big) \affDist(\rmd\markVar_1)\delta_1(\rmd\markVar_2) \\
    &= \momentVar_1 \hawkesRate(\mu-\stateVar_1) + e^{\ode_\affDist(\momentVar_1)+\momentVar_2} - 1
  \end{align*}
  Almost identical calculus to Example \ref{example:compound-poisson} also provides us with the following expression, which agrees with \ref{eq:rf-compound-hawkes}.
  \begin{equation*}
    \ode^*(\hat\diffVar, \hat\stateVar) = \dot\stateVar_2 \log\Big(\frac{\dot\stateVar_2}{\affInt(\stateVar_1)} \Big) - \dot\stateVar_2 + \affInt(\stateVar_1) + \dot\stateVar_2 \ode_\affDist^*\Big( \frac{\dot\stateVar_1 - \hawkesRate(\mu - \stateVar_1)}{\dot\stateVar_2} \Big)
  \end{equation*}
\end{example}
